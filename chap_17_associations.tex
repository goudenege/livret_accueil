%%%%%%%%%%%%%%%%%%%%%%%%%%%%%%%%%%%%%%%%%%%%%%%%%%
%%%%%%%%%%%%%%%%%%%%%%%%%%%%%%%%%%%%%%%%%%%%%%%%%%
\chapter{Les associations}


%%%%%%%%%%%%%%%%%%%%%%%%%%%%%%
\section{L'AND\`eS}
\index{Association nationale des docteurs \\(AND\`eS)}

\emph{Pr\'esident\mp e actuel\mp le : \verifier{Linda Lahleh}} \hfill Site web : \url{https://www.andes.asso.fr}
%E-mail : {\tt andes.contact@andes.asso.fr}\\
\smallskip

L'Association Nationale des Docteurs est une association r\'egie par la loi du 1er juillet 1901.
Fond\'ee en 1970, elle est reconnue d'utilit\'e publique depuis 1975.

L'AND\`es a trois missions principales :
\begin{itemize}
\item promouvoir le doctorat : mettre en avant la valeur ajout\'ee que repr\'esente l'exp\'erience professionnelle du doctorat pour r\'ev\'eler les comp\'etences des docteur\mp e\mp s ;
\item mettre les talents des docteur\mp e\mp s au service de la soci\'et\'e : contribuer au d\'ecloisonnement des sph\`eres professionnelles en positionnant les docteur\mp e\mp s comme \og{}passeurs de fronti\`eres\fg{},
    tirer parti de l'expertise et des savoirs-faire des docteur\mp e\mp s pour relever les d\'efis du monde de demain ;
\item cr\'eer et mettre en synergie les r\'eseaux de docteur\mp e\mp s : augmenter la visibilit\'e collective des docteur\mp e\mp s,
    permettre \`a chacun\mp e de d\'evelopper son r\'eseau professionnel,
    favoriser les interactions entre cr\'eateurs de r\'eseaux.
\end{itemize}


%%%%%%%%%%%%%%%%%%%%%%%%%%
\section{Animath}
\label{animath}
\index{Animath}
 
\emph{Pr\'esident\mp e actuel\mp le~: \verifier{Fabrice Rouillier}} \hfill Site web~: \url{https://www.animath.fr/}
\smallskip

Animath est une association dont le r\^ole est de promouvoir l’activit\'e math\'ematique chez les jeunes, sous toutes ses
formes, dans les coll\'eges, lyc\'ees et universit\'es, tout en d\'eveloppant le plaisir de faire des math\'ematiques. 

En 1998, les soci\'et\'es savantes (SMF, SMAI), l’Association des professeurs de math\'ematiques de l’enseignement public, l’Inspection g\'en\'erale de math\'ematiques et les diff\'erents acteurs de l’animation math\'ematique (associations comme Maths en Jeans, la FFJM, le CIJM, Kangourou, acteurs institutionnels comme les IREM) ont d\'ecid\'e de cr\'eer l’association Animath, charg\'ee de \og favoriser \fg{} l’introduction, le fonctionnement, le 
d\'eveloppement, la mise en r\'eseau et la valorisation d’activit\'es math\'ematiques dans les \'ecoles, coll`eges, lyc\'ees et \'etablissements de niveau universitaire”. Animath est soutenue par le CNRS et INRIA.

Le premier r\^ole d’Animath est donc d'\^etre la \og maison commune\fg{} des activit\'es math\'ematiques p\'eriscolaires, donc de coordination, d’incitation, d'information et de mise en r\'eseau et de d\'eveloppement des synergies 
entre les diff\'erents acteurs.  Animath a \'et\'e, entre 2012 et 2016, porteuse du consortium Cap’Maths, 
cr\'e\'e dans le cadre de l’appel \`a projet \og Culture scientifique et technique et \'egalit\'e des chances\fg{} dans la cadre du programme \og Ìnvestissements d'avenir\fg{} ; Cap'Maths a apport\'e 2,1 millions d'Euros \ \`a des actions de popularisation des math\'ematiques pendant cette p\'eriode, pour un financement total de 5,7 millions. Le reliquat du financement 
Cap'Maths, soit 900 000 Euros a \'et\'e vers\'e \`a la fondation Blaise Pascal lui donnant ainsi une base de d\'epart. 

Les principaux projets qu'Animath porte, seule ou avec des partenaires, sont : 
\begin{itemize} 
\item les {\it journ\'ees Filles et math\'ematiques, 
une \'equation lumineuse} et les {\it Rendez-vous des jeunes math\'ematiciennes} (avec {\it femmes et math\'ematiques}) 
\footnote{\url{https://filles-et-maths.fr/}}, 
\item les {\it stages MathC2+} (avec la FSMP)
\footnote{\url{https://www.mathc2plus.fr/}},
\item {\it Mathmosph\`ere}, un club et des stages virtuels de math\'ematiques \footnote{\url{https://www.animath.fr/actions/mathmosphere}}
\item un programme de coop\'eration internationale, visant au d\'eveloppement de clubs et de math\'ematiques dans les pays moins riches ou en voie de d\'eveloppement
\footnote{\url{https://www.animath.fr/actions/international/}}
\item les {\it Correspondances math\'ematiques}, qui permettent \`a des \'equipes de lyc\'ennes et lyc\'eens de travailler sur des probl\`emes ouverts de math\'ematiques et d'\'echanger leurs solutions par vid\'eo \footnote{\url{https://correspondances-maths.fr/}} 
\item le {\it concours Alkindi} de cryptographie (avec France IOI) pour \'le\`eves de 4\`eme, 3\`eme, 2nde 
\footnote{\url{http://www.concours-alkindi.fr}} (plus de 60000 participants en 2018-2019)
\item le {\it Tournoi français des jeunes math\'ematiciennes et math\'ematiciens} ($\text{TFJM}^2$) 
\footnote{\url{https://www.tfjm.org/}}, 
\item la {\it Pr\'eparation olympique fran\c caise de math\'ematiques}, organisant pr\'eparation et participation aux comp\'etitions math\'ematiques internationales (de type olympiades)
\footnote{\url{http://maths-olympiques.fr/}}, 
\item un encouragement aux clubs de math\'ematiques \footnote{\url{https://www.animath.fr/actions/clubs/}} et, en coop\'eration avec le Minist\`ere de l'\'education nationale, un soutien au recensement des clubs de math\'ematiques entrepris dans le cadre du plan Villani-Torossian
\footnote{\url{http://eduscol.education.fr/cid139417/clubs-de-mathematiques.html}}
\item la participation des lyc\'een\mp nes aux conf\'erences {\it Un texte, un math\'ematicien} organis\'ees par la SMF et la Biblioth\`que nationale de France \footnote{\url{https://smf.emath.fr/la-smf/cycle-un-texte-un-mathematicien}}. 
\end{itemize}

Pour une contextualisation dans la promotion des math\'ematiques vers le public, voir le paragraphe \ref{jeunes}.\\




%%%%%%%%%%%%%%%%%%%%%%%%%%%%%%%%%%%%%%%%
\section{L'Association Femmes et Math\'ematiques}
\index{Femmes et math\'ematiques (association)}

\emph{Pr\'esident\mp e actuel\mp lee~: \verifier{Anne Boyé}} \hfill Site web~: \url{https://www.femmes-et-maths.fr}
\smallskip

Cr\'e\'ee en 1987 par des math\'ematiciennes, l'association \textit{femmes et math\'ematiques} compte actuellement environ cent cinquante membres (femmes et hommes), principalement des chercheuses et des enseignantes du sup\'erieur ou du secondaire. Parmi ses objectifs:
\begin{itemize}
\item encourager les filles \`a s'orienter vers des \'etudes scientifiques et techniques,
\item promouvoir les femmes dans le milieu scientifique, en particulier math\'ematique,
\item agir pour plus de parit\'e en math\'ematiques,
\item \^etre un lieu de rencontre entre math\'ematiciennes,
\item coop\'erer avec les associations ayant un but analogue en France ou \`a l'\'etranger.
\end{itemize}

L'association \textit{femmes et math\'ematiques}: \\
\textbf{R\'ealise}
\begin{itemize}
\item  des interventions dans des \'etablissements scolaires et universitaires sur le double th\`eme des math\'ematiques et de la place des femmes dans les professions scientifiques,
\item des journ\'es \og Filles et maths : une \'equation lumineuse  \fg destin\'ees \`a encourager les jeunes \`a se lancer dans des \'etudes de math\'ematiques et \`a lutter contre les st\'er\'eotypes sexistes en sciences,
\item des statistiques sexu\'ees sur la pr\'esence des femmes en math\'ematiques,
\item un livret \og Femmes et sciences... au-del\`a des id\'ees re\c cues \fg  avec les associations Femmes et Sciences et Femmes Ing\'enieurs,
\item une brochure \og Zoom sur les m\'etiers des math\'ematiques et de l'informatique \fg, avec les soci\'et\'es savantes de math\'ematiques et d'informatique,
\item Participe \`a des forums de m\'etiers et des salons de l'\'education ou des math\'ematiques dans plusieurs villes en France.
 \end{itemize}

\textbf{Participe \`a}
\begin{itemize}
\item des groupes de travail (Minist\`ere de l'Education Nationale et de la Recherche, Rectorats, Service des droits des femmes et de l'\'egalit\'e),
\item l'\'elaboration de rapports officiels,
\item des colloques math\'ematiques et sur l'\'egalit\'e des sexes, en France et \`a l'\'etranger,
 des manifestations diverses, F\^ete de la Science, Journ\'ee internationale des droits des Femmes le 8 mars, Mondial des m\'etiers, Colloques d'associations amies,
\item des auditions par la commission des affaires culturelles, familiales et sociales de l'Assembl\'ee Nationale, par le Haut conseil de la science et de la technologie, etc.
\item  des op\'erations de \og\ marrainage\fg\  qui se d\'eclinent principalement sous deux formes:
\begin{itemize}
\item des jeunes lyc\'eennes contactent l'association pour des TPE,
\item de jeunes \'etudiantes posent des questions \`a l'association \`a propos de leur orientation.
\end{itemize}
\end{itemize}
\textbf{Organise r\'eguli\`erement}
\begin{itemize}
\item des colloques \`a l'Institut Henri Poincar\'e \`a Paris,
\item des journ\'ees r\'egionales dans des universit\'es diff\'erentes : expos\'es de math\'ematiques et table ronde li\'ee \`a l'\'egalit\'e des chances,
\item un forum des jeunes math\'ematicien\mp nes tous les ans \`a l'automne (en 2018 il a eu lieu à Orléans, et en 2019, il aura lieu à l'IHP à Paris),
\end{itemize}
\textbf{Publie}
\begin{itemize}
\item une newsletter trimestrielle,
\item des articles dans des revues,
\item des statistiques sexu\'ees.
\end{itemize}
\textbf{Anime}
\begin{itemize}
\item
une liste de diffusion : femmes-et-maths@listes.math.cnrs.fr
\item
un compte twitter : @femmesetmaths
\end{itemize}

En 2001, l'association est l'une des laur\'eates du premier \textbf{Prix Ir\`ene Joliot-Curie}. \\
En 2006 une \textbf{mention sp\'eciale du Prix Ir\`ene Joliot-Curie} du Minist\`ere d\'el\'egu\'e \`a l'Enseignement sup\'erieur et \`a la Recherche a r\'ecompens\'e l'une de nos membres pour son initiative remarquable dans le domaine du mentorat. \\






%%%%%%%%%%%%%%%%%%%%%%%%%%%%%
\section{CIMPA}\index{Centre international de math\'ematiques pures et appliqu\'ees (CIMPA)}

\emph{Pr\'esident\mp e actuel\mp le : \verifier{Christophe Ritzenthaler}}\hfill Site web~: \url{https://www.cimpa.info/fr}

%\emph{Vice pr\'esident\mp e : \verifier{Alain Damlamian}, Secr\'etaire : \verifier{Jean-Marc Bardet},
%Tr\'esorier : \verifier{Marc Aubry}}

%\emph{Directeur\mp trice : \verifier{Claude Cibils}}
\smallskip


Le CIMPA est un organisme international
\oe uvrant pour l'essor des
math\'ematiques dans les pays en
voie de d\'eveloppement.
Fond\'e en 1978, le CIMPA est bas\'e
\`a Nice. Il a pour vocation de promouvoir
la coop\'eration internationale
dans le domaine de l'enseignement
sup\'erieur et de la recherche en math\'ematiques
pures et appliqu\'ees et
leurs interactions, ainsi que dans les
disciplines connexes.

Cr\'e\'e en France et reconnu par
l'UNESCO, le CIMPA b\'en\'eficie du
soutien du MESR (France), de l'UNS
(France), du MICINN (Espagne) et
du CNRS (France). Disposant du
statut d'association (loi fran\c caise de
1901), il s'appuie sur de nombreux
math\'ematicien\mp nes et membres institutionnels
du monde entier.

En 2007, le Conseil d'administration
du CIMPA a exprim\'e la volont\'e
de le faire \'evoluer en un centre
europ\'een afin que d'autres pays
puissent lui apporter un soutien
financier et participer \`a ses activit\'es
scientifiques. Aujourd'hui en
marche, cette \'evolution permettra
de mieux r\'epondre aux nombreuses
demandes des pays en voie de d\'eveloppement
que les moyens actuels
ne permettent pas de satisfaire.


%%%%%%%%%%%%%%%%%%%%%%%%%%%%%%%%%
\section{La Conf\'ed\'eration des jeunes chercheurs}
\emph{Pr\'esident\mp e actuel\mp le : \verifier{Julie Crabot}} \hfill Site web : \url{https://cjc.jeunes-chercheurs.org}

\smallskip

\index{Conf\'ed\'eration des jeunes chercheurs (CJC)}

La Conf\'ed\'eration des Jeunes Chercheurs (CJC) regroupe des associations
de doctorant\mp e\mp s et de nouveaux\mp elles docteur\mp e\mp s de toute la France et de toutes
les disciplines. Elle a pour but de repr\'esenter et d\'efendre les int\'er\^ets
des jeunes chercheuses et chercheurs et de promouvoir le doctorat comme une exp\'erience
professionelle de la recherche et de l'innovation. Elle se positionne
comme force de proposition sur les questions de la recherche, de
l'enseignement sup\'erieur et de la formation doctorale.


%%%%%%%%%%%%%%%%%%%%%%%%%%%%%%%%%%%%%%%%
\section{La Fondation Blaise Pascal}
\index{Blaise Pascale (association)}

\emph{Pr\'esident\mp e actuel\mp le~: \verifier{Laurence Devillers}} \hfill Site web~: \url{https://www.fondation-blaise-pascal.org}
\smallskip

La fondation Blaise Pascal est une fondation nationale qui a pour vocation de promouvoir, soutenir, d\'evelopper et p\'erenniser les actions de m\'ediation scientifique en math\'ematiques et informatique \`a destination de tout citoyen, et plus sp\'ecifiquement aupr\`es des jeunes et des femmes. Elle a \'et\'e cr\'e\'ee sous \'egide de la Fondation pour l’Universit\'e de Lyon en novembre 2016. Ses fondateurs sont le CNRS et l’Universit\'e de Lyon.
 
La fondation Blaise Pascal poursuit cinq objectifs majeurs :
am\'eliorer la perception g\'en\'erale des sciences formelles par le grand public et notamment par les jeunes scolaris\'es, en am\'eliorant la compr\'ehension de leur impact, de leur utilit\'e et de leur vitalit\'e ; lutter contre les pr\'ejug\'es et les st\'er\'eotypes sociaux et de genre qui empêchent certains jeunes de se lancer dans des \'etudes en informatique et en math\'ematiques ; augmenter globalement le flux d'\'etudiant\mp e\mp s effectuant des \'etudes longues dans un domaine scientifique ;
d\'emultiplier les moyens par le partage des ressources et la structuration des offres des acteurs de m\'ediation ; att\'enuer les disparit\'es sociales et g\'eographiques, grâce \`a une meilleure r\'epartition des projets \`a l'\'echelle nationale.


Depuis 2017 : 6 appels \`a projets r\'ealis\'es 745000 euros allou\'es, 142 projets financ\'es sur l’ensemble du territoire national, 83 structures soutenues dans 13 r\'egions, plus d’un million de personnes ont b\'en\'efici\'e des actions que soutien la fondation :
\begin{itemize}
\item Environ 1 535 000 \'el\`eves de la primaire \`a la terminale ;
\item Environ 60 000 adultes;
\item Plus de 700 enseignant\mp e\mp s et enseignant\mp e\mp s-chercheur\mp ses.
\end{itemize}


%%%%%%%%%%%%%%%%%%%%%%%%%%%%%%%%%%%%%%%%%%%%%
%\section{Matexo}
%
%
%Matexo est le portail p\'edagogique du domaine emath.fr, destin\'e aux \'etudiants et
%enseignants du sup\'erieur en math\'ematiques.
%Il est compos\'e de plusieurs sites~:
%\begin{itemize}
%\item la base de documents r\'eserv\'es aux enseignants~: notes de cours, feuilles
%d'exercices avec ou sans correction...
%\item ExeMaAlt qui est un serveur d'exercices alternatifs,
%\item Exo7 qui est un site d'exercices en libre service \`a destination des \'etudiants
%et des enseignants (avec notamment la possibilit\'e de cr\'eer des feuilles de TD \`a la
%vol\'ee).
%\end{itemize}
%
%
%Site web~: \url{https://matexo.emath.fr}


%%%%%%%%%%%%%%%%%%%%%%%%%%%%%%%%%%%%%%%%%%%%
\section{MATh.en.JEANS}

\emph{Pr\'esident\mp e actuel\mp le : \verifier{Aviva Szpirglas}}\hfill Site web : \url{https://www.mathenjeans.fr}
%Contact: {\tt mathenjeans@free.fr}
\smallskip

Depuis 1989, MATh.en.JEANS, ({\em M\'ethode d'Apprentissage des Th\'eories math\'ematiques en Jumelant des \'Etablissements pour une Approche Nouvelle du Savoir}), fait vivre les math\'ematiques aux jeunes suivant les principes de la recherche, au sein d'ateliers dans les \'etablissements scolaires et au contact de chercheur\mp ses professionnel\mp les. Elle permet \`a des jeunes  de tous niveaux et de toutes origines de pratiquer une authentique d\'emarche scientifique, avec ses dimensions aussi bien th\'eoriques qu'appliqu\'ees et si possible en prise avec des th\`emes de recherche actuels.

\subsection*{Le principe} 
Chaque semaine, d\`es le mois de septembre, des \'el\`eves volontaires et des enseignant\mp e\mp s de deux \'etablissements scolaires jumel\'es pour l'occasion travaillent en parall\`ele en petits groupes, pendant une ou deux heures hebdomadaires, sur des sujets de recherche math\'ematique \`a la fois attractifs et s\'erieux propos\'es par un\mp e chercheur\mp se professionnel\mp le, souvent proches de ses propres probl\'ematiques.

Trois ou quatre fois dans l'ann\'ee, les \'el\`eves, les enseignant\mp e\mp s et le\mp la chercheur\mp se impliqu\'e\mp e\mp s dans les deux ateliers se rencontrent \`a l'occasion de \og s\'eminaires\fg{} o\`u ils\mp elles \'echangent leurs points de vue, d\'ebattent et partagent leurs id\'ees, critiquent et font avancer leur travail.

Les enseignant\mp e\mp s veillent au bon d\'eroulement mat\'eriel des ateliers. Elles ou ils incitent aux \'echanges et aident les \'el\`eves \`a pr\'eciser leurs pens\'ees, \`a les reformuler, en leur laissant le temps n\'ecessaire. Ils ou elles accompagnent la pr\'eparation de la pr\'esentation orale puis d'un \'ecrit. Mais ils\mp elles ne r\'esolvent pas le probl\`eme \`a la place des \'el\`eves, ils\mp elles ne le traduisent pas, ils\mp elles ne le r\'eduisent pas \`a des petites questions.

Le\mp La chercheur\mp se a pour r\^ole de r\'ediger les sujets propos\'es \`a l'atelier, en tenant compte du niveau des \'el\`eves. Il\mp Elle accompagne la recherche des \'el\`eves en suivant leur progression \`a l'occasion des s\'eminaires. Au besoin, il\mp elle compl\`ete ou r\'eactualise les questions pos\'ees.

Chaque ann\'ee, entre mars et avril, les \'el\`eves pr\'esentent leurs r\'esultats et les soumettent \`a la critique dans les congr\`es qui regroupent l'ensemble des ateliers MATh.en.JEANS existants. Moment fort de l'ann\'ee, le congr\`es annuel, r\'eunit ses acteur\mp trices, jeunes, professeur\mp e\mp s et chercheur\mp ses, dans un lieu choisi pour son dynamisme scientifique.

Le congr\`es pass\'e,  les \'el\`eves sont incit\'es \`a r\'ediger un article, qui sera publi\'e par l'association apr\`es validation.

\subsection*{L'association}
L'association a \'et\'e cr\'e\'ee en 1990, par Pierre Audin et Pierre Duchet - respectivement enseignant et chercheur en math\'ematiques - suite \`a l'op\'eration \og 1000 classes - 1000 chercheurs\fg{} men\'ee en 1985-1986, et \`a un projet pilote sur l'ann\'ee scolaire 1989-1990.

Elle a pour principales missions d'impulser la mise en place des ateliers dans les \'etablissements scolaires, de mettre en contact les enseignant\mp e\mp s et les chercheur\mp ses, de les coordonner, d'organiser les congr\`es annuels o\`u les \'el\`eves pr\'esentent leurs travaux, de valider et de publier leurs productions \'ecrites. Elle met l'accent sur les \'echanges entre pair\mp e\mp s et le contact avec la recherche vivante.

Elle a obtenu en 1990 le prix de la d\'emarche scientifique au Salon PERIF (r\'eunissant des projets scientifiques en Ile de France), et en 1992, le prix d'Alembert de la Soci\'et\'e Math\'ematique de France. Elle est agr\'e\'ee par le Minist\`ere de l'Education Nationale et soutenue par le CNRS et plusieurs autres partenaires institutionnels ou associatifs. Elle est partie prenante du Consortium Cap'Maths.

Depuis quelques ann\'ees MATh.en.JEANS est en forte expansion ; actuellement environ 300 ateliers fonctionnent en France et dans le monde (notamment le r\'eseau des \'etablissements fran\c{c}ais \`a l'\'etranger), regroupant environ 5000 \'el\`eves, 600 professeur\mp e\mp s et 200 chercheur\mp ses. Pour 2019, 12 congr\`es sont organis\'es dont 9 en France et 3 \`a l'\'etranger. MATH.en.JEANS f\^ete ses 30 ans cette m\^eme ann\'ee.


%%%%%%%%%%%%%%%%%%%%%%%%%%%%%%%
\section{L'Op\'eration Postes}

L'Op\'eration Postes (OP) n'est pas une association, mais elle a sa
place dans cette liste pour tous les services rendus \`a la
communaut\'e math\'ematique. Vous la connaissez d\'ej\`a tous,
mais voici tout de m\^eme quelques rappels~: \index{Op\'eration
postes (OP)}
\begin{itemize}
\item l'OP existe depuis 1998 et est constitu\'ee
d'(enseignant\mp e\mp s)-chercheur\mp ses b\'en\'evoles~;

\item elle b\'en\'eficie du soutien de la SMAI (qui a permis son
lancement et assure l'h\'ebergement de son serveur), de la SMF et de
la SFdS, ainsi que de SIF et de l'AFIF, soci\'et\'es savantes en
informatique~;

\item elle est soutenue financi\`erement par le minist\`ere (pour
le remboursement des missions)~;

%\item elle est partenaire de la guilde des doctorants~;

\item elle a pour but de diffuser le maximum d'informations sur
les concours de recrutement d'enseignant\mp e\mp s-chercheur\mp ses et de
chercheur\mp ses en math\'ematiques (sections CNU 25 et 26) et
informatique (section 27).
\end{itemize}
 \index{Soci\'et\'e des personnels enseignants et
chercheurs en informatique de France (SPECIF)}
\index{Association fran\c caise d'informatique \\fondamentale (AFIF)}

Vous pouvez y contribuer en
\begin{itemize}
\item transmettant des
informations (profils de postes, dates des Comit\'es de S\'election, des auditions, r\'esultats des
concours, AMI (Academic Mobility Index) de votre laboratoire, {\em
etc.})~;
\item informant les candidat\mp e\mp s en faisant la publicit\'e de
l'OP, de MARS (machine d'aide au recrutement dans le sup\'erieur),
{\em etc.}~;
\item informant vos coll\`egues {\bf y compris d'autres
disciplines} de l'existence de MOUVE (Machine Ouverte aux
Universitaires qui Veulent
\'Echanger).
\index{Op\'eration postes (OP)!Machine Ouverte aux Universitaires
\\qui Veulent \'Echanger (MOUVE)} \index{Op\'eration postes
(OP)!Machine d'aide au recrutement dans le su\-p\'e\-rieur (MARS)}
\index{Op\'eration postes (OP)!Academic Mobility Index (AMI)}
\item contribuant au wiki de conseils aux candidat\mp e\mp s \`a l'adresse : \url{https://postes.smai.emath.fr/2021/OUTILS/conseils/index.php}.
\end{itemize}

N'h\'esitez pas \`a (re)d\'ecouvrir
et \`a faire conna\^\i  tre son site web~:
\url{http://postes.smai.emath.fr/}