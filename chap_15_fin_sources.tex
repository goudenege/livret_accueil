%%%%%%%%%%%%%%%%%%%%%%%%%%%%%%%%%%%%%%%%%%%%%%%%%%%%
%%%%%%%%%%%%%%%%%%%%%%%%%%%%%%%%%%%%%%%%%%%%%%%%%%%%

\chapter*{Les sources de financement}

On peut ais\'ement distinguer deux types de financement pour un
laboratoire~:

\begin{itemize}
\item[$\bullet$] les financements r\'ecurrents, qui peuvent \'emaner
\begin{itemize}
\item du(es) minist\`ere(s) de tutelle {\em via} le(s)
\'etablissement(s) dont rel\`eve le laboratoire,
\item des organismes de recherche (on \'evoquera ici seulement le CNRS
et INRIA mais, dans d'autres disciplines, on peut trouver des
financements r\'ecurrents provenant de l'Inserm, l'INRA, le CEA, {
{\em etc.}})~;
\end{itemize}
\item[$\bullet$] les financements sur projet, venant (la liste n'est
pas exhaustive!)
\begin{itemize}
\item de l'agence nationale de la recherche (ANR),
\item du minist\`ere des affaires \'etrang\`eres,
\item de la Communaut\'e europ\'eenne pour les diff\'erents programmes europ\'eens,
\item de contrats avec des partenaires industriels.
\end{itemize}
\end{itemize}

