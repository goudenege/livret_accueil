\documentclass[11pt]{book}
%%%%%%%%%%%%%%%%%%%%%%%%%%%%%
% Ancienne version
%\usepackage[latin1]{inputenc}% pour les accents
%\usepackage[scale=.85]{geometry}
%\usepackage[french]{babel}
%%%%%%%%%%%%%%%%%%%%%%%%%%%%%
% Nouvelle version pour les accents
\usepackage[T1]{fontenc}
\usepackage[utf8]{inputenc}
\usepackage[french]{babel}
\usepackage{textcomp}
% \usepackage{siunitx}
\usepackage[scale=.85]{geometry}
%%%%%%%%%%%%%%%%%%%%%%%%%%%%%

\frenchbsetup{StandardLists=true} % pour eviter les conflits entre babel et enumitem
\usepackage{makeidx,xcolor}
\usepackage[colorlinks]{hyperref}   % choisir pour la version pdf
\hypersetup{linkcolor=blue,urlcolor=blue}
%\usepackage{hyperref}    % choisir pour l'impression (noir et blanc)
\usepackage{amsmath}
%\uspackage{amssymb,amsmath,amsfonts,mathptmx}
\usepackage{eurosym}
\usepackage{wrapfig,pifont,fancybox,fancyhdr,multicol,enumitem,subfig,booktabs}

\usepackage{graphicx}
\graphicspath{{../Images/}}

%\renewcommand{\chaptermark}[1]{\bsc{\chaptername~\thechapter{} : #1}}
\pagestyle{fancy}
\fancyhead{}
\fancyhead[RO,LE]{\thepage}
\fancyhead[RE]{\leftmark}
\fancyhead[LO]{\rightmark}
\fancyfoot{}
\renewcommand{\headrulewidth}{0.4pt}
\renewcommand{\footrulewidth}{0.pt}




\newcommand {\lien}[1]{{\small\tt \href{http\string://#1}{http\string://#1}}}
\newcommand {\liens}[1]{{\small\tt \href{https\string://#1}{https\string://#1}}}
\newcommand {\lienftp}[1]{{\small\tt \href{ftp\string://#1}{http\string://#1}}}
\makeindex



%%%%%%%%%%%%%%
% La commande ci-dessous ne fait rien mais permet par un simple ctrl+F de retrouver rapidement les
% informations a verifier a chaque nouvelle edition (noms des directeur.trices par exemple).
%%%%%%%%%%%%%%

\newcommand{\verifier}[1]{\textcolor{black}{#1}}



\begin{document}
\renewcommand{\labelitemi}{$\bullet$}
\setlist{noitemsep,nosep}
\setlength{\parindent}{0cm}

{\pagestyle{empty}
%
~\hfill\includegraphics[height=1.25cm]{Images/CNRS}\hfill\includegraphics[height=1.25cm]{Images/sfds}\hfill\includegraphics[height=1.25cm]{Images/SMAI}
\hfill\includegraphics[height=1.25cm]{Images/smf}\hfill\includegraphics[height=1.cm]{Images/inria_fr}\hfill~

~

\vfill

\textbf{\Huge Livret d'accueil des math\'ematicien$\cdot$nes}

\hrule

\begin{flushright}
\textbf{\Large \verifier{$9^\text{\`eme}$ \'edition -- Avril 2021}}
\end{flushright}

\vskip 2cm

\begin{center}

\textbf{\large \href{http\string://postes.smai.emath.fr/apres/accueil/index.php}{http\string://postes.smai.emath.fr/apres/accueil/index.php}}

\vfill

\includegraphics[height=2cm]{Images/IHP} 

\bigskip

\textit{\textbf{Ce livret a \'et\'e r\'edig\'e par des b\'en\'evoles
et n'a aucune valeur officielle.}}

\textit{\textbf{La journ\'ee d'accueil est h\'eberg\'ee par l'IHP.}}

%\textit{\textbf{La reproduction de ce livret est assur\'ee par la FSMP.}}
\end{center}

\newpage

~

\newpage

\tableofcontents

\newpage
}

\setlength{\parskip}{.15cm}

\chapter*{Introduction}

La journ\'ee d'accueil des nouveaux et nouvelles recrut\'e$\cdot$es en math\'ematiques a pour vocation de regrouper
tou$\cdot$tes les chercheuses et chercheurs ainsi que les enseignant$\cdot$es-chercheur$\cdot$ses nouvellement en poste afin de leur faire d\'ecouvrir
les rouages de la communaut\'e math\'ematique~: pr\'esentation des diff\'erentes carri\`eres, 
intervention des grands acteurs (organismes de recherche, soci\'et\'es savantes, associations)
et des solutions de financement de projets. 

Organis\'ee par des b\'en\'evoles math\'ematicien$\cdot$nes, cette journ\'ee permet 
\`a ces jeunes recrut\'e$\cdot$es de s'int\'egrer et de nourrir leur sentiment d'appartenance
\`a la communaut\'e des math\'ematiques dans lequel ils ou elles s'investissent pour leur bien et le bien de leurs coll\`egues.
L'investissement b\'en\'evole n'est en effet pas contradictoire avec la r\'eussite de la carri\`ere professionnelle,
il y concourt et l'enrichit.

Pour aider les nouveaux et nouvelles recrut\'e$\cdot$es, les \'equipes successives d'organisation de cette journ\'ee
ont r\'edig\'e et mis \`a jour ce livret, dont l'ambition est d'\^etre factuel mais qui n'a aucune valeur officielle. 
Les diff\'erentes versions sont consultables sur le site de l'Op\'eration Postes, rubrique ``APRES''. 
Si l'organisation future de notre communaut\'e se trouvait profond\'ement boulevers\'ee,
au point qu'une simple mise \`a jour ne soit pas
possible, ce texte aura au moins le m\'erite de rappeler
ce qu'elle \'etait avant ces r\'eformes.

M\^eme s'il vise les nouveaux et nouvelles ma\^\i  tres de conf\'erences et
charg\'e$\cdot$es de recherche, ce texte est bien entendu destin\'e \`a une
large diffusion sous forme \'electronique. Il contient notamment
beaucoup de liens vers des sites web qui pr\'esentent en d\'etail
des points particuliers. Nous pensons pr\'ef\'erable de renvoyer le
lecteur vers la source de l'information, lorsqu'elle existe,
plut\^ot que d'int\'egrer cette source dans un texte qui serait tr\`es long et
risquerait de devenir rapidement obsol\`ete.

Nous invitons les personnes qui constateraient des liens rompus ou
des informations n\'eces\-si\-tant une mise \`a jour \`a nous en
faire part par courrier \'electronique adress\'e \`a~:
\href{mailto:accueil.mcf.cr@gmail.com}{\texttt{accueil.mcf.cr@gmail.com}}.

Nous avons essay\'e de d\'etailler l'organisation de chaque structure~:
qui la dirige, qui en fait partie, quel en est le processus de
nomination~; de plus, dans la mesure du possible nous avons tent\'e
d'obtenir une information sur le bilan des actions ant\'erieures.
Les textes de pr\'e\-sen\-ta\-tion des
soci\'et\'es savantes nous ont \'et\'e
fournis par leurs pr\'esidents respectifs.

Signalons qu'un guide est \'egalement disponible sur le site du CNRS~:

\lien{www.dgdr.cnrs.fr/drh/concours/guide/guide.htm}.



%%%%%%%%%%%%%%%%%%%%%%%%%%%%%%%%%%%%%%%%%%
%%%%%%%%%%%%%%%%%%%%%%%%%%%%%%%%%%%%%%%%%%
\part{\^Etre enseignant-chercheur ou chercheur}

\include{chap_MCF}
\include{chap_CR_CNRS}
\include{chap_CR_INRIA}
%%%%%%%%%%%%%%%%%%%%%%%%%%%%%%%%%
\chapter{Le m\'etier de chercheur$\cdot$se \`a l'INRA}
\emph{Attention, ce chapitre n'est peut-\^etre plus \`a jour.}

\section{L'Institut : statut, structures, personnels}

L'Institut National de la Recherche Agronomique (\lien{www.inra.fr/}), cr\'e\'e en 1946 est, depuis 1984, un Etablissement Public \`a Caract\`ere Scientifique et Technologique (EPST). Il est plac\'e sous la double tutelle du Minist\`ere charg\'e de l'Agriculture et du Minist\`ere charg\'e de la Recherche.\\
Il a pour missions de :
\begin{itemize}
\item \oe uvrer au service de l'int\'er\^et public tout en maintenant l'\'equilibre entre les exigences de la recherche et les demandes de la soci\'et\'e ;
\item produire et diffuser des connaissances scientifiques et des innovations, principalement dans les domaines de l'agriculture, de l'alimentation et de l'environnement ;
\item contribuer \`a l'expertise, \`a la formation, \`a la promotion de la culture scientifique et technique, au d\'ebat science/soci\'et\'e.
\end{itemize}

\medskip

Les recherches de l'INRA ont pour but de parfaire et d'exploiter la connaissance du monde vivant au service de l'agriculture, de l'alimentation et de l'environnement rural de l'homme.\\

L'INRA est administr\'e par un Conseil d'Administration pr\'esid\'e par le ou la Pr\'esident$\cdot$e de l'Institut. Ce$\cdot$tte dernier$\cdot$e assure \'egalement la direction g\'en\'erale. Il ou elle est assist\'e$\cdot$e du Conseil Scientifique de l'Institut. Les recherches sont conduites au sein d'unit\'es de recherches (200 unit\'es de recherche dont 103 associ\'ees \`a d'autres organismes) r\'eunies au sein de 13 d\'epartements de recherche\footnote{Alimentation humaine, Biologie et am\'elioration des plantes, Caract\'erisation et \'elaboration des produits issus de l'agriculture, Ecologie des for\^ets, prairies et milieux aquatiques, Environnement et agronomie, G\'en\'etique animale, Math\'ematiques et informatique appliqu\'ees, Microbiologie et cha\^ine alimentaire, Physiologie animale et syst\`emes d'\'elevage, Sant\'e animale, Sant\'e des plantes et environnement, Sciences pour l'action et le d\'eveloppement, Sciences sociales, agriculture et alimentation, espace et environnement} ; ceux-ci sont eux-m\^emes coordonn\'es par 3 directeur$\cdot$trices scientifiques \footnote{Alimentation et Bio\'economie, Agriculture, Environnement}. Dix-sept centres r\'egionaux INRA et un centre si\`ege sont r\'epartis en 148 sites dans toute la France (m\'etropole et Antilles-Guyane). L'INRA comprend \'egalement 49 unit\'es exp\'erimentales et 109 unit\'es d'appui et de service. Son budget \'etait de 882 millions d'euros en 2013.\\

Les personnels de l'INRA sont des fonctionnaires de l'Etat r\'egis par le statut g\'en\'eral de la fonction publique. Ce statut est fix\'e par les lois \no83-634 du 13 juillet 1983 et \no84-16 du 11 janvier 1984 modifi\'ees relatives au statut g\'en\'eral des fonctionnaires, les d\'ecrets \no83-1260 du 30 d\'ecembre 1983 et \no84-1207 du 28 d\'ecembre 1984 modifi\'es relatifs respectivement aux fonctionnaires des EPST et \`a celles et ceux de l'INRA. Ces textes r\'eglementent les diff\'erentes \'etapes de la carri\`ere des agents : recrutement, avancement, cong\'es, cessation de fonctions.\\

Aujourd'hui l'INRA compte plus de 10 000 agents titulaires (dont 4458 chercheur$\cdot$ses et ing\'enieur$\cdot$es), 19\% des chercheur$\cdot$ses recrut\'e$\cdot$es en 2013 \'etaient \'etrangers. Chaque ann\'ee, l'INRA accueille \'egalement plus de 2000 jeunes scientifiques parmi lesquel$\cdot$les des doctorant$\cdot$es et des jeunes docteur$\cdot$es. Les jeunes scientifiques en contrat avec l'Institut sont des membres \`a part enti\`ere du personnel scientifique de l'unit\'e de recherche dans laquelle ils et elles \'evoluent. Elles et ils contribuent activement aux recherches conduites tout en se formant \`a la recherche. Les directeur$\cdot$trices d'unit\'e veillent \`a ce que d\`es leur arriv\'ee, les doctorant$\cdot$es et les jeunes chercheur$\cdot$ses s'inscrivent dans un projet professionnel clair et construit.


\section{ Le recrutement}
Conform\'ement aux missions imparties aux personnels de la recherche, les chercheur$\cdot$es de l'organisme doivent contribuer non seulement \`a l'acquisition de connaissances nouvelles dans les domaines de leurs comp\'etences mais aussi au transfert des r\'esultats de leurs travaux dans la soci\'et\'e : valorisation \'economique et sociale, diffusion des informations scientifiques et techniques, formation \`a et par la recherche, d\'eveloppement des \'echanges scientifiques avec l'\'etranger.\\ 
Quelle que soit leur discipline de formation, les chercheur$\cdot$es s'appuient sur des activit\'es de laboratoire ou de "terrain". Ils et elles sont fortement impliqu\'e$\cdot$es dans des r\'eseaux scientifiques, r\'epondent \`a des questions environnementales, \'economiques, sociales. Recherche personnelle et projet collectif s'imbriquent \'etroitement pour faire progresser les connaissances et pour participer au d\'eveloppement de l'innovation.\\

L'INRA emploie plus de 1900 chercheur$\cdot$es. Ils$\cdot$elles sont des fonctionnaires de l'Etat et sont r\'eparti$\cdot$es selon deux cat\'egories : les {\bf directeur$\cdot$trices de recherche} (seniors) et les {\bf charg\'e$\cdot$es de recherche} (juniors). Ils$\cdot$elles \'evoluent dans des {\bf disciplines scientifiques} vari\'ees. Pr\`es de 44\% des chercheur$\cdot$ses de l'INRA sont des femmes. En sa qualit\'e d'\'etablissement public, l'Inra recrute ses chercheur$\cdot$ses par voie de concours. L'Institut a re\c{c}u la reconnaissance officielle de la  Commission Europ\'eenne pour l'excellence de sa  politique de ressources humaines \`a l'\'egard des  chercheurs. Ainsi, une  charte interne \`a l'Institut  pr\'ecise les conditions d'accueil et d'insertion des jeunes  docteur$\cdot$es en termes de recrutement, de positionnement dans  les unit\'es d'accueil, de formation, de publication  et  de valorisation des r\'esultats

\subsection{ Les charg\'e$\cdot$es de recherche}
Plus de 1200 Charg\'e$\cdot$es de recherche \'evoluent au sein des \'equipes de recherche de l'Inra. En d\'ebut de carri\`ere, les {\bf charg\'e$\cdot$es de recherche} (jeunes docteur$\cdot$es) se consacrent \`a l'avancement de la th\'ematique de recherche qui leur a \'et\'e confi\'ee et \`a la publication syst\'ematique des r\'esultats acquis. Elles et ils b\'en\'eficient de l'environnement de chercheur$\cdot$ses confirm\'e$\cdot$es. Par la suite, ils$\cdot$elles encadrent eux$\cdot$elles-m\^emes des personnels techniques et des stagiaires qui vont concourir au d\'eveloppement de leur projet. Les fonctions d'animation et d'encadrement prennent progressivement davantage de place, ainsi que les activit\'es d'enseignement, mais la priorit\'e reste focalis\'ee sur la production scientifique.\\

Les chercheur$\cdot$ses sont recrut\'e$\cdot$es par voie de concours organis\'es par discipline ou groupe de disciplines.  Chaque ann\'ee, l'INRA organise une campagne de concours pour le recrutement de {\bf charg\'e$\cdot$es de recherche de 2e classe (CR2)}. Le recrutement s'effectue, en r\`egle g\'en\'erale, parmi les {\bf chercheur$\cdot$ses d\'ebutant$\cdot$es} ayant soutenu une th\`ese depuis peu. Les candidat$\cdot$es sont recrut\'e$\cdot$es pour leurs comp\'etences scientifiques qu'ils$\cdot$elles mettront au service des grandes orientations de l'Inra en r\'epondant \`a une th\'ematique de recherche. Les candidats doivent avoir valoris\'e les r\'esultats de leur th\`ese par des publications. Les recrutements sont ouverts dans de nombreuses th\'ematiques scientifiques telles que la biologie cellulaire et mol\'eculaire, l'\'ecologie, l'\'economie, la g\'en\'etique, la g\'enomique et autres approches "omiques", l'informatique et l'intelligence artificielle, les math\'ematiques, la nutrition, la physiologie, la physico-chimie, les sciences m\'edicales et v\'et\'erinaires et la sociologie. Le {\bf calendrier de la campagne} est en g\'en\'eral le suivant (sous r\'eserve de la publication de l'arr\^et\'e d'ouverture au Journal Officiel) : ouverture des inscriptions fin janvier, cl\^oture des inscriptions fin f\'evrier, admissibilit\'e (sur dossier) en avril-mai, admission (\'epreuve orale) en juin-juillet. Les candidat$\cdot$es admis$\cdot$es sont nomm\'e$\cdot$es en qualit\'e de stagiaires pour une dur\'ee d'un an et sont titularis\'e$\cdot$es, apr\`es avis de la Commission scientifique sp\'ecialis\'ee (CSS) comp\'etente. Toutefois, le stage peut \^etre prolong\'e de 18 mois au maximum ou il peut \^etre mis fin aux fonctions du chercheur ou de la chercheuse apr\`es avis de l'instance d'\'evaluation et de la Commission Administrative Paritaire (CAP) comp\'etente \`a l'\'egard du corps des Charg\'es de Recherche.\\

Chaque ann\'ee, l'INRA organise une campagne de concours pour le recrutement de {\bf charg\'e$\cdot$es de recherche de 1\`ere classe (CR1) sur projet}. Le concours de Charg\'e$\cdot$e de Recherche de 1\`ere classe s'adresse \`a des {\bf chercheur$\cdot$ses confirm\'e$\cdot$es} (aptitude \`a concevoir, pr\'esenter et conduire un projet de recherche : capacit\'e \`a prendre des responsabilit\'es d'animation et d'encadrement dans un cadre collectif). Ainsi, les candidat$\cdot$es doivent avoir fait preuve, par leur parcours, de leur autonomie professionnelle et de leur ouverture sur des r\'eseaux de collaboration. Ils$\cdot$Elles doivent poss\'eder leur culture scientifique propre, leur r\'eseau de collaborations et avoir \`a leur actif une production scientifique de bonne qualit\'e et de bon niveau. Au grade de CR1, les fonctions d'animation et d'encadrement prennent progressivement davantage de place, ainsi que les activit\'es d'enseignement, mais la priorit\'e reste focalis\'ee sur la production scientifique. Le {\bf calendrier de la campagne} est en g\'en\'eral le suivant : ouverture des inscriptions fin juin, cl\^oture des inscriptions fin ao\^ut, admissibilit\'e (sur dossier) octobre, admission (\'epreuve orale) en novembre-d\'ecembre.  Les nominations, effectu\'ees \`a l'issue des \'epreuves d'admission, sont d\'ecid\'ees par la ou le Pr\'esident$\cdot$e de l'Institut dans l'ordre de la liste des admis$\cdot$es. Les candidat$\cdot$es admis$\cdot$es sont nomm\'e$\cdot$es en qualit\'e de stagiaires pour une dur\'ee d'un an et sont titularis\'e$\cdot$es, apr\`es avis de la Commission scientifique sp\'ecialis\'ee (CSS) comp\'etente. Toutefois, le stage peut \^etre prolong\'e d'un an au maximum ou il peut \^etre mis fin aux fonctions du chercheur apr\`es avis de l'instance d'\'evaluation et de la Commission Administrative Paritaire (CAP) comp\'etente \`a l'\'egard du corps des Charg\'e$\cdot$es de Recherche.

\subsection{ Les directeur$\cdot$trices de recherche}
Plus de 700 directeur$\cdot$trices de recherche conduisent les grands projets et les \'equipes de recherche de l'INRA. Les directeur$\cdot$trices de recherche (chercheur$\cdot$ses confirm\'e$\cdot$es), reconnu$\cdot$es par la qualit\'e de leurs publications scientifiques et l'excellence des projets qu'elles ou ils ont conduits, animent et dirigent de grands projets ou des unit\'es de recherche. Ils$\cdot$Elles ont la capacit\'e d'animer, sous tous leurs aspects, des programmes europ\'eens ou des \'equipes de recherche de taille significative. Leur capacit\'e d'expertise av\'er\'ee est appr\'eci\'ee dans les instances r\'eglementaires ou aupr\`es de structures porteuses d'importants enjeux socio-\'economiques. Chaque ann\'ee, l'INRA organise une campagne de concours pour le {\bf recrutement de directeur$\cdot$trices de recherche}. Le {\bf calendrier de cette campagne} est le suivant : ouverture des inscriptions fin juin, cl\^oture des inscriptions fin ao\^ut, admissibilit\'e (sur dossier) octobre, admission (\'epreuve orale) en novembre-d\'ecembre.

\subsection{ S'informer sur l'ouverture des concours : publicit\'e et contacts}

{\bf Publicit\'e}\\
L'ouverture de chaque session de concours est fix\'ee par arr\^et\'e publi\'e au Journal officiel. Le nombre de postes propos\'es et la date limite de d\'ep\^ot des dossiers sont \'egalement fix\'es par arr\^et\'e. L'ouverture des concours fait par ailleurs l'objet d'une publicit\'e sur Internet :\lien{www.inra.fr } (rubrique "Carri\`eres \& emplois").\\

{\bf Contacts}\\
Toutes les informations utiles (conditions pour concourir, documents \`a fournir pour s'inscrire, d\'eroulement des \'epreuves) peuvent \^etre obtenues aupr\`es de la Direction des Ressources Humaines.

\section{L'\'evaluation}
Conform\'ement au d\'ecret qui r\'egit l'\'evaluation des chercheur$\cdot$ses des EPST et \`a celui sp\'ecifiant les instances d'\'evaluation des chercheurs pour l'INRA, des Commissions Scientifiques Sp\'ecialis\'ees (CSS) \'evaluent les chercheur$\cdot$ses, \`a un rythme biennal, sur la base d'un dossier. {\bf Treize commissions} \'evaluent les chercheur$\cdot$ses de l'INRA. Douze d'entre elles sont d\'efinies par les disciplines et les m\'ethodes de recherche et sont transversales aux d\'epartements. Une treizi\`eme commission \'evalue les chercheur$\cdot$ses ayant des activit\'es de direction, d'animation ou de gestion de la recherche. Les p\'erim\`etres des CSS ont \'et\'e progressivement adapt\'es aux dynamiques scientifiques de l'Institut de fa\c{c}on \`a favoriser les interactions scientifiques jug\'ees strat\'egiques pour l'INRA. Ces p\'erim\`etres sont valid\'es par le Conseil Scientifique de l'Institut. Chaque chercheur$\cdot$se choisit sa commission d'\'evaluation apr\`es une discussion avec son ou sa directeur$\cdot$trice d'unit\'e. Les chercheur$\cdot$ses qui ont un profil pluridisciplinaire et dont les disciplines scientifiques ne sont pas suffisamment repr\'esent\'ees au sein d'une seule commission, peuvent soumettre leur dossier \`a deux commissions. Enfin, une commission peut demander \`a une chercheuse ou un chercheur  de soumettre son dossier \`a une autre commission, qu'elle jugera plus comp\'etente.\\

Ces commissions r\'ealisent une {\bf \'evaluation-conseil}. Elles produisent pour la direction de l'Institut un avis sur chaque dossier \'evalu\'e. Ces avis sont utilis\'es par la direction pour diff\'erentes d\'ecisions concernant la gestion des personnels. Ils sont statutairement requis pour les demandes de titularisation des charg\'e$\cdot$es de recherche, les candidatures de promotion en CR1 et en DR de classe exceptionnelle. Ils seront aussi disponibles pour la direction lors de son examen des candidatures \`a la prime d'excellence scientifique. Elles formulent des recommandations sur les aspects de l'activit\'e qui doivent \^etre am\'elior\'es. Elles r\'edigent un message personnel destin\'e \`a chaque chercheur$\cdot$se qui concr\'etise l'attention port\'ee \`a son profil d'activit\'e et \`a sa production et formulent d'\'eventuels conseils.\\

L'\'evaluation des chercheuses et chercheurs de l'Inra, r\'ealis\'ee par les CSS, porte sur l'ensemble des activit\'es des chercheur$\cdot$ses et prend en compte leur environnement, les missions qui leur sont confi\'ees et les objectifs des collectifs auxquels elles et ils appartiennent. L'\'evaluation par les CSS est une {\bf  \'evaluation ind\'ependante} de la hi\'erarchie et de l'environnement proche des chercheur$\cdot$ses. Enfin, cette \'evaluation est coll\'egiale : les avis et les messages sont le r\'esultat du travail de l'ensemble de la commission sous la responsabilit\'e de son ou sa pr\'esident$\cdot$e.

\section{ Les carri\`eres et les r\'emun\'erations}
Votre \'echelon d\'etermine l'indice auquel vous allez \^etre r\'emun\'er\'e$\cdot$e. La valeur d'un point d'indice est \'egale \`a 4,6303 euros depuis le 01/01/2010. L'indice de r\'emun\'eration auquel le ou la candidat$\cdot$e est recrut\'e$\cdot$e est d\'etermin\'e en fonction de ses dipl\^omes et de ses activit\'es professionnelles ant\'erieures.

\subsection{ Progression de carri\`ere pour les chercheur$\cdot$ses}
L'avancement d'\'echelon \`a l'int\'erieur d'un m\^eme grade intervient en fonction de l'anciennet\'e, selon les tableaux ci-dessous. Les Charg\'e$\cdot$es de recherche de 2\`eme classe (CR2) peuvent \^etre promu$\cdot$es au choix \`a la 1\`ere classe (CR1), apr\`es avis de la Commission scientifique sp\'ecialis\'ee (CSS) comp\'etente, sous r\'eserve de justifier d'au
moins 4 ann\'ees d'anciennet\'e dans le grade de CR2. Les Charg\'e$\cdot$es de recherche de 1\`ere classe (CR1), justifiant d'une anciennet\'e minimale de 3 ann\'ees dans le grade, peuvent se pr\'esenter aux concours pour l'acc\`es au corps des Directeur$\cdot$trices de recherche de 2\`eme classe (DR2). Il s'agit d'un v\'eritable changement de m\'etier. La pr\'esentation et l'argumentation d'un projet sont indispensables. \`a titre tr\`es exceptionnel, tout$\cdot$e Charg\'e$\cdot$e de Recherche peut concourir pour l'acc\`es au corps des Directeurs de Recherche sans condition d'anciennet\'e sous r\'eserve d'y avoir \'et\'e autoris\'e$\cdot$e par le Conseil Scientifique de l'\'etablissement, au vu de la contribution notoire qu'il ou elle aura apport\'ee \`a la recherche. Une bonification d'anciennet\'e d'un an est accord\'ee aux Charg\'e$\cdot$es de Recherche qui effectuent une mobilit\'e dont la dur\'ee est au moins \'egale \`a 2 ans :
\begin{itemize}
\item dans un autre organisme de recherche ou d'enseignement sup\'erieur \`a l'\'etranger,
\item aupr\`es d'une administration, d'une collectivit\'e locale ou d'une entreprise publique ou priv\'ee.
\end{itemize}
Les Directeur$\cdot$trices de recherche de 2\`eme classe (DR2) peuvent acc\'eder \`a la 1\`ere classe (DR1) apr\`es examen de leur dossier de candidature par une commission d'avancement.

\subsubsection*{Charg\'e$\cdot$es de recherche de 2\ieme{} classe}
\begin{center}
\begin{tabular}{lccc}
\toprule
& Indice brut& Indice major\'e (01/01/2013)& Anciennet\'e requise dans l'\'echelon \\
\midrule
1\ier{} \'echelon &530&454& 1 an \\

2\ieme{} \'echelon &542&461& 1 an \\

3\ieme{} \'echelon &580&490& 1 an\\

4\ieme{} \'echelon &618&518 &1 an et 4 mois\\

5\ieme{} \'echelon &653&545 & 2ans\\

6\ieme{} \'echelon &677&564&Terminal \\
\bottomrule
\end{tabular}
\end{center}


\subsubsection*{Charg\'e$\cdot$es de recherche de 1\iere{} classe}
\begin{center}
\begin{tabular}{lccc}
\toprule
& Indice brut& Indice major\'e (01/01/2013)& Anciennet\'e requise dans l'\'echelon \\
\midrule
1\ier{} \'echelon &562&476& 2 ans \\

2\ieme{} \'echelon &600&505& 2 ans et 6 mois \\

3\ieme{} \'echelon &678&564& 2 ans et 6 mois\\

4\ieme{} \'echelon &755&623 &2 ans et 6 mois\\

5\ieme{} \'echelon &821&673 & 2ans et 6 mois\\

6\ieme{} \'echelon &882&719& 2 ans et 6 mois\\

7\ieme{} \'echelon &920&749 &2 ans et 9 mois\\

8\ieme{} \'echelon &966&783 & 2ans et 10 mois\\

9\ieme{} \'echelon &1015&821&Terminal \\
\bottomrule
\end{tabular}
\end{center}

\subsubsection*{Directeur$\cdot$trices de recherche de 2\ieme{} classe}
\begin{center}
\begin{tabular}{lccc}
\toprule
& Indice brut& Indice major\'e (01/01/2013)& Anciennet\'e requise dans l'\'echelon \\
\midrule
1\ier{} \'echelon &801&658& 1 an et 3 mois\\

2\ieme{} \'echelon &852&696& 1 an et 3 mois\\

3\ieme{} \'echelon &901&734& 1 an et 3 mois\\

4\ieme{} \'echelon &958&776 &1 an et 3mois\\

5\ieme{} \'echelon &1015&821 & 3 ans et 6 mois\\

6\ieme{} \'echelon &Hors \'echelle A&A1, A2, A3&Terminal \\
\bottomrule
\end{tabular}
\end{center}


\subsubsection*{Directeur$\cdot$trices de recherche de 1\iere{} classe}
\begin{center}
\begin{tabular}{lccc}
\toprule
& Indice brut& Indice major\'e (01/01/2013)& Anciennet\'e requise dans l'\'echelon \\
\midrule
1\ier{} \'echelon &1015&821& 3 ans \\

2\ieme{} \'echelon &Hors \'echelle B&B1, B2, B3& 3 ans \\

3\ieme{} \'echelon &Hors \'echelle C&C1, C2, C3& Terminal\\
\bottomrule
\end{tabular}
\end{center}

\subsubsection*{Directeur$\cdot$trices de recherche classe exceptionnelle}
\begin{center}
\begin{tabular}{lccc}
\toprule
& Indice brut& Indice major\'e (01/01/2013)& Anciennet\'e requise dans l'\'echelon \\
\midrule
1\ier{} \'echelon &Hors \'echelle D&D1, D2, D3& 1 an et 6 mois \\

2\ieme{} \'echelon &Hors \'echelle E&E1, E2& Terminal \\
\bottomrule
\end{tabular}
\end{center}

\subsection{ Les primes et indemnit\'es}
{\bf La prime de recherche (987 \euro{}  annuels pour les CR et 796 \euro{}{}  annuels pour les DR) est attribu\'ee mensuellement \`a toutes les chercheuses et tous les chercheurs selon leur grade et leur corps. L'indemnit\'e d'enseignement (42.72 \euro{}{}  annuels) est vers\'ee mensuellement \`a l'agent exer\c{c}ant une activit\'e d'enseignement. L'indemnit\'e de r\'esidence est vers\'ee mensuellement selon l'affectation g\'eographique de l'agent. Un suppl\'ement familial de traitement peut \^etre ajout\'e en fonction du nombre d'enfants \`a charge.}\\

La prime d'encadrement doctoral et de recherche (PEDR) concerne les chercheur$\cdot$ses en activit\'e. Elle peut leur \^etre attribu\'ee en reconnaissance de leur contribution individuelle \`a l'activit\'e scientifique. L'objectif de cette prime est de r\'ecompenser financi\`erement l'excellence de l'activit\'e des chercheurs et chercheuses en allouant \`a certains d'entre eux et elles une prime individuelle sur la base de r\'esultats av\'er\'es. Trois cas d'attribution sont pr\'evus par le d\'ecret d'application :
\begin{itemize}
\item pour les laur\'eats d'une distinction scientifique de niveau international ou national conf\'er\'ee par un organisme de recherche et dont la liste est fix\'ee par arr\^et\'e minist\'eriel ; l'attribution est alors automatique sans condition d'enseignement (prime type 1) ;
\item pour les chercheurs et chercheuses apportant une contribution exceptionnelle \`a la recherche (prime type 2 : 2A et 2B) ;
\item pour les chercheuses et chercheurs dont l'activit\'e scientifique est jug\'ee d'un niveau \'elev\'e, sous r\'eserve qu'elles et ils remplissent la condition d'enseignement pr\'evue (prime type 3). Le ou la candidat$\cdot$e \`a cette prime s'engage \`a remplir, d\`es la premi\`ere ann\'ee de versement de la prime, la condition d'enseignement correspondant \`a 64 heures de travaux dirig\'es annuels (ou activit\'e de formation \'equivalente). Les chercheur$\cdot$ses s'engagent \`a fournir chaque ann\'ee avant le 1er f\'evrier, le d\'ecompte des heures effectu\'ees l'ann\'ee pr\'ec\'edente.
\end{itemize}

\section{La mobilit\'e}

La mobilit\'e, qu'elle soit th\'ematique, g\'eographique ou effectu\'ee vers d'autres \'etablissements ou entreprises du secteur public ou priv\'e, fait partie int\'egrante du parcours professionnel. Elle offre l'opportunit\'e de concilier les \'evolutions et les besoins de l'Institut avec les comp\'etences et les aspirations individuelles des agents.
Diff\'erentes dispositions sont offertes au fonctionnaire pour effectuer une mobilit\'e. \\

La {\bf mise \`a disposition} au cours de laquelle le fonctionnaire demeure dans son corps d'origine. Elle ou il est r\'eput\'e$\cdot$e occuper son emploi et continue de percevoir sa r\'emun\'eration,
mais elle ou il effectue tout ou partie de son service aupr\`es d'un ou de plusieurs organismes d'accueil. La mise \`a disposition peut \^etre prononc\'ee au profit d'administrations des trois fonctions publiques, ainsi que d'organismes contribuant \`a la mise en \oe uvre de la politique de l'Etat, des collectivit\'es territoriales ou de leurs \'etablissements publics administratifs, pour l'exercice des seules missions de service public confi\'ees \`a ces organismes. La mise \`a disposition intervient avec l'accord du fonctionnaire. Elle est prononc\'ee pour une dur\'ee maximale de trois ans et peut \^etre renouvel\'ee. Elle est g\'en\'eralement encadr\'ee par une convention ou un contrat  de partenariat sign\'e par les partenaires et les agents concern\'es. Ce document pr\'ecise notamment la nature des activit\'es qu'il va exercer et ses conditions d'emploi.\\

Dans le cas du {\bf d\'etachement}, le ou la fonctionnaire est plac\'e hors de son corps ou cadre d'emploi initial pour travailler au sein d'un autre organisme que son administration d'origine. Il ou elle continue toutefois \`a jouir des droits \`a l'avancement et \`a la retraite attach\'es \`a son corps d'origine. D'un point de vue administratif, son \'evolution de carri\`ere se poursuit de mani\`ere parall\`ele dans les deux \'etablissements (\'etablissement d'origine et \'etablissement d'accueil). La dur\'ee du d\'etachement peut \^etre courte (6 mois port\'es \`a un an pour ceux qui exercent une mission \`a l'\'etranger) ou longue (jusqu'\`a 5 ans renouvelables).\\

Pour les chercheur$\cdot$ses, {\bf il n'existe pas de proc\'edure de mobilit\'e interne} avec une campagne d'affichage des profils \`a pourvoir. Lorsqu'un chercheur ou une chercheuse exprime un souhait de mobilit\'e (g\'eographique, th\'ematique), il ou elle en informe son directeur ou sa directrice d'unit\'e et son ou sa chef de d\'epartement, puis cette mobilit\'e se construit sur la base d'un projet scientifique, en interaction avec l'unit\'e d'accueil. C'est \'egalement le cas lorsque les chercheur$\cdot$ses r\'ealisent une mobilit\'e cons\'ecutive \`a la restructuration, la d\'elocalisation ou la fermeture de leur unit\'e. Si cette mobilit\'e se traduit par un changement de d\'epartement de recherche, elle n\'ecessitera aussi une n\'egociation entre les chefs des d\'epartements concern\'es, puis l'arbitrage final de la direction g\'en\'erale.

\chapter{L'{\'e}dition scientifique\protect\footnote{Chapitre r\'edig\'e par Fr\'ed\'eric H\'elein, Directeur scientifique du \href{http://www.rnbm.org/}{R\'eseau National des Biblioth\`eques de Math\'ematiques}.}}
 
\section{L'{\'e}dition scientifique en pleine mutation}

L'{\'e}dition scientifique et, notamment, le syst{\`e}me des revues publiant des articles de recherche {\'e}voluent constamment
depuis 30 ans, et cette {\'e}volution est loin d'{\^e}tre termin{\'e}e.

L'{\'e}v{\'e}nement majeur des ann{\'e}es 1980 fut l'introduction des logiciels de traitement de texte TeX et LaTeX. On passa ainsi de la frappe
{\`a} la machine {\`a} {\'e}crire des pr{\'e}publications {\`a} la saisie de fichiers {\'e}lectroniques que l'on peut transmettre directement aux {\'e}diteurs des
revues. Puis, au milieu des ann{\'e}es 90, le d{\'e}veloppement d'Internet permit la naissance des premi{\`e}res revues {\'e}lectroniques publiant les
articles en ligne. Ce mode de diffusion s'est g{\'e}n{\'e}ralis{\'e} et est devenu la norme durant la d{\'e}cennie 2000 suivant plusieurs {\'e}tapes~: dans un
premier temps les {\'e}diteurs ont continu{\'e} {\`a} vendre aux biblioth{\`e}ques des abonnements sur papier, en proposant, contre un suppl{\'e}ment financier,
un acc{\`e}s en ligne {\`a} une version {\'e}lectronique des articles. Dans un deuxi{\`e}me temps, les usages ont {\'e}t{\'e} invers{\'e}s et les {\'e}diteurs ont vendu des
abonnements {\'e}lectroniques avec la possibilit{\'e} de recevoir des fascicules imprim{\'e}s sur papier, {\`a} nouveau contre paiement d'un suppl{\'e}ment.
Enfin, dans une troisi{\`e}me phase, qui est arriv{\'e}e {\`a} terme dans de nombreuses disciplines mais pas encore en math{\'e}matiques, on a assist{\'e} {\`a}
une d{\'e}saffection des fascicules papier, les biblioth{\`e}ques concentrant l'essentiel de leurs moyens au r{\`e}glement des acc{\`e}s {\'e}lectroniques,
sauf pour certaines biblioth{\`e}ques soucieuses de constituer un patrimoine d'archives sur papier. Ainsi en pratique lorsqu'un chercheur
ou un {\'e}tudiant veut lire ou t{\'e}l{\'e}charger un article {\'e}lectronique, il faut qu'il soit reconnu comme {\'e}tant un <<~ayant-droit~>> par le portail
{\'e}lectronique de l'{\'e}diteur, ce qui signifie que l'institution ou la biblioth{\`e}que via laquelle il se connecte a pay{\'e} un abonnement. Sinon
l'acc{\`e}s lui sera refus{\'e}, {\`a} moins qu'il accepte de payer en ligne.

Cependant le fonctionnement {\'e}ditorial de la revue n'a pas chang{\'e}, du moins en ce qui concerne les <<~vraies~>> revues scientifiques (ce qui
exclut les revues <<~pirates~>> dont nous parlerons plus loin)~: chaque revue est supervis{\'e}e scientifiquement par un comit{\'e} {\'e}ditorial, constitu{\'e}
de chercheurs, dont la responsabilit{\'e} est de solliciter des rapporteurs anonymes, en leur demandant d'expertiser les articles soumis. Le but
est bien s{\^u}r de s{\'e}lectionner les articles qui seront publi{\'e}s, en veillant {\`a} ce qu'ils soient corrects, originaux et dignes d'int{\'e}r{\^e}t,
t{\^a}che difficile et parfois tributaire de crit{\`e}res subjectifs ou sociologiques. L'organisation de ce travail n{\'e}cessite une part de secr{\'e}tariat
assez importante qui, pour les revues les mieux dot{\'e}es, est assur{\'e} par un ou une secr{\'e}taire. Ce ou cette secr{\'e}taire est le plus souvent r{\'e}mun{\'e}r{\'e}(e)
par des institutions acad{\'e}miques, avec, parfois, une aide financi{\`e}re de l'{\'e}diteur. Mais dans de nombreux cas ce travail est assur{\'e} par des chercheurs
membres du comit{\'e} {\'e}ditorial, parfois indemnis{\'e}s par une petite r{\'e}tribution financi{\`e}re de la part de l'{\'e}diteur\footnote{Cet aspect
 des relations entre comit{\'e}s {\'e}ditoriaux et maisons d'{\'e}dition est toutefois assez opaque en g{\'e}n{\'e}ral.}

Ces {\'e}volutions ont permis de r{\'e}duire consid{\'e}rablement les co{\^u}ts de fonctionnement pour les {\'e}diteurs, ceux-ci n'ayant plus {\`a} composer les fascicules
en imprimerie et n'ayant pratiquement plus {\`a} les imprimer, ni {\`a} les stocker et les exp{\'e}dier. N{\'e}anmoins ces co{\^u}ts restent non nuls car les {\'e}diteurs
continuent {\`a} contribuer partiellement au fonctionnement du comit{\'e} {\'e}ditorial de certaines revues, {\`a} mettre aux normes les fichiers {\'e}lectroniques
avant leur publication. Ils doivent aussi concevoir des plate-formes pour la mise en ligne (y compris les syst{\`e}mes de p{\'e}age ou d'identification
des <<~ayant-droit~>> et excluant les <<~pas-ayant-droit~>> !)
et, pour les revues soucieuses d'un bon niveau de qualit{\'e}, assurer une relecture de la langue des articles. De plus comme
le chercheur du 21{\`e}me si{\`e}cle veut pouvoir acc{\'e}der {\`a} un article en quelques clics {\`a} partir de donn{\'e}es partielles (l{\`a} o{\`u} son anc{\^e}tre du 20{\`e}me si{\`e}cle
devait faire des d{\'e}marches dans sa biblioth{\`e}que et parfois attendre que celle-ci lui procure ce dont il avait besoin), les {\'e}diteurs doivent aussi
produire une sorte de carte d'identit{\'e} {\'e}lectronique de l'article contenant des informations sur l'article et ses auteurs, que l'on appelle <<~m{\'e}tadonn{\'e}es~>>.
Ces co{\^u}ts peuvent repr{\'e}senter une charge non n{\'e}gligeable pour les petits {\'e}diteurs (lesquels, dans les cas o{\`u}, par exemple, ils ne peuvent pas assurer
le travail de mise en ligne de leur revues, sont oblig{\'e}s de payer des prestataires de service comme JSTOR). En revanche les gros {\'e}diteurs, publiant des
centaines de revues, r{\'e}alisent de tr{\`e}s grandes {\'e}conomies d'{\'e}chelle sur ces co{\^u}ts de mise en ligne. 

Cette baisse des co{\^u}ts de production et la d{\'e}mat{\'e}rialisation des publications sous une forme {\'e}lectronique ont eu plusieurs cons{\'e}quences. 

\subsection{Les avantages des bouquets}

\subsubsection{Pour les chercheurs et leurs institutions} 
La premi{\`e}re cons{\'e}quence fut un gain pour les institutions de recherche (mais nous verrons le revers de la m{\'e}daille un peu plus loin... )~:
au lieu de s'abonner {\`a} une liste restreinte de revues, {\`a} la mesure de leur budget et choisies suivant les priorit{\'e}s scientifiques,
les biblioth{\`e}ques, en se regroupant en consortia, ont pu s'abonner {\`a} des bouquets de revues (dont le principe est analogue {\`a}
celui des bouquets de cha{\^\i}nes de t{\'e}l{\'e}vision) pour un co{\^u}t qui sembla raisonnable au d{\'e}but. Cela fut particuli{\`e}rement b{\'e}n{\'e}fique aux
petites universit{\'e}s ou aux petites institutions qui eurent ainsi acc{\`e}s {\`a} des dizaines, voire des centaines de revues {\'e}lectroniques,
alors qu'auparavant, elle ne pouvaient s'offrir que quelques revues au mieux. Cela repose sur des accords qui sont n{\'e}goci{\'e}s en amont
par le consortium et dans lesquels chaque biblioth{\`e}que adh{\'e}rente s'engage {\`a} payer une partie de la facture globale. En France les
premiers accords de ce type furent conclus pour les math{\'e}matiques par le \href{http://www.rnbm.org/}{RNBM}
(R{\'e}seau National des Biblioth{\`e}ques de Math{\'e}matiques)
avec l'{\'e}diteur Springer. Aujourd'hui les n{\'e}gociations men{\'e}es par le RNBM sont chapeaut{\'e}es par le Consortium
\href{http://www.couperin.org/}{\emph{Couperin}}, qui regroupe
toutes les Universit{\'e}s et la plupart des Institutions de recherche en France. En effet le RNBM est maintenant un Groupement de Service de
l'INSMI, qui est lui-m{\^e}me un Institut au sein du CNRS, lequel CNRS est membre du consortium \emph{Couperin}...


\subsubsection{Pour les {\'e}diteurs}
La deuxi{\`e}me cons{\'e}quence fut un gain pour les gros {\'e}diteurs: gr{\^a}ce aux importantes {\'e}conomies d'{\'e}chelle qu'elles ont pu faire,
les tr{\`e}s grosses
compagnies comme \emph{Reed Elsevier} (maintenant \emph{RELX Group}), \emph{Springer Nature}, \emph{Wiley}, \emph{Wolters Kluwer},
\emph{Informa} (\emph{Taylor \& Francis})
r{\'e}alisent toutes aujourd'hui des marges op{\'e}rationnelles sup{\'e}rieures {\`a} 24\% (l'industrie pharmaceutique est d{\'e}pass{\'e}e!)
et d{\'e}passant m{\^e}me 37\% pour
\emph{Reed Elsevier} et \emph{Springer Nature}, ce qui constitue un record toutes cat{\'e}gories (on pourra consulter  {\`a} ce sujet la
\href{http://www.eprist.fr/wp-content/uploads/2016/03/I-IST_16_R{\'e}sultatsFinanciers2015EditionScientifique.pdf}{note de l'EPRIST du 30 mars 2016}).

Mais un des probl{\`e}mes, c'est que ces b{\'e}n{\'e}fices spectaculaires ne s'expliquent pas uniquement par la baisse des co{\^u}ts de production, car ils sont
r{\'e}alis{\'e}s sur le dos des institutions publiques (universit{\'e}s, organismes de recherche) ou de certaines industries de pointe. En effet ils doivent
beaucoup aux augmentations fortes et incessantes des prix des abonnements, lesquelles semblent difficiles {\`a} justifier. Comment se fait-il
que les biblioth{\`e}ques acceptent de payer chaque ann{\'e}e des sommes toujours plus grandes ? La r{\'e}ponse est que, bien que le march{\'e} soit
partag{\'e} entre une multitude d'entreprises, des plus petites aux plus grosses comme \emph{Reed Elsevier} ou \emph{Springer Nature},
ce march{\'e} est sans concurrence,
car chaque revue est unique.

\subsection{Les effets pervers des bouquets}

De plus, les grosses entreprises tirent profit du syst{\`e}me de bouquets de revues. La strat{\'e}gie consiste {\`a} proposer aux biblioth{\`e}ques le choix entre
des abonnements {\`a} la carte aux revues qui les int{\'e}ressent, mais {\`a} des tarifs prohibitifs, et un abonnement {\`a} un bouquet (<<~big deal~>>). Pour plusieurs
raisons, les n{\'e}gociateurs choisissent le plus souvent la seconde option. La premi{\`e}re de ces raisons est que la diff{\'e}rence de prix est {\'e}norme~: pour une
somme comparable, les biblioth{\`e}ques ont ainsi acc{\`e}s {\`a} des centaines de revues au lieu de quelques unes ou de quelques dizaines (suivant la taille de 
l'universit{\'e}). Enfin, c'est beaucoup plus simple pour le n{\'e}gociateur (et il faut reconna{\^\i}tre que les contrats propos{\'e}s par les {\'e}diteurs sont au moins
aussi opaques que les forfaits des op{\'e}rateurs t{\'e}l{\'e}phoniques). Mais une fois que les n{\'e}gociateurs sont ainsi <<~guid{\'e}s~>> (pour ne pas dire <<~forc{\'e}s~>>) vers
le choix d'un <<~big deal~>>, il devient alors tr{\`e}s difficile de n{\'e}gocier son montant global, puisque la discussion porte sur <<~tout ou rien~>>~: le n{\'e}gociateur
ne peut pas prendre la responsabilit{\'e} de revenir vers les universit{\'e}s en disant qu'il n'y aura pas d'acc{\`e}s aux revues Elsevier ou Springer l'ann{\'e}e
prochaine et l'{\'e}diteur\footnote{N{\'e}anmoins les Pays-Bas ont r{\'e}cemment boycott{\'e} Elsevier et l'Allemagne fait de m{\^e}me depuis d{\'e}but 2017,
mais il s'agit plus de moyens de pression pour une n{\'e}gociation que de v{\'e}ritables boycotts.}
le sait tr{\`e}s bien... Voir par exemple  \href{http://alambic.hypotheses.org/6245}{L'Alambic num{\'e}rique du 6 d{\'e}cembre 2016}

Un autre effet ind{\'e}sirable du syst{\`e}me de bouquets est que, comme les contrats en question avec les gros {\'e}diteurs sont en g{\'e}n{\'e}ral sign{\'e}s pour plusieurs
ann{\'e}es (parce qu'ainsi l'{\'e}diteur propose un prix plus bas sur une plus longue p{\'e}riode) et comme les budgets des biblioth{\`e}ques sont {\`a} la baisse d'une
ann{\'e}e sur l'autre, les biblioth{\`e}ques sont oblig{\'e}es de se d{\'e}sabonner aux revues des petits {\'e}diteurs pour {\'e}quilibrer leur budget. Du coup, ce sont pr{\'e}cis{\'e}ment
ces petits {\'e}diteurs, lesquels n'ont pas acc{\`e}s aux {\'e}conomies d'{\'e}chelle, qui font les frais de ce syst{\`e}me in fine. Cela concourt {\`a} faire dispara{\^\i}tre les
petites maisons d'{\'e}dition, ainsi rachet{\'e}es par les gros {\'e}diteurs, qui s'en trouvent ainsi renforc{\'e}s, alimentant de la sorte un cercle vicieux.

Les bouquets n'ont pas seulement des effets pervers sur les d{\'e}penses des institutions qui financent la recherche et sur les petites revues, mais aussi
sur le plan scientifique. Auparavant les biblioth{\`e}ques se devaient de s{\'e}lectionner rigoureusement les revues auxquelles elles s'abonnaient, ce qui
obligeait les revues {\`a} maintenir un niveau et une qualit{\'e} scientifique suffisantes pour survivre. Aujourd'hui ce m{\'e}canisme de s{\'e}lection naturelle des
revues n'existe plus et on assiste {\`a} une prolif{\'e}ration des revues, dont certaines n'auraient pas fait long feu dans l'ancien {\'e}cosyst{\`e}me, ce que l'on
peut regretter.

Beaucoup de chercheurs et de biblioth{\'e}caires, ainsi que les consortia et les {\'e}tablissements, sont aujourd'hui pleinement
conscients de ces probl{\`e}mes et {\`a} la recherche de solutions.

\subsection{Un autre effet ind{\'e}sirable de l'{\'e}lectronique~: la r{\'e}tention de l'information}

En th{\'e}orie le passage {\`a} l'{\'e}lectronique permet de d{\'e}cupler les possibilit{\'e}s de diffusion des informations scientifiques et d'offrir l'acc{\`e}s aux r{\'e}sultats
de la recherche {\`a} plus de chercheurs et de citoyens. C'est en grande partie vrai, mais, paradoxalement, {\c c}a n'est pas forc{\'e}ment le cas pour beaucoup
d'articles publi{\'e}s dans des revues scientifiques. En effet une biblioth{\`e}que ne paye plus pour acqu{\'e}rir et conserver ind{\'e}finiment un document sur papier,
mais pour acc{\'e}der {\`a} une information d{\'e}mat{\'e}rialis{\'e}e, dont certains {\'e}diteurs gardent abusivement le contr{\^o}le. Ceux-ci peuvent alors demander un droit de p{\'e}age
pour chaque usage~: lire les publications de l'ann{\'e}e en cours, consulter des archives des ann{\'e}es pr{\'e}c{\'e}dentes ou effectuer de la fouille de donn{\'e}es.

\subsection{La r{\'e}action du monde de la recherche}

Face {\`a} ces abus des chercheurs ont cherch{\'e} {\`a} r{\'e}agir. En 2012 le math{\'e}maticien Tim Gowers a lanc{\'e} une p{\'e}tition et un appel au boycott de l'{\'e}diteur
Elsevier (\href{http://www.thecostofknowledge.com/}{\emph{The Cost of Knowledge}}~: l'engagement {\`a}
ne plus publier, ni accepter d'{\^e}tre rapporteur ou {\'e}diteur pour une revue Elsevier), qui a
recueilli une assez forte adh{\'e}sion, mais dont les effets restent limit{\'e}s. En effet la situation ne pourra pas {\'e}voluer tant que des mod{\`e}les
{\'e}conomiques stables permettant de s'affranchir du joug des {\'e}diteurs commerciaux ne seront pas en place. Diff{\'e}rents projets ont vu le jour en ce sens.

\section{L'\emph{Open Access} ou l'acc{\`e}s libre}

\subsection{L'\emph{Open Access} r{\^e}v{\'e} par les chercheurs}

La r{\'e}ponse id{\'e}ale consisterait {\`a} profiter des possibilit{\'e}s d'internet pour rendre accessible {\`a} tout le monde tous les  contenus des revues scientifiques.
Etant donn{\'e} que les co{\^u}ts de mise en ligne sont beaucoup plus bas que ceux de l'{\'e}dition traditionnelle, ceux-ci pourraient {\^e}tre support{\'e}s par les
institutions publiques et celle-ci pourraient ainsi r{\'e}aliser des {\'e}conomies. Ce r{\^e}ve, qui avait pour nom \emph{Open Access}, avait {\'e}t{\'e} publiquement formul{\'e}
en 2001 dans la d{\'e}claration de Budapest (\href{http://www.budapestopenaccessinitiative.org/}{\emph{Budapest Open Access Initiative}}),
laquelle d{\'e}claration avait
suscit{\'e} une belle frayeur chez les {\'e}diteurs.

\subsection{L'\emph{Open Access} revisit{\'e} par les {\'e}diteurs}

Des ann{\'e}es plus tard, les {\'e}diteurs se sont appropri{\'e}s le concept d'\emph{Open Access} et l'ont retourn{\'e} {\`a} leur avantage. Leur id{\'e}e est de rendre les articles
accessibles gratuitement en ligne, certes, mais en faisant payer l'auteur, ou son institution des frais de publication, appel{\'e}s souvent Publications
fees ou APC, pour Article Processing Charges. Pr{\'e}cisons que les frais de publication sont le plus souvent autour de 2000 {\`a} 3000 \euro (mais peuvent
atteindre 6 000 ou 7 000 \euro pour certaines revues). Mentionnons {\'e}galement le premier probl{\`e}me, {\'e}vident, que pose le paiement des APC par l'auteur~:
il cr{\'e}e une in{\'e}galit{\'e} entre les chercheurs pour faire reconna{\^\i}tre leur travaux, in{\'e}galit{\'e} reposant sur les finances de leur laboratoire, de leur universit{\'e},
de leur pays ou des plans de financement nationaux ou europ{\'e}ens dont ils peuvent b{\'e}n{\'e}ficier. Ce point crucial est malheureusement souvent omis dans beaucoup
d'analyses que l'on peut lire.

Nous sommes donc en face de plusieurs projets se r{\'e}clamant tous de l'\emph{Open Access}~: d'une part, ceux propos{\'e}s par les {\'e}diteurs, d'autre part, ceux que les
chercheurs ou leurs institutions tentent de mettre en place, afin d'{\'e}viter un mod{\`e}le dans lequel les chercheurs sont oblig{\'e}s de payer ou, au moins,
d'en limiter les d{\'e}g{\^a}ts. De plus nous sommes en Europe plus ou moins oblig{\'e}s de trouver une ou plusieurs solutions, car, parmi les objectifs fix{\'e}s dans
\href{https://ec.europa.eu/research/science-society/document_library/pdf_06/recommendation-access-and-preservation-scientific-information_fr.pdf}{Horizon 2020}
par la communaut{\'e} europ{\'e}enne, figurent celui de diffuser tous les r{\'e}sultats de la recherche en \emph{Open Access}!


\subsection{O{\`u} en est-on?}

La situation est en fait plus complexe encore que ce que l'on pourrait croire d'apr{\`e}s ce qui pr{\'e}c{\`e}de, tant pour les chercheurs et leurs institutions
que pour les {\'e}diteurs. A nouveau dans ce qui suit, il convient de distinguer parmi les {\'e}diteurs les groupes importants, disposant de moyens financiers
gigantesques pour investir et s'adapter {\`a} toutes ces {\'e}volutions, des petits {\'e}diteurs, qui ont le plus grand mal {\`a} suivre.

Le premier constat, paradoxal, est que, malgr{\'e} le d{\'e}veloppement rapide de l'\emph{Open Access} depuis une d{\'e}cennie, la principale d{\'e}pense des institutions et
principale source de revenues pour les {\'e}diteurs reste, de loin, les abonnements aux revues. Cela est vrai m{\^e}me au Royaume-Uni, pourtant engag{\'e} depuis
2013 dans une politique de publication syst{\'e}matique des articles de ses chercheurs en \emph{Open Access}. Le mod{\`e}le {\'e}conomique traditionnel, fond{\'e} sur les abonnements,
est donc remarquablement stable. On peut d{\'e}celer plusieurs raisons pour expliquer la lenteur d'une transition de l'ancien mod{\`e}le vers un ou plusieurs
mod{\`e}les \emph{Open Access}. Du c{\^o}t{\'e} des {\'e}diteurs, cette lenteur n'est pas un probl{\`e}me mais une b{\'e}n{\'e}diction, car la transition a un co{\^u}t mais, comme nous le
verrons plus loin, ce co{\^u}t suppl{\'e}mentaire est support{\'e} par les institutions publiques, ce qui se traduit par une majoration des b{\'e}n{\'e}fices qu'ils r{\'e}alisent.

Pour les institutions, cette lenteur est souvent due au fait que les d{\'e}cideurs tardent {\`a} prendre position, comme c'est le cas notamment en France.
Il faut {\`a} ce sujet reconna{\^\i}tre la difficult{\'e} de mettre en place une politique~: d'abord parce que nous sommes dans une situation transitoire et qu'il
est difficile de prendre du recul et encore plus d'anticiper sur les {\'e}volutions futures. Ensuite parce que les diff{\'e}rents acteurs ne sont pas toujours
d'accord~: le probl{\`e}me ne se pose de la m{\^e}me fa{\c c}on pour les universit{\'e}s, pour les organismes comme le CNRS et, \emph{a fortiori}, pour les industries de pointe
(quant {\`a} l'Acad{\'e}mie des Sciences, laquelle n'est pas concern{\'e}e par le r{\`e}glement des factures des biblioth{\`e}ques, mais au contraire per{\c c}oit de l'argent
d'un {\'e}diteur comme Elsevier, ses avis trahissent une vision un peu trop simplifi{\'e}e du probl{\`e}me). De plus, {\`a} l'int{\'e}rieur de chacune de ces instances,
plusieurs sensibilit{\'e}s diff{\'e}rentes peuvent s'opposer. Enfin les points de vue, les besoins et les habitudes peuvent varier radicalement selon des
disciplines comme la biologie et la m{\'e}decine, d'un c{\^o}t{\'e}, et les math{\'e}matiques et les sciences humaines et sociales, dont les budgets sont beaucoup
plus chiches, de l'autre.

En tr{\`e}s gros, pour les institutions, le d{\'e}bat porte sur le choix entre confier aux {\'e}diteurs priv{\'e}s la gestion de la transition vers l'\emph{Open Access},
en n{\'e}gociant avec eux pour tenter d'obtenir les meilleurs conditions financi{\`e}res possibles, ou bien construire des mod{\`e}les d'{\'e}dition g{\'e}r{\'e}s par les
institutions publiques. Pour cette deuxi{\`e}me voie, les bonnes volont{\'e}s et les id{\'e}es ne manquent pas parmi les chercheurs et les professionnels de
l'information scientifique (biblioth{\'e}caires, documentalistes, informaticiens), mais, en g{\'e}n{\'e}ral, des moyens financiers importants et une r{\'e}elle
volont{\'e} politique font d{\'e}faut.

Du point de vue des {\'e}diteurs, {\'e}tant donn{\'e} qu'il est en g{\'e}n{\'e}ral difficile de basculer instantan{\'e}ment une revue financ{\'e}e par des abonnements en une
revue \emph{Open Access} et, ce, d'autant plus que cette revue est prestigieuse, deux types de strat{\'e}gie sont mises en place. La premi{\`e}re consiste {\`a} cr{\'e}er
\emph{ex nihilo} de nouvelles revues \emph{Open Access}, financ{\'e}es par les paiements des APC. Mais alors la difficult{\'e} est que ces revues ne b{\'e}n{\'e}ficient pas
\emph{a priori} de la renomm{\'e}e et du prestige des revues d{\'e}j{\`a} bien {\'e}tablies, et donc les {\'e}diteurs ne peuvent pr{\'e}tendre que les auteurs accepteront de payer
des APC {\'e}lev{\'e}s pour y publier (mais cette transition a toutefois {\'e}t{\'e} <<~r{\'e}ussie~>> en m{\'e}decine). De ce fait, le montant des APC pour ces revues se situe
en moyenne entre 300 et 600 \euro. La deuxi{\`e}me strat{\'e}gie est une pratique beaucoup plus contestable, qui consiste {\`a} transformer des revues anciennement
financ{\'e}es par les abonnements en revues hybrides. Expliquons cela.

\subsection{Les revues hybrides ou comment les institutions payent deux fois des articles offerts gratuitement par ses chercheurs}

Une revue hybride est une revue {\`a} laquelle il est n{\'e}cessaire de payer un abonnement, si l'on veut acc{\'e}der {\`a} la totalit{\'e} de son contenu, mais qui contient
des articles en OpenAccess, {\`a} condition que les auteurs de ces articles aient vers{\'e} des APC pour cela. En somme, non seulement l'auteur c{\`e}de gratuitement
les droits patrimoniaux de son article {\`a} l'{\'e}diteur, et non seulement sa biblioth{\`e}que doit payer pour qu'il puisse acc{\'e}der {\`a} tous les articles de la revue
en question, mais encore l'auteur paye pour {\^e}tre lu gratuitement par d'autres. Les {\'e}diteurs ont coutume de justifier cette pratique en expliquant que les
versements des APC contribuent {\`a} baisser les co{\^u}ts des abonnements, mais en r{\'e}alit{\'e}, on n'observe pas de baisses de ces co{\^u}ts. C'est pourquoi,
comme l'a
\href{http://www.cnrs.fr/comitenational/doc/recommandations/2016/Recommandation-csi-INSMI-au-sujet-des-frais-de-publication-(APC).pdf}{recommand{\'e} le Conseil Scientifique de l'INSMI}
en 2016, \textbf{cette option doit {\^e}tre {\'e}vit{\'e}e},
d'autant plus que, comme nous le verrons plus loin, il existe
un moyen pour rendre gratuitement accessible les contenus de ses publications~: la voie verte.

Ajoutons que le syst{\`e}me des revues hybrides pr{\'e}sente pour les gros {\'e}diteurs d'autres avantages que ceux imm{\'e}diats d'une source compl{\'e}mentaire de revenus,
dans la mesure o{\`u} il permet d'{\'e}viter une transition brutale du mod{\`e}le avec abonnements vers un mod{\`e}le exclusivement \emph{Open Access} avec paiement d'APC.
En effet, dans une telle transition, si des entreprises comme Elsevier souhaitaient garder leur chiffre d'affaire, elles seraient oblig{\'e}es de r{\'e}clamer des
APC sup{\'e}rieurs {\`a} 7~000 \euro par article, ce qui mettrait au grand jour la r{\'e}alit{\'e}
des prix auxquels on est arriv{\'e} aujourd'hui et causerait une certaine {\'e}motion. Cela comporterait ainsi
le risque de remettre en question ces tarifs, que les {\'e}diteurs pr{\'e}f{\`e}rent certainement {\'e}viter de courir. 

\textbf{Attention~!} Le choix de publier un article sous une forme hybride, contre paiement d'un APC, n'est pas toujours clairement expliqu{\'e} lorsque l'{\'e}diteur
vous demande de remplir en ligne un contrat d'{\'e}dition. Cette option est appel{\'e}e <<~Open choice~>> chez certains {\'e}diteurs. 
De plus \textbf{il est parfois difficile de revenir sur ce choix, et certains {\'e}diteurs interdisent
de le faire}, ce qui semble contredire l'article
\href{https://www.legifrance.gouv.fr/affichCodeArticle.do?cidTexte=LEGITEXT000006069565&idArticle=LEGIARTI000006292075&dateTexte=&categorieLien=cid}{L121-20-12}
du Code de la consommation. 
Il convient donc d'{\^e}tre vigilant au moment o{\`u} l'on renseigne un contrat de
publication en ligne, surtout d{\`e}s qu'on voit le mot <<~\emph{Open Access}~>>.
Si le mal est fait, il est pr{\'e}f{\'e}rable de contacter l'{\'e}diteur commercial avant 14 jours pour lui demander d'annuler la commande
(cf. les articles
\href{https://www.legifrance.gouv.fr/affichCodeArticle.do?cidTexte=LEGITEXT000006069565&idArticle=LEGIARTI000006292075&dateTexte=&categorieLien=cid}{L121-20-12}
et \href{https://www.legifrance.gouv.fr/affichCodeArticle.do?cidTexte=LEGITEXT000006069565&idArticle=LEGIARTI000024039758}{L122-3}
du Code de la Consommation).

\subsection{Les revues pirates}

Un autre effet ind{\'e}sirable de l'\emph{Open Access} est le d{\'e}veloppement de <<~revues pirates~>>~: dans les pires des cas, il s'agit de revues tout {\`a} fait
factices, sans comit{\'e} de r{\'e}daction, ni processus de review et qui donc publient n'importe quoi, dans un simple but commercial. Il est clair que
l'existence m{\^e}me de ces revues n'est possible que parce que des chercheurs (ou des personnes souhaitant se faire passer pour des chercheurs) payent
pour publier. Nous devons mettre en garde contre ces revues, qui choisissent des noms ronflants {\'e}voquant ceux de revues prestigieuses et qui
affichent des adresses qui inspirent confiance (comme par exemple, une ville universitaire du
monde anglo-saxon), lesquelles peuvent n'{\^e}tre que de simple bo{\^\i}tes {\`a} lettres.
Enfin, entre ces revues pirates et des revues tout {\`a} fait recommandables, s'{\'e}tale une zone grise de revues, qui ne m{\'e}ritent pas le qualificatif
de <<~pirate~>> et dont le fonctionnement satisfait plus ou moins les r{\`e}gles habituelles, mais dont le niveau scientifique et l'exigence sont largement
discutables. On peut retrouver des revues de ce type dans les bouquets auxquelles les biblioth{\`e}ques sont abonn{\'e}es, mais aussi bien s{\^u}r parmi les revues
\emph{Open Access}. La prolif{\'e}ration de ces revues pose probl{\`e}me, dans la mesure o{\`u}, le plus souvent, elles contribuent {\`a} diluer la connaissance
dans un corpus o{\`u} l'on ne se retrouve plus (et que personne ne lit), augmentant le risque de publier des r{\'e}p{\'e}titions, des plagiats, quand il ne
s'agit pas d'articles faux.

\section{Que font les institutions en France et dans le Monde?}

\subsection{D{\'e}velopper des revues Open Access sans frais de publications}

\subsubsection{En France}

En math{\'e}matiques, l'INSMI, via la cellule Mathdoc, d{\'e}veloppe la plate-forme  \href{http://www.cedram.org}{\emph{Cedram}}
proposant des revues de math{\'e}matiques
en \emph{Open Access}~: ainsi les \emph{Annales Blaise Pascal}, de l'Universit{\'e} de Clermont-Ferrand,
les \emph{Annales de l'Institut Fourier} sont devenues d'acc{\`e}s libres,
sans paiement d'APC. Le \emph{Journal de l'Ecole Polytechnique} est publi{\'e} {\`a} nouveau, suivant le m{\^e}me mod{\`e}le et d'autres revues acad{\'e}miques de math{\'e}matiques
devraient suivre le mouvement. Cependant le nombre de revues concern{\'e}es est pour l'instant limit{\'e}, peut-{\^e}tre parce que les moyens mis en place sont pour
l'instant plus modestes qu'en Sciences Humaines (voir plus loin). 

Le projet \href{https://www.episciences.org/}{\emph{Episciences}} a pour ambition de proposer des revues dans toutes les disciplines. La particularit{\'e} de ce
projet est d'utiliser les d{\'e}p{\^o}ts de pr{\'e}publications comme HAL et arXiv comme support de publication~: un {\'e}pijournal est une structure dot{\'e}e d'un comit{\'e}
{\'e}ditorial fonctionnant suivant les m{\^e}mes r{\`e}gles qu'un journal traditionnel, mais dont les articles sont simplement mis en ligne sur une banque de
pr{\'e}publications. Cela permet de r{\'e}duire les co{\^u}ts de publication au minimum (m{\^e}me s'ils restent non nuls). Ce projet est mis en {\oe}uvre par le CCSD,
une Unit{\'e} Mixte de Service du CNRS (dont par ailleurs l'activit{\'e} principale est la gestion du portail HAL), avec un soutien d'Inria. Cependant,
malgr{\'e} une certaine publicit{\'e}, ce projet d{\'e}marre lentement, faute, pour l'instant, de moyens suffisants et donc d'une r{\'e}elle volont{\'e} politique de le soutenir.

Mais le plus grand succ{\`e}s en France est celui des sciences humaines, qui ont r{\'e}ussi {\`a} se doter de moyens importants et {\`a} d{\'e}velopper un mod{\`e}le
d'{\'e}dition en \emph{Open Access} qui f{\'e}d{\`e}re la moiti{\'e} de revues francophones (regroup{\'e}es dans le portail Revues.org, lui-m{\^e}me int{\'e}gr{\'e} au plus vaste
projet \href{https://www.openedition.org/}{\emph{OpenEdition}}.

\subsubsection{\`A l'{\'e}tranger}

Ce sont {\`a} nouveau les Sciences Humaines qui sont {\`a} la pointe, avec, par exemple, le projet \href{http://www.knowledgeunlatched.org/}{\emph{Knowledge Unlatched}}, dont le but est la publication de
livres {\'e}lectroniques (e-books) d'acc{\`e}s gratuits, financ{\'e}s par des souscriptions aupr{\`e}s de biblioth{\`e}ques ou encore \href{https://www.openlibhums.org/}{\emph{Open Library of Humanities}},
un projet similaire pour les revues. En math{\'e}matiques, la fondation \href{https://compositio.nl/#}{\emph{Compositio Mathematica}}, n{\'e}e aux Pays-Bas, poursuit le m{\^e}me objectif. Enfin
le projet le plus important est  \href{http://www.scielo.org}{\emph{SciELO}}, initiative br{\'e}silienne {\`a} laquelle ont adh{\'e}r{\'e} la plupart des pays d'Am{\'e}rique
latine, qui h{\'e}berge plus d'un millier de revues dont beaucoup sont \emph{Open Access} sans frais de publication. Ce projet souffre n{\'e}anmoins du fait
que beaucoup de revues y sont nouvelles, b{\'e}n{\'e}ficient donc d'un prestige limit{\'e} et peinent {\`a} attirer les bons articles, notamment en math{\'e}matiques.

Afin de se rep{\'e}rer dans ce paysage complexe, en pleine {\'e}volution, on pourra consulter des sites proposant des listes de revues \emph{OpenAccess}
avec des informations sur leurs pratiques comme \href{http://www.sherpa.ac.uk/romeo/index.php}{\emph{SHERPA/RoMEO}}.

\subsection{La voie verte et la loi <<~pour une R{\'e}publique Num{\'e}rique~>>}

Il s'agit d'une solution pour rendre accessibles les r{\'e}sultats de la recherche et r{\'e}pondre ainsi aux objectifs de l'Horizon 2020, sans avoir {\`a}
payer des APC pour cela, mais en perp{\'e}tuant le financement des revues par des abonnements. L'id{\'e}e est de poster sur des portails comme arXiv ou
HAL les contenus des articles publi{\'e}s par ailleurs dans des revues. Pour permettre aux revues de continuer {\`a} vendre des abonnements et, ainsi,
de vivre, ces contenus sont mis en lignes une fois qu'une p{\'e}riode minimale, appel{\'e}e p{\'e}riode d'embargo, s'est {\'e}coul{\'e}e depuis la publication de l'article.
C'est ce qu'on appelle la voie verte ou celle du Green \emph{Open Access}.

Suivant d'autres pays, la France s'est dot{\'e}e en 2016 d'un texte l{\'e}gislatif pr{\'e}cisant le mode d'application de ce principe pour les chercheurs financ{\'e}s
au moins pour moiti{\'e} par les deniers publics fran{\c c}ais. Il s'agit de l'article 30 de la loi <<~pour une R{\'e}publique Num{\'e}rique~>> promulgu{\'e}e le 7 octobre 2016
devenu \href{https://www.legifrance.gouv.fr/affichCodeArticle.do;jsessionid=B2842ED6626DE6921F9FD32EA69A3C93.tpdila15v_2?idArticle=LEGIARTI000033205794&cidTexte=LEGITEXT000006071190&dateTexte=20170124}{Article L533-4}
du \emph{Code de la recherche}.
Ainsi, voici ce que vous avez le droit de faire~:
\begin{itemize}
\item lorsque vous avez {\'e}crit une pr{\'e}publication, vous pouvez la poster sur HAL, arXiv ou toute autre banque d'articles accessible {\`a} tous imm{\'e}diatement
et vous pouvez y laisser ce texte ind{\'e}finiment. Vous pouvez remplacer cet article par une version plus r{\'e}cente, tant que celle-ci est ant{\'e}rieure {\`a} la date
{\`a} laquelle vous signez de contrat de cession de droits {\`a} un {\'e}diteur pour le publier. Cette disposition {\'e}tait d{\'e}j{\`a} valable avant la loi du 7 octobre 2016,
car elle d{\'e}coule du code de la propri{\'e}t{\'e} intellectuelle.
\item lorsque, une fois votre article r{\'e}vis{\'e} par le comit{\'e} de r{\'e}daction de la revue et une fois celui-ci accept{\'e}, vous signez un contrat pour sa publication,
alors, le plus souvent (mais cela d{\'e}pend de la politique de l'{\'e}diteur), vous n'avez pas le droit de mettre tout de suite en ligne le contenu de votre article
mot pour mot (et formule pour formule). Mais (et c'est l{\`a} une disposition de la nouvelle loi), pass{\'e} un d{\'e}lai maximum de 6 mois, vous avez le droit de le faire
(ce d{\'e}lai maximal est de 12 mois pour les Sciences Humaines). La loi ne s'applique pas  au fichier produit par l'{\'e}diteur (il est possible que vous n'ayez
jamais le droit de mettre en ligne).
\end{itemize}
\medskip
\noindent
Cette disposition est-elle r{\'e}tro-active~? 
En g{\'e}n{\'e}ral ce n'est pas le cas. Cependant il existe des exceptions\footnote{notamment si
\emph{la loi est d'ordre public et r{\'e}pond {\`a} des motifs imp{\'e}rieux d'int{\'e}r{\^e}t g{\'e}n{\'e}ral}}
et une {\'e}tude juridique montre que l'article
\href{https://www.legifrance.gouv.fr/affichCodeArticle.do;jsessionid=B2842ED6626DE6921F9FD32EA69A3C93.tpdila15v_2?idArticle=LEGIARTI000033205794&cidTexte=LEGITEXT000006071190&dateTexte=20170124}{L533-4}
peut {\^e}tre consid{\'e}r{\'e} comme en faisant partie. Ainsi
le Conseil Scientifique du CNRS a vot{\'e} le \href{http://www.cnrs.fr/comitenational/doc/recommandations/2017/Reco_Interpetation_de_la_loi_numerique.pdf}{24 janvier 2017}, entre autres
recommandations, celle d'appliquer cet article 
de fa{\c c}on r{\'e}tro-active~!


La m{\^e}me loi contient un article (num{\'e}ro 38, int{\'e}gr{\'e} au \emph{Code de la propri{\'e}t{\'e} intellectuelle} dans l'article
\href{https://www.legifrance.gouv.fr/affichCodeArticle.do;jsessionid=6B4C28C0CE24D0F735EB9C7F3B081B16.tpdila15v_3?cidTexte=LEGITEXT000006069414&idArticle=LEGIARTI000033219347&dateTexte=20170124&categorieLien=cid#LEGIARTI000033219347}{L342-3})
dont le but est d'autoriser un chercheur {\`a} pratiquer la fouille de textes et de donn{\'e}es (\emph{Text \& Data Mining})
dans les contenus auxquels son institution est abonn{\'e}e~:  d{\`e}s qu'un abonnement sera conclu, les {\'e}diteurs devront mettre {\`a} la disposition d'organismes publics
d{\'e}sign{\'e}s par d{\'e}cret les donn{\'e}es concern{\'e}es par cet abonnement, afin de permettre cette fouille de textes et de donn{\'e}es. Toutefois les modalit{\'e}s d'application
de cet article ne sont pas encore connues, tant que le d{\'e}cret d'application ne sera pas publi{\'e} (cela est pr{\'e}vu courant janvier 2017).

\subsection{Accord globaux pour financer l'\emph{Open Access}}

Plut{\^o}t {\`a} l'oppos{\'e} des projets pr{\'e}c{\'e}dents, une tendance se dessine dans les pays du nord de l'Europe (le Royaume-Uni depuis
2013, suivi, {\`a} partir de 2016, par les Pays-Bas, l'Allemagne\footnote{Ces m{\^e}mes pays sont aussi ceux o{\`u} les g{\'e}ants
de l'{\'e}dition sont implant{\'e}s.}, l'Autriche...)~:
n{\'e}gocier au niveau national ou d'une institution des accords commerciaux avec les {\'e}diteurs pr{\'e}voyant de payer une somme forfaitaire une fois pour
toute {\`a} un {\'e}diteur, pour que les chercheurs de l'institution concern{\'e}e puisse publier chez cet {\'e}diteur en \emph{Open Access}.
Ces accords peuvent {\'e}galement
inclure des abonnements aux bouquets. Ce type d'accord pr{\'e}sente quelques avantages~: {\'e}liminer les possibles in{\'e}galit{\'e}s entre chercheurs au sein d'une m{\^e}me
institution, contr{\^o}ler le co{\^u}t de l'\emph{Open Access} (m{\^e}me si, en g{\'e}n{\'e}ral, il n'emp{\^e}che pas une augmentation globale des co{\^u}ts
comme cela est \href{http://www.eprist.fr/wp-content/uploads/2016/11/I-IST_24EtudeJISC.pdf}{observ{\'e}} au Royaume-Uni). Cependant ils comportent
bien des risques~: les plans de <<~basculement~>> propos{\'e}s reposent sur des analyses macro-{\'e}conomiques grossi{\`e}res, sans disposer de donn{\'e}es fines et pr{\'e}cises
(les montants des abonnements pay{\'e}s par les institutions sont en g{\'e}n{\'e}ral tenus secrets, quant aux prix que co{\^u}tent les APC pour les articles en
\emph{Open Access}, aucune institution n'est capable d'en donner une estimation! sauf peut-{\^e}tre au Royaume-Uni --- seuls les {\'e}diteurs connaissent
les chiffres). De plus, du fait que ces contrats sont pr{\'e}vus prioritairement avec certains {\'e}diteurs (en l'occurrence les plus gros) et pas les autres,
cela risque de fausser le march{\'e} de l'\emph{Open Access} en d{\'e}faveur des petits {\'e}diteurs (une fois de plus), puisque les APC pour ceux-ci devraient
{\^e}tre pay{\'e}s s{\'e}par{\'e}ment. Tout cela ne ferait qu'accro{\^\i}tre davantage la concentration de l'industrie de l'{\'e}dition
contre laquelle nous mettent en garde la COAR et
l'UNESCO dans leur 
\href{http://www.unesco.org/new/fileadmin/MULTIMEDIA/HQ/CI/CI/pdf/news/coar_unesco_oa_statement.pdf}{appel}.

Ainsi ces accords engageraient de fa{\c c}on irr{\'e}versible et massive les budgets des biblioth{\`e}ques, d{\'e}tournant ainsi ces moyens
de politiques de d{\'e}veloppement des mod{\`e}les d'{\'e}dition plus vertueux que nous avons mentionn{\'e}s plus haut et aboutissant {\`a}
une situation dans laquelle on n'aura pas rem{\'e}di{\'e} aux effets ind{\'e}sirables observ{\'e}s actuellement, notamment sur le plan scientifique,
et on aura confi{\'e} la gestion de ces probl{\`e}mes {\`a} de grandes entreprises commerciales.

\section{Au del{\`a} des publications}


\subsection{L'{\'e}valuation}

Apr{\`e}s cet {\'e}tat des lieux, il est bon de se demander pourquoi nous publions dans des revues dont le fonctionnement est si on{\'e}reux. Il appara{\^\i}t
clairement aujourd'hui que la raison premi{\`e}re n'est plus la diffusion des connaissances et des r{\'e}sultats de la recherche comme on pouvait le
clamer nagu{\`e}re, puisque, pour cela, il suffit de d{\'e}poser nos articles sur une banque de pr{\'e}publications comme \href{https://hal.archives-ouvertes.fr/}{HAL}
ou \href{https://arxiv.org/}{arXiv}. La raison est donc
la n{\'e}cessit{\'e} d'{\^e}tre {\'e}valu{\'e} par un comit{\'e} de r{\'e}daction et d'{\^e}tre ainsi reconnu. Une autre raison essentielle et r{\'e}elle pour publier dans des revues
est la constitution d'un corpus de connaissances stable et auquel les g{\'e}n{\'e}rations futures pourront se r{\'e}f{\'e}rer sans ambigu{\"\i}t{\'e}, mais il faut reconna{\^\i}tre
que cette seconde raison, beaucoup plus noble, n'est certainement pas la motivation premi{\`e}re du chercheur qui soumet un article {\`a} une revue.
Repenser le processus de l'{\'e}valuation, en s'affranchissant du joug des {\'e}diteurs priv{\'e}s, vendant cher journaux et outils d'{\'e}valuation <<~clefs en main~>>
mais mal ficel{\'e}s (par la bibliom{\'e}trie), est le d{\'e}fi des scientifiques de demain!

\subsection{Les r{\'e}seaux sociaux}

Les r{\'e}seaux sociaux scientifiques comme \emph{ResearchGate} (ou \emph{Academia}) offrent des possibilit{\'e}s tr{\`e}s int{\'e}ressantes pour acc{\'e}der
{\`a} des articles, des pr{\'e}publications~: l'inscription {\`a} ces r{\'e}seaux donne acc{\`e}s {\`a} un nombre croissant de tels documents,
ainsi qu'{\`a} des projets et des {\'e}changes scientifiques et permet d'y participer. Ces r{\'e}seaux sont tr{\`e}s efficaces, ainsi ils
op{\`e}rent automatiquement une fouille des publications se rapportant {\`a} un auteur sur la toile, l'aidant ainsi {\`a} constituer une banque
de textes dont il est l'auteur. On peut donc les utiliser avec profit.

Mais il faut cependant rester prudent et s'interroger sur certains points. Par exemple~: si un auteur d{\'e}pose une pr{\'e}publication sur un
tel r{\'e}seau, en conserve-t-il la propri{\'e}t{\'e}? S'il s'agit de la propri{\'e}t{\'e} intellectuelle et si le droit fran{\c c}ais s'applique,
la r{\'e}ponse est claire, car, gr{\^a}ce au code de la propri{\'e}t{\'e} intellectuelle, l'auteur est prot{\'e}g{\'e} et garde ind{\'e}finiment
la propri{\'e}t{\'e} d'un texte ou d'une {\oe}uvre.
En revanche la situation est plus floue en ce qui concerne la propri{\'e}t{\'e} patrimoniale~: l'auteur a-t-il le droit de signer un contrat de
publication avec un {\'e}diteur pour publier un texte r{\'e}dig{\'e} sur un tel r{\'e}seau social? Et inversement, le r{\'e}seau social peut-il
pr{\'e}tendre avoir des droits de publication sur ce texte? Il n'y a pas de r{\'e}ponse claire {\`a} ces questions {\`a} cause du vide juridique
sur le statut patrimonial de ces documents. Un risque est que ces r{\'e}seaux sociaux, dont l'usage est gratuit pour l'instant soient un jour
vendus {\`a} un gros {\'e}diteur, qui r{\'e}cup{\'e}rera ainsi des donn{\'e}es pr{\'e}cieuses comme les contenus scientifiques et aussi des
indices d'{\'e}valuation des chercheurs (tels que ceux produits automatiquement \emph{ResearchGate}). Le cas s'est d{\'e}j{\`a} produit avec notamment
le r{\'e}seau social \emph{Mendeley}, rachet{\'e} par Elsevier. 

D'autres moyens <<~libres~>>, mis au point par des {\'e}quipes qui n'ont pas de finalit{\'e} commerciale, sont offerts au chercheur. Ceux-ci sont
encore {\`a} l'{\'e}tat de projets et on peut esp{\'e}rer qu'ils se d{\'e}velopperont, si les institutions publiques donnent un petit coup de pouce.
L'un d'eux est le projet  \href{http://dissem.in/}{\emph{dissemin}}, il permet {\`a} un chercheur de se constituer tr{\`e}s rapidement une banque
d'articles dont il est l'auteur. Un autre projet int{\'e}ressant est le  \href{http://sjscience.org/}{\emph{Self Journal of Science}}, qui repose
sur un concept original et qui pourrait {\^e}tre une alternative int{\'e}ressante au processus d'{\'e}valuation traditionnel.

\subsection{Les portails}
Outre les syst{\`e}mes d'acc{\`e}s {\'e}lectronique {\`a} la documentation mis {\`a} la disposition des chercheurs
par leurs biblioth{\`e}ques ou leurs Services Communs de la Documentation, des portails nationaux ou europ{\'e}ens leur sont {\'e}galement propos{\'e}s.
Le \href{https://portail.math.cnrs.fr/}{\emph{Portail Math}} est d{\'e}velopp{\'e} par l'INSMI via la cellule \href{http://www.mathdoc.fr/}{\emph{Mathdoc}},
avec le soutien du r{\'e}seau \href{https://www.mathrice.fr/}{\emph{Mathrice}} et du \href{http://www.rnbm.org/}{\emph{RNBM}}.
Un de ses objectifs est d'offrir un acc{\`e}s personnalis{\'e} et simple {\`a} la documentation math{\'e}matique.
Il offre {\'e}galement des services num{\'e}riques.

A un niveau interdisciplinaire, le portail \href{https://bib.cnrs.fr/}{\emph{BibCnrs}},
refond{\'e} r{\'e}cemmment, donne acc{\`e}s aux ressources documentaires acquises par
l'\href{http://www.inist.fr/}{\emph{Inist}} pour le compte du CNRS.

Enfin le portail \href{https://eudml.org/}{\emph{EuDML}} offre une collection d'articles
en acc{\`e}s libre mise {\`a} disposition par un r{\'e}seau europ{\'e}en d'institutions.

%%%%%%%%%%%%%%%%%%%%%%%%%%%%%%%%%%%

\chapter{Concilier travail et vie de famille}

Vous pouvez trouver la plupart des informations r\'esum\'ees ici sur
le portail de l'administration fran\c{c}aise:\\
\lien{www.service-public.fr/}

%%%%%%%%%%%%%%%%%%%%%%%%%%%%%%%%%%%%%%
\section{Le cong\'e de maternit\'e}

Toutes les salari\'ees, du priv\'e comme du public, ont droit au
cong\'e de maternit\'e.
Il est \`a noter que vous pouvez d\'ecaler ce cong\'e. Ceci veut dire
que, par exemple pour un premier enfant, vous n'\^etes pas oblig\'ee de
respecter six semaines d'arr\^et pr\'enatal et dix semaines d'arr\^et
postnatal~: vous pouvez reporter une partie du cong\'e pr\'enatal en
cong\'e postnatal apr\`es accord de votre m\'edecin et
\`a condition de conserver un minimum de
2 semaines d'arr\^et pr\'enatal.\\

Si vous \^etes enseignante-chercheuse, vous vous inqui\'eterez ensuite
de savoir quel volume horaire vous aurez \`a enseigner l'ann\'ee de
votre cong\'e. Quelle que soit la date d'accouchement, la d\'echarge de service est de 96h \'eq TD. Une fiche r\'ecapitulative est ici\\
{\footnotesize\lien{www.snesup.fr/conge-de-maternite}}\\
et elle fait r\'ef\'erence \`a la circulaire 2012-0009 du 30-4-2012 valable pour les cong\`es de maternit\'e, de paternit\'e et les cong\`es de maladie

\url{http://www.enseignementsup-recherche.gouv.fr/pid20536/bulletin-officiel.html?cid_bo=60265&cbo=1},


%Auparavant, chaque \'etablissement faisait sa propre cuisine interne. Le seul et unique texte qui pouvait faire r\'ef\'erence \'etait la circulaire DPE A2/FD 892 du 7 novembre 2001, que vous pouvez t\'el\'echarger \`a l'adresse \\ \lien{www.snesup.fr/webuploads/download/492\_0.}\\

%Ce texte vous garantit que, si votre cong\'e tombe int\'egralement pendant l'ann\'ee scolaire, on ne peut vous demander plus de la moiti\'e de votre charge si c'est votre premier ou votre deuxi\`eme enfant,  plus du cinqui\`eme de votre charge si c'est au moins votre troisi\`eme enfant, et aucun service s'il s'agit de naissances multiples. Les probl\`emes commencent bien \'evidemment quand votre cong\'e tombe \`a cheval sur les vacances scolaires...{}\\

% Voici une petite synth\`ese de ce que nous avons pu constater.
%\begin{itemize}
%\item Le d\'egr\`evement horaire simple~: on vous calcule une moyenne
%$N$ d'heures par semaine ``ouverte" (hors vacances
%scolaires). Votre cong\'e de maternit\'e peut alors contenir un minimum
%de six semaines ouvertes et vous obtiendrez donc une d\'echarge de
%$6N$ heures. Appliqu\'e tel quel, ce calcul respecte rarement la
%circulaire.

%\item Le d\'egr\`evement horaire forfaitaire~: quel que soit le moment
%auquel tombe votre cong\'e, on vous retire un nombre forfaitaire
%d'heures, typiquement un tiers de service (cas fr\'equemment
%rencontr\'e). Ce calcul peut \^etre avantageux, sauf si votre
%cong\'e tombe int\'egralement pendant l'ann\'ee scolaire~; il est
%alors en opposition avec la circulaire (sauf dans les rares cas o\`u
%le d\'egr\`evement est d'un demi-service).

%\item Le d\'egr\`evement horaire avec coefficient multiplicateur~: c'est le
%seul qui semble respecter la circulaire, mais c'est de loin le moins
%appliqu\'e. Le calcul est le m\^eme que pour le d\'egr\`evement
%simple, mais le volume horaire restant est multipli\'e par un
%coefficient (0,75 par exemple) pour \^etre en conformit\'e avec la
%circulaire. Ce coefficient est parfois justifi\'e par un
%am\'enagement du temps de travail de la femme enceinte.
%\end{itemize}
%Quel que soit le mode de calcul retenu par votre \'etablissement,
%n'h\'esitez pas \`a contacter le service du personnel pour faire
%respecter vos droits et l'application de la circulaire.

Enfin, sachez que vous pouvez pr\'etendre \`a un CRCT de 6 mois \`a la suite d'un cong\'e maternit\'e, voir la section  \ref{CRCT}. Dans le d\'ecret 84-431 du 6 juin 1984 - Article 19, il est mentionn\'e que ``Un cong\'e pour recherches ou conversions th\'ematiques, d'une dur\'ee de six mois, peut \^etre accord\'e apr\`es un cong\'e maternit\'e ou un cong\'e parental, \`a la demande de l'enseignant$\cdot$e-chercheur$\cdot$se."\\

Nous terminons cette section par quelques liens int\'eressants :\\
%\lien{postes.smai.emath.fr/parite/journee/ConfLBroze.pdf}\\
\lien{postes.smai.emath.fr/apres/parite/}\\
\lien{listes.mathrice.fr/math.cnrs.fr/info/forum-parite}\\
%\lien{postes.smai.emath.fr/parite/ipa.php}\\
%\lien{postes.smai.emath.fr/parite/journee/journee\_parite.php}\\

%%%%%%%%%%%%%%%%%%%%%%%%%%%%%%%%%
\section{Cong\'e parental et temps partiel}

Tout salari\'e a droit de demander un cong\'e parental (dans les trois
premi\`eres ann\'ees suivant une naissance ou une adoption) ou \`a travailler \`a temps
partiel. Ceci est bien s\^ur valable pour les chercheur$\cdot$ses et les
enseignant$\cdot$es-chercheur$\cdot$ses. En cas de cong\'e parental, vous n'\^etes plus
r\'emun\'er\'e$\cdot$e mais vos ann\'ees de cong\'e compteront pour la
retraite. En cas de temps partiel, vous \^etes alors pay\'e$\cdot$e au {\em
pro rata} de votre temps de travail, \`a une exception pr\`es~: si
vous souhaitez vous mettre \`a 80\,\%. Dans ce cas, vous toucherez
85,7\,\% de votre salaire. Il est \`a noter que les primes (par exemple
d'enseignement sup\'erieur et de recherche) ou le suppl\'ement
familial de traitement seront aussi calcul\'es au {\em pro rata}.

Dans le cas o\`u vous avez des enfants en bas \^age, votre Caisse
d'allocations familiales (CAF) pourra vous verser un compl\'ement de r\'emun\'eration. En d\'ebut
de carri\`ere, il est parfois plus avantageux financi\`erement de travailler \`a
80\,\% tant que le compl\'ement CAF peut vous \^etre vers\'e. Une bonne fa\c con
de reprendre l'enseignement en douceur apr\`es un cong\'e maternit\'e, un cong\'e
parental ou simplement l'arriv\'ee d'un enfant puisque le compl\'ement CAF peut \^etre
vers\'e aux jeunes mamans comme aux jeunes papas!

De m\^eme qu'\`a la suite d'un cong\'e maternit\'e, vous pouvez pr\'etendre,
apr\`es un cong\'e parental, \`a un CRCT de 6 mois, voir la section \ref{CRCT}.

%%%%%%%%%%%%%%%%%%%%%%%%%%%%%%%%
\section{Arr\^et maladie ou cong\'e de paternit\'e}

Toujours dans la circulaire 2012-0009 du 30-4-2012 cit\'ee plus haut, il est pr\'ecis\'e qu'on ne peut demander \`a un$\cdot$e
enseignant$\cdot$e-chercheur$\cdot$se de rattraper les heures qu'il n'aurait pu
effectuer suite \`a un arr\^et maladie. Typiquement, si vous \^etes
malade un jour o\`u vous deviez effectuer 10 heures d'enseignement,
ces heures sont consid\'er\'ees comme ayant \'et\'e effectu\'ees et
doivent vous \^etre comptabilis\'ees, tout comme \`a la personne qui
vous a remplac\'e$\cdot$e le cas \'ech\'eant. Et toute heure effectu\'ee en
plus de votre service doit vous \^etre pay\'ee en heure
suppl\'ementaire. Nous ne pouvons donc que vous conseiller de
d\'eposer vos arr\^ets maladie, m\^eme de courte dur\'ee.

Le cong\'e de paternit\'e est constituer de 11 jours (ou de 18 jours en cas de naissances multiples) 
\`a poser de mani\`ere cons\'ecutive dans les 4 mois qui suivent la naissance. 




%%%%%%%%%%%%%%%%%%%%%%%%%%%%%%%%%%%%%%%%%%
%%%%%%%%%%%%%%%%%%%%%%%%%%%%%%%%%%%%%%%%%%
\part{Les instances officielles}

%%%%%%%%%%%%%%%%%%%%%%%%%%%%%%%%%%%%%%%%%%%%%%%%%%%%%%
%%%%%%%%%%%%%%%%%%%%%%%%%%%%%%%%%%%%%%%%%%%%%%%%%%%%%%

\chapter{Le minist\`ere}
\label{chapMinistere}
\index{Minist{\`e}re de l'Enseignement Sup{\'e}rieur, de la Recherche et de l'Innovation (MESRI)}

Depuis mai 2017 le Minist{\`e}re de l'\'Education Nationale,  l'Enseignement Sup{\'e}rieur et de la Recherche (MENESR) a \'et\'e remplac\'e par le Minist{\`e}re  de l'Enseignement Sup{\'e}rieur, de la Recherche et de l'Innovation (MESRI).
\newline
Pour un organigramme complet, on peut se reporter \`a la page du minist\`ere.\\
{\small\lien{www.enseignementsup-recherche.gouv.fr/pid24542/index.html}}

Cet organigramme contient diff{\'e}rentes directions dont le r\^ole est de proposer et de mettre en \oe{}uvre, dans leur champ de comp\'etences, la politique du minist\`ere.
\newline
\newline
Le MESRI interagit avec de nombreux organismes, \'etablisse\-ments, agences et conseils tels que:
\begin{itemize}
 \item \textbf{Organismes sous tutelle :} Etablissements d'enseignement sup{\'e}rieur, Grandes {\'e}coles, Universit{\'e}s,
 Centre national des oeuvres universitaires et scolaires (CNOUS), Centres r{\'e}gionaux des oeuvres universitaires et scolaires (CROUS).
 
 \item \textbf{Organismes de recherche :} Etablissements publics {\`a} caract{\`e}re scientifique et technologique (EPST),
 Etablissements publics {\`a} caract{\`e}re industriel et commercial (EPIC), Etablissements publics {\`a} caract{\`e}re administratif (EPCA),  Groupements d'int{\'e}r{\^e}t public (GIP), Fondations.
 
 \item \textbf{Haut conseil d'{\'e}valuation :} Haut Conseil de l'{\'e}valuation de la recherche et de l'enseignement sup{\'e}rieur (HCERES) (voir le chapitre \ref{HCERES}).
 
 \item \textbf{Agences de financement :} Bpifrance ({\small\lien{www.bpifrance.fr}}) pour l'accompagnement des entreprises, Agence nationale de la recherche (ANR,
 {\small\lien{www.agence-nationale-recherche.fr}}).

 \item \textbf{Structures de consultation :} Conf{\'e}rence des pr{\'e}sidents d'universit{\'e} (CPU), 
 Conf{\'e}rence des directeurs d'{\'e}coles fran\c caises d'ing{\'e}nieurs (CDEFI),
 Conseil national de l'enseignement sup{\'e}rieur et de la recherche (CNESER), Haut conseil des biotechnologies, 
 Conseil strat{\'e}gique de la recherche (CSR) (a remplac{\'e} le Haut conseil de la science et de la technologie).
\end{itemize}
Pour plus d'exhaustivit\'e, on pourra se r\'ef\'erer au lien suivant: \\
{\small \lien{www.enseignementsup-recherche.gouv.fr/pid24572/index.html}}
\newline
\newline
Les paragraphes suivants d\'ecrivent certaines instances, internes ou externes au minist\`ere, 
intervenant directement sur les questions d'enseignement sup\'erieur et de recherche.
\newline
\newline
Outre ces fonctions ``strat\'egiques", le minist\`ere a \'egalement d'autres activit\'es qui concernent directement les
jeunes math\'ematiciennes et jeunes math\'ematiciens comme l'{\bf expertise des dossiers de coop\'eration} (tels que les PHC, voir le chapitre \ref{PHC}).

\section{La DGESIP} \label{DGESIP}
La Direction g\'en\'erale de l'enseignement sup\'erieur et de l'insertion professionnelle (DGESIP) a
pour principale mission l'\'elabo\-ration et la mise en \oe{}uvre
de la politique relative \`a l'ensemble des formations
post\'erieures au baccalaur\'eat initiales (Licence, Master, Doctorat) et continues
relevant du ministre en charge de l'enseignement sup\'erieur.
\index{Direction g\'en\'erale de l'enseignement sup\'erieur et de l'insertion professionnelle\\(DGESIP)}
\index{Licence, Master, doctorat (LMD)}
Pour plus de d\'etails, on pourra consulter le site du minist\`ere:\\
{\small \lien{www.enseignementsup-recherche.gouv.fr/cid24149/index.html}}

\section{La DGRI}
L'activit\'e de la Direction g\'en\'erale de la recherche et de l'innovation (DGRI) s'articule principalement autour de deux axes~: 
l'\'elaboration et la mise en \oe uvre de la politique de l'\'Etat en mati\`ere de recherche et d'emploi scientifique
et le pilotage des programmes de la mission interminist\'erielle de recherche et d'enseignement sup\'erieur (MIRES).
\index{Direction g\'en\'erale de la recherche et de l'innovation (DGRI)}
\index{Mission interminist\'erielle de recherche et d'enseignement sup\'erieur (MIRES)}

La DGRI veille d'abord \`a la coh\'erence et \`a la qualit\'e du
syst\`eme fran\c cais de recherche et d'innovation, en liaison avec
l'ensemble des minist\`eres concern\'es (finances, industrie,
affaires \'etrang\`eres, {\em etc.}). Elle d\'efinit les
orientations de la politique scientifique nationale ainsi que les
priorit\'es de recherche des \'etablissements d'enseignement
sup\'erieur. Elle assure leur mise en \oe uvre par la tutelle
strat\'egique des organismes relevant du minist\`ere en charge de la
recherche et contribue \`a la politique de l'innovation et de la
recherche industrielle.

Enfin, la DGRI assure le secr\'etariat permanent du Conseil strat\'egique de la recherche (CSR) dont elle pr\'epare les travaux.
Le CSR un organisme cr\'e\'e en 2013 et plac\'e aupr\`es du Premier ministre fran\c cais pour proposer les grandes orientations de la strat\'egie nationale 
de recherche scientifique, et participer \`a l'\'evaluation de leur mise en \oe uvre.
\index{Conseil strat\'egique de la recherche (CSR)}

\`A l'\'echelle europ\'eenne et internationale, la DGRI d\'efinit les mesures n\'ecessaires \`a la construction de l'espace europ\'een de l'enseignement sup\'erieur et de la recherche en liaison avec la DGESIP et la Direction des relations europ\'eennes, internationales et de la coop\'eration (DREIC).
\index{Direction des relations europ\'eennes, internationa\-les et de la coop\'eration (DREIC)}

Au titre de la politique territoriale de la recherche, la DGRI est charg\'ee de la politique d'organisation territoriale des activit\'es de recherche, en liaison avec la DGESIP. Elle assure le suivi des contrats de plan \'Etat-R\'egions pour ce qui concerne les \'etablissements de recherche dont elle a la tutelle.
\index{Contrats de plan \'Etat-R\'egions (CPER)}
Elle coordonne aussi l'activit\'e des d\'el\'egu\'es r\'egionaux \`a la recherche et \`a la technologie charg\'es de l'action d\'econcentr\'ee de l'\'Etat pour la recherche et l'innovation.
\index{D\'el\'egation r\'egionale \`a la recherche et \`a la technologie (DRRT)}

La DGRI r\'epartit entre les organismes dont elle a la tutelle (la
plupart des EPST et EPIC) les moyens n\'ecessaires \`a
l'accomplissement de leurs missions, met en place et entretient en
concertation avec ces organismes les indicateurs de performance afin
de rendre compte de l'efficacit\'e des moyens engag\'es. Cela
concerne, entre autres, le BRGM, le CEA, le CNRS,
l'IFPEN, l'Ifremer, l'IFSTTAR, l'Inra, Inria, l'Inserm, l'IRD, Irstea,
l'Onera, {\em etc.}

\index{Etablissement public \`a caract\`ere scientifique \\et technologique (EPST)}
\index{Etablissement public \`a caract\`ere industriel et commercial (EPIC)}
\index{Bureau de recherches g\'eologiques et mini\`eres \\(BRGM)}
\index{Commissariat \`a l'\'energie atomique (CEA)}
\index{Conseil national de la recherche scientifique \\(CNRS)}
\index{IFP Energies nouvelles (IFPEN)}
\index{Institut fran\c cais de recherche pour l'exploitation de la mer (Ifremer)}
\index{Institut fran\c ais de sciences et technologies des transports, de l'am\'enagement et des r\'eseaux (IFSTTAR)}
\index{Institut national de recherche agronomique (Inra)}
\index{Institut national de recherche en informatique et en automatique (Inria)}
\index{Institut national de la sant\'e et de la recherche m\'edicale (Inserm)}
\index{Institut de recherche \\pour le d\'eveloppement (IRD)}
\index{Institut national de recherche en sciences et technologies pour l'environnement et l'agriculture (Irstea)}
\index{Office national d'\'etudes \\et de recherches a\'erospatiales (Onera)}

Pour une description plus d\'etaill\'ee, nous renvoyons au site du minist\`ere: \\
{\small \lien{www.enseignementsup-recherche.gouv.fr/cid24148/index.html}}

\section{La DREIC}\label{DREIC}
La Direction des relations europ\'eennes, internationales, et de la coop\'eration (DREIC) d\'epend du Secr\'etariat g\'en\'eral du minist\`ere (voir paragraphe~\ref{sgm}). Elle coordonne le d\'eveloppement, les \'echanges et la coop\'eration avec les syst\`emes scolaires, universitaires et de recherche \'etrangers. \`A cette fin, elle contribue \`a la pr\'eparation des accords bilat\'eraux (voir, par exemple, les partenariats Hubert-Curien (PHC) au paragraphe \ref{PHC}), ainsi qu'\`a celle des projets conduits dans le cadre des organisations europ\'eennes ou internationales. Elle apporte son concours \`a la DGESIP et \`a la DGRI pour la d\'efinition des mesures n\'ecessaires \`a la construction de l'espace europ\'een de l'enseignement sup\'erieur et de la recherche. Elle pr\'epare les positions du minist\`ere et assure sa repr\'esentation dans les instances et rencontres internationales, notamment dans les conseils et comit\'es europ\'eens de l'\'education.
La DREIC travaille en \'etroite collaboration avec le minist\`ere des affaires \'etrang\`eres.
\index{Direction des relations europ\'eennes, internationa\-les et de la coop\'eration (DREIC)}

{\small \lien{www.enseignementsup-recherche.gouv.fr/cid20297/index.html}}

\section{La DEPP}
La Direction de l'\'evaluation, de la prospective et de la performance (DEPP) est charg\'ee de la conception et de la gestion du syst\`eme d'information statistique en mati\`ere d'enseignement et de recherche. Elle con\c coit et met en \oe uvre, \`a la demande des autres directions, un programme d'\'evaluations, d'enqu\^etes et d'\'etudes sur tous les aspects du syst\`eme de recherche.
\index{Direction de l'\'evaluation, de la prospective et de la performance (DEPP)}

{\small \lien{www.enseignementsup-recherche.gouv.fr/cid20296/index.html}}

\section{Le Secr\'etariat g\'en\'eral} \label{sgm}

Le Secr\'etariat g\'en\'eral, plac\'e sous l'autorit\'e du minist\`ere, regroupe l'ensemble des directions et services venant en soutien des directions op\'erationnelles des minist\`eres (DGESIP, DGRI pour ce qui nous concerne). On y trouve aussi la DREIC, la DEPP, mais \'egalement la Direction g\'en\'erale des ressources humaines (DGRH), dont tous les enseignants-chercheurs d\'ependent, {\it via} leur \'etablissement d'affectation. \\
{\small \lien{www.enseignementsup-recherche.gouv.fr/cid20292/index.html}} \\
{\small \lien{www.education.gouv.fr/cid1173/index.html}}
\index{Direction g\'en\'erale des ressources humaines\\(DGRH)}

\section{Le CNESER}\label{CNESER}

\index{Conseil national de l'enseignement sup\'erieur et de la recherche (CNESER)}

Parmi les structures consultatives du minist\`ere cit\'ees dans le sch\'ema externe,
le Conseil national de l'enseignement sup\'erieur et de recherche (CNESER) est l'instance de r\'ef\'erence pour le minist\`ere sur toutes les questions d'enseignement
sup\'erieur et de recherche \`a l'exception de celles touchant au statut des personnels. Y sont abord\'es, entre autres,
\begin{itemize}
\item la politique g\'en\'erale de l'enseignement sup\'erieur~;
\item les grands projets de r\'eforme (lors du passage au LMD par
exemple)~;
\index{Licence, Master, doctorat (LMD)}
\item les budgets des universit\'es, les programmes et
demandes de cr\'edits~;
\item les habilitations des divers
dipl\^omes (Licence, Master, {\em etc.})~;
\item les reconnaissances
des \'ecoles doctorales~;
\item l'ensemble des textes de loi et
d\'ecrets concernant l'enseignement sup\'erieur et la recherche.
\end{itemize}
{\small \lien{www.enseignementsup-recherche.gouv.fr/cid53497/index.html}}
% Outre le ministre, il comprend 68 membres, dont 45 repr\'esentants
% \'elus des universit\'es et \'etablissements assimil\'es, r\'epartis
% comme suit~:
% \begin{itemize}
% \item 5 repr\'esentants des chefs d'\'etablissements~;
% \item 22 enseignants-chercheurs, enseignants ou chercheurs (dont 11
% professeurs des universit\'es ou assimil\'es)~;
% \item 11 \'etudiants~;
% \item 7 repr\'esentants des personnels non-enseignants dont un
% conservateur des biblioth\`eques.
% \end{itemize}
% Il est pr\'esid\'e par
% le ministre charg\'e de l'enseignement sup\'erieur. Il regroupe en
% son sein une commission scientifique permanente charg\'ee de
% pr\'eparer les travaux du conseil en mati\`ere de recherche,
% d'enseignement et de dipl\^omes de 3\ieme{} cycle, et une
% section permanente qui assure l'ensemble des sessions du conseil
% national en dehors des sessions pl\'eni\`eres.


%%%%%%%%%%%%%%%%%%%%%%%%%%%%%%%%%%%%%%%%%%%%
%%%%%%%%%%%%%%%%%%%%%%%%%%%%%%%%%%%%%%%%%%%%
\chapter{Les universit\'es}
\label{universite}
\index{Universit\'e}
Les universit\'es participent, en tant qu'\'etablissements d'enseignement sup\'erieur et de recherche,
au service public de l'enseignement sup\'erieur, dont les missions sont ainsi d\'efinies par la 
 \href{https://www.legifrance.gouv.fr/affichCodeArticle.do?cidTexte=LEGITEXT000006071191&idArticle=LEGIARTI000027747739&dateTexte=20170113}{loi n$ \textdegree$2013-660 du 22 juillet 2013 - art. 7 :}
\begin{enumerate}
 \item la formation initiale et continue,
 \item la recherche scientifique et technologique, la diffusion et la valorisation de ses r\'esultats,
 \item l'orientation, la promotion sociale et l'insertion professionnelle,
 \item la diffusion de la culture humaniste, scientifique, technique et industrielle,
 \item  la coop\'eration internationale.
\end{enumerate}

\quad

Les \'etablissements d'enseignement sup\'erieur et de recherche sont regroup\'es sous l'appellation Etablissements publics \`a
caract\`ere scientifique, culturel et professionnel (EPCSCP), et sont constitu\'es de~:
\begin{itemize}
\item 71 universit\'es,
\item 1 institut national polytechnique,
\item 19 instituts et \'ecoles ext\'erieurs aux universit\'es (INSA, \'Ecoles Centrales, Universit\'es Technologiques, {\em etc.}) ;
\item 20 grands \'etablissements de statuts divers~;
\item 4 \'ecoles normales sup\'erieures,
\item 5 \'ecoles fran\c caises \`a l'\'etranger~;
\item 21 communaut\'es d'universit\'es et \'etablissements  (COMUE ou ComUE)
\end{itemize}
\index{\'etablissements publics \`a caract\`ere scientifique, culturel et professionnel (EPCSCP)}
\index{communaut\'es d'universit\'es et \'etablissements (COMUE)}
Plus d'information sur le \href{http://www.enseignementsup-recherche.gouv.fr/cid20263/index.html}{site du minist\`ere}.\\

Nous utiliserons pour tous ces \'etablissement le mot \textit{universit\'es} pour plus de commodit\'e,
m\^eme s'il faut garder \`a l'esprit qu'il subsiste certaines diff\'erences au sein des EPSCP entre ceux
qui sont des universit\'es et les autres : un$\cdot$e pr\'esident$\cdot$e d'universit\'e est un$\cdot$e directeur$\cdot$trice pour d'autres
\'etablissements (ils ou elles ne sont pas nomm\'e$\cdot$es de la m\^eme mani\`ere, mais leurs pr\'erogatives sont
tr\`es proches), les noms et r\^oles des conseils peuvent diff\'erer, {\em etc.}\\

L'organisation et le fonctionnement des universit\'es sont r\'egis
par le \href{http://www.legifrance.gouv.fr/affichCode.do?cidTexte=LEGITEXT000006071191}{Code de l'\'education}. \\

Nous allons pr\'esenter maintenant bri\`evement les diff\'erentes instances de l'universit\'e.


%%%%%%%%%%%%%%%%%%%%%%%%%%%%%%%%%%%%%%%

\section{La Pr\'esidence}
Le ou la pr\'esident$\cdot$e de l'universit\'e est \'elu$\cdot$e \`a la majorit\'e absolue par les membres \'elu$\cdot$es du Conseil d'administration, 
pour un mandat de quatre ans (renouvelable une fois). 

Pour les d\'etails on peut se r\'ef\'erer au code de l'\'education :
\begin{enumerate}
 \item \href{https://www.legifrance.gouv.fr/affichCodeArticle.do;jsessionid=C8BFF801F2976E9272297AB33338C553.tpdila14v_3?idArticle=LEGIARTI000027747943&cidTexte=LEGITEXT000006071191&dateTexte=20170113}{Article L712-1, modifi\'e par Loi n$\textdegree$2013-660 du 22 juillet 2013 - art. 45}
 \item \href{https://www.legifrance.gouv.fr/affichCodeArticle.do;jsessionid=C8BFF801F2976E9272297AB33338C553.tpdila14v_3?idArticle=LEGIARTI000027747947&cidTexte=LEGITEXT000006071191&dateTexte=20170113}{Article L712-2, modifi\'e par Loi n$\textdegree$2013-660 du 22 juillet 2013 - art. 46}
\end{enumerate}

Au niveau national, les pr\'esident$\cdot$es d'universit\'e sont regroup\'e$\cdot$es en Conf\'erence des pr\'esidents d'universit\'e (CPU), cf. chapitre \ref{chapMinistere}.

\section{Conseil centraux}
Depuis la \href{https://www.legifrance.gouv.fr/affichTexte.do;jsessionid=C8BFF801F2976E9272297AB33338C553.tpdila14v_3?cidTexte=JORFTEXT000027735009&dateTexte=20170113}{loi du 22 juillet 2013} relative {\`a} l'enseignement sup{\'e}rieur et {\`a} la recherche, 
deux conseils contribuent \`a la gouvernance des universit\'es~: le Conseil d'administration et le Conseil acad{\'e}mique. Ces conseils sont consult\'es et votent sur 
l'orientation politique de l'universit{\'e}. Ils sont constitu\'es de repr{\'e}sentants des enseignant$\cdot$es, des chercheur$\cdot$ses, des personnels administratifs et techniques, 
des {\'e}tudiant$\cdot$es et de personnalit{\'e}s ext{\'e}rieures. 


\subsection{Conseil d'Administration (CA)}
Le r\^ole du Conseil d'administration est de d\'elib\'erer et de voter les d\'ecisions relevant de la politique de l'\'etablissement. Il doit ainsi se prononcer sur
le contrat d'\'etablissement, les accords et les conventions. Il lui revient de voter le budget et la r\'epartition des subventions et des emplois. 

Le nombre de ses membres est de l'ordre d'une trentaine. Les repr\'esentants des enseignant$\cdot$es-chercheur$\cdot$ses et enseignant$\cdot$es  sont pour moiti\'e des professeur$\cdot$es des universit\'es ou assimil\'es et pour moiti\'e des personnels d'autre statut (ma\^itres de conf\'erence, PRAG). Il comprend des personnalit\'es ext\'erieures, des repr\'esentant$\cdot$es des usagers (\'etudiant$\cdot$es et personnes b\'en\'eficiant de la formation continue inscrits dans l'\'etablissement),
ainsi que des repr\'esentant$\cdot$es des personnels IATOS (ing\'enieur$\cdot$es, administratifs, techniques et des biblioth\'eques), en exercice dans l'\'etablissement. 

\href{https://www.legifrance.gouv.fr/affichCodeArticle.do;jsessionid=C8BFF801F2976E9272297AB33338C553.tpdila14v_3?idArticle=LEGIARTI000027747951&cidTexte=LEGITEXT000006071191&dateTexte=20170113}{Article L712-3 modifi\'e par Loi n$\textdegree$2013-660 du 22 juillet 2013 - art. 47}

\subsection{Conseil acad{\'e}mique}
Lorsqu'elles se r\'eunissent ensemble, la Commission de la formation et de la vie universitaire 
(CFVU, ancien CEVU, Conseil des {\'e}tudes et de la vie universitaire) et la Commission de la recherche (ancien Conseil scientifique) constituent le Conseil acad\'emique.

Cet organisme est consult\'e par l'\'equipe de direction de l'universit\'e, pour se prononcer sur les orientations des politiques de formation, de recherche, ou sur tout autre sujet touchant la vie universitaire. Lorsqu'elles ont un impact budg\'etaire, les d{\'e}cisions du Conseil acad{\'e}mique doivent \^etre valid\'ees par le Conseil d'administration. 

\href{https://www.legifrance.gouv.fr/affichCodeArticle.do;jsessionid=C8BFF801F2976E9272297AB33338C553.tpdila14v_3?idArticle=LEGIARTI000027747976&cidTexte=LEGITEXT000006071191&dateTexte=20170113}{Article L712-4 modifi\'e par Loi n$\textdegree$2013-660 du 22 juillet 2013 - art. 49}


\subsubsection*{Commission de la formation et de la vie universitaire}
Tout ce qui touche aux formations d\'elivr\'ees par l'universit\'e est du ressort de la Commission de la Formation et de la Vie Universitaire (CFVU). La commission se prononce en particulier sur les programmes de formation des diff\'erentes composantes de l'universit\'e, mais aussi sur la r{\'e}partition des moyens attribu\'es \`a la formation au sein de l'enveloppe vot\'ee par le Conseil d'administration. Il lui revient aussi de soutenir et d\'evelopper les activit{\'e}s culturelles, sportives, sociales, associatives et veiller \`a la qualit\'e des conditions de vie et de travail des {\'e}tudiant$\cdot$es. Elle est garante des libert{\'e}s universitaires et des libert{\'e}s syndicales et politiques des {\'e}tudiant$\cdot$es. 

\href{https://www.legifrance.gouv.fr/affichCodeArticle.do;jsessionid=C8BFF801F2976E9272297AB33338C553.tpdila14v_3?idArticle=LEGIARTI000027747967&cidTexte=LEGITEXT000006071191&dateTexte=20170113}{Article L712-6 modifi{\'e} par Loi n$\textdegree$2013-660 du 22 juillet 2013 - art. 49}

\href{https://www.legifrance.gouv.fr/affichCodeArticle.do;jsessionid=C8BFF801F2976E9272297AB33338C553.tpdila14v_3?idArticle=LEGIARTI000027747991&cidTexte=LEGITEXT000006071191&dateTexte=20170113}{Article L712-6-1 modifi{\'e} par Loi n$\textdegree$2013-660 du 22 juillet 2013 - art. 50}

\subsubsection*{Commission de la recherche (CR)}
La r\'epartition des moyens destin{\'e}s \`a la recherche tels qu'allou{\'e}s par le Conseil d'administration est d\'ecid\'ee par la Commission de la Recherche (CR), qui d\'ecide \'egalement des r{\`e}gles de fonctionnement des laboratoires, et qui donne un avis consultatif sur les conventions avec les organismes de recherche. La CR a plus g\'en\'eralement la responsabilit\'e de favoriser le d\'eveloppement et la diffusion de la culture scientifique, technique et industrielle.

\href{https://www.legifrance.gouv.fr/affichCodeArticle.do;jsessionid=C8BFF801F2976E9272297AB33338C553.tpdila14v_3?idArticle=LEGIARTI000027747971&cidTexte=LEGITEXT000006071191&dateTexte=20170113}{Article L712-5   modifi{\'e} par Loi n$\textdegree$22013-660 du 22 juillet 2013 - art. 49}

\section{Agence comptable}
L'agent comptable, nomm\'e par deux minist\`eres \`a la fois (Education Nationale et Budget), a la responsabilit\'e de la comptabilit\'e de l'universit\'e dont il doit se porter garant. Il \'etablit le compte financier, et contr\^ole la gestion budg\'etaire (d{\'e}cret du 7 novembre 2012 relatif {\`a} la gestion budg{\'e}taire et comptable publique).


\section{Composantes}
Les composantes d'une universit\'e peuvent \^etre :
\begin{enumerate}
 \item des unit\'es de formation et de recherche (UFR), des d\'epartements, laboratoires et centres de recherche,
 \item des \'ecoles ou des instituts,
 \item des regroupements de composantes.
\end{enumerate}

Pour plus d'informations: 
\href{https://www.legifrance.gouv.fr/affichCode.do;jsessionid=6C65A5741617DEF72E5BAC5710E2EE64.tpdila20v_1?idSectionTA=LEGISCTA000006166682&cidTexte=LEGITEXT000006071191&dateTexte=20170114}{Article L713-1, modifi\'e par ORDONNANCE n$\textdegree$2014-807 du 17 juillet 2014 - art. 3} 

Ces composantes sont libres de fixer leur statut (qui doivent \^etre approuv\'es par le Conseil d'administration) et leur budget. 
Concernant les regroupements de composantes, les universit\'es fusionn\'ees sont organis\'ees en coll\`eges qui regroupent plusieurs UFR et instituts. 

A noter que les \'ecoles et les instituts, tels que les instituts universitaires de technologie (IUT), 
disposent de pr\'erogatives qui leur sont propres. Pour plus de d\'etails, un lecteur avis\'e pourra consulter :
\index{Institut universitaire de technologie (IUT)}
% Section 3 : Les instituts et les \'ecoles.
\href{https://www.legifrance.gouv.fr/affichCode.do;jsessionid=6C65A5741617DEF72E5BAC5710E2EE64.tpdila20v_1?idSectionTA=LEGISCTA000006182446&cidTexte=LEGITEXT000006071191&dateTexte=20170114}{Article L713-9 modifi\'e par Loi n$\textdegree$2005-380 du 23 avril 2005 - art. 44 JORF 24 avril 2005}

\subsection{Unit\'es de formation et de recherche (UFR)}
\index{Unit\'e de formation et de recherche (UFR)}
Les unit\'es de formation et de recherche (UFR) associent des d\'epartements de formation (par exemple, d\'epartement des trois ann\'ees de Licence) et des laboratoires ou centres de recherche. Elles correspondent \`a un projet \'educatif et \`a un programme de recherche mis en \oe uvre par des enseignant$\cdot$es-chercheur$\cdot$ses, des enseignant$\cdot$es et des chercheur$\cdot$ses relevant d'une ou de plusieurs disciplines fondamentales. Les UFR sont administr\'ees par un conseil \'elu et dirig\'ees par un$\cdot$e directeur$\cdot$trice \'elu$\cdot$e par ce conseil.

Pour de plus amples informations: 
\href{https://www.legifrance.gouv.fr/affichCode.do;jsessionid=6C65A5741617DEF72E5BAC5710E2EE64.tpdila20v_1?idSectionTA=LEGISCTA000006182444&cidTexte=LEGITEXT000006071191&dateTexte=20170114}{Article L713-3 modifi\'e par Loi n$\textdegree$2003-339 du 14 avril 2003 - art. 2 JORF 15 avril 2003}

\subsection{Laboratoire de recherche}
\index{Unit\'e de de recherche}
\index{Laboratoire de recherche}
\index{Unit\'e mixte de de recherche (UMR)}
Les structures permettant aux chercheur$\cdot$ses d'effectuer leur travail, en leur fournissant en particulier les moyens financiers, informatiques et administratifs) sont les unit\'es de recherche. Celles-ci peuvent \^etre des laboratoires relevant d'une ou de plusieurs universit\'es et d'organismes de recherche scientifique (CNRS, INRIA,\ldots), comportant des \'equipes ayant la responsabilit\'e de la vie scientifique (s\'eminaires, groupes de travail,\ldots). Ils comprennent donc des personnels de diff\'erents statuts et appartenances. Le statut de ces laboratoires d\'epend des organismes dont ils rel\`event (par exemple, unit\'e mixte de recherche ou UMR lorsqu'il y a un contrat d'association entre laboratoires, universit\'es et organismes de recherche). La direction du laboratoire est r\'egie par les statuts de ce laboratoire, avec la possibilit\'e d'un conseil de laboratoire qui est charg\'e de d\'efinir la strat\'egie de recherche.

La gestion des moyens financiers de la recherche, ne concerne g\'en\'eralement pas les salaires des personnels (qui sont \`a la charge des organismes employeurs). La question du financement de la recherche fait l'objet du chapitre \ref{financement-projets}. \\

\section{Regroupements}
\label{Regroupements}
\subsection{Communaut\'es, associations et fusions}
\index{communaut\'es d'universit\'es et \'etablissements (COMUE)}
Suite \`a la loi n$\circ$2013-660 du 22 juillet 2013 relative \`a  l'enseignement sup\'erieur et \`a la recherche,
tous les \'etablissements publics d'enseignement sup\'erieur sous tutelle du MESRI(Chapitre \ref{chapMinistere}) sont amen\'es \`a  se regrouper et \`a se coordonner \`a l'\'echelle de leur territoire, que ces regroupements aient statut de COMUE (pour communaut\'es d'universit\'es et \'etablissements), d'universit\'e fusionn\'ee ou d'association \`a  un EPSCP.
Ce dispositif succ\`ede aux P\^oles de recherche et d'enseignement sup\'erieur (PRES). 

Pour le lecteur ou la lectrice avide de plus amples informations, voici le lien minist\'eriel  \`a ce sujet, 
sous forme de FAQ et contenant \'egalement  une liste des \'etablissements membres des COMUE et des associations : \\
\lien{www.enseignementsup-recherche.gouv.fr/cid94756/index.html}

Et pour compl\'eter, des liens moins officiels : \\
\lien{www.campusfrance.org/fr/page/les-universites-et-les-comue} \\
\lien{www.sauvonsluniversite.com/spip.php?article6553}

Au moment de la mise \`a jour de ce livret, les projets de regroupements font partie de l'actualit\'e de la communaut\'e.

\subsection{Fondation de coop\'eration scientifique (FCS)}

Les Fondations de coop\'eration scientifique (FCS) succ\'edent aux R\'eseaux th\'ematiques de recherche avanc\'ee (RTRA), anciennes structures supprim\'ees par la loi relative \`a l'enseignement sup\'erieur et \`a la recherche de 2013.
\index{R\'eseau th\'ematique de recherche avanc\'ee (RTRA)}
\index{Fondation de coop\'eration scientifique (FCS)}

Leur noble objectif : rassembler, sur un th\`eme donn\'e, une masse critique de chercheur$\cdot$ses de tr\`es haut niveau, autour d'un noyau dur d'unit\'es de recherche g\'eographiquement proches, afin d'\^etre comp\'etitif avec les meilleurs centres de recherche au niveau mondial.

\href{https://www.legifrance.gouv.fr/affichCode.do?idSectionTA=LEGISCTA000027748290&cidTexte=LEGITEXT000006071190}{Article L344-11 modifi\'e par Loi n$\textdegree$2013-660 du 22 juillet 2013 - art. 66}

\subsubsection{La Fondation Sciences Math\'ematiques de Paris (FSMP)}
\lien{www.sciencesmaths-paris.fr/} \\

La Fondation Sciences Math\'ematiques de Paris (FSMP) a \'et\'e cr\'e\'ee en 2006, sur le feu statut de RTRA. 
Depuis 2011, la FSMP est \'egalement porteuse du LabEx Sciences Math\'ematiques de Paris. 

Les \'equipes participantes, rattach\'ees au CNRS, \`a INRIA, \`a l'ENS Paris,
au Coll\`ege de France et aux universit\'es Paris Descartes (Paris 5), Pierre et Marie Curie (Paris 6), Paris Panth\'eon-Sorbonne (P1), Paris Diderot (Paris 7), Paris-Dauphine (Paris 9) et Paris Nord (P13).

Les moyens mis en \oe uvre sont notamment des bourses, des chaires, des positions post-doctorales, des invitations de chercheur$\cdot$se$\cdot$s...

\subsubsection{La Fondation Math\'ematique Jacques Hadamard}
\lien{www.fondation-hadamard.fr} \\ \\
Cr\'e\'ee en 2011, la Fondation Math\'ematique Jacques Hadamard est h\'eberg\'ee par la FCS Campus Paris Saclay, charg\'ee de porter l'Op\'eration du m\^eme nom, cf. \ref{trucenex}.
Ses membres fondateurs sont l'Universit\'e Paris-Sud, l'Ecole Polytechnique, l'ENS Cachan, l'IHES et le CNRS. Elle porte le projet LabEx Math\'ematiques Hadamard. Ses objectifs
et moyens sont similaires \`a ceux de la Fondation Sciences Math\'ematiques de Paris.

\quad

\section*{Quelques liens :}
\begin{itemize}
\item \href{https://www.legifrance.gouv.fr/affichCode.do;jsessionid=C8BFF801F2976E9272297AB33338C553.tpdila14v_3?cidTexte=LEGITEXT000006071191&dateTexte=20170113}{Code de l'\'education} 
\item Le site de la Maison des universit\'es : \lien{www.amue.fr/}
\item Strat\'egie nationale dans l'enseignement sup\'erieur : \\
\lien{www.enseignementsup-recherche.gouv.fr/pid25092/strategie.html}
\end{itemize}
 
\include{chap_CNRS}
\include{chap_IINRIA}
\include{chap_INRAE}
\include{chap_HCERES}
\include{chap_CNU}
\include{chap_CN_CNRS}

%%%%%%%%%%%%%%%%%%%%%%%%%%%%%%%%%%%%%%%%%%%%%%%%%%%%
%%%%%%%%%%%%%%%%%%%%%%%%%%%%%%%%%%%%%%%%%%%%%%%%%%%%
\part{Le financement de la recherche}

\include{chap_fin_sources}
\chapter[Les financements r\'ecurrents]{Les financements r\'ecurrents des \'etablissements d'enseignement sup\'erieur}


\section{La loi LRU sur l'autonomie des universit\'es}
\label{sec. quad}
\index{loi LRU}

La  \href{www.legifrance.gouv.fr/affichTexte.do?cidTexte=JORFTEXT000000824315}{loi relative aux Libert\'es et Responsabilit\'es des Universit\'es}
(dite loi LRU ou loi P\'ecresse), r\'egit depuis le 10 ao\^ut 2007 les relations 
entre les ``grands \'etablissements'' et l'\'Etat, donc notamment les relations financi\`eres. Il y est dit que les \'etablissements 
concluent avec l'\'etat un `` contrat pluriannuel d'\'etablissement'' (en pratique, pluriannuel signifie pour cinq ans maintenant, 
c'\'etait quatre ans il y a quelques ann\'ees). Ces contrats pr\'ecisent des modalit\'es d'\'evaluation des personnels, et la mani\`ere dont 
l'\'etablissement contribue \`a un ``p\^ole de recherche et d'enseignement sup\'erieur''. Ils ne constituent en aucun cas un engagement financier 
pluriannuel de l'\'Etat, qui d\'etermine annuellement l'attribution des moyens par la loi de finances.

Il est dit dans la loi LRU que les \'etablissements rendent compte p\'eriodiquement de l'ex\'ecution de leurs engagements, 
qui est \'evalu\'ee par le Haut Conseil de l'\'evaluation de la recherche et de l'enseignement sup\'erieur (HCERES), cf chapitre \ref{HCERES}. 
Cette évaluation a des cons\'equences sur la vie des enseignants-chercheurs, d\'etaill\'ees ci-apr\`es. Il est alors dit dans la loi LRU que 
l'\'Etat tient compte des r\'esultats de cette \'evaluation pour d\'eterminer 
les engagements financiers qu'il prend envers les \'etablissements dans le cadre des contrats pluriannuels.
Ces engagements concernent en grande partie la masse salariale des personnels de l'universit\'e 
(qui peut \^etre de l'ordre de 70\% du budget total) et donc la possibilit\'e d'augmenter (ou l'obligation de diminuer) 
les effectifs des enseignant$\cdot$es et des enseignant$\cdot$es-chercheur$\cdot$ses.

Pour la tr\`es grande majorit\'e des laboratoires de math\'ematiques,
le minist\`ere est le principal support
financier (\emph{via} les universit\'es).
Le financement r\'ecurrent doit permettre l'achat de mat\'eriel
(informatique et fournitures de bureau essentiellement), ainsi que
le paiement de frais de mission pour les membres permanents et non
permanents reconnus du laboratoire.

\subsection{Le BQR}
% \label{BQR}
\index{Bonus qualit\'e recherche (BQR)}

Historiquement, les financements r\'ecurrents dans les \'etablissements d'enseignement
sup\'erieur sont soumis au BQR (bonus qualit\'e recherche)~: ces
\'etablissements pr\'el\`event une quote-part repr\'esentant 15\,\% de
toutes les sommes vers\'ees par l'\'Etat et les organismes de
recherche, pour mener \`a bien leur politique scientifique.
Ainsi le BQR est pr\'elev\'e sur les subventions
minist\'erielles affect\'ees aux laboratoires, et il est redistribu\'e par
l'interm\'ediaire d'appels d'offres discut\'es et vot\'es au sein de l'universit\'e. 
Ces appels d'offres peuvent
proposer, par exemple, des soutiens \`a l'acquisition d'\'equipements de recherche,
\`a l'organisation de colloques, soutien aux jeunes arrivant$\cdot$es (d\'echarge des jeunes EC).\\

Avec la loi LRU, il semble que le BQR ne soit plus systématique et ce au profit d'une organisation locale à l'établissement.
Ainsi il a parfois été remplacé par plusieurs dotation budgétaires, aux laboratoires, aux départements de formation, aux UFR, etc.
Il est semble alors difficile de donner une information générique sur ce sujet.

\section{Le financement par les organismes de recherche}

\subsection{Le CNRS}
\index{CNRS}


En math\'ematiques, pr\`es des deux tiers des laboratoires (les UMR)
sont associ\'es au CNRS. Le CNRS est aussi signataire des contrats pluriannuels avec les
\'etablissements d'enseignement su\-p\'e\-rieur, lorsqu'il est
tutelle d'au moins un laboratoire de cet \'etablissement. Cela
signifie, entre autres, qu'il s'engage \`a fournir, pendant la
dur\'ee du contrat, une dotation dont le montant est revu
annuellement par la direction du CNRS. Le contrat entre l'\'etablissement et le CNRS peut \'eventuellement
\^etre renforc\'e, si le CNRS d\'ecide d'adjoindre aux moyens
financiers et aux agents admistratifs d'autres \'el\'ements, comme des
d\'el\'egations (voir \ref{delegation}).

\subsection{Inria}
\index{Inria}

Inria peut financer des \'equipes de recherche de deux fa\c cons
diff\'erentes.

\begin{itemize}
\item Il peut s'associer \`a des \'etablissements d'enseignement
sup\'erieur et/ou d'autres organismes de recherche, auquel cas le
fonctionnement s'apparente au cas du CNRS.
\item Il peut financer des \'equipes propres, les \'equipes-projets, \`a dur\'ee
de vie plus limit\'ee (quatre ann\'ees, \'eventuel\-le\-ment
reconductibles), au sein de ses centres de recherche.
\end{itemize}

\include{chap_fin_non_rec_2019}

%%%%%%%%%%%%%%%%%%%%%%%%%%%%%%%%%%%%%%%%%%%%%%%%%%
%%%%%%%%%%%%%%%%%%%%%%%%%%%%%%%%%%%%%%%%%%%%%%%%%%
\part{La communaut\'e math\'ematique}

\include{chap_soc_sav}
%%%%%%%%%%%%%%%%%%%%%%%%%%%%%%%%%%%%%%%%%%%%%%%%%%
%%%%%%%%%%%%%%%%%%%%%%%%%%%%%%%%%%%%%%%%%%%%%%%%%%
\chapter{Les associations}


%%%%%%%%%%%%%%%%%%%%%%%%%%%%%%
\section{L'AND\`eS}
\index{Association nationale des docteurs \\(AND\`eS)}

\emph{Pr\'esident$\cdot$e actuel$\cdot$le : \verifier{Godefroy Lem\'enager}} \hfill Site web : \lien{www.andes.asso.fr/}
%E-mail : {\tt andes.contact@andes.asso.fr}\\
\smallskip

L'Association Nationale des Docteurs est une association r\'egie par la loi du 1er juillet 1901.
Fond\'ee en 1970, elle est reconnue d'utilit\'e publique depuis 1975.

L'AND\`es a trois missions principales :
\begin{itemize}
\item promouvoir le doctorat :\\
mettre en avant la valeur ajout\'ee que repr\'esente l'exp\'erience professionnelle du doctorat pour r\'ev\'eler les comp\'etences des docteur$\cdot$es ;
\item mettre les talents des docteur$\cdot$es au service de la soci\'et\'e :\\
    contribuer au d\'ecloisonnement des sph\`eres professionnelles en positionnant les docteur$\cdot$es comme \og{}passeurs de fronti\`eres\fg{},
    tirer parti de l'expertise et des savoirs-faire des docteur$\cdot$es pour relever les d\'efis du monde de demain ;
\item cr\'eer et mettre en synergie les r\'eseaux de docteur$\cdot$es :\\
    augmenter la visibilit\'e collective des docteur$\cdot$es,
    permettre \`a chacun$\cdot$e de d\'evelopper son r\'eseau professionnel,
    favoriser les interactions entre cr\'eateurs de r\'eseaux.
\end{itemize}


%%%%%%%%%%%%%%%%%%%%%%%%%%
\section{Animath}
\label{animath}
\index{Animath}
 
\emph{Pr\'esident$\cdot$e actuel$\cdot$le~: \verifier{Fabrice Rouillier}} \hfill Site web~: \lien{www.animath.fr/}
\smallskip

Animath est une association dont le r\^ole est de promouvoir l’activit\'e math\'ematique chez les jeunes, sous toutes ses
formes, dans les coll\'eges, lyc\'ees et universit\'es, tout en d\'eveloppant le plaisir de faire des math\'ematiques. 

En 1998, les soci\'et\'es savantes (SMF, SMAI), l’Association des professeurs de math\'ematiques de l’enseignement public, l’Inspection g\'en\'erale de math\'ematiques et les diff\'erents acteurs de l’animation math\'ematique (associations comme Maths en Jeans, la FFJM, le CIJM, Kangourou, acteurs institutionnels comme les IREM) ont d\'ecid\'e de cr\'eer l’association Animath, charg\'ee de ``favoriser l’introduction, le fonctionnement, le 
d\'eveloppement, la mise en r\'eseau et la valorisation d’activit\'es math\'ematiques dans les \'ecoles, coll`eges, lyc\'ees et \'etablissements de niveau universitaire”. Animath est soutenue par le CNRS et INRIA.

Le premier r\^ole d’Animath est donc d'\^etre la ``maison commune" des activit\'es math\'ematiques p\'eriscolaires, donc de coordination, d’incitation, d'information et de mise en r\'eseau et de d\'eveloppement des synergies 
entre les diff\'erents acteurs.  Animath a \'et\'e, entre 2012 et 2016, porteuse du consortium Cap’Maths, 
cr\'e\'e dans le cadre de l’appel \`a projet ``Culture scientifique et technique et \'egalit\'e des chances" dans la cadre du programme ``Ìnvestissements d'avenir" ; Cap'Maths a apport\'e 2,1 millions d'Euros \ \`a des actions de popularisation des math\'ematiques pendant cette p\'eriode, pour un financement total de 5,7 millions. Le reliquat du financement 
Cap'Maths, soit 900 000 Euros a \'et\'e vers\'e \`a la fondation Blaise Pascal lui donnant ainsi une base de d\'epart. 

Les principaux projets qu'Animath porte, seule ou avec des partenaires, sont : 
\begin{itemize} 
\item les {\it journ\'ees Filles et math\'ematiques, 
une \'equation lumineuse} et les {\it Rendez-vous des jeunes math\'ematiciennes} (avec {\it femmes et math\'ematiques}) 
\footnote{\url{https://filles-et-maths.fr/}}, 
\item les {\it stages MathC2+} (avec la FSMP)
\footnote{\url{https://www.mathc2plus.fr/}},
\item {\it Mathmosph\`ere}, un club et des stages virtuels de math\'ematiques \footnote{\url{https://www.animath.fr/actions/mathmosphere}}
\item un programme de coop\'eration internationale, visant au d\'eveloppement de clubs et de math\'ematiques dans les pays moins riches ou en voie de d\'eveloppement
\footnote{\url{https://www.animath.fr/actions/international/}}
\item les {\it Correspondances math\'ematiques}, qui permettent \`a des \'equipes de lyc\'ennes et lyc\'eens de travailler sur des probl\`emes ouverts de math\'ematiques et d'\'echanger leurs solutions par vid\'eo \footnote{\url{https://correspondances-maths.fr/}} 
\item le {\it concours Alkindi} de cryptographie (avec France IOI) pour \'le\`eves de 4\`eme, 3\`eme, 2nde 
\footnote{\url{http://www.concours-alkindi.fr}} (plus de 60000 participants en 2018-2019)
\item le {\it Tournoi français des jeunes math\'ematiciennes et math\'ematiciens} ($\text{TFJM}^2$) 
\footnote{\url{https://www.tfjm.org/}}, 
\item la {\it Pr\'eparation olympique fran\c caise de math\'ematiques}, organisant pr\'eparation et participation aux comp\'etitions math\'ematiques internationales (de type olympiades)
\footnote{\url{http://maths-olympiques.fr/}}, 
\item un encouragement aux clubs de math\'ematiques \footnote{\url{https://www.animath.fr/actions/clubs/}} et, en coop\'eration avec le Minist\`ere de l'\'education nationale, un soutien au recensement des clubs de math\'ematiques entrepris dans le cadre du plan Villani-Torossian
\footnote{\url{http://eduscol.education.fr/cid139417/clubs-de-mathematiques.html}}
\item la participation des lyc\'een$\cdot$nes aux conf\'erences {\it Un texte, un math\'ematicien} organis\'ees par la SMF et la Biblioth\`que nationale de France \footnote{\url{https://smf.emath.fr/la-smf/cycle-un-texte-un-mathematicien}}. 
\end{itemize}

Pour une contextualisation dans la promotion des math\'ematiques vers le public, voir le paragraphe \ref{jeunes}.\\




%%%%%%%%%%%%%%%%%%%%%%%%%%%%%%%%%%%%%%%%
\section{L'Association Femmes et Math\'ematiques}
\index{Femmes et math\'ematiques (association)}

\emph{Pr\'esident$\cdot$e actuel$\cdot$lee~: \verifier{Anne Boyé}} \hfill Site web~: \lien{www.femmes-et-maths.fr/}
\smallskip

Cr\'e\'ee en 1987 par des math\'ematiciennes, l'association \textit{femmes et math\'ematiques} compte actuellement environ cent cinquante membres (femmes et hommes), principalement des chercheuses et des enseignantes du sup\'erieur ou du secondaire. \\
Parmi ses objectifs:
\begin{itemize}
\item encourager les filles \`a s'orienter vers des \'etudes scientifiques et techniques,
\item promouvoir les femmes dans le milieu scientifique, en particulier math\'ematique,
\item agir pour plus de parit\'e en math\'ematiques,
\item \^etre un lieu de rencontre entre math\'ematiciennes,
\item coop\'erer avec les associations ayant un but analogue en France ou \`a l'\'etranger.
\end{itemize}

L'association \textit{femmes et math\'ematiques} \\
\textbf{R\'ealise}
\begin{itemize}
\item  des interventions dans des \'etablissements scolaires et universitaires sur le double th\`eme des math\'ematiques et de la place des femmes dans les professions scientifiques,
\item des journ\'es \og Filles et maths : une \'equation lumineuse  \fg destin\'ees \`a encourager les jeunes \`a se lancer dans des \'etudes de math\'ematiques et \`a lutter contre les st\'er\'eotypes sexistes en sciences,
\item des statistiques sexu\'ees sur la pr\'esence des femmes en math\'ematiques,
\item un livret \og Femmes et sciences... au-del\`a des id\'ees re\c cues \fg  avec les associations Femmes et Sciences et Femmes Ing\'enieurs,
\item une brochure \og Zoom sur les m\'etiers des math\'ematiques et de l'informatique \fg, avec les soci\'et\'es savantes de math\'ematiques et d'informatique,
\item Participe \`a des forums de m\'etiers et des salons de l'\'education ou des math\'ematiques dans plusieurs villes en France.
 \end{itemize}

\textbf{Participe \`a}
\begin{itemize}
\item des groupes de travail (Minist\`ere de l'Education Nationale et de la Recherche, Rectorats, Service des droits des femmes et de l'\'egalit\'e),
\item l'\'elaboration de rapports officiels,
\item des colloques math\'ematiques et sur l'\'egalit\'e des sexes, en France et \`a l'\'etranger,
 des manifestations diverses, F\^ete de la Science, Journ\'ee internationale des droits des Femmes le 8 mars, Mondial des m\'etiers, Colloques d'associations amies,
\item des auditions par la commission des affaires culturelles, familiales et sociales de l'Assembl\'ee Nationale, par le Haut conseil de la science et de la technologie, etc.
\item  des op\'erations de \og\ marrainage\fg\  qui se d\'eclinent principalement sous deux formes:
\begin{itemize}
\item des jeunes lyc\'eennes contactent l'association pour des TPE,
\item de jeunes \'etudiantes posent des questions \``a l'association \`a propos de leur orientation.
\end{itemize}
\end{itemize}
\textbf{Organise r\'eguli\`erement}
\begin{itemize}
\item des colloques \`a l'Institut Henri Poincar\'e \`a Paris,
\item des journ\'ees r\'egionales dans des universit\'es diff\'erentes : expos\'es de math\'ematiques et table ronde li\'ee \`a l'\'egalit\'e des chances,
\item un forum des jeunes math\'ematicien$\cdot$nes tous les ans \`a l'automne (en 2018 il a eu lieu à Orléans, et en 2019, il aura lieu à l'IHP à Paris),
\end{itemize}
\textbf{Publie}
\begin{itemize}
\item une newsletter trimestrielle,
\item des articles dans des revues,
\item des statistiques sexu\'ees.
\end{itemize}
\textbf{Anime}
\begin{itemize}
\item
une liste de diffusion : femmes-et-maths@listes.math.cnrs.fr
\item
un compte twitter : @femmesetmaths
\end{itemize}

En 2001, l'association est l'une des laur\'eates du premier \textbf{Prix Ir\`ene Joliot-Curie}. \\
En 2006 une \textbf{mention sp\'eciale du Prix Ir\`ene Joliot-Curie} du Minist\`ere d\'el\'egu\'e \`a l'Enseignement sup\'erieur et \`a la Recherche a r\'ecompens\'e l'une de nos membres pour son initiative remarquable dans le domaine du mentorat. \\






%%%%%%%%%%%%%%%%%%%%%%%%%%%%%
\section{CIMPA}\index{Centre international de math\'ematiques pures et appliqu\'ees (CIMPA)}

\emph{Pr\'esident$\cdot$e actuel$\cdot$le : \verifier{TSOU Sheung Tsun}}\hfill Site web~: \lien{www.cimpa-icpam.org}

\emph{Vice pr\'esident$\cdot$e : \verifier{Alain Damlamian}, Secr\'etaire : \verifier{Jean-Marc Bardet},
Tr\'esorier : \verifier{Marc Aubry}}

\emph{Directeur$\cdot$trice : \verifier{Claude Cibils}}
\smallskip



Le CIMPA est un organisme international
\oe uvrant pour l'essor des
math\'ematiques dans les pays en
voie de d\'eveloppement.
Fond\'e en 1978, le CIMPA est bas\'e
\`a Nice. Il a pour vocation de promouvoir
la coop\'eration internationale
dans le domaine de l'enseignement
sup\'erieur et de la recherche en math\'ematiques
pures et appliqu\'ees et
leurs interactions, ainsi que dans les
disciplines connexes.

Cr\'e\'e en France et reconnu par
l'UNESCO, le CIMPA b\'en\'eficie du
soutien du MESR (France), de l'UNS
(France), du MICINN (Espagne) et
du CNRS (France). Disposant du
statut d'association (loi fran\c caise de
1901), il s'appuie sur de nombreux
math\'ematicien$\cdot$nes et membres institutionnels
du monde entier.

En 2007, le Conseil d'administration
du CIMPA a exprim\'e la volont\'e
de le faire \'evoluer en un centre
europ\'een afin que d'autres pays
puissent lui apporter un soutien
financier et participer \`a ses activit\'es
scientifiques. Aujourd'hui en
marche, cette \'evolution permettra
de mieux r\'epondre aux nombreuses
demandes des pays en voie de d\'eveloppement
que les moyens actuels
ne permettent pas de satisfaire.







%%%%%%%%%%%%%%%%%%%%%%%%%%%%%%%%%
\section{La Conf\'ed\'eration des jeunes chercheurs}
\emph{Pr\'esident$\cdot$e actuel$\cdot$le : \verifier{K\'evin Bonnot}} \hfill Site web : \lien{cjc.jeunes-chercheurs.org}

\smallskip

\index{Conf\'ed\'eration des jeunes chercheurs (CJC)}

La Conf\'ed\'eration des Jeunes Chercheurs (CJC) regroupe des associations
de doctorant$\cdot$es et de nouveaux$\cdot$elles docteur$\cdot$es de toute la France et de toutes
les disciplines. Elle a pour but de repr\'esenter et d\'efendre les int\'er\^ets
des jeunes chercheuses et chercheurs et de promouvoir le doctorat comme une exp\'erience
professionelle de la recherche et de l'innovation. Elle se positionne
comme force de proposition sur les questions de la recherche, de
l'enseignement sup\'erieur et de la formation doctorale.


%%%%%%%%%%%%%%%%%%%%%%%%%%%%%%%%%%%%%%%%
\section{La Fondation Blaise Pascale}
\index{Blaise Pascale (association)}

\emph{Pr\'esident$\cdot$e actuel$\cdot$le~: \verifier{Serge Abiteboul}} \hfill Site web~: \lien{www.fondation-blaise-pascal.org}
\smallskip

La fondation Blaise Pascal est une fondation nationale qui a pour vocation de promouvoir, soutenir, d\'evelopper et p\'erenniser les actions de m\'ediation scientifique en math\'ematiques et informatique \`a destination de tout citoyen, et plus sp\'ecifiquement aupr\`es des jeunes et des femmes. Elle a \'et\'e cr\'e\'ee sous \'egide de la Fondation pour l’Universit\'e de Lyon en novembre 2016. Ses fondateurs sont le CNRS et l’Universit\'e de Lyon.
 
La fondation Blaise Pascal poursuit cinq objectifs majeurs :
am\'eliorer la perception g\'en\'erale des sciences formelles par le grand public et notamment par les jeunes scolaris\'es, en am\'eliorant la compr\'ehension de leur impact, de leur utilit\'e et de leur vitalit\'e ; lutter contre les pr\'ejug\'es et les st\'er\'eotypes sociaux et de genre qui empêchent certains jeunes de se lancer dans des \'etudes en informatique et en math\'ematiques ; augmenter globalement le flux d'\'etudiant$\cdot$es effectuant des \'etudes longues dans un domaine scientifique ;
d\'emultiplier les moyens par le partage des ressources et la structuration des offres des acteurs de m\'ediation ; att\'enuer les disparit\'es sociales et g\'eographiques, grâce \`a une meilleure r\'epartition des projets \`a l'\'echelle nationale.


Depuis 2017 : 6 appels \`a projets r\'ealis\'es 745000 euros allou\'es, 142 projets financ\'es sur l’ensemble du territoire national, 83 structures soutenues dans 13 r\'egions, plus d’un million de personnes ont b\'en\'efici\'e des actions que soutien la fondation :
\begin{itemize}
\item Environ 1 535 000 \'el\`eves de la primaire \`a la terminale ;
\item Environ 60 000 adultes;
\item Plus de 700 enseignant$\cdot$es et enseignant$\cdot$es-chercheur$\cdot$ses.
\end{itemize}


%%%%%%%%%%%%%%%%%%%%%%%%%%%%%%%%%%%%%%%%%%%%%
%\section{Matexo}
%
%
%Matexo est le portail p\'edagogique du domaine emath.fr, destin\'e aux \'etudiants et
%enseignants du sup\'erieur en math\'ematiques.
%Il est compos\'e de plusieurs sites~:
%\begin{itemize}
%\item la base de documents r\'eserv\'es aux enseignants~: notes de cours, feuilles
%d'exercices avec ou sans correction...
%\item ExeMaAlt qui est un serveur d'exercices alternatifs,
%\item Exo7 qui est un site d'exercices en libre service \`a destination des \'etudiants
%et des enseignants (avec notamment la possibilit\'e de cr\'eer des feuilles de TD \`a la
%vol\'ee).
%\end{itemize}
%
%
%Site web~: \lien{matexo.emath.fr}


%%%%%%%%%%%%%%%%%%%%%%%%%%%%%%%%%%%%%%%%%%%%
\section{MATh.en.JEANS}

\emph{Pr\'esident$\cdot$e actuel$\cdot$le : \verifier{Aviva Szpirglas}}\hfill Site web : {\tt http://www.mathenjeans.fr}
%Contact: {\tt mathenjeans@free.fr}
\smallskip

Depuis 1989, MATh.en.JEANS, ({\em M\'ethode d'Apprentissage des Th\'eories math\'ematiques en Jumelant des \'Etablissements pour une Approche Nouvelle du Savoir}), fait vivre les math\'ematiques aux jeunes suivant les principes de la recherche, au sein d'ateliers dans les \'etablissements scolaires et au contact de chercheur$\cdot$ses professionnel$\cdot$les. Elle permet \`a des jeunes  de tous niveaux et de toutes origines de pratiquer une authentique d\'emarche scientifique, avec ses dimensions aussi bien th\'eoriques qu'appliqu\'ees et si possible en prise avec des th\`emes de recherche actuels.

\subsection*{Le principe} 
Chaque semaine, d\`es le mois de septembre, des \'el\`eves volontaires et des enseignant$\cdot$es de deux \'etablissements scolaires jumel\'es pour l'occasion travaillent en parall\`ele en petits groupes, pendant une ou deux heures hebdomadaires, sur des sujets de recherche math\'ematique \`a la fois attractifs et s\'erieux propos\'es par un$\cdot$e chercheur$\cdot$se professionnel$\cdot$le, souvent proches de ses propres probl\'ematiques.

Trois ou quatre fois dans l'ann\'ee, les \'el\`eves, les enseignant$\cdot$es et le$\cdot$la chercheur$\cdot$se impliqu\'e$\cdot$es dans les deux ateliers se rencontrent \`a l'occasion de \og{}s\'eminaires\fg{} o\`u ils$\cdot$elles \'echangent leurs points de vue, d\'ebattent et partagent leurs id\'ees, critiquent et font avancer leur travail.

Les enseignant$\cdot$es veillent au bon d\'eroulement mat\'eriel des ateliers. Elles ou ils incitent aux \'echanges et aident les \'el\`eves \`a pr\'eciser leurs pens\'ees, \`a les reformuler, en leur laissant le temps n\'ecessaire. Ils ou elles accompagnent la pr\'eparation de la pr\'esentation orale puis d'un \'ecrit. Mais ils$\cdot$elles ne r\'esolvent pas le probl\`eme \`a la place des \'el\`eves, ils$\cdot$elles ne le traduisent pas, ils$\cdot$elles ne le r\'eduisent pas \`a des petites questions.

Le$\cdot$La chercheur$\cdot$se a pour r\^ole de r\'ediger les sujets propos\'es \`a l'atelier, en tenant compte du niveau des \'el\`eves. Il$\cdot$Elle accompagne la recherche des \'el\`eves en suivant leur progression \`a l'occasion des s\'eminaires. Au besoin, il$\cdot$elle compl\`ete ou r\'eactualise les questions pos\'ees.

Chaque ann\'ee, entre mars et avril, les \'el\`eves pr\'esentent leurs r\'esultats et les soumettent \`a la critique dans les congr\`es qui regroupent l'ensemble des ateliers MATh.en.JEANS existants. Moment fort de l'ann\'ee, le congr\`es annuel, r\'eunit ses acteur$\cdot$trices, jeunes, professeur$\cdot$es et chercheur$\cdot$ses, dans un lieu choisi pour son dynamisme scientifique.

Le congr\`es pass\'e,  les \'el\`eves sont incit\'es \`a r\'ediger un article, qui sera publi\'e par l'association apr\`es validation.

\subsection*{L'association}
L'association a \'et\'e cr\'e\'ee en 1990, par Pierre Audin et Pierre Duchet - respectivement enseignant et chercheur en math\'ematiques - suite \`a l'op\'eration ``1000 classes - 1000 chercheurs" men\'ee en 1985-1986, et \`a un projet pilote sur l'ann\'ee scolaire 1989-1990.

Elle a pour principales missions d'impulser la mise en place des ateliers dans les \'etablissements scolaires, de mettre en contact les enseignant$\cdot$es et les chercheur$\cdot$ses, de les coordonner, d'organiser les congr\`es annuels o\`u les \'el\`eves pr\'esentent leurs travaux, de valider et de publier leurs productions \'ecrites. Elle met l'accent sur les \'echanges entre pair$\cdot$es et le contact avec la recherche vivante.

Elle a obtenu en 1990 le prix de la d\'emarche scientifique au Salon PERIF (r\'eunissant des projets scientifiques en Ile de France), et en 1992, le prix d'Alembert de la Soci\'et\'e Math\'ematique de France. Elle est agr\'e\'ee par le Minist\`ere de l'Education Nationale et soutenue par le CNRS et plusieurs autres partenaires institutionnels ou associatifs. Elle est partie prenante du Consortium Cap'Maths.

Depuis quelques ann\'ees MATh.en.JEANS est en forte expansion ; actuellement environ 300 ateliers fonctionnent en France et dans le monde (notamment le r\'eseau des \'etablissements fran\c{c}ais \`a l'\'etranger), regroupant environ 5000 \'el\`eves, 600 professeur$\cdot$es et 200 chercheur$\cdot$ses. Pour 2019, 12 congr\`es sont organis\'es dont 9 en France et 3 \`a l'\'etranger. MATH.en.JEANS f\^ete ses 30 ans cette m\^eme ann\'ee.


%%%%%%%%%%%%%%%%%%%%%%%%%%%%%%%
\section{L'Op\'eration Postes}

L'Op\'eration Postes (OP) n'est pas une association, mais elle a sa
place dans cette liste pour tous les services rendus \`a la
communaut\'e math\'ematique. Vous la connaissez d\'ej\`a tous,
mais voici tout de m\^eme quelques rappels~: \index{Op\'eration
postes (OP)}
\begin{itemize}
\item l'OP existe depuis 1998 et est constitu\'ee
d'(enseignant$\cdot$es)-chercheur$\cdot$ses b\'en\'evoles~;

\item elle b\'en\'eficie du soutien de la SMAI (qui a permis son
lancement et assure l'h\'ebergement de son serveur), de la SMF et de
la SFdS, ainsi que de SIF et de l'AFIF, soci\'et\'es savantes en
informatique~;

\item elle est soutenue financi\`erement par le minist\`ere (pour
le remboursement des missions)~;

%\item elle est partenaire de la guilde des doctorants~;

\item elle a pour but de diffuser le maximum d'informations sur
les concours de recrutement d'enseignant$\cdot$es-chercheur$\cdot$ses et de
chercheur$\cdot$ses en math\'ematiques (sections CNU 25 et 26) et
informatique (section 27).
\end{itemize}
 \index{Soci\'et\'e des personnels enseignants et
chercheurs en informatique de France (SPECIF)}
\index{Association fran\c caise d'informatique \\fondamentale (AFIF)}

Vous pouvez y contribuer en
\begin{itemize}
\item transmettant des
informations (profils de postes, dates des Comit\'es de S\'election, des auditions, r\'esultats des
concours, AMI (Academic Mobility Index) de votre laboratoire, {\em
etc.})~;
\item informant les candidat$\cdot$es en faisant la publicit\'e de
l'OP, de MARS (machine d'aide au recrutement dans le sup\'erieur),
{\em etc.}~;
\item informant vos coll\`egues {\bf y compris d'autres
disciplines} de l'existence de MOUVE (Machine Ouverte aux
Universitaires qui Veulent
\'Echanger).
\index{Op\'eration postes (OP)!Machine Ouverte aux Universitaires
\\qui Veulent \'Echanger (MOUVE)} \index{Op\'eration postes
(OP)!Machine d'aide au recrutement dans le su\-p\'e\-rieur (MARS)}
\index{Op\'eration postes (OP)!Academic Mobility Index (AMI)}
\item contribuant au wiki de conseils aux candidat$\cdot$es \`a l'adresse :\\
\lien{postes.smai.emath.fr/2021/OUTILS/conseils/index.php}.
\end{itemize}

N'h\'esitez pas \`a (re)d\'ecouvrir
et \`a faire conna\^\i  tre son site web~:
\lien{postes.smai.emath.fr/}
%%%%%%%%%%%%%%%%%%%%%%%%%%%%%%%%%%%%%%%%%%%%%%
%%%%%%%%%%%%%%%%%%%%%%%%%%%%%%%%%%%%%%%%%%%%%%
\chapter{La communication}

%%%%%%%%%%%%%%%%%%%%%%%%%%%%%%%%%%%%%%%%%%%%%%
\section{Vulgarisation}

Vous vous demandez peut-\^etre pourquoi faire de la vulgarisation
\footnote{Nous utilisons ce terme de vulgarisation pour d\'ecrire l'activit\'e d'explication de travaux
 savants \`a des publics non sp\'ecialis\'es. On comprend bien que la notion de public non-sp\'ecialis\'e est relative :
 parler de ses travaux \`a une commission d'audition 25-26 n'est pas la m\^eme chose que dans un colloque sp\'ecialis\'e ;
 parler de son domaine de recherche \`a un public scientifiquement averti (ing\'enieur$\cdot$es, lecteur$\cdot$trice de la
{\it Recherche} ou de {\it Pour la science}...), \`a des enseignant$\cdot$es de math\'ematiques, \`a des
coll\'egien$\cdot$nes ou lyc\'een$\cdot$nes appellent des p\'edagogies diff\'erentes...  D'autres termes sont utilis\'es
au lieu de vulgarisation, par exemple : diss\'emination scientifique, communication scientifique.} scientifique,
alors que vos recherches, l'enseignement (et sans doute bient\^ot l'administration ?) prennent d\'ej\`a tout votre temps.
Les math\'ematicien$\cdot$nes ne sont pas habitu\'e$\cdot$es \`a expliquer leurs travaux au public.
Pourtant, que vous soyez enseignant$\cdot$e-chercheur$\cdot$se ou chercheur$\cdot$se, la diffusion de la culture et l'information
scientifique et technique fait partie de vos missions, et ce texte cherche \`a vous expliquer pourquoi c'est
 important\footnote{Il faut \^etre tout \`a fait clair : de telles
activit\'es ne sont pas correctement prises en compte dans les carri\`eres, ni au niveau national par le CNU
ou le comit\'e national du CNRS, ni au niveau local par les universit\'es.}.
De plus, vous vous apercevrez en tentant l'exp\'erience que communiquer son savoir et sa passion, et
par l\`a changer l'image que la soci\'et\'e a des math\'ematiques et des math\'ematicien$\cdot$nes, est aussi une activit\'e gratifiante.

Voici quelques fa\c cons de r\'eduire le manque de communication entre les math\'ematicien$\cdot$nes et le public.
Choisissez l'activit\'e qui vous convient selon vos pr\'ef\'erences, vos aptitudes et surtout votre disponibilit\'e.
Profitez des initiatives existantes !

\medskip\par
{\bf F\^ete de la Science} De plus en plus de laboratoires de math\'ematiques y participent en proposant
des conf\'erences, des ateliers, ou en animant un stand
\footnote{Voir \url{http://smf.emath.fr/content/fete-de-la-science-2016} pour quelques actions propos\'ees en 2016.}.
La plupart proposent des manipulations ludiques et des \'enigmes qui ne demandent aucune connaissance particuli\`ere.
Vous seriez surpris$\cdot$e de l'entrain suscit\'e par ce type d'activit\'es et du succ\`es qu'elles remportent aupr\`es du public.
Parlez-en \`a vos coll\`egues des autres universit\'es pour trouver des id\'ees de manipulations simples \`a mettre en place.

Tous les ans a lieu \`a Paris (en mai) le salon des jeux et de la culture math\'ematique, organis\'e par le CIJM\footnote{Comit\'e International des Jeux Math\'ematiques \url{http://www.cijm.org/}}, auquel vous pouvez aussi participer.

\medskip\par
{\bf Images des math\'ematiques\footnote{\url{http://images.math.cnrs.fr/}}}.
Ce site pr\'esente la recherche contemporaine et le m\'etier de math\'ematicien$\cdot$ne \`a l'ext\'erieur de la communaut\'e
scientifique, afin de rapprocher les chercheur$\cdot$ses en math\'ematiques et le public.
Tous les articles sont \'ecrits par des chercheur$\cdot$ses. Vos contributions seront donc les bienvenues.
Il est possible d'\'ecrire des articles \`a diff\'erents niveaux, mais l'id\'ee est toujours de parler de maths \`a des gens
qui n'en connaissent pas ou presque pas et d'essayer de montrer ce que fait un$\cdot$e math\'ematicien$\cdot$ne aujourd'hui.

\medskip\par
\textbf{Interstices\footnote{\url{http://interstices.info}}}
Ce site de culture scientifique a lui aussi \'et\'e cr\'e\'e par des chercheur$\cdot$ses, pour rendre accessibles \`a un large public
 les sciences et technologies de l'information et de la communication.

\medskip\par
\textbf{Culture math}\footnote{\url{http://www.math.ens.fr/culturemath/}}. Ce site s'adresse aux professeur$\cdot$es
de math\'ematiques du secondaire ; financ\'e par la direction g\'en\'erale de l'enseignement scolaire du minist\`ere de
 l'\'education nationale, il propose des documents permettant aux professeur$\cdot$es d'enrichir les contenus de leurs cours.
Culture Math accepte volontiers des textes de chercheur$\cdot$ses.

\medskip\par
\textbf{MADD Maths}\footnote{\url{http://maddmaths.smai.math.cnrs.fr/}}  MADD Maths est l'acronyme de Math\'ematiques Appliqu\'ees Divulgu\'ees et Didactiques, est une initiative de la SMAI en direction du grand public et notamment des lyc\'een$\cdot$nes.
L'objectif de MADD Maths est de montrer que les math\'ematiques constituent un domaine tr\`es dynamique o\`u il y a encore beaucoup de choses \`a d\'ecouvrir, qui est tr\`es utile, avec des applications parfois inattendues ou amusantes, et de donner envie de vouloir en savoir plus. Puisque les maths peuvent sembler quelquefois compliqu\'ees, le but du site web est de les rendre accessibles.
C'est aussi l'occasion de d\'ecouvrir de nouvelles facettes des maths que l'on n'a pas l'occasion d'apercevoir au lyc\'ee ou au coll\`ege.
Au menu, interviews de math\'ematicien$\cdot$nes passionnant$\cdot$es, math\'ematicien$\cdot$nes inattendu$\cdot$es, rubrique "culture maths", courts articles de vulgarisation de la recherche en maths.

\medskip
{\bf Presse}. Plusieurs magazines ou revues de vulgarisation scientifique publient des articles r\'edig\'es par des chercheur$\cdot$ses.
Il peut s'agir de publications grand public ou de revues destin\'ees \`a un lectorat plus restreint, mais non sp\'ecialis\'e.
Citons par exemple les magazines
\textit{La Recherche}\footnote{\url{http://www.larecherche.fr/}} et
\textit{Pour la Science}\footnote{\url{http://www.pourlascience.com/}}, ou encore
\textit{Science et vie}\footnote{\url{http://www.science-et-vie.com/}},
\textit{Science et vie junior}\footnote{\url{http://www.labosvj.fr/}},
\textit{Science et avenir}\footnote{\url{http://www.sciencesetavenir.fr/}},
\textit{Tangente}\footnote{\url{http://tangente.poleditions.com/}} (en kiosque)
et \textit{Quadrature}\footnote{\url{http://www.quadrature-journal.org/}}
(niveau TS ou Licence 1, sur abonnement seulement).

Tout comme pour les sites pr\'ec\'edents, il est conseill\'e de contacter les responsables du
 magazine pour proposer un sujet avant de se lancer dans l'\'ecriture d'un texte.

%\medskip
%{\bf Expos\'es}. Les Promenades math\'ematiques
%\footnote{\url{http://smf.emath.fr/MathGrandPublic/PromenadesMathematiques/}}
%sont une initiative destin\'ee \`a favoriser la diffusion de la culture math\'ematique aupr\`es
% de tous les publics en organisant des ateliers ou des conf\'erences de vulgarisation dans des cadres divers.
%Organis\'ees conjointement par la Soci\'et\'e Math\'emati\-que de France (cf \ref{smf}) et l'association Animath (cf \ref{animath}),
% elles b\'en\'eficient du soutien du CNRS et d'INRIA et s'appuient sur les laboratoires de
% math\'ematiques CNRS, INRIA et universitaires.

\medskip
{\bf Audimath}. Audimath \footnote{\url{http://audimath.math.cnrs.fr/}} est un r\'eseau cr\'e\'e par l’Institut National Sciences Math\'ematiques et de leurs Interactions (INSMI) du CNRS et destin\'e \`a apporter un soutien \`a tous les acteurs de la communaut\'e universitaire investis dans le d\'eveloppement des activit\'es de diffusion des math\'ematiques aupr\`es des publics extra-universitaires.



%%%%%%%%%%%%%%%%%%%%%%%%%%%%%%%%%%%%%%%%%%%%%
\section{Action vers les jeunes} \label{jeunes}

La communaut\'e de la recherche en math\'ematiques peut s'impliquer dans des actions en direction des jeunes et de
nos coll\`egues de l'enseignement secondaire, voire primaire.
On peut distinguer plusieurs types d'activit\'es dites ``p\'eriscolaires" qui permettent de toucher les jeunes :
\begin{itemize}
\item actions de culture math\'ematique : expositions, sites de culture math\'ema\-ti\-que, conf\'erences,
 rencontres avec des chercheur$\cdot$ses (voir ci-dessus) ;
\item comp\'etitions et concours en temps limit\'e (rallyes math\'ematiques, Championnat international
des jeux math\'ematiques et logiques, Kangourou...) ;
\item projets scientifiques permettant une initiation \`a la recherche, parfois sous forme de concours
ou comp\'etition (ateliers Maths en jeans\footnote{\url{http://www.mathenjeans.fr/}}, ateliers
 hippocampe maths \footnote{\lien{www.irem.univ-mrs.fr/-Hippocampe-.html}}, concours
Faites de la science\footnote{\url{http://www.faitesdelascience.fr/}}, concours C.G\'enial\footnote{\url{http://www.sciencesalecole.org/les-concours/concours-c-genial.html}}) ;
\item ateliers et clubs de math\'ematiques dans les coll\`eges et lyc\'ees ;
\item accompagnement de jeunes fortement motiv\'e$\cdot$es et au talent pr\'ecoce par un tutorat, des stages,
des clubs de math\'ematiques comme il en existe maintenant dans plusieurs universit\'es \footnote{Club de mathe\'ematiques discr\`etes (Lyon) \url{http://math.univ-lyon1.fr/~lass/club.html}, club math\'ematique de Nancy \url{http://depmath-nancy.univ-lorraine.fr/club/}, club Parismaths \url{http://www.parimaths.fr/}, cercle mathe\'ematique de Strasbourg \url{http://www-math.u-strasbg.fr/CercleMath/}, cercle Sofia Kovalevskaia de Toulouse \url{http://www2.animath.fr/spip.php?article2706}} ;
\item organisation de dispositifs sp\'ecifiques en direction de jeunes des zones d\'efavoris\'ees :
tutorat, stages pendant les vacances centr\'es sur les math\'ema\-ti\-ques\footnote{Par exemple, le Centre Galois (Orl\'eans Tours) \url{http://www.centre-galois.fr} ou Science ouverte (Paris 13) \url{http://scienceouverte.fr/-Stages-vacances-}, le stage Mat'les vacances \url{http://paestel.fr/} ; voir aussi les stages MathC2+ dans le paragraphe Animath} ;
\item organisation de tutorat, de mentorat, et de journ\'ees sp\'ecifiques... destin\'es aux filles  \footnote{\url{http://www.animath.fr/spip.php?rubrique160}} ;
\item Le Salon de la culture et des jeux math\'ematiques, qui a lieu tous les ans \`a la fin mai \`a Paris \url{https://www.cijm.org/salon} ;
\item Les diff\'erents \'ev\'enements qui sont organis\'es chaque dans le cadre de la Semaine des math\'ematiques, au mois de mars, ou de la F\^ete de la science au mois de septembre.
\end{itemize}

Voir aussi ci-dessus la pr\'esentation d'Animath (paragraphe \ref{animath}) et des autres acteurs. 

Le rapport Villani-Torossian \url{https://ww.education.gouv.fr/cid126423/21-mesures-pour-l-enseignement-des-mathematiques.html} pr\'econise, dans sa mesure 7 P\'eriscolaire et Clubs,	d'encourager les partenariats institutionnels avec le p\'eriscolaire et favoriser le d\'eveloppement de ce secteur, de recenser et de p\'erenniser les clubs en lien avec les math\'ematiques (de mod\'elisation, d’informatique, de jeux intelligents, etc.) ainsi que de r\'emun\'erer les intervenant$\cdot$es et adapter les emplois du temps des enseignant$\cdot$es. Le rappot apporte des pr\'ecisions dans les Recommandations 41 \`a 45.



%%%%%%%%%%%%%%%%%%%%%%%%%%%%%%%%%%%%%%%%
\section{Valorisation de la recherche}

La valorisation de la recherche comporte plusieurs aspects. Contrairement \`a la vulgarisation scientifique,
qui ressort d'une d\'emarche culturelle, la valorisation rel\`eve d'une d\'emarche plus utilitaire. Curieusement,
ce sont les m\^emes services des universit\'es et des organismes qui s'occupent de vulgarisation et de valorisation.

\subsection{Mise en valeur de travaux dans la communaut\'e math\'ematique}
Vous avez sans doute cr\'e\'e votre page web personnelle en vue de candidater.
Pensez maintenant \`a la mettre \`a jour r\'eguli\`erement : elle est la premi\`ere vitrine de vos recherches.
Pensez aussi \`a mettre vos articles sur \textbf{HAL}: \lien{hal.archives-ouvertes.fr/}.

\subsection{Mise en valeur de travaux en dehors de la communaut\'e math\'ematique}
Cette d\'emarche est compl\'ementaire de la vulgarisation de la recherche dont on a parl\'e plus haut.
Il est par exemple utile de promouvoir la recherche
en math\'ematiques pour que les tutelles prennent conscience de la valeur de leurs \'equipes de recherche
et puissent \`a leur tour utiliser cette information dans leur politique de communication ; il faut bien
comprendre que les r\'esultats de la recherche en math\'ematiques sont moins visibles et surtout moins
compr\'ehensibles que ceux de la quasi-totalit\'e des domaines de recherche ; en plus, et contrairement
 aux autres domaines, il est moins facile de justifier la recherche par des applications mirifiques \'eventuelles
comme gu\'erir le cancer, trouver des sources illimit\'ees d'\'energie etc.

La m\'ediatisation peut utiliser plusieurs angles : reconnaissance scientifique par la publication dans une
revue de premier plan, invitation dans un tr\`es grand congr\`es, r\'esultat facilement explicable ou donnant
 lieu \`a de belles images, preuve d'une conjecture un peu ancienne dont on peut mettre l'histoire en relief,
obtention d'une distinction particuli\`ere (IUF, prix...), collaboration internationale inhabituelle, contrats et brevets...

\medskip\par
Si vous pensez que vos \textbf{r\'esultats} peuvent \^etre \textbf{m\'ediatisables}, le ou la correspondant$\cdot$e
 communication de votre laboratoire\footnote{Coordonn\'ees des correspondant$\cdot$es communication des
 laboratoires de math\'ematiques \url{http://www.cnrs.fr/insmi/spip.php?article256}} vous aidera \`a prendre
 contact avec les services de communication\footnote{\`a l'INSMI \texttt{insmi-equipecom@cnrs-dir.fr}\\
Coordonn\'ees des communicants dans les d\'el\'egations r\'egionales du CNRS sur \url{http://www.cnrs.fr/fr/organisme/dircom/comdelegations.htm}} de vos tutelles qui chercheront \`a les valoriser
aupr\`es de la presse, des \'elu$\cdot$es, des jeunes, du grand public... (n'oubliez pas de pr\'evenir votre directeur$\cdot$trice d'unit\'e).

Pour cela, r\'edigez, si possible avant publication, un court texte en fran\c cais (environ une demi-page)
 repla\c cant le travail dans son contexte et explicitant votre r\'esultat.
Ce document, \'eventuellement accompagn\'e d'une illustration ou d'un sch\'ema, permettra de d\'eterminer
 l'audience susceptible d'\^etre int\'eress\'ee et de pouvoir b\'en\'eficier de diverses chambres de r\'esonances
 au niveau local, r\'egional ou national.
En effet, m\^eme si l'information ne fait pas l'objet d'un communiqu\'e de presse national, elle peut \^etre mise
en avant, par exemple dans :
\begin{itemize}
 \item des sites web (laboratoire ou institut, d\'el\'egation r\'egionale...) ;
 \item la lettre bi-mensuelle aux m\'edias "En direct des labos", diffusant les actualit\'es scientifiques
 des instituts du CNRS ;
 \item le journal du CNRS\footnote{\url{http://www2.cnrs.fr/presse/journal/}} (magazine mensuel
 tir\'e \`a 50 000 exemplaires, envoy\'e \`a tous les agents CNRS ainsi qu'\`a 2000 journalistes, \'elus, partenaires...) ;
 \item CNRS Hebdo (lettre \'electronique diffus\'ee par courriel chaque vendredi, regroupant des informations
nationales et les actualit\'es de la d\'el\'egation r\'egionale et de ses laboratoires) ;
 \item journal ou site web de l'universit\'e.
\end{itemize}

\smallskip\par
Il ne faut pas oublier que les sites web sont aujourd'hui la principale source d'information qu'utilisent
 les \'etudiants pour choisir une universit\'e et un laboratoire pour faire un master ou un doctorat. Avoir
 un site qui pr\'esente les activit\'es du laboratoire, avec une partie en anglais, est un atout important.
Cela peut aussi susciter des collaborations avec des scientifiques travaillant dans d'autres secteurs, des industriels.


\medskip
Si vous produisez des \textbf{images scientifiques}, elles peuvent \^etre d\'epos\'ees dans la banque d'images de CNRS Images\footnote{\url{http://phototheque.cnrs.fr/}}, en acc\`es libre sur Internet (exemples d'utilisations : exposition, presse, plaquette et marque-pages de l'INSMI, sites internet, etc.).

\medskip
Enfin, pour toute communication ou publication, n'oubliez pas de \textbf{mentionner
l'ensemble des tutelles} de votre laboratoire, et pas seulement votre organisme employeur.

%%%%%%%%%%%%%%%%%%%%%%%%%%%%%%%%%%%%%%%%%%%%%%%%%%
%%%%%%%%%%%%%%%%%%%%%%%%%%%%%%%%%%%%%%%%%%%%%%%%%%

\chapter{Listes de diffusion}


Nous donnons  pour conclure une liste (\textbf{non exhaustive}) de description de quelques listes de diffusion susceptibles de vous int\'eresser (peut-\^etre \^etes-vous
d\'ej\`a abonn\'e$\cdot$e \`a certaines d'entre elles) qui vous permettront d'\^etre
tenu inform\'e$\cdot$e r\'eguli\`erement des points qui vous int\'eressent le plus. \\

\begin{itemize}
\item
{\bf Liste SMF}~: r\'eserv\'ee aux adh\'erent$\cdot$es SMF, environ 15 courriels/an.
\\
\item {\bf Liste SMAI-Info}~: il s'agit d'une lettre \'electronique
mensuelle. On peut choisir ses rubriques parmi 15 th\`emes. Ouverte
\`a toutes et \`a tous.

\lien{smai.emath.fr/smai-info}
\\
%\item{\bf Liste SMAI~}: r\'eserv\'ee aux adh\'erents SMAI, environ 20 courriels/an.
%\\
\item {\bf Forums SMF}~: plusieurs espaces de discussions. Ouverts \`a tous et \`a toutes.

\lien{smf.emath.fr/}
\\
\item {\bf OP koi29}~: liste de diffusion d'informations relatives
aux concours MCF et PR notamment, par l'Op\'eration Postes. Ouverte \`a
tou$\cdot$tes.

\lien{postes.smai.emath.fr/2021/OUTILS/koi29/index.php}
\\
\item{\bf APRES KOI29}~:
liste de diffusion d'informations pour les chercheuses, chercheuses,
enseignants-chercheurs et enseignantes-chercheuses (appel d'offres, primes, d\'el\'egations,
{\em etc.}). Ouverte \`a tous et \`a toutes.

\lien{postes.smai.emath.fr/apres/apres-koi29/apres-koi29.php}
\\
\item {\bf Mathdoc}~: lettre d'information trimestrielle,

\lien{mathdoc.emath.fr/news/}
\\
\item {\bf Mathrice}~: plusieurs listes de discussion du GdS
mathrice \`a propos de l'administration syst\`eme (mathrice), des serveurs web
 (mathtoile), de l'annuaire (mathldap).

\lien{mathrice.org/}
\\
\item {\bf Calcul}~: liste de discussion orient\'ee sur les
probl\`emes li\'es \`a l'utilisation de l'informatique pour le calcul au
sens le plus large. Cette liste multi-disciplinaire est largement
ouverte \`a tou$\cdot$tes les acteur$\cdot$trices du calcul, institutionnels et
industriels.

\lien{calcul.math.cnrs.fr/spip.php?rubrique3}
\\
%\item {\bf CLORA (Europe)}
%
%\lien{www.clora.net/public/membres/inscription.html}
%\\
\item{\bf \'Egide PHC  - programmes bilat\'eraux }

\lien{www.campusfrance.org/fr/formulaire/listes-de-diffusion}
\\
\item{\bf Lettre de l'ANR}

\lien{www.agence-nationale-recherche.fr\string:80/LettreAgence}
\\
\item{\bf NA-digest}~:
The NA Digest is a collection of articles on topics related to numerical analysis and those who practice it.

\lien{www.netlib.org/na-digest-html/}
\\
\item{\bf Lettre d'informations de la Fondation Sciences math\'ematiques de Paris}

Accessible depuis la page de la fondation : \lien{www.sciencesmath-paris.fr}
\\
\item {\bf Le forum Parit\'e}~:

\lien{listes.mathrice.fr/math.cnrs.fr/info/forum-parite}
\\
\item{\bf Lettre d'informations d'AMIES}
\\
\lien{www.agence-maths-entreprises.fr/a/?q=fr/newsletters}
\\
\item{\bf Veille d'information de la conf\'erence des pr\'esidents d'universit\'e}
\\
\lien{https://listes.cpu.fr/sympa/subscribe/veillecpu}
\end{itemize}




\printindex

\end{document}
