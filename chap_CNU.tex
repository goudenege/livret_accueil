%%%%%%%%%%%%%%%%%%%%%%%%%%%%%%%%%%%%%%%%%%%
%%%%%%%%%%%%%%%%%%%%%%%%%%%%%%%%%%%%%%%%%%%

\chapter{Le CNU} \label{CNU}


Le conseil national des
universit\'es (CNU) est l'instance nationale comp\'etente pour le
recrutement et la carri\`ere des enseignant$\cdot$es-chercheur$\cdot$ses.
Il est en particulier charg\'e d'examiner les demandes de
qualification (MCF et PR), de promotion, et de cong\'e pour
recherche ou conversion th\'ematique (CRCT). 
Depuis 2014, il est charg\'e d'\'emettre un avis sur les demandes de PEDR et bient\^ot, il pourrait \'egalement \^etre en charge d'effectuer le suivi de carri\`ere des enseignant$\cdot$es-chercheur$\cdot$ses.

\index{Conseil national des universit\'es (CNU)}
\index{Cong\'e pour recherche ou conversion th\'e\-ma\-ti\-que
(CRCT)}
\index{Professeur d'universit\'e (PR)!qualification aux fonctions
de}
\index{Ma\^\i tre de conf\'erences (MCF)!qualification aux fonctions de}

%%%%%%%%%%%%%%%%%%%%%%%%%%%%%%%%%%%%%%
\section{Sa composition}

En 25\ieme{} (math\'ematiques) et 26\ieme{}
(math\'ematiques appliqu\'ees et
applications des math\'e\-ma\-ti\-ques) sections, le CNU est
compos\'e de 96 membres~: 48 titulaires et 48 suppl\'eant$\cdot$es, les rangs A (PR
et assimil\'es) et rangs B (MCF et assimil\'es) \'etant
repr\'esent\'es \`a parit\'e. %Les membres nomm\'es le sont par le minist\`ere. 
Chaque conseil si\`ege pour quatre ans et poss\`ede un
bureau constitu\'e de six personnes~: un$\cdot$e pr\'esident$\cdot$e (PR), deux
vice-pr\'esident$\cdot$es (un PR et un MCF) et trois assesseurs (1 PR et 2 MCF). Vous
trouverez la composition actuelle des CNU 25 et 26 sur les sites\\
\lien{cnu25.emath.fr/} et \lien{cnu26.emath.fr/}.

\index{Professeur$\cdot$e d'universit\'e (PR)} \index{Ma\^\i tre de
conf\'erences (MCF)}
%%%%%%%%%%%%%%%%%%%%%%%%%%%%%%%%%%%%%%%%%
\section{Ses missions}
\subsection{La qualification} \label{kalif}

La qualification est une des \'etapes n\'ecessaires pour postuler (voir \ref{recrutement})
aux fonctions de ma\^\i  tre de conf\'erences ou de professeur$\cdot$e des
universit\'es (sauf pour les postes r\'eserv\'es aux MCF habilit\'e$\cdot$es
ayant plus de dix ans d'anciennet\'e). Le nombre de qualifi\'e$\cdot$es
n'est pas li\'e au nombre de postes offerts au concours. La
qualification reste valable quatre ans et, chaque ann\'ee, un
arr\^et\'e pr\'ecise les modalit\'es et les conditions d'inscription
sur la liste de qualification.  Un lien vers ces arr\^et\'es peut \^etre trouv\'e \`a l'adresse\\
\lien{cnu25.emath.fr/qualif/index.html}. \\

La proc\'edure est la suivante (les dates sont donn\'ees \`a titre
indicatif)~:
\begin{itemize}

\item septembre/octobre~: inscription sur les listes de demande
de qualification. L'inscription se fait sur l'application ANTARES.
Vous obtenez ainsi un num\'ero de candidat$\cdot$e (indispensable).
Attention~: la cl\^oture des inscriptions est d\'efinitive~! Si vous
ratez cette \'etape, il vous faut attendre l'ann\'ee suivante~;

\index{ANTARES}

\item novembre/d\'ecembre~: d\'esignation par le bureau du CNU des
rapporteurs (2 par candidat$\cdot$e) ;
\item mi-d\'ecembre~: date \`a laquelle la th\`ese ou l'habilitation doit avoir \'et\'e soutenue~;
\item mi-d\'ecembre~: envoi des dossiers aux
rapporteur$\cdot$trices. Les titulaires de dipl\^omes
universitaires, qualifications et titres de niveau \'equivalent
peuvent \^etre dispens\'e$\cdot$es du doctorat (ou de l'habilitation) par le
CNU. Dans la pratique, cette dispense peut \^etre accord\'ee pour
les candidat$\cdot$es ayant effectu\'e leurs \'etudes et/ou une partie de
leur carri\`ere \`a l'\'etranger~;

\index{Habilitation \`a diriger des recherches (HDR)}

\item janvier/f\'evrier~: examen des dossiers~;

\item janvier/f\'evrier~: r\'eunion et d\'ecisions du CNU. Lors de
cette r\'eunion, le dossier de chaque candidat$\cdot$e est d\'ecrit par les
rapporteur$\cdot$trices et l'ensemble des membres du CNU d\'ecide de la
qualification. Seuls les rangs A du CNU examinent et d\'ecident des
qualifications aux fonctions de professeur$\cdot$e~;

\item f\'evrier~: les candidat$\cdot$es consultent leurs r\'esultats sur ANTARES et impriment l'\'ecran pour en conserver une copie.\\
\end{itemize}

En cas de refus de qualification, la ou le candidat$\cdot$e peut demander les
rapports \'ecrits des deux rapporteur$\cdot$trices, ainsi que celui du CNU.
L'arr\^et\'e  pr\'ecise les
modalit\'es d'obtention des motifs de refus.
%
De plus, la ou le candidat$\cdot$e pourra prendre contact avec la ou le pr\'esident$\cdot$e de la section CNU,
qui pr\'ecisera les raisons du refus.
%
Dans le cas de deux refus cons\'ecutifs, le d\'ecret de 1984
pr\'evoit une possibilit\'e de r\'eexamen~:

 {\it Les candidats dont la qualification a fait l'objet de deux
refus successifs de la part d'une section du conseil national des
universit\'es peuvent saisir de leur candidature le groupe
comp\'etent du conseil national des universit\'es en formation
restreinte aux bureaux de section. Cette formation se prononce dans
les m\^emes conditions de proc\'edure que la section comp\'etente du
conseil national des universit\'es. Elle proc\`ede
toutefois \`a l'audition des candidats.}\\

Un$\cdot$e candidat$\cdot$e qualifi\'e$\cdot$e n'ayant pas obtenu de poste au bout de quatre
ans doit demander une nouvelle qualification s'il ou elle veut candidater
\`a nouveau.\\

Les crit\`eres de qualification varient d'une section \`a l'autre.
Nous renvoyons aux pages des CNU 25 et 26 pour plus de d\'etails~:
\lien{cnu25.emath.fr/} et \lien{cnu26.emath.fr/}.

\subsection*{Quelques chiffres}

Voici quelques chiffres sur les qualifications par les CNU 25 et
26~:
\begin{center}
\begin{tabular}{*{11}{c}}
\toprule
 & 2006& 2007 & 2008&2009&2010&2011&2012&2013&2014&2016\\
\midrule
MCF 25  &  217/288 & 207 & 216/275 & 249/309 & 207/257 &200/250 & 222 & 242/285 &238/290 &223/307\\
PR 25 &   103/125& 101/123& 101/118 & 120/145 & 89/102 &97/101 & 104& 112/120 & 100/111&68/75\\
MCF 26  &  284/410& 252/385& 247/384&259/466&249/392& 289/426&271/396& 291/442&310/458\\
PR 26 & 96/118 &96/125&108/146&102/144&83/115&100/138 &111/129&97/139 & 110/145 \\
\bottomrule
\end{tabular}
\end{center}

Lorsqu'il y a deux chiffres, le premier chiffre correspond au nombre
de qualifi\'e$\cdot$es et le deuxi\`eme chiffre correspond au nombre de
dossiers \'etudi\'es par le CNU.


\subsection{Les promotions}

Les possibilit\'es de promotion sont~:
\begin{itemize}
\item la hors-classe pour les ma\^\i  tres de conf\'erences, \`a laquelle on peut postuler
\`a partir du septi\`eme \'echelon (soit avec 16 ans d'anciennet\'e !) ;
\item la premi\`ere classe et la classe exceptionnelle
(1\ier{} \'echelon et 2\ieme{} \'echelon)
pour les professeur$\cdot$es.\\
\end{itemize}

Le nombre de promotions est calcul\'e chaque ann\'ee en fonction,
entre autres, des choix budg\'etaires, mais aussi des textes
l\'egislatifs. Ce nombre est d\'efini globalement, pour l'ensemble
des sections, sous forme d'un pourcentage de promotions par
rapport au nombre de ``promouvables\footnote{Au sens de
``susceptibles d'\^etre promus".}"
dans chaque grade. Voir par exemple :\\
{\footnotesize \lien{www.legifrance.gouv.fr/affichTexte.do?cidTexte=JORFTEXT000025756970\&dateTexte=\&categorieLien=id}}

Il existe trois voies de promotion.
\begin{itemize}
\item La voie 1, ou voie normale, concerne la grande
majorit\'e des cas. Les promotions sont attribu\'ees pour moiti\'e par
le CNU, et pour moiti\'e par les \'etablissements eux-m\^emes. Ces
derni\`eres ann\'ees, les universit\'es traitaient les promotions
avant le CNU, mais \`a pr\'esent le CNU si\`ege avant que les
promotions locales ne soient accord\'ees.

Le contingent de promotions accord\'ees par le CNU est r\'eparti
\'equitablement entre les sections~: le pourcentage global d\'efini
{\em a priori}, divis\'e par deux, multipli\'e par le nombre de
promouvables dans chaque grade et chaque section, donne le nombre de
promotions g\'er\'ees au niveau du CNU. En revanche, le contingent de
promotions affect\'e \`a une universit\'e n'est pas partag\'e en
sections~: chaque universit\'e (conseil scientifique ou conseil
d'administration) peut r\'epartir les promotions dont elle dispose
sans contrainte d'\'equilibre entre les sections.

\item La voie 2 ne concerne que les petits \'etablissements
pour lesquels le nombre de promouvables est trop faible. Les
promotions sont alors enti\`erement attribu\'ees par le CNU.

\item La voie 3, ou voie sp\'ecifique, est r\'eserv\'ee \`a
ceux qui exercent des responsabilit\'es administratives
particuli\`eres (chefs d'\'etablissement). Les promotions sont
globalis\'ees pour toutes les sections et attribu\'ees par une
instance sp\'eciale.
\end{itemize}

\subsection*{Quelques chiffres}

Voici quelques chiffres sur les promotions par le CNU 25
\begin{center}
\begin{tabular}{*{12}{c}}
\toprule
CNU 25 &  2005 & 2006& 2007 & 2008 & 2009 & 2010&2011&2012&2013& 2014 &2016\\
\midrule
MCF HC &  11/88& 11/74 &11/67&12/74&16&19 & 24& 21&19/48 & 18/46 &18/49\\
PR1 &  11/121 & 11/102&10/103& 10/100&11&14 & 15 & 16& 15/80& 14/79&12/75\\
PRCE 1 &  4/79 & 7/67&7/69&8/81&9 &11& 12 & 11&10/46 & 11/50 &10/51 \\
PRCE 2  & 4/20& 5/16&4/15&4/15&4&4& 3& 4& 5/24& 5/26 & 7/36\\
\bottomrule
\end{tabular}
\end{center}

\noindent et par le CNU 26

\begin{center}
\begin{tabular}{*{11}{c}}
\toprule
CNU 26  & 2005 & 2006& 2007 & 2008 & 2009 & 2010& 2011& 2012 & 2013 & 2014\\
\midrule
MCF HC &  12/122 & 12/122&12/99&13/106 &19/100& 21/94 &26/85 &24/70 &  22/81 & 22/77 \\
PR1 &  14/161 & 15/148&12/139&13/148& 16/153& 19/133 &19/116 &18/101&17/105 & 16/101\\
PRCE 1 & 4/87 & 7/86&6/91&9/99& 10/89& 11/82& 13/67& 13/64& 14/72 & 14/62\\
PRCE 2   & 3/17& 4/15&3/11&  3/9& 3/14& 3/13 &4/21 &5/29& 5/27 & 6/42\\
\bottomrule
\end{tabular}
\end{center}

Lorsqu'il y a deux chiffres, le premier chiffre correspond au nombre
de promu$\cdot$es et le deuxi\`eme chiffre correspond au nombre de dossiers
\'etudi\'es par le CNU. Les listes nominatives des promu$\cdot$es sont consultables sur les sites respectifs
des CNU.

%%%%%Nouvelle section%%%%%%


\subsection{La PEDR}

L'attribution  de  la  PEDR est  du  ressort  des  universit\'es,  mais  la  plupart  font  appel  \`a  l'expertise  des diff\'erentes  sections  des  CNU    pour  l'\'evaluation  des  dossiers  des  candidat$\cdot$es. Chaque  section  devra attribuer  aux  dossiers  des  avis    A,  B  ou  C,  avec  un  contingentement  d\'efini  par  le  minist\`ere (20\%  de  A,  30\%  de  B  et  50\%  de  C)
Pour  l'examen  des  dossiers,  des  avis    (A,  B  ou  C)  seront  attribu\'es  dans  quatre  rubriques  distinctes que  les  candidat$\cdot$es  sont  invit\'e$\cdot$es  \`a  mettre  en  valeur
\begin{itemize}
\item la production  scientifique;
\item l'encadrement  doctoral  et  scientifique;
\item les responsabilit\'es  scientifiques;
\item le rayonnement.
\end{itemize}

\begin{enumerate}
\item Parmi  ces  quatre  rubriques,  la  production  scientifique  jouera  un  r\^ole  pr\'epond\'erant  dans  l'\'evaluation  des  dossiers.  La  publication d'articles  dans  des  revues  s\'electives  joue un  r\^ole important  dans  l'\'evaluation  de  la  production  scientifique,  la qualit\'e  des  articles  \'etant  plus  importante   que   leur   nombre,   les   brevets   et   logiciels   \'eventuels   auront   une   influence importante.
\item Pour l'encadrement  doctoral,  le  nombre  et  le  taux d'encadrement  des  th\`eses  est  un  \'el\'ement d'appr\'eciation  central  mais  \'egalement  le  devenir  des  docteur$\cdot$es.    Pour  les  MCF l'encadrement  de  m\'emoires  de  M2,  le  co-encadrement  de  th\`eses  seront  consid\'er\'es.
\item Pour  les responsabilit\'es  scientifiques  seront  consid\'er\'ees  les  activit\'es  de  direction  de  grands  programmes,  organisation  de  congr\`es,  directions d'unit\'es  de  recherche,  d'\'ecoles  doctorales,  responsabilit\'es de  masters,  de  contrats  industriels  ou  publics.  
\item Pour   le   rayonnement   seront   consid\'er\'ees   les   activit\'es   \'editoriales,     invitations   dans   des  universit\'es  \'etrang\`eres,  expertises  nationales  ou  internationales  et  les  participations  \`a  des  jurys  de  th\`ese  ou  d'HDR.
\end{enumerate}

Ces  quatre rubriques  seront  \'evalu\'ees  de  mani\`ere diff\'erenci\'ee suivant  que  le ou la  candidat$\cdot$e  appartienne  \`a l'une  des  trois  cat\'egories  suivantes:  MCF,  PR2  ou  
PR1-PREX. Elles sont susceptibles d'\'evoluer et nous vous conseillons de vous renseigner sur les pages des CNU 25 et 26 avant de d\'eposer un dossier.

En 2016, la section 25 a \'emis un avis A pour 23 MCF et 21 PR, un avis B pour 33 MCF et 32 PR et qu'un avis C pour 55 MCF et 54 PR.



%Pour  les  ma\^itres  de  conf\'erences    r\'ecemment  nomm\'es  (dans  les  six  derni\`eres    ann\'ees)  les  rubriques  encadrement  doctoral  et  responsabilit\'es  scientifiques  n'ont  en  g\'en\'eral  pas  grand  sens.  Cependant,  la  pr\'esence  d'\'el\'ements    comme  les  encadrements  de  M2,  co-encadrements  de  th\`ese,  responsabilit\'e  d'un  s\'eminaire,  etc  ...    sera  un  \'el\'ement  crucial  d'appr\'eciation    pour  certains  jeunes  MCF  particuli\`erement  actifs.   De   mani\`ere   g\'en\'erale,   pour   les   jeunes   MCF,   l'autonomie   acquise   par rapport   au directeur/travaux  de  th\`ese  est  un  \'el\'ement  d'appr\'eciation  important. Les rubriques  encadrement  doctoral  et  responsabilit\'es  scientifiques  sont  particuli\`erement  prises  en  compte   pour   les   professeurs.  L'absence   de   responsabilit\'es   administratives ou  d'encadrement  doctoral     dans   le   dossier   d'un   PR2   et   surtout   d'un   PR1-PREX   est   une   anomalie   qui   peut  \'eventuellement  \^etre  compens\'ee  par  une  activit\'e  scientifique  particuli\`erement  brillante.    Il  n'est  pas  du  ressort  de  la  PEDR  de  r\'ecompenser  une  activit\'e administrative  particuli\`erement  intense  (non  accompagn\'ee  d'une  production  scientifique  brillante)  mais  il  est  
%anormal  qu'un  PR  ne  prenne  pas  sa  part  d'activit\'es  administratives.  La  m\^eme    analyse  sera  appliqu\'ee  aux  MCF  ``exp\'eriment\'es''  (recrut\'es  depuis  au  moins  6  ans).
%
%
%Les   candidats   sont   invit\'es   \`a   mettre   clairement   ces   \'el\'ements   en   avant   dans   leur   dossier.  Pour son  \'evaluation,  le  CNU  s�attachera  quasi  exclusivement  
%\`a  l'examen  des  activit\'es  dans  les  quatre  derni\`eres  ann\'ees. A  noter  cependant  :  la  section  est  souveraine  dans  ses  choix  et  ses  d\'elib\'erations  ont  lieu  \`a  huis  
%clos.  En  aucun  cas  les  crit\`eres  d\'ecrits  ci-dessus  ne  font  l'objet  d'une  application  automatique.

%%%%%%%%%


\subsection{L'examen des demandes de CRCT}

Le CNU examine \'egalement, chaque ann\'ee, les demandes de Cong\'e
pour Recherche ou Conversion Th\'ematique (CRCT) et propose un
classement des candidat$\cdot$es. Une partie des cong\'es est g\'er\'ee
nationalement par le CNU, l'autre \'etant g\'er\'ee localement par
chaque universit\'e. Maintenant que le CNU si\`ege avant les
conseils des universit\'es (conseil scientifique), les dossiers des
candidats qui n'ont pas obtenu de CRCT sur le contingent national
peuvent \^etre transmis aux universit\'es.
\index{Cong\'e pour recherche ou conversion th\'e\-ma\-ti\-que
(CRCT)}
 En 2016, le CNU 25 disposait de 8 semestres de CRCT %et le CNU 26 de 8 semestres
 !

\subsection{La transformation de postes}

Le CNU donne son avis sur les transformations de postes d'assistant$\cdot$e
en ma\^ \i tre de conf\'erences, ou de ma\^\i  tre de conf\'erences
en professeur$\cdot$e. Notamment, il donne son avis {\it a posteriori} pour
les postes de professeur$\cdot$es r\'eserv\'e$\cdot$es aux ma\^\i  tres de
conf\'erences habilit\'e$\cdot$es ayant plus de dix ans d'anciennet\'e,
pour lesquels l'inscription sur les listes de qualification n'est
pas n\'ecessaire.

\index{Assistan$\cdot$e}

\subsection{Le reclassement}

 Le CNU examine les demandes de
validation de services d'enseignement ou/et de recherche effectu\'es
\`a l'\'etranger pour une prise en compte dans l'anciennet\'e. Il
faut faire parvenir au CNU, par l'interm\'ediaire du service du
personnel, vos contrats de travail (certifi\'es, et \'eventuellement
traduits).

\index{Validation des services effectu\'es \`a l'\'etranger}

\subsection{Liens}

Vous pouvez vous reporter \`a la page du minist\`ere

{\small \lien{www.enseignementsup-recherche.gouv.fr/cid22711/le-conseil-national-des-universites.html}}

et aux sites des sections CNU 25 et 26
{\small \lien{cnu25.emath.fr/}} et {\small \lien{cnu26.emath.fr/}}.


