%%%%%%%%%%%%%%%%%%%%%%%%%%%%%%%%%
\chapter{Le m\'etier de chercheur$\cdot$se \`a l'INRA}
\emph{Attention, ce chapitre n'est peut-\^etre plus \`a jour.}

\section{L'Institut : statut, structures, personnels}

L'Institut National de la Recherche Agronomique (\lien{www.inra.fr/}), cr\'e\'e en 1946 est, depuis 1984, un Etablissement Public \`a Caract\`ere Scientifique et Technologique (EPST). Il est plac\'e sous la double tutelle du Minist\`ere charg\'e de l'Agriculture et du Minist\`ere charg\'e de la Recherche.\\
Il a pour missions de :
\begin{itemize}
\item \oe uvrer au service de l'int\'er\^et public tout en maintenant l'\'equilibre entre les exigences de la recherche et les demandes de la soci\'et\'e ;
\item produire et diffuser des connaissances scientifiques et des innovations, principalement dans les domaines de l'agriculture, de l'alimentation et de l'environnement ;
\item contribuer \`a l'expertise, \`a la formation, \`a la promotion de la culture scientifique et technique, au d\'ebat science/soci\'et\'e.
\end{itemize}

\medskip

Les recherches de l'INRA ont pour but de parfaire et d'exploiter la connaissance du monde vivant au service de l'agriculture, de l'alimentation et de l'environnement rural de l'homme.\\

L'INRA est administr\'e par un Conseil d'Administration pr\'esid\'e par le ou la Pr\'esident$\cdot$e de l'Institut. Ce$\cdot$tte dernier$\cdot$e assure \'egalement la direction g\'en\'erale. Il ou elle est assist\'e$\cdot$e du Conseil Scientifique de l'Institut. Les recherches sont conduites au sein d'unit\'es de recherches (200 unit\'es de recherche dont 103 associ\'ees \`a d'autres organismes) r\'eunies au sein de 13 d\'epartements de recherche\footnote{Alimentation humaine, Biologie et am\'elioration des plantes, Caract\'erisation et \'elaboration des produits issus de l'agriculture, Ecologie des for\^ets, prairies et milieux aquatiques, Environnement et agronomie, G\'en\'etique animale, Math\'ematiques et informatique appliqu\'ees, Microbiologie et cha\^ine alimentaire, Physiologie animale et syst\`emes d'\'elevage, Sant\'e animale, Sant\'e des plantes et environnement, Sciences pour l'action et le d\'eveloppement, Sciences sociales, agriculture et alimentation, espace et environnement} ; ceux-ci sont eux-m\^emes coordonn\'es par 3 directeur$\cdot$trices scientifiques \footnote{Alimentation et Bio\'economie, Agriculture, Environnement}. Dix-sept centres r\'egionaux INRA et un centre si\`ege sont r\'epartis en 148 sites dans toute la France (m\'etropole et Antilles-Guyane). L'INRA comprend \'egalement 49 unit\'es exp\'erimentales et 109 unit\'es d'appui et de service. Son budget \'etait de 882 millions d'euros en 2013.\\

Les personnels de l'INRA sont des fonctionnaires de l'Etat r\'egis par le statut g\'en\'eral de la fonction publique. Ce statut est fix\'e par les lois \no83-634 du 13 juillet 1983 et \no84-16 du 11 janvier 1984 modifi\'ees relatives au statut g\'en\'eral des fonctionnaires, les d\'ecrets \no83-1260 du 30 d\'ecembre 1983 et \no84-1207 du 28 d\'ecembre 1984 modifi\'es relatifs respectivement aux fonctionnaires des EPST et \`a celles et ceux de l'INRA. Ces textes r\'eglementent les diff\'erentes \'etapes de la carri\`ere des agents : recrutement, avancement, cong\'es, cessation de fonctions.\\

Aujourd'hui l'INRA compte plus de 10 000 agents titulaires (dont 4458 chercheur$\cdot$ses et ing\'enieur$\cdot$es), 19\% des chercheur$\cdot$ses recrut\'e$\cdot$es en 2013 \'etaient \'etrangers. Chaque ann\'ee, l'INRA accueille \'egalement plus de 2000 jeunes scientifiques parmi lesquel$\cdot$les des doctorant$\cdot$es et des jeunes docteur$\cdot$es. Les jeunes scientifiques en contrat avec l'Institut sont des membres \`a part enti\`ere du personnel scientifique de l'unit\'e de recherche dans laquelle ils et elles \'evoluent. Elles et ils contribuent activement aux recherches conduites tout en se formant \`a la recherche. Les directeur$\cdot$trices d'unit\'e veillent \`a ce que d\`es leur arriv\'ee, les doctorant$\cdot$es et les jeunes chercheur$\cdot$ses s'inscrivent dans un projet professionnel clair et construit.


\section{ Le recrutement}
Conform\'ement aux missions imparties aux personnels de la recherche, les chercheur$\cdot$es de l'organisme doivent contribuer non seulement \`a l'acquisition de connaissances nouvelles dans les domaines de leurs comp\'etences mais aussi au transfert des r\'esultats de leurs travaux dans la soci\'et\'e : valorisation \'economique et sociale, diffusion des informations scientifiques et techniques, formation \`a et par la recherche, d\'eveloppement des \'echanges scientifiques avec l'\'etranger.\\ 
Quelle que soit leur discipline de formation, les chercheur$\cdot$es s'appuient sur des activit\'es de laboratoire ou de "terrain". Ils et elles sont fortement impliqu\'e$\cdot$es dans des r\'eseaux scientifiques, r\'epondent \`a des questions environnementales, \'economiques, sociales. Recherche personnelle et projet collectif s'imbriquent \'etroitement pour faire progresser les connaissances et pour participer au d\'eveloppement de l'innovation.\\

L'INRA emploie plus de 1900 chercheur$\cdot$es. Ils$\cdot$elles sont des fonctionnaires de l'Etat et sont r\'eparti$\cdot$es selon deux cat\'egories : les {\bf directeur$\cdot$trices de recherche} (seniors) et les {\bf charg\'e$\cdot$es de recherche} (juniors). Ils$\cdot$elles \'evoluent dans des {\bf disciplines scientifiques} vari\'ees. Pr\`es de 44\% des chercheur$\cdot$ses de l'INRA sont des femmes. En sa qualit\'e d'\'etablissement public, l'Inra recrute ses chercheur$\cdot$ses par voie de concours. L'Institut a re\c{c}u la reconnaissance officielle de la  Commission Europ\'eenne pour l'excellence de sa  politique de ressources humaines \`a l'\'egard des  chercheurs. Ainsi, une  charte interne \`a l'Institut  pr\'ecise les conditions d'accueil et d'insertion des jeunes  docteur$\cdot$es en termes de recrutement, de positionnement dans  les unit\'es d'accueil, de formation, de publication  et  de valorisation des r\'esultats

\subsection{ Les charg\'e$\cdot$es de recherche}
Plus de 1200 Charg\'e$\cdot$es de recherche \'evoluent au sein des \'equipes de recherche de l'Inra. En d\'ebut de carri\`ere, les {\bf charg\'e$\cdot$es de recherche} (jeunes docteur$\cdot$es) se consacrent \`a l'avancement de la th\'ematique de recherche qui leur a \'et\'e confi\'ee et \`a la publication syst\'ematique des r\'esultats acquis. Elles et ils b\'en\'eficient de l'environnement de chercheur$\cdot$ses confirm\'e$\cdot$es. Par la suite, ils$\cdot$elles encadrent eux$\cdot$elles-m\^emes des personnels techniques et des stagiaires qui vont concourir au d\'eveloppement de leur projet. Les fonctions d'animation et d'encadrement prennent progressivement davantage de place, ainsi que les activit\'es d'enseignement, mais la priorit\'e reste focalis\'ee sur la production scientifique.\\

Les chercheur$\cdot$ses sont recrut\'e$\cdot$es par voie de concours organis\'es par discipline ou groupe de disciplines.  Chaque ann\'ee, l'INRA organise une campagne de concours pour le recrutement de {\bf charg\'e$\cdot$es de recherche de 2e classe (CR2)}. Le recrutement s'effectue, en r\`egle g\'en\'erale, parmi les {\bf chercheur$\cdot$ses d\'ebutant$\cdot$es} ayant soutenu une th\`ese depuis peu. Les candidat$\cdot$es sont recrut\'e$\cdot$es pour leurs comp\'etences scientifiques qu'ils$\cdot$elles mettront au service des grandes orientations de l'Inra en r\'epondant \`a une th\'ematique de recherche. Les candidats doivent avoir valoris\'e les r\'esultats de leur th\`ese par des publications. Les recrutements sont ouverts dans de nombreuses th\'ematiques scientifiques telles que la biologie cellulaire et mol\'eculaire, l'\'ecologie, l'\'economie, la g\'en\'etique, la g\'enomique et autres approches "omiques", l'informatique et l'intelligence artificielle, les math\'ematiques, la nutrition, la physiologie, la physico-chimie, les sciences m\'edicales et v\'et\'erinaires et la sociologie. Le {\bf calendrier de la campagne} est en g\'en\'eral le suivant (sous r\'eserve de la publication de l'arr\^et\'e d'ouverture au Journal Officiel) : ouverture des inscriptions fin janvier, cl\^oture des inscriptions fin f\'evrier, admissibilit\'e (sur dossier) en avril-mai, admission (\'epreuve orale) en juin-juillet. Les candidat$\cdot$es admis$\cdot$es sont nomm\'e$\cdot$es en qualit\'e de stagiaires pour une dur\'ee d'un an et sont titularis\'e$\cdot$es, apr\`es avis de la Commission scientifique sp\'ecialis\'ee (CSS) comp\'etente. Toutefois, le stage peut \^etre prolong\'e de 18 mois au maximum ou il peut \^etre mis fin aux fonctions du chercheur ou de la chercheuse apr\`es avis de l'instance d'\'evaluation et de la Commission Administrative Paritaire (CAP) comp\'etente \`a l'\'egard du corps des Charg\'es de Recherche.\\

Chaque ann\'ee, l'INRA organise une campagne de concours pour le recrutement de {\bf charg\'e$\cdot$es de recherche de 1\`ere classe (CR1) sur projet}. Le concours de Charg\'e$\cdot$e de Recherche de 1\`ere classe s'adresse \`a des {\bf chercheur$\cdot$ses confirm\'e$\cdot$es} (aptitude \`a concevoir, pr\'esenter et conduire un projet de recherche : capacit\'e \`a prendre des responsabilit\'es d'animation et d'encadrement dans un cadre collectif). Ainsi, les candidat$\cdot$es doivent avoir fait preuve, par leur parcours, de leur autonomie professionnelle et de leur ouverture sur des r\'eseaux de collaboration. Ils$\cdot$Elles doivent poss\'eder leur culture scientifique propre, leur r\'eseau de collaborations et avoir \`a leur actif une production scientifique de bonne qualit\'e et de bon niveau. Au grade de CR1, les fonctions d'animation et d'encadrement prennent progressivement davantage de place, ainsi que les activit\'es d'enseignement, mais la priorit\'e reste focalis\'ee sur la production scientifique. Le {\bf calendrier de la campagne} est en g\'en\'eral le suivant : ouverture des inscriptions fin juin, cl\^oture des inscriptions fin ao\^ut, admissibilit\'e (sur dossier) octobre, admission (\'epreuve orale) en novembre-d\'ecembre.  Les nominations, effectu\'ees \`a l'issue des \'epreuves d'admission, sont d\'ecid\'ees par la ou le Pr\'esident$\cdot$e de l'Institut dans l'ordre de la liste des admis$\cdot$es. Les candidat$\cdot$es admis$\cdot$es sont nomm\'e$\cdot$es en qualit\'e de stagiaires pour une dur\'ee d'un an et sont titularis\'e$\cdot$es, apr\`es avis de la Commission scientifique sp\'ecialis\'ee (CSS) comp\'etente. Toutefois, le stage peut \^etre prolong\'e d'un an au maximum ou il peut \^etre mis fin aux fonctions du chercheur apr\`es avis de l'instance d'\'evaluation et de la Commission Administrative Paritaire (CAP) comp\'etente \`a l'\'egard du corps des Charg\'e$\cdot$es de Recherche.

\subsection{ Les directeur$\cdot$trices de recherche}
Plus de 700 directeur$\cdot$trices de recherche conduisent les grands projets et les \'equipes de recherche de l'INRA. Les directeur$\cdot$trices de recherche (chercheur$\cdot$ses confirm\'e$\cdot$es), reconnu$\cdot$es par la qualit\'e de leurs publications scientifiques et l'excellence des projets qu'elles ou ils ont conduits, animent et dirigent de grands projets ou des unit\'es de recherche. Ils$\cdot$Elles ont la capacit\'e d'animer, sous tous leurs aspects, des programmes europ\'eens ou des \'equipes de recherche de taille significative. Leur capacit\'e d'expertise av\'er\'ee est appr\'eci\'ee dans les instances r\'eglementaires ou aupr\`es de structures porteuses d'importants enjeux socio-\'economiques. Chaque ann\'ee, l'INRA organise une campagne de concours pour le {\bf recrutement de directeur$\cdot$trices de recherche}. Le {\bf calendrier de cette campagne} est le suivant : ouverture des inscriptions fin juin, cl\^oture des inscriptions fin ao\^ut, admissibilit\'e (sur dossier) octobre, admission (\'epreuve orale) en novembre-d\'ecembre.

\subsection{ S'informer sur l'ouverture des concours : publicit\'e et contacts}

{\bf Publicit\'e}\\
L'ouverture de chaque session de concours est fix\'ee par arr\^et\'e publi\'e au Journal officiel. Le nombre de postes propos\'es et la date limite de d\'ep\^ot des dossiers sont \'egalement fix\'es par arr\^et\'e. L'ouverture des concours fait par ailleurs l'objet d'une publicit\'e sur Internet :\lien{www.inra.fr } (rubrique "Carri\`eres \& emplois").\\

{\bf Contacts}\\
Toutes les informations utiles (conditions pour concourir, documents \`a fournir pour s'inscrire, d\'eroulement des \'epreuves) peuvent \^etre obtenues aupr\`es de la Direction des Ressources Humaines.

\section{L'\'evaluation}
Conform\'ement au d\'ecret qui r\'egit l'\'evaluation des chercheur$\cdot$ses des EPST et \`a celui sp\'ecifiant les instances d'\'evaluation des chercheurs pour l'INRA, des Commissions Scientifiques Sp\'ecialis\'ees (CSS) \'evaluent les chercheur$\cdot$ses, \`a un rythme biennal, sur la base d'un dossier. {\bf Treize commissions} \'evaluent les chercheur$\cdot$ses de l'INRA. Douze d'entre elles sont d\'efinies par les disciplines et les m\'ethodes de recherche et sont transversales aux d\'epartements. Une treizi\`eme commission \'evalue les chercheur$\cdot$ses ayant des activit\'es de direction, d'animation ou de gestion de la recherche. Les p\'erim\`etres des CSS ont \'et\'e progressivement adapt\'es aux dynamiques scientifiques de l'Institut de fa\c{c}on \`a favoriser les interactions scientifiques jug\'ees strat\'egiques pour l'INRA. Ces p\'erim\`etres sont valid\'es par le Conseil Scientifique de l'Institut. Chaque chercheur$\cdot$se choisit sa commission d'\'evaluation apr\`es une discussion avec son ou sa directeur$\cdot$trice d'unit\'e. Les chercheur$\cdot$ses qui ont un profil pluridisciplinaire et dont les disciplines scientifiques ne sont pas suffisamment repr\'esent\'ees au sein d'une seule commission, peuvent soumettre leur dossier \`a deux commissions. Enfin, une commission peut demander \`a une chercheuse ou un chercheur  de soumettre son dossier \`a une autre commission, qu'elle jugera plus comp\'etente.\\

Ces commissions r\'ealisent une {\bf \'evaluation-conseil}. Elles produisent pour la direction de l'Institut un avis sur chaque dossier \'evalu\'e. Ces avis sont utilis\'es par la direction pour diff\'erentes d\'ecisions concernant la gestion des personnels. Ils sont statutairement requis pour les demandes de titularisation des charg\'e$\cdot$es de recherche, les candidatures de promotion en CR1 et en DR de classe exceptionnelle. Ils seront aussi disponibles pour la direction lors de son examen des candidatures \`a la prime d'excellence scientifique. Elles formulent des recommandations sur les aspects de l'activit\'e qui doivent \^etre am\'elior\'es. Elles r\'edigent un message personnel destin\'e \`a chaque chercheur$\cdot$se qui concr\'etise l'attention port\'ee \`a son profil d'activit\'e et \`a sa production et formulent d'\'eventuels conseils.\\

L'\'evaluation des chercheuses et chercheurs de l'Inra, r\'ealis\'ee par les CSS, porte sur l'ensemble des activit\'es des chercheur$\cdot$ses et prend en compte leur environnement, les missions qui leur sont confi\'ees et les objectifs des collectifs auxquels elles et ils appartiennent. L'\'evaluation par les CSS est une {\bf  \'evaluation ind\'ependante} de la hi\'erarchie et de l'environnement proche des chercheur$\cdot$ses. Enfin, cette \'evaluation est coll\'egiale : les avis et les messages sont le r\'esultat du travail de l'ensemble de la commission sous la responsabilit\'e de son ou sa pr\'esident$\cdot$e.

\section{ Les carri\`eres et les r\'emun\'erations}
Votre \'echelon d\'etermine l'indice auquel vous allez \^etre r\'emun\'er\'e$\cdot$e. La valeur d'un point d'indice est \'egale \`a 4,6303 euros depuis le 01/01/2010. L'indice de r\'emun\'eration auquel le ou la candidat$\cdot$e est recrut\'e$\cdot$e est d\'etermin\'e en fonction de ses dipl\^omes et de ses activit\'es professionnelles ant\'erieures.

\subsection{ Progression de carri\`ere pour les chercheur$\cdot$ses}
L'avancement d'\'echelon \`a l'int\'erieur d'un m\^eme grade intervient en fonction de l'anciennet\'e, selon les tableaux ci-dessous. Les Charg\'e$\cdot$es de recherche de 2\`eme classe (CR2) peuvent \^etre promu$\cdot$es au choix \`a la 1\`ere classe (CR1), apr\`es avis de la Commission scientifique sp\'ecialis\'ee (CSS) comp\'etente, sous r\'eserve de justifier d'au
moins 4 ann\'ees d'anciennet\'e dans le grade de CR2. Les Charg\'e$\cdot$es de recherche de 1\`ere classe (CR1), justifiant d'une anciennet\'e minimale de 3 ann\'ees dans le grade, peuvent se pr\'esenter aux concours pour l'acc\`es au corps des Directeur$\cdot$trices de recherche de 2\`eme classe (DR2). Il s'agit d'un v\'eritable changement de m\'etier. La pr\'esentation et l'argumentation d'un projet sont indispensables. \`a titre tr\`es exceptionnel, tout$\cdot$e Charg\'e$\cdot$e de Recherche peut concourir pour l'acc\`es au corps des Directeurs de Recherche sans condition d'anciennet\'e sous r\'eserve d'y avoir \'et\'e autoris\'e$\cdot$e par le Conseil Scientifique de l'\'etablissement, au vu de la contribution notoire qu'il ou elle aura apport\'ee \`a la recherche. Une bonification d'anciennet\'e d'un an est accord\'ee aux Charg\'e$\cdot$es de Recherche qui effectuent une mobilit\'e dont la dur\'ee est au moins \'egale \`a 2 ans :
\begin{itemize}
\item dans un autre organisme de recherche ou d'enseignement sup\'erieur \`a l'\'etranger,
\item aupr\`es d'une administration, d'une collectivit\'e locale ou d'une entreprise publique ou priv\'ee.
\end{itemize}
Les Directeur$\cdot$trices de recherche de 2\`eme classe (DR2) peuvent acc\'eder \`a la 1\`ere classe (DR1) apr\`es examen de leur dossier de candidature par une commission d'avancement.

\subsubsection*{Charg\'e$\cdot$es de recherche de 2\ieme{} classe}
\begin{center}
\begin{tabular}{lccc}
\toprule
& Indice brut& Indice major\'e (01/01/2013)& Anciennet\'e requise dans l'\'echelon \\
\midrule
1\ier{} \'echelon &530&454& 1 an \\

2\ieme{} \'echelon &542&461& 1 an \\

3\ieme{} \'echelon &580&490& 1 an\\

4\ieme{} \'echelon &618&518 &1 an et 4 mois\\

5\ieme{} \'echelon &653&545 & 2ans\\

6\ieme{} \'echelon &677&564&Terminal \\
\bottomrule
\end{tabular}
\end{center}


\subsubsection*{Charg\'e$\cdot$es de recherche de 1\iere{} classe}
\begin{center}
\begin{tabular}{lccc}
\toprule
& Indice brut& Indice major\'e (01/01/2013)& Anciennet\'e requise dans l'\'echelon \\
\midrule
1\ier{} \'echelon &562&476& 2 ans \\

2\ieme{} \'echelon &600&505& 2 ans et 6 mois \\

3\ieme{} \'echelon &678&564& 2 ans et 6 mois\\

4\ieme{} \'echelon &755&623 &2 ans et 6 mois\\

5\ieme{} \'echelon &821&673 & 2ans et 6 mois\\

6\ieme{} \'echelon &882&719& 2 ans et 6 mois\\

7\ieme{} \'echelon &920&749 &2 ans et 9 mois\\

8\ieme{} \'echelon &966&783 & 2ans et 10 mois\\

9\ieme{} \'echelon &1015&821&Terminal \\
\bottomrule
\end{tabular}
\end{center}

\subsubsection*{Directeur$\cdot$trices de recherche de 2\ieme{} classe}
\begin{center}
\begin{tabular}{lccc}
\toprule
& Indice brut& Indice major\'e (01/01/2013)& Anciennet\'e requise dans l'\'echelon \\
\midrule
1\ier{} \'echelon &801&658& 1 an et 3 mois\\

2\ieme{} \'echelon &852&696& 1 an et 3 mois\\

3\ieme{} \'echelon &901&734& 1 an et 3 mois\\

4\ieme{} \'echelon &958&776 &1 an et 3mois\\

5\ieme{} \'echelon &1015&821 & 3 ans et 6 mois\\

6\ieme{} \'echelon &Hors \'echelle A&A1, A2, A3&Terminal \\
\bottomrule
\end{tabular}
\end{center}


\subsubsection*{Directeur$\cdot$trices de recherche de 1\iere{} classe}
\begin{center}
\begin{tabular}{lccc}
\toprule
& Indice brut& Indice major\'e (01/01/2013)& Anciennet\'e requise dans l'\'echelon \\
\midrule
1\ier{} \'echelon &1015&821& 3 ans \\

2\ieme{} \'echelon &Hors \'echelle B&B1, B2, B3& 3 ans \\

3\ieme{} \'echelon &Hors \'echelle C&C1, C2, C3& Terminal\\
\bottomrule
\end{tabular}
\end{center}

\subsubsection*{Directeur$\cdot$trices de recherche classe exceptionnelle}
\begin{center}
\begin{tabular}{lccc}
\toprule
& Indice brut& Indice major\'e (01/01/2013)& Anciennet\'e requise dans l'\'echelon \\
\midrule
1\ier{} \'echelon &Hors \'echelle D&D1, D2, D3& 1 an et 6 mois \\

2\ieme{} \'echelon &Hors \'echelle E&E1, E2& Terminal \\
\bottomrule
\end{tabular}
\end{center}

\subsection{ Les primes et indemnit\'es}
{\bf La prime de recherche (987 \euro{}  annuels pour les CR et 796 \euro{}{}  annuels pour les DR) est attribu\'ee mensuellement \`a toutes les chercheuses et tous les chercheurs selon leur grade et leur corps. L'indemnit\'e d'enseignement (42.72 \euro{}{}  annuels) est vers\'ee mensuellement \`a l'agent exer\c{c}ant une activit\'e d'enseignement. L'indemnit\'e de r\'esidence est vers\'ee mensuellement selon l'affectation g\'eographique de l'agent. Un suppl\'ement familial de traitement peut \^etre ajout\'e en fonction du nombre d'enfants \`a charge.}\\

La prime d'encadrement doctoral et de recherche (PEDR) concerne les chercheur$\cdot$ses en activit\'e. Elle peut leur \^etre attribu\'ee en reconnaissance de leur contribution individuelle \`a l'activit\'e scientifique. L'objectif de cette prime est de r\'ecompenser financi\`erement l'excellence de l'activit\'e des chercheurs et chercheuses en allouant \`a certains d'entre eux et elles une prime individuelle sur la base de r\'esultats av\'er\'es. Trois cas d'attribution sont pr\'evus par le d\'ecret d'application :
\begin{itemize}
\item pour les laur\'eats d'une distinction scientifique de niveau international ou national conf\'er\'ee par un organisme de recherche et dont la liste est fix\'ee par arr\^et\'e minist\'eriel ; l'attribution est alors automatique sans condition d'enseignement (prime type 1) ;
\item pour les chercheurs et chercheuses apportant une contribution exceptionnelle \`a la recherche (prime type 2 : 2A et 2B) ;
\item pour les chercheuses et chercheurs dont l'activit\'e scientifique est jug\'ee d'un niveau \'elev\'e, sous r\'eserve qu'elles et ils remplissent la condition d'enseignement pr\'evue (prime type 3). Le ou la candidat$\cdot$e \`a cette prime s'engage \`a remplir, d\`es la premi\`ere ann\'ee de versement de la prime, la condition d'enseignement correspondant \`a 64 heures de travaux dirig\'es annuels (ou activit\'e de formation \'equivalente). Les chercheur$\cdot$ses s'engagent \`a fournir chaque ann\'ee avant le 1er f\'evrier, le d\'ecompte des heures effectu\'ees l'ann\'ee pr\'ec\'edente.
\end{itemize}

\section{La mobilit\'e}

La mobilit\'e, qu'elle soit th\'ematique, g\'eographique ou effectu\'ee vers d'autres \'etablissements ou entreprises du secteur public ou priv\'e, fait partie int\'egrante du parcours professionnel. Elle offre l'opportunit\'e de concilier les \'evolutions et les besoins de l'Institut avec les comp\'etences et les aspirations individuelles des agents.
Diff\'erentes dispositions sont offertes au fonctionnaire pour effectuer une mobilit\'e. \\

La {\bf mise \`a disposition} au cours de laquelle le fonctionnaire demeure dans son corps d'origine. Elle ou il est r\'eput\'e$\cdot$e occuper son emploi et continue de percevoir sa r\'emun\'eration,
mais elle ou il effectue tout ou partie de son service aupr\`es d'un ou de plusieurs organismes d'accueil. La mise \`a disposition peut \^etre prononc\'ee au profit d'administrations des trois fonctions publiques, ainsi que d'organismes contribuant \`a la mise en \oe uvre de la politique de l'Etat, des collectivit\'es territoriales ou de leurs \'etablissements publics administratifs, pour l'exercice des seules missions de service public confi\'ees \`a ces organismes. La mise \`a disposition intervient avec l'accord du fonctionnaire. Elle est prononc\'ee pour une dur\'ee maximale de trois ans et peut \^etre renouvel\'ee. Elle est g\'en\'eralement encadr\'ee par une convention ou un contrat  de partenariat sign\'e par les partenaires et les agents concern\'es. Ce document pr\'ecise notamment la nature des activit\'es qu'il va exercer et ses conditions d'emploi.\\

Dans le cas du {\bf d\'etachement}, le ou la fonctionnaire est plac\'e hors de son corps ou cadre d'emploi initial pour travailler au sein d'un autre organisme que son administration d'origine. Il ou elle continue toutefois \`a jouir des droits \`a l'avancement et \`a la retraite attach\'es \`a son corps d'origine. D'un point de vue administratif, son \'evolution de carri\`ere se poursuit de mani\`ere parall\`ele dans les deux \'etablissements (\'etablissement d'origine et \'etablissement d'accueil). La dur\'ee du d\'etachement peut \^etre courte (6 mois port\'es \`a un an pour ceux qui exercent une mission \`a l'\'etranger) ou longue (jusqu'\`a 5 ans renouvelables).\\

Pour les chercheur$\cdot$ses, {\bf il n'existe pas de proc\'edure de mobilit\'e interne} avec une campagne d'affichage des profils \`a pourvoir. Lorsqu'un chercheur ou une chercheuse exprime un souhait de mobilit\'e (g\'eographique, th\'ematique), il ou elle en informe son directeur ou sa directrice d'unit\'e et son ou sa chef de d\'epartement, puis cette mobilit\'e se construit sur la base d'un projet scientifique, en interaction avec l'unit\'e d'accueil. C'est \'egalement le cas lorsque les chercheur$\cdot$ses r\'ealisent une mobilit\'e cons\'ecutive \`a la restructuration, la d\'elocalisation ou la fermeture de leur unit\'e. Si cette mobilit\'e se traduit par un changement de d\'epartement de recherche, elle n\'ecessitera aussi une n\'egociation entre les chefs des d\'epartements concern\'es, puis l'arbitrage final de la direction g\'en\'erale.
