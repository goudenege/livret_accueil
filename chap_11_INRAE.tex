 
%%%%%%%%%%%%%%%%%%%%%%%%%%%%%%%%%%%%%%%%%%%%%%%%%%
 
 \chapter{INRAE}
 
 \section{Structure et fonctionnement}
 
L'Institut National de Recherche pour l'Agriculture, l'alimentation et l'Environnement (INRAE), issu de la fusion au 1er janvier 2020 entre l'Institut national de Recherche en Sciences et Technologies pour l'Environnement et l'Agriculture (IRSTEA) et de l’Institut National de la Recherche Agronomique (INRA), est un organisme de recherche qui mobilise de nombreuses disciplines scientifiques r\'eparties dans 14 d\'epartements scientifiques de recherche\footnote{
\og Action, transition et territoires \fg{},
\og  Agro\'ecosyst\`emes \fg{},
\og  Alimentation Humaine \fg{},
\og  Aliments, produis biosourc\'es et d\'echets \fg{},
\og  Biologie et am\'elioration des Plantes \fg{},
\og  \'Ecologie et biodiversit\'e \fg{},
\og  \'Economie et sciences sociales \fg{},
\og  \'Ecosyst\`emes aquatiques, ressources en eau et risques \fg{},
\og  G\'en\'etique Animale \fg{},
\og  Math\'ematiques et num\'erique \fg{},
\og  Microbiologie et cha\^ine alimentaire \fg{},
\og  Physiologie animale et syst\`emes d'\'elevage \fg{},
\og  Sant\'e animale \fg{},
\og  Sant\'e des plantes et environnement \fg{}.}.
Ces d\'epartements se r\'epartissent sur 18 centres sur toute la France et regroupe un peu plus de 200 unit\'es de recherche.

\textbf{Lien utile\hspace{.5em}}\url{https://www.inrae.fr}

\section{Le d\'epartement MathNum} 

Le d\'epartement MathNum m\`ene des recherches en math\'ematique et en sciences et technologies du num\'erique (informatique, bioinformatique et intelligence artificielle, en sciences de l'information et de la communication, sur les technologies de robotiques, capteurs et imagerie satellitaire). Il se distingue par sa position transversale et collabore r\'eguli\`erement avec d'autres d\'epartements d'INRAE, ainsi qu'avec des partenaires scientifiques nationaux et internationaux tels qu'INRIA, le CNRS, des universit\'es et des grandes \'ecoles Paris-Saclay, AgroParisTech, Montpellier, Toulouse,  {\em etc}). Grâce \`a ses collaborations interdisciplinaires, le d\'epartement contribue \`a la compr\'ehension, la pr\'ediction et l'aide \`a la d\'ecision sur des syst\`emes complexes relevant des sciences du vivant et de l'environnement

MathNum est impliqu\'e dans des communaut\'es disciplinaires en math\'ematique, informatique, sciences et technologies du num\'erique, sous la forme de Labex, fondations, instituts convergences (DigitAg et Dataia), institut interdisciplinaire en intelligence artificielle (3IA Aniti), f\'ed\'erations ou groupes de recherche.

Exer\c{c}ant nos recherches dans le cadre d’une science ouverte motiv\'ee par l’acquisition des connaissances et sensibles aux grands enjeux soci\'etaux d’aujourd’hui, les \'equipes du D\'epartement partagent l’ambition de d\'evelopper un num\'erique responsable, attentif \`a des enjeux tels que la reproductibilit\'e, l’innovation ouverte, la mesure de l‘impact et la frugalit\'e num\'erique.

\textbf{Lien utile\hspace{.5em}}\url{https://www.inrae.fr/departements/mathnum}

\subsection{Champs th\'ematiques}

Les activit\'es du D\'epartement incluent de la recherche th\'eorique, m\'ethodologique et appliqu\'ee, avec des collaborations notamment en biologie et \'ecologie pr\'edictives, en \'epid\'emiologie, en agriculture num\'erique, et plus g\'en\'eralement pour l’\'etude et la mod\'elisation des syst\`emes complexes rencontr\'es dans les domaines d’int\'er\^et d’INRAE. Les recherches de MathNum s’organisent selon cinq champs th\'ematiques (CT), refl\'etant ses grands domaines de comp\'etence m\'ethodologiques et technologiques par rapport à la gestion de la donn\'ee et au traitement de l’information :
\begin{itemize}
\item Optique et t\'el\'ed\'etection, m\'etrologie, capteurs, traitement du signal ;
\item Repr\'esentation et extraction des connaissances, syst\`emes d'information ;
\item Probabilit\'es, statistique et apprentissage automatique ;
\item Mod\'elisation des syst\`emes complexes et syst\`emes dynamiques ;
\item Technologies pour l'action. automatique et contr\^ole, recherche op\'erationnelle.
\end{itemize}

\textbf{Lien utile\hspace{.5em}}\url{https://www.inrae.fr/departements/mathnum/recherches-du-departement-mathnum}

\subsection{Dispositif de recherche}

Le d\'epartement regroupe 24 \'equipes scientifiques r\'eparties dans 11 unit\'es de recherche sur 7 centres r\'egionaux INRAE. Parmi ces unit\'es de recherche, on compte 4 Unit\'es Mixtes de Recherche (UMR) : MIA Paris - Saclay (avec AgroParisTech), ITAP, MISTEA et TETIS (avec SupAgro \`a Montpellier) et une unit\'e sous contrat \`a Evry avec le CNRS et l'Universit\'e d'Evry.

\textbf{Lien utile\hspace{.5em}}\url{https://www.inrae.fr/departements/mathnum/structures-du-departement-mathnum}

\subsection{Ressources humaines et comp\'etences}

Le d\'epartement MathNum est en taille le plus petit d\'epartement d'INRAE. La population des chercheur\mp euse\mp s du d\'epartement se r\'epartit au sein de trois grandes familles disciplinaires : probabilit\'es et statistique, informatique et syst\`emes dynamiques. En 2023, le d\'epartement comptait pr\`es de 190 titulaires (INRAE ou en accueil) et 160 contractuels dont une centaine de doctorants. 

Toutes les chercheuses et tous les chercheurs du d\'epartement sont \'evalu\'es par la Commission Scientifique Sp\'ecialis\'ee Math\'ematiques, Bio-Informatique, Intelligence Artificielle.

\subsection{Les infrastructures et r\'eseaux de MathNum}

Le Département MathNum est fortement impliqué dans le pilotage d’infrastructures scientifiques d’intérêt collectif.

\textbf{Lien utile\hspace{.5em}}\url{https://www.inrae.fr/departements/mathnum/infrastructures-reseaux-mathnum}



%\section{ Le d\'epartement de Math\'ematiques et Informatique Appliqu\'ees}
%Le d\'epartement Math\'ematiques et Informatique Appliqu\'ees (MIA, \lien{www.mia.inra.fr/ } ou\\  \lien{www.mathinfo.inra.fr/fr} ) partage avec les autres d\'epartements de recherche de l'INRAE la mission principale de production de connaissances g\'en\'eriques et finalis\'ees, de mise au point de m\'ethodes, d'outils et de savoir-faire, dans ses champs de comp\'etences que sont les math\'ematiques et l'informatique appliqu\'ees aux domaines de l'alimentation, l'agriculture et l'environnement. \\
%L'emploi des math\'ematiques et de l'informatique est aujourd'hui fondamental pour relever les d\'efis scientifiques et technologiques auxquels fait face la recherche agronomique et les besoins en comp\'etences en math-info (m\'ethodes et ing\'enierie) augmentent dans tous les domaines de l'INRAE et ne se limitent plus au p\'erim\`etre du d\'epartement MIA. Ce nouveau contexte a ainsi conduit \`a actualiser r\'ecemment le r\^ole du d\'epartement MIA au sein de l'institut \`a travers trois familles de missions:
%\begin{itemize}
%\item Mission I : Le d\'epartement a pour mission de mener des recherches en math-info sur des verrous m\'ethodologiques qui \'emergent des enjeux prioritaires de la recherche agronomique (sciences du vivant, de l'environnement, etc.), et de mettre en oeuvre ces recherches via des partenariats (projets, th\`eses, etc.).
%\item Mission II : Le d\'epartement a \'egalement pour mission de conduire dans un cadre inter-disciplinaire des recherche \`a l'interface sur des enjeux prioritaires de l'INRAE pour lesquels le r\^ole des math-info, nouveau ou g\'en\'erique, est incontournable.
%\item Mission III : Le d\'epartement a enfin pour mission d'accompagner le d\'eveloppement des math\'ematiques et informatique \`a l'INRAE, concernant en particulier : 
%\begin{itemize}
%\item[(i)] l'ing\'enierie du dispositif INRAE en mati\`ere de traitement, gestion et analyse de donn\'ees, de calcul et de simulation, en particulier dans le cadre de plates-formes ;
%\item[(ii)] l'expertise en m\'ethodologie math\'ematiques-informatique et en ing\'enierie informatique et calcul intensif en direction des d\'epartements et des programmes ;
%\item[(iii)] la formation, l'entretien de la comp\'etence m\'etier, la diffusion et la promotion de la culture math\'e\-matiques-informatique ; 
%\item[(iv)] le suivi des partenariats entre l'INRAE et les autres organismes concernant les math\'ematiques et l'informatique.
%\end{itemize}
%\end{itemize}
%En termes m\'ethodologiques, les priorit\'es du d\'epartement MIA se d\'eclinent \`a l'heure actuelle selon deux axes li\'es \`a la gestion et \`a l'analyse des masses de donn\'ees h\'et\'erog\`enes et \`a la construction, analyse et simulation de mod\`eles complexes.
%
%\subsection{ Dispositif de recherche}
%
% Le d\'epartement MIA pilote ou co-pilote 7 unit\'es de recherche, pr\'esentes sur six sites INRAE en m\'etropole : trois sont des unit\'es dites "propres" et constitu\'ees quasi-exclusivement de   personnes rattach\'ees \`a MIA (MIA Toulouse \lien{carlit.toulouse.inra.fr/wikiz/index.php/Accueil} , MIA Jouy \\  \lien{www6.jouy.inra.fr/mia} et BioSP \lien{www.biosp.org/} en Avignon) une unit\'e, MIG\\ \lien{mig.jouy.inra.fr/ } \`a Jouy-en-Josas, est commune avec les d\'epartements PHASE et MICA , et deux unit\'es sont des Unit\'es Mixtes de Recherche (UMR) avec d'autres organismes de recherche ou d'enseignement (l'unit\'e MISTEA de Montpellier \lien{www6.montpellier.inra.fr/mistea/}  avec l'\'ecole SupAgro et l'INRIA et l'unit\'e MIA de Paris \lien{www.agroparistech.fr/mia/ } avec l'\'ecole AgroParisTech. Enfin, le d\'epartement MIA est impliqu\'e dans une unit\'e sous contrat \`a Evry avec le CNRS et l'Universit\'e d'Evry (\lien{www.math-evry.cnrs.fr/sg/welcome}).

%\subsection{Les r\'eseaux scientifiques soutenus par le d\'epartement MIA}
%Le d\'epartement soutient fortement plusieurs r\'eseaux scientifiques sur des th\'ematiques vari\'ees : Elicitations de dires d'experts, Algorithmic Issues for Inference in Graphical Models (AIGM), Exploration num\'erique des propri\'et\'es des mod\`eles (MEXICO), Inf\'erence de R\'eseaux Biologiques (NETBIO), Mod\'elisation de paysage agricole (PAYOTE), Taxonomie num\'erique mol\'eculaire (TANUMO), Mod\'elisation et statistique en sant\'e des animaux et des plantes (ModStatSAP), Statistique pour les trajectoires, Int\'egration de sources/masses de donn\'ees h\'et\'erog\`enes et ontologies, Statistiques pour les Sciences Participatives (CiSStats), Mod\'elisation et simulation informatique des agro-\'ecosyst\`emes (RECORD), formalisme Discrete Event System (DEVS), Mod\`eles et M\'ethodes statistiques pour les variables spatio-temporelles, Syst\`emes d'\'equations diff\'erentielles et autres syst\`emes dynamiques pour l'\'ecologie (MEDIA), R\'eduction et simplification de mod\`eles (REM), Optimisation : m\'ethodes et applications dans les sciences de la vie.
