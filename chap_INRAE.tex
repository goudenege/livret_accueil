 
%%%%%%%%%%%%%%%%%%%%%%%%%%%%%%%%%%%%%%%%%%%%%%%%%%
 
 \chapter{L'INRA}
 
 \section{L'INRA et les math\'ematiques}

Ses ancrages \`a la soci\'et\'e font de l'INRA un organisme de recherche "finalis\'ee" qui mobilise de nombreuses disciplines scientifiques : principalement les sciences de la vie (68 \% des comp\'etences scientifiques de l'Institut), mais aussi les sciences des milieux et des proc\'ed\'es (12 \%), l'ing\'enierie \'ecologique, les \'ecotechnologies et les biotechnologies (8 \%), ainsi que les sciences \'economiques et sociales (8 \%) et les sciences du num\'erique et mod\'elisation (4 \%). Les sciences du num\'erique et mod\'elisation regroupent pour l'essentiel des comp\'etences dans deux domaines principaux, d'une part les biostatistiques et bioinformatique et d'autre part la mod\'elisation, simulation et analyse de syst\`emes. L'\'eco-informatique, les math\'ematiques appliqu\'ees au calcul num\'erique, l'optimisation, la commande, l'algorithmique et l'aide \`a la d\'ecision ainsi que la repr\'esentation des connaissances sont moins repr\'esent\'ees. Des chercheuses et chercheurs en sciences du num\'erique et mod\'elisation sont pr\'esent$\cdot$es dans quasiment tous les d\'epartements de  recherche de l'INRA et plus particuli\`erement au sein du d\'epartement de Math\'ematiques et Informatique Appliqu\'ees.

\section{ Le d\'epartement de Math\'ematiques et Informatique Appliqu\'ees}
Le d\'epartement Math\'ematiques et Informatique Appliqu\'ees (MIA, \lien{www.mia.inra.fr/ } ou\\  \lien{www.mathinfo.inra.fr/fr} ) partage avec les autres d\'epartements de recherche de l'INRA la mission principale de production de connaissances g\'en\'eriques et finalis\'ees, de mise au point de m\'ethodes, d'outils et de savoir-faire, dans ses champs de comp\'etences que sont les math\'ematiques et l'informatique appliqu\'ees aux domaines de l'alimentation, l'agriculture et l'environnement. \\
L'emploi des math\'ematiques et de l'informatique est aujourd'hui fondamental pour relever les d\'efis scientifiques et technologiques auxquels fait face la recherche agronomique et les besoins en comp\'etences en math-info (m\'ethodes et ing\'enierie) augmentent dans tous les domaines de l'INRA et ne se limitent plus au p\'erim\`etre du d\'epartement MIA. Ce nouveau contexte a ainsi conduit \`a actualiser r\'ecemment le r\^ole du d\'epartement MIA au sein de l'institut \`a travers trois familles de missions:
\begin{itemize}
\item Mission I : Le d\'epartement a pour mission de mener des recherches en math-info sur des verrous m\'ethodologiques qui \'emergent des enjeux prioritaires de la recherche agronomique (sciences du vivant, de l'environnement, etc.), et de mettre en oeuvre ces recherches via des partenariats (projets, th\`eses, etc.).
\item Mission II : Le d\'epartement a \'egalement pour mission de conduire dans un cadre inter-disciplinaire des recherche \`a l'interface sur des enjeux prioritaires de l'INRA pour lesquels le r\^ole des math-info, nouveau ou g\'en\'erique, est incontournable.
\item Mission III : Le d\'epartement a enfin pour mission d'accompagner le d\'eveloppement des math\'ematiques et informatique \`a l'INRA, concernant en particulier : 
\begin{itemize}
\item[(i)] l'ing\'enierie du dispositif INRA en mati\`ere de traitement, gestion et analyse de donn\'ees, de calcul et de simulation, en particulier dans le cadre de plates-formes ;
\item[(ii)] l'expertise en m\'ethodologie math\'ematiques-informatique et en ing\'enierie informatique et calcul intensif en direction des d\'epartements et des programmes ;
\item[(iii)] la formation, l'entretien de la comp\'etence m\'etier, la diffusion et la promotion de la culture math\'e\-matiques-informatique ; 
\item[(iv)] le suivi des partenariats entre l'INRA et les autres organismes concernant les math\'ematiques et l'informatique.
\end{itemize}
\end{itemize}
En termes m\'ethodologiques, les priorit\'es du d\'epartement MIA se d\'eclinent \`a l'heure actuelle selon deux axes li\'es \`a la gestion et \`a l'analyse des masses de donn\'ees h\'et\'erog\`enes et \`a la construction, analyse et simulation de mod\`eles complexes.

\subsection{ Dispositif de recherche}

 Le d\'epartement MIA pilote ou co-pilote 7 unit\'es de recherche, pr\'esentes sur six sites INRA en m\'etropole : trois sont des unit\'es dites "propres" et constitu\'ees quasi-exclusivement de   personnes rattach\'ees \`a MIA (MIA Toulouse \lien{carlit.toulouse.inra.fr/wikiz/index.php/Accueil} , MIA Jouy \\  \lien{www6.jouy.inra.fr/mia} et BioSP \lien{www.biosp.org/} en Avignon) une unit\'e, MIG\\ \lien{mig.jouy.inra.fr/ } \`a Jouy-en-Josas, est commune avec les d\'epartements PHASE et MICA , et deux unit\'es sont des Unit\'es Mixtes de Recherche (UMR) avec d'autres organismes de recherche ou d'enseignement (l'unit\'e MISTEA de Montpellier \lien{www6.montpellier.inra.fr/mistea/}  avec l'\'ecole SupAgro et l'INRIA et l'unit\'e MIA de Paris \lien{www.agroparistech.fr/mia/ } avec l'\'ecole AgroParisTech. Enfin, le d\'epartement MIA est impliqu\'e dans une unit\'e sous contrat \`a Evry avec le CNRS et l'Universit\'e d'Evry (\lien{www.math-evry.cnrs.fr/sg/welcome}).

\subsection{ Ressources humaines et comp\'etences}
Avec un peu plus d'une centaine de personnes, le d\'epartement MIA est en taille le plus petit d\'epartement de l'INRA. La population des chercheurs et chercheuses du d\'epartement se r\'epartit au sein de trois grandes familles disciplinaires : probabilit\'es et statistique (int\'egrant statistique pour l'image, probabilit\'es et processus stochastiques), informatique (algorithmique, repr\'esentation des connaissances) et syst\`emes dynamiques (au sens de mod\'elisation, analyse et conduite des syst\`emes dynamiques, incluant donc les forces du d\'epartement en automatique, et une partie de celles en intelligence artificielle).\\
Toutes les chercheuses et tous les chercheurs du d\'epartement sont \'evalu\'es par la Commission Scientifique Sp\'ecialis\'ee Math\'ematiques, Bio-Informatique, Intelligence Artificielle.

\subsection{Les r\'eseaux scientifiques soutenus par le d\'epartement MIA}
Le d\'epartement soutient fortement plusieurs r\'eseaux scientifiques sur des th\'ematiques vari\'ees : Elicitations de dires d'experts, Algorithmic Issues for Inference in Graphical Models (AIGM), Exploration num\'erique des propri\'et\'es des mod\`eles (MEXICO), Inf\'erence de R\'eseaux Biologiques (NETBIO), Mod\'elisation de paysage agricole (PAYOTE), Taxonomie num\'erique mol\'eculaire (TANUMO), Mod\'elisation et statistique en sant\'e des animaux et des plantes (ModStatSAP), Statistique pour les trajectoires, Int\'egration de sources/masses de donn\'ees h\'et\'erog\`enes et ontologies, Statistiques pour les Sciences Participatives (CiSStats), Mod\'elisation et simulation informatique des agro-\'ecosyst\`emes (RECORD), formalisme Discrete Event System (DEVS), Mod\`eles et M\'ethodes statistiques pour les variables spatio-temporelles, Syst\`emes d'\'equations diff\'erentielles et autres syst\`emes dynamiques pour l'\'ecologie (MEDIA), R\'eduction et simplification de mod\`eles (REM), Optimisation : m\'ethodes et applications dans les sciences de la vie.
