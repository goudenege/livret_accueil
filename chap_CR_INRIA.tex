%%%%%%%%%%%%%%%%%%%%%%%%%%%%%%%%%%%%%%%%%%%%%%%
%%%%%%%%%%%%%%%%%%%%%%%%%%%%%%%%%%%%%%%%%%%%%%%
\chapter{Le m\'etier de chercheur$\cdot$se Inria}
\index{Inria}
\index{Charg\'e$\cdot$e de recherche (CR)! Inria}
\index{Directeur$\cdot$trice de recherche (DR)! Inria}
R\'eparti$\cdot$es entre les corps de charg\'e$\cdot$es de recherche (CR) et de
directeur$\cdot$trices de recherche (DR), les chercheur$\cdot$ses d'Inria ont pour
missions g\'en\'erales de~:
\begin{itemize}
\item concevoir et mener les activit\'es de recherche scientifique,
\item participer aux transferts de r\'esultats dans les entreprises,
\item diffuser l'information et la culture scientifiques et techniques,
\item participer \`a la formation initiale et continue,
\item animer et coordonner les activit\'es de recherche.
\end{itemize}
Il y a actuellement 583 chercheuses et chercheurs permanent$\cdot$es r\'emun\'er\'e$\cdot$es par Inria (328 CR et 255 DR). 
La proportion de femmes est d'environ 17\,\% (chiffres de d\'ecembre 2015).

\section{Les concours}
\label{concours Inria}
Le recrutement des chercheur$\cdot$ses aux grades de CR2, CR1, DR2 et DR1 se
fait par concours une fois par an. Il existe \'egalement des
opportunit\'es d'emploi d'accueil en d\'etachement pour une dur\'ee
d\'etermin\'ee ({\em cf.} \ref{detachement}). 
Il existe enfin des postes en CDD, \og Starting Research Positions\fg\ et 
\og Advanced Research Positions \fg. 
%En 2013, respectivement 6 et 2 telles offres d'emploi sont propos\'ees par Inria.
Les titulaires d'un doctorat (ou d'un dipl\^ome de docteur$\cdot$e ing\'enieur$\cdot$e) peuvent concourir 
pour l'acc\`es au grade de CR2, et~:
\begin{itemize}
\item pour l'acc\`es au grade de CR1 s'ils justifient de 2 ans
d'exercice des m\'etiers de la recherche,
\item pour l'acc\`es au grade de DR2 s'ils justifient de 8 ans
d'exercice des m\'etiers de la recherche,
\item pour l'acc\`es au grade de DR1 s'ils justifient de 12 ans
d'exercice des m\'etiers de la recherche.
\end{itemize}

On peut indiquer qu'il n'y a pas de poste de CR1 ouverts \`a l'ext\'erieur chaque ann\'ee. 
Quand il y en a, ils peuvent \^etre soient fl\'ech\'es dans les centres (le plus souvent), 
soit au niveau national (plus rare). 
Toutes les informations utiles sur les recrutements figurent sur la
page suivante:

\lien{www.inria.fr/institut/recrutement-metiers}

Le concours s'effectue en quatre \'etapes~:
\begin{itemize}
\item une premi\`ere pr\'es\'election \'etablit la liste des candidat$\cdot$es admis$\cdot$es \`a
concourir~;
\item une deuxi\`eme \'etape de pr\'es\'election \'etablit la liste des candidat$\cdot$es
auditionn\'e$\cdot$es, puis il est proc\'ed\'e \`a l'audition des candidat$\cdot$es~;
\item \`a la suite de l'audition des candidat$\cdot$es, le jury \'etablit la
liste des candidat$\cdot$es admissibles~;
\item le jury d'admission, pr\'esid\'e par le ou la directeur$\cdot$trice g\'en\'eral$\cdot$e d'Inria, \'etablit la liste des candidates et des candidats admis$\cdot$es.
\end{itemize}
Plus souvent qu'au CNRS (pour l'informatique et les math\'ematiques), il arrive que les listes d'admissibilit\'e soient 
remani\'ees lors de l'\'etape d'admission.

\subsection{Le calendrier}
Les concours sont en g\'en\'eral ouverts mi-d\'ecembre, les dossiers
devant \^etre envoy\'es avant la mi-janvier, et les auditions ont lieu \`a peu pr\`es
en m\^eme temps que les auditions MCF.

Depuis 2008, Inria fait en sorte que le concours CR2 soit
d\'efinitivement termin\'e d\'ebut juin, de sorte que les candidat$\cdot$es
qui le souhaitent puissent se d\'esister de la proc\'edure MCF avant
la saisie Antar\`es.

\subsection{Quelques chiffres}
Il y a eu, en 2015, 18 recrutements de CR2, 1 de CR1 et 11 de DR2. 
Ces postes sont majoritairement affect\'es \`a un centre de recherche Inria (CRI).
Ces derni\`eres ann\'ees, les
CRI r\'ecents (Bordeaux, Lille et Saclay) ont b\'en\'efici\'e d'un
plus grand nombre de postes, notamment en CR2. \index{Centre de
recherche Inria (CRI)} De plus, 8 recrutements ont \'et\'e effectu\'es sur des \og Starting Research Positions\fg, 2 autres sur des \og Advanced Research Positions \fg.

\section{L'affectation}
Les candidats et candidates CR2 doivent pr\'eciser dans quelle(s) \'equipe(s)-projet(s)
ils et elles souhaitent \^etre affect\'e$\cdot$es ({\em cf.} \ref{sec. projets INRIA}). La situation est
un peu diff\'erente pour les concours CR1 et DR2~: les candidat$\cdot$es
peuvent demander \`a \^etre affect\'e$\cdot$es dans une \'equipe-projet existante, mais
il est aussi possible qu'ils$\cdot$elles proposent de mener une activit\'e de
recherche nouvelle au sein d'Inria et soient alors recrut\'e$\cdot$es en
dehors des \'equipes-projets existantes. Pour les concours d'acc\`es au grade
de DR1, les candidat$\cdot$es doivent indiquer le CRI dans lequel ils$\cdot$elles
souhaitent cr\'eer une \'equipe-projet. En cas de r\'eussite au concours, la
d\'ecision d'affectation est prise par le$\cdot$la directeur$\cdot$trice g\'en\'eral$\cdot$e d'Inria.

Nous rappelons que tout nouveau chercheur et toute nouvelle chercheuse Inria (comme tout
fonctionnaire) a le droit de pr\'esenter une demande de
reconstitution de carri\`ere~: elle permet de faire
reconna\^\i tre tout emploi comportant une activit\'e de
recherche pr\'ec\'edant l'embauche \`a Inria \`a fin d'avancement
d'\'echelon \`a l'anciennet\'e.

\section{Les carri\`eres et r\'emun\'erations}
Vous trouverez ci-apr\`es les grilles indiciaires des diff\'erents
grades de chercheur$\cdot$ses. Il est important de noter que ces grilles sont exactement les m\^emes que celles concernant les chercheur$\cdot$ses au CNRS. Le passage d'un \'echelon \` a l'autre se fait
\` a l'anciennet\'e, tandis que le passage d'un grade \` a l'autre
se fait au choix~: pour passer au grade de CR1, un CR2 Inria
voit sa candidature examin\'ee par la commission
d'\'evaluation ({\em cf.} \ref{CE-INRIA}).

Pour tous les indices, \`a compter du 1\ier{} juillet 2010, la valeur
du point d'indice est port\'ee \`a 55,5635\,\euro{} ({\em cf.}
\ref{salairesEC}).

\subsubsection*{Charg\'e$\cdot$es de recherche de 2\ieme{} classe}
\begin{center}
\begin{tabular}{lccc}
\toprule
& Indice (INM)& Dur\'ee& R\'emun\'eration annuelle brute \\
\midrule

1\ier{} \'echelon &454&1 an& 25\,225,83\,\euro{} \\

2\ieme{} \'echelon &461&1 an& 25\,614,77\,\euro{}\\

3\ieme{} \'echelon &490&1 an & 27\,226,12\,\euro{}\\

4\ieme{} \'echelon &518&1 an et 4 mois&28\,781,89\,\euro{}\\

5\ieme{} \'echelon &545&2 ans & 30\,282,11\,\euro{}\\

6\ieme{} \'echelon &564&Terminal&31\,337,81\,\euro{}\\
\bottomrule
\end{tabular}
\end{center}

\subsubsection*{Charg\'e$\cdot$es de recherche de 1\iere{} classe}
\begin{center}
\begin{tabular}{lccc}
\toprule
& Indice (INM)& Dur\'ee& R\'emun\'eration annuelle brute \\
\midrule
1\ier{} \'echelon &476&2 ans& 26\,448,23\,\euro{}\\

2\ieme{} \'echelon &505&2 ans 6 mois&28\,059,57\,\euro{}\\

3\ieme{} \'echelon &564& 2 ans 6 mois&31\,337,81\,\euro{}\\

4\ieme{} \'echelon & 623 &2 ans 6 mois&34\,616,06\,\euro{}\\

5\ieme{} \'echelon & 673& 2 ans 6 mois&37\,394,24\,\euro{}\\

6\ieme{} \'echelon &719&2 ans 6 mois&39\,950,16\,\euro{}\\

7\ieme{} \'echelon &749&2 ans 9 mois&41\,617,06\,\euro{}\\

$8^{\mbox{\tiny e}}$ \'echelon &783&2 ans et 10 mois&43\,506,22\,\euro{}\\

$9^{\mbox{\tiny e}}$ \'echelon &821&Terminal&45\,617,63\,\euro{}\\
\bottomrule
\end{tabular}
\end{center}

\subsubsection*{Directeur$\cdot$trices de recherche de 2\ieme{} classe}
\begin{center}
\begin{tabular}{lccc}
\toprule
& Indice (INM)& Dur\'ee& R\'emun\'eration annuelle brute \\
\midrule
1\ier{} \'echelon &658&1 an 3 mois& 36\,560,78\,\euro{} \\

2\ieme{} \'echelon &696&1 an 3 mois&38\,762,20\,\euro{}\\

3\ieme{} \'echelon &734&1 an 3 mois&40\,783,61\,\euro{}\\

4\ieme{} \'echelon &776 &1 an 3 mois&43\,117,28\,\euro{}\\

5\ieme{} \'echelon & 821 & 3 ans 6 mois&45\,617,63\,\euro{}\\

6\ieme{} \'echelon - A1 &881&1 an&48\,951,44\,\euro{}\\

6\ieme{} \'echelon - A2 &916&1 an&50\,896,17\,\euro{}\\

6\ieme{} \'echelon - A3 &963&Terminal &53\,507,65\,\euro{}\\
\bottomrule
\end{tabular}
\end{center}


\subsubsection*{Directeur$\cdot$trices de recherche de 1\iere{}
classe et de classe exceptionnelle}
L'indice major\'e des directrices et directeurs
de recherche de premi\`ere classe est compris entre 821 et 1164, ce
qui correspond \`a un salaire terminal de 64\,675,91\,\euro{} brut
annuel, et celui des directrices et directeurs de recherche de classe
exceptionnelle est compris entre 1164 et 1320, ce qui correspond \`a
un salaire terminal de 73\,343,82\,\euro{} brut annuel.

Pour plus d'informations, voir sur l'intranet d'Inria.
%\lien{www.INRIA.fr/interne/drh/}.
\subsubsection*{La prime de recherche annuelle}
Pour les CR2, CR1, DR2 et DR1, une prime de recherche annuelle (d'une valeur de quelques centaines d'euros) est vers\'ee semestriellement (en juin et d\'ecembre).

\subsubsection*{La prime d'encadrement doctoral et de recherche}
\index{Prime d'encadrement doctoral et de recherche (PEDR)}
\label{PEDR INRIA}

La prime d'encadrement doctoral et de recherche peut \^etre attribu\'ee, pour une p\'eriode de quatre ans, aux chercheuses et chercheurs d'Inria et aux enseignants chercheurs et enseignantes-chercheuses d\'etach\'e$\cdot$es dans un corps d'Inria, et ce \`a diff\'erents titres :
\begin{itemize}
\item Cat\'egorie [1] : Aux laur\'eats d'une distinction scientifique nationale ou internationale (voir \ref{PEDR CNRS}). Pour justifier une candidature au titre de la cat\'egorie [1], la chercheuse ou le chercheur aura \'et\'e distingu\'e$\cdot$e sur la p\'eriode des 8 ans pr\'ec\'edant l'ann\'ee de r\'ef\'erence.
\item Cat\'egorie [2] : Aux chercheur$\cdot$ses ``apportant une contribution exceptionnelle \`a la recherche''. Cette cat\'egorie concerne les directeurs et directrices de recherche de classe exceptionnelle et les laur\'eats d'autres distinctions scientifiques.
\item Cat\'egorie [3] : Aux chercheur$\cdot$ses dont le niveau d'activit\'e scientifique est particuli\`erement \'elev\'e et qui exercent une activit\'e d'encadrement doctoral, dans la mesure o\`u ils$\cdot$elles s'engagent \`a effectuer, en moyenne sur la p\'eriode de quatre ans, un service annuel d'enseignement \'equivalent \`a 64 heures de travaux dirig\'es.
\end{itemize}
~\\
{\bf Processus d'attribution}\\
\index{Prime d'encadrement doctoral et de recherche (PEDR)!crit\`eres d'attribution}

La PEDR peut \^etre accord\'ee sur pr\'esentation d'un dossier de candidature qui est identique quelle que soit la cat\'egorie au titre de laquelle le chercheur en sollicite l'attribution.\\

{\bf Crit\`eres d'appr\'eciation}
\begin{enumerate}
\item Les contributions \`a la recherche
\item Les contributions au transfert technologique et \`a l'innovation
\item Les contributions \`a l'enseignement, \`a la formation et \`a la diffusion de l'information scientifique
\item La reconnaissance nationale et internationale
\end{enumerate}
Pour l'appr\'eciation de chacun de ces crit\`eres, l'accent est mis sur la qualit\'e et l'impact des travaux.
Dans tous les cas, au-del\`a de l'excellence propre personnelle du candidat, sont prises en consid\'eration l'implication dans des recherches collectives et la capacit\'e \`a emmener une \'equipe ou un groupe vers le succ\`es.\\

{\bf Montant}\\
\index{Prime d'encadrement doctoral et de recherche (PEDR)!montants}
\\La PEDR est attribu\'ee aux b\'en\'eficiaires sur une p\'eriode de quatre ans, renouvelable. Le bar\`eme retenu par Inria a r\'ecemment chang\'e, et utilise maintenant 3 cat\'egories, \`a savoir \og juniors\fg\ (jusqu'\`a 6 ans apr\`es la th\`ese), \og confirm\'es \fg\ (entre th\`ese + 6 et th\`ese +14) et \og s\'eniors\fg\ (au moins 14 ans apr\`es la th\`ese). Les montants respectifs s'\'el\`event \`a (environ) 5000, 7000 et 9000 \euro{} bruts par an.

\section{L'\'evaluation} \label{CE-INRIA}
\index{Evaluation Inria}
Sauf circonstance particuli\`ere, les chercheur$\cdot$ses ne sont pas \'evalu\'es individuellement.
En revanche, les \'equipes-projets r\'edigent, chaque ann\'ee, un rapport d'activit\'e. Les
\'evaluations individuelles interviennent au moment
\begin{itemize}
\item de la titularisation \`a la fin de la premi\`ere ann\'ee suivant
le recrutement (le dossier est alors examin\'e par la commission
d'\'evaluation),
\item du passage CR2-CR1 (le dossier de promotion est alors examin\'e
par la commission d'\'evaluation),
\item du passage de l'habilitation \`a diriger des recherches,
\item des concours de recrutement.
\end{itemize}
\index{Habilitation \`a diriger des recherches (HDR)}

\section{Cumul d'activit\'es}
\index{Cumul d'activit\'e ! \`a Inria}
\label{sec. cumul INRIA}
Comme pour le CNRS, la ``r\`egle" veut que les fonctionnaires ne cumulent pas plusieurs emplois,
cf. la section \ref{sec. cumul CNRS}.
Toutefois, Inria peut autoriser ses agents \`a exercer, \`a titre accessoire, une activit\'e, lucrative ou non,
aupr\`es d'une personne ou d'un organisme public ou priv\'e, d\`es lors que cette activit\'e
est compatible avec les fonctions qui leur sont confi\'ees et n'affecte pas leur exercice.

\subsection{L'enseignement}
Inria encourage les activit\'es d'enseignement de ses
chercheurs et chercheuses dans la mesure o\`u cette activit\'e ne nuit pas aux
missions premi\`eres des chercheurs. Le fait d'enseigner est soumis
\`a autorisation de cumul.\\
Notez que pour \^etre \'eligible \`a la PEDR, un$\cdot$e chercheur$\cdot$se Inria doit en g\'en\'eral effectuer un service annuel \'equivalent \`a 64 heures de travaux dirig\'es (voir \ref{PEDR INRIA}).

\subsection{L'expertise et le conseil}
De m\^eme que pour l'enseignement, et \'egalement avec
l'autorisation de la direction, un$\cdot$e chercheur$\cdot$se Inria peut effectuer
une expertise ou du conseil aupr\`es d'une entreprise ou d'un
organisme priv\'e.

\section{La mobilit\'e}
\index{Mobilit\'e ! des chercheur$\cdot$ses Inria}
\subsection{Interne}
Il peut s'agir de mobilit\'e th\'ematique ou de mobilit\'e
g\'eographique. La mobilit\'e th\'ematique est motiv\'ee par des
raisons scientifiques, tandis que la mobilit\'e g\'eographique se
caract\'erise par une mutation dans un autre site Inria. Elle peut
s'accompagner d'une mobilit\'e th\'ematique ou fonctionnelle.

\subsection{Externe}
\label{sec. mobex INRIA}
Un$\cdot$e chercheur$\cdot$se peut choisir d'exercer temporairement une autre
activit\'e professionnelle en dehors d'Inria, en gardant un lien
plus ou moins fort avec l'Institut. Il$\cdot$Elle peut \'egalement, sous
certaines conditions, choisir d'interrompre son activit\'e
professionnelle \`a Inria pour raisons familiales ou pour r\'ealiser
un projet personnel. Comme pour les chercheuses et chercheurs CNRS ou les
enseignant$\cdot$es-chercheur$\cdot$ses, il existe trois possibilit\'es que nous
rappelons ici.

\subsubsection{La mise \`a disposition}
La mise \`a disposition aupr\`es d'un autre organisme est la position
qui permet de conserver le lien le plus fort avec Inria. En effet,
le ou la chercheur$\cdot$se demeure rattach\'e$\cdot$e \`a son corps d'origine \`a Inria, qui
continue donc \`a le r\'emun\'erer. Il ou elle continue \'egalement \`a
b\'en\'eficier de ses droits \`a l'avancement et \`a la retraite \`a
Inria. Il s'agit souvent d'une position de transition (en moyenne 6
mois, mais cela peut aller jusqu'\`a trois ans renouvelables), avant
une p\'eriode de d\'etachement ou de disponibilit\'e.

\subsubsection{Le d\'etachement}
Le d\'etachement aupr\`es d'un autre organisme est une position
m\'ediane entre l'organisme d'accueil et l'organisme d'origine. La ou le
chercheur$\cdot$se est r\'emun\'er\'e$\cdot$e par l'organisme d'accueil, mais continue \`a
b\'en\'eficier de ses droits \`a l'avancement et \`a la retraite \`a
Inria. Ce type de d\'etachement peut \^etre de courte dur\'ee (6
mois) ou de longue dur\'ee (jusqu'\`a cinq ans renouvelables).

\subsubsection{La disponibilit\'e}
La disponibilit\'e est une position par laquelle le$\cdot$la chercheur$\cdot$se
interrompt momentan\'ement sa carri\`ere \`a Inria. Cette interruption
peut \^etre motiv\'ee soit pour des motifs familiaux ou personnels, soit
pour r\'ealiser des \'etudes ou recherches d'int\'er\^et g\'en\'eral,
soit pour cr\'eer ou reprendre une entreprise valorisant les
r\'esultats de la recherche. Le$\cdot$La chercheur$\cdot$se ne per\c coit plus de
r\'emun\'eration de la part de  et les droits \`a l'avancement
et \`a la retraite sont suspendus. Toutefois, il$\cdot$elle reste soumis$\cdot$e \`a
certaines obligations vis-\`a-vis d'Inria, notamment en terme
d'autorisation de cumul d'emploi et de r\'emun\'eration.
