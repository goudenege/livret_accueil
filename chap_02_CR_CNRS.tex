

%%%%%%%%%%%%%%%%%%%%%%%%%%%%%%%%%%%%%%%%%%%%%%%%
%%%%%%%%%%%%%%%%%%%%%%%%%%%%%%%%%%%%%%%%%%%%%%%%%%%%%%%%%%%%%%%%%%%%%%%%%%%%%%%%%%%%%%%%%%%%%%%%%%%%%%
%%%%%%%%%%%%%%%%%%%%%%%%%%%%%%%%%%%%%%%%%%%%%%%%%%%%%%
\chapter{Le m\'etier de chercheur$\cdot$se au CNRS}

Le Centre National de la Recherche Scientifique (CNRS) emploie
 environ 400 chercheur$\cdot$ses au sein de l'Insmi (Institut National des Sciences
 Math\'ematiques et de leurs Interactions).
Pour comparaison,
les sections 25 et 26 du CNU comptent environ 3200 enseignant$\cdot$es-chercheur$\cdot$ses (EC).
Une tr\`es grande majorit\'e des chercheur$\cdot$ses en math\'{e}matiques
travaillant au CNRS sont dans des laboratoires de l'Insmi mais quelques math\'ematicien$\cdot$nes se trouvent \'egalement dans des laboratoires relevant d'autres instituts~: {\it Institut des sciences de l'information et de leurs interactions}, {\it Institut des sciences biologiques}, {\it Institut de Physique}, {\it Institut des sciences humaines et sociales}\ldots Quasiment tous les instituts du CNRS h\'ebergent au moins un$\cdot$e math\'ematicien$\cdot$ne !

Les chercheur$\cdot$ses sont \'electeur$\cdot$trices de l'universit\'e qui les h\'eberge.
Elles ou ils sont \'egalement \'eligibles, y compris \`a la fonction de pr\'esident$\cdot$e d'universit\'e.


\index{Labintel}

%%%%%%%%%%%%%%%%%%%%%%%%%%%%%%%%%%%%%%%
\section{Les missions}

Comme on l'apprend sur le site web du CNRS, les cinq missions statutaires de ses
chercheur$\cdot$ses  sont les suivantes~:
\begin{itemize}
\item la production de nouvelles connaissances scientifiques (articles dans des revues, livres, participation \`a des congr\`es) ;
\item l'application et  la valorisation des r\'esultats ;
\item la diffusion de l'information scientifique ;
\item la participation \`a la formation des doctorant$\cdot$es ;
\item l'expertise nationale et internationale de la recherche.
\end{itemize}

%%%%%%%%%%%%%%%%%%%%%%%%%%%%%%%%%%%%%%%%
\section{Le recrutement}
\label{sec. recrutCNRS}

\index{Charg\'e$\cdot$e de recherche (CR)! CNRS}
\index{Directeur$\cdot$trice de recherche (DR)! CNRS}

Les chercheur$\cdot$ses au CNRS, qu'ils$\cdot$elles soient charg\'e$\cdot$es de recherche (CR)
ou directeur$\cdot$trices de recherche (DR), sont des fonctionnaires. Leur
recrutement se fait par un concours pour lequel il faut faire acte
de candidature, entre d\'ecembre et janvier de
chaque ann\'ee, et qui se d\'eroule au printemps : \\
\lien{www.cnrs.fr/fr/travailler/concours.htm}

La s\'election des candidat$\cdot$es se d\'eroule  en deux phases : la phase d'admissibilit\'e et la phase d'admission. Les jurys d'admissibilit\'e sont compos\'es de chercheur$\cdot$ses, membres des sections et des commissions interdisciplinaires du Comit\'e National. La phase d'admissibilit\'e est la premi\`ere \'etape avant la phase d'admission. La phase d'admissibilit\'e est diff\'erente en fonction du corps postul\'e.
\begin{itemize}
\item    Pour les concours de charg\'e$\cdot$es de recherche, la phase d'admissibilit\'e comporte deux \'etapes : une pr\'es\'election des candidat$\cdot$es sur dossier par le jury d'admissibilit\'e comp\'etent, puis, pour les candidat$\cdot$es pr\'es\'electionn\'e$\cdot$es, une audition par le jury d'admissibilit\'e comp\'etent.
\item  Pour les concours de directeur$\cdot$trices de recherche, elle consiste en l'examen des dossiers et en une audition facultative, selon la d\'ecision de chaque section du Comit\'e National. 
  \end{itemize}
    \`A l'issue de cette phase, les candidats sont d\'eclar\'es ou non admissibles.
    La phase d'admission consiste en l'arr\^et de la liste des candidats d\'efinitivement admis sur la base de l'examen du dossier des candidat$\cdot$es admissibles.

Les nominations prennent effet au 1er octobre de l'ann\'ee du concours. Les candidat$\cdot$es peuvent toutefois demander un report de prise de fonctions qui ne peut exc\'eder le 1er f\'evrier de l'ann\'ee suivant le concours.


%%%%%%%%%%%%%%%%%%%%%%%%%%%%%%%%%%%%%%
\section{L'affectation}

\index{Affectation des chercheur$\cdot$ses CNRS}

L'affectation des chercheur$\cdot$ses entrant$\cdot$es est faite par l'Insmi ou par l'institut qui a mis le poste au concours s'il s'agit d'un poste pour une unit\'e d'une discipline autre que les math\'ematiques. 
L'affectation est pr\' epar\' ee par une phase de concertation entre la direction de l'Insmi, les chercheur$\cdot$ses, et les directeur$\cdot$trices d'unit\' e. Les propositions de la direction de l'Insmi tiennent compte de trois facteurs principaux : v\oe ux et projets individuels des candidat$\cdot$es, politiques des laboratoires, et mission d'animation de l'Insmi vis-\`a-vis du r\' eseau national des laboratoires, cette derni\` ere impliquant l'irrigation d'un grand nombre de laboratoires et, d'autre part, une mobilit\' e au recrutement CR et au passage CR-DR.


Cette affectation peut \^etre en accord avec l'un des v\oe ux exprim\' es par la ou le candidat$\cdot$e
dans son dossier de candidature, ou pas. Dans le dossier de candidature, il est demand\'e aux candidat$\cdot$es de pr\'esenter
leur projet de recherche en se r\'ef\'erant \`a quelques laboratoires dans lesquels leur activit\'e pourrait s'incrire. Plusieurs choix sont demand\'es dont obligatoirement un hors de la r\'egion parisienne.

Apr\`es le concours, les CR sont nomm\'e$\cdot$es en qualit\'e de stagiaire.  Ils et elles sont
titularis\'e$\cdot$es au bout d'un an,  apr\`es avis de la section du Comit\'e National et de la ou du
directeur$\cdot$trice de leur unit\'e, au vu d'un rapport d'activit\'e
\'etabli par le chercheur lui-m\^eme ou la chercheuse elle-m\^eme. Les DR sont titularis\'e$\cdot$es imm\'ediatement, sans stage.

Le directeur ou la directrice de l'unit\'e o\`u le$\cdot$la chercheur$\cdot$se est affect\'e$\cdot$e est son$\cdot$sa sup\'erieur$\cdot$e hi\'erarchique. Il ou elle aura \`a se prononcer
sur l'activit\'e du chercheur ou de la chercheuse au sein de l'unit\'e \`a chaque
\'etape de son parcours professionnel~: son avis sera tout d'abord
l'un des \'el\'ements du dossier de titularisation et il
interviendra ensuite lors des \'evaluations p\'eriodiques ({{\em
cf.}} \ref{sec. evalcnrs}).

Noter qu'en tant qu'instance comp\'etente pour le recrutement et l'\'evaluation,
la section du Comit\'e National du CNRS donne un avis informel sur l'affectation des nouveaux$\cdot$elles
recrut\'e$\cdot$es et d\' esigne un$\cdot$e Directeur$\cdot$trice des recherches pour chaque nouveau CR.

\index{Comit\'e National de la recherche scientifique}

%%%%%%%%%%%%%%%%%%%%%%%%%%%%%%%%%%
\section{L'\'evaluation des chercheur$\cdot$ses au CNRS}

Elle est assur\'ee par la section idoine du Comit\'e National du CNRS. Voir le chapitre \ref{sec. evalcnrs}.

\index{Evaluation ! des chercheur$\cdot$ses CNRS}


%%%%%%%%%%%%%%%%%%%%%%%%%%%%%%%%%%%%%%%%%
\section{Les carri\`eres et r\'emun\'erations}

Comme tous les fonctionnaires, les chercheur$\cdot$ses au CNRS
b\'en\'eficient d'avancement d'\'echelon \`a
l'anciennet\'e et d'avancement de grade.

Le corps des charg\'e$\cdot$es de recherche se divise en deux cat\'egories~: la Classe Normale et l'Hors Classe.

Le corps des directeur$\cdot$trices de recherche se divise en trois
cat\'egories~: la deuxi\`eme classe (DR2), la premi\`ere classe
(DR1) et la classe exceptionnelle, elle-m\^eme divis\'ee en
1\ier{} puis 2\ieme{} \'{e}chelon (DRCE1,
DRCE2).
%VRM 2010: info donn\'ee par F. Planchon dans le matapli 88 de 2009
Tous les passages de grade se font au choix, sur dossier
scientifique, comme le passage de CR \`a DR.

Il est \`a noter qu'il existe aussi des concours externes pour devenir CR1 ou DR1 ouverts \`a
tou$\cdot$tes les chercheur$\cdot$ses. Le concours DR2 est syst\'ematiquement ouvert \`a toutes et à tous. 

Les chercheur$\cdot$ses au CNRS b\'{e}n\'{e}ficient d'une prime d'environ
350\,\euro{}; en juin et d\'{e}cembre de chaque ann\'ee. Les grilles
des salaires des IR, CR et DR sont disponibles aux adresses suivantes. 

\lien{www.dgdr.cnrs.fr/drh/remuneration/grilles/ir.htm}

\lien{www.dgdr.cnrs.fr/drh/remuneration/grilles/cr.htm}

\lien{www.dgdr.cnrs.fr/drh/remuneration/grilles/dr.htm}


Nous rappelons qu'en tant que nouveau fonctionnaire, tout$\cdot$e nouveau$\cdot$elle chercheur$\cdot$se au CNRS
a le droit de pr\'esenter une demande de
reconstitution de carri\`ere~: elle permet de faire
reconna\^\i tre tout emploi comportant une activit\'e de
recherche pr\'ec\'edant l'embauche au CNRS \`a fin d'avancement
d'\'echelon \`a l'anciennet\'e.

\subsection{Quelques documents sur la carri\`ere des chercheur$\cdot$ses
au CNRS}

Nous avons r\'esum\'e dans ce livret quelques \'el\'ements
 concernant les missions des chercheur$\cdot$ses au CNRS, leur
carri\`ere et les diff\'erentes possibilit\'es d'activit\'e
que celle-ci permet d'envisager. Le CNRS \'edite d\'ej\`a de
tr\`es bons textes sur ces sujets~; nous y renvoyons le$\cdot$la lecteur$\cdot$trice
int\'eress\'e$\cdot$e~:
\begin{itemize}
\item le guide {\em Bienvenue au CNRS}~:\\
\lien{www.dgdr.cnrs.fr/drh/concours/guide/bienvenue.htm}
\item les fiches des m\'etiers du CNRS~:\\
\lien{www.dgdr.cnrs.fr/drh/omes/default.htm}
\end{itemize}

Nous signalons aussi que le CNRS \'edite chaque ann\'ee des
brochures d'int\'er\^et plus g\'en\'eral qui contien\-nent
des informations tr\`es int\'eressantes pour les chercheur$\cdot$ses,
notamment le {\em Bilan social du CNRS}, dans sa derni\`ere
\'edition :~\\
\lien{http://bilansocial.dsi.cnrs.fr/}


%Enfin, sur le site de la Mission pour la place des femmes au CNRS,
%on trouve des informations. \\
%\lien{www.cnrs.fr/mpdf/}\\
%
Enfin, sur le site de la Mission pour la place des femmes au CNRS,
on trouve des informations  qui m\'eritent le d\'etour. En particulier, un grand nombre de chiffres r\'ecents sur la proportion des chercheuses au CNRS
et l'indice de parit\'e~:\\
\lien{www.cnrs.fr/mpdf/}

Sans surprise (mais avec r\'evolte!), on y apprendra que les femmes sont peu repr\'esent\'ees
\`a l'Insmi (le bonnet d'\^ane du CNRS avec la physique et la m\'ecanique) et que la proportion de math\'ematiciennes
baisse depuis une vingtaine d'ann\'ees... On notera n\'eanmoins que la proportion de femmes DR est quasiment la m\^eme que celle des CR, ce qui n'est pas forc\'ement le cas d'autres instituts plus f\'eminis\'es. 


%%%%%%%%%%%%%%%%%%%%%%%%%%%%%%%%%%%%
\subsection{Une prime: la PEDR}
\label{PEDR CNRS}
\index{Prime d'encadrement doctoral et de recherche (PEDR)}

La prime d'encadrement doctoral et recherche (PEDR) s'inscrit dans le contexte du contrat CNRS-Etat et du Plan Carri\`eres
accompagnant la mise en place de la LRU. Elle correspond \`a l'extension \`a tou$\cdot$tes les chercheur$\cdot$ses depuis 2009
de la prime d'encadrement doctoral r\'eserv\'ee par le pass\'e aux seul$\cdot$es enseignant$\cdot$es-chercheur$\cdot$ses. Notons qu'un$\cdot$e CR nouvellement recrut\'e$\cdot$e peut demander la PEDR d\`es sa premi\`ere ann\'ee et a de fortes chances de l'obtenir.
 La PEDR est vers\'ee pour une dur\'ee de quatre ans.
Les crit\`eres et les modalit\'es de s\'election des b\'en\'eficiaires ont \'et\'e adopt\'es par le Conseil d'administration
du CNRS lors de sa s\'eance du 1er avril 2010.
\index{Prime d'encadement doctoral et recherche (PEDR)!crit\`eres d'attribution}
Ainsi, seront retenus pour la PEDR \`a partir de la campagne de candidature 2010
\begin{itemize}
\item les personnels laur\'eats d'une distinction scientifique de niveau national ou international figurant
dans l'arr\^et\'e en date du 20 janvier 2010 (1. Prix Nobel, 2. M\'edaille Fields, 3. Prix Crafoord, 4. Prix Turing,
5. Prix Albert Lasker, 6. Prix Wolf, 7. M\'edaille d'or du CNRS, 8. M\'edaille d'argent du CNRS,
9. Lauriers de l'INRA, 10. Grand Prix de l'INSERM, 11. Prix Balzan, 12. Prix Abel, 13. Les prix scientifiques
attribu\'es par l'Institut de France et ses acad\'emies, 14. Japan Prize, 15. Prix Gairdner, 16. Prix Claude L\'evi-Strauss) ;
\item les autres chercheur$\cdot$ses en fonction de quatre grands crit\`eres analogues \`a ceux retenus
 par les universitaires : production scientifique, rayonnement et diffusion scientifique, responsabilit\'es
collectives, encadrement doctoral et activit\'e d'enseignement (avec engagement d'enseigner 64 h ETD par an).
\end{itemize}

\index{Prime d'encadement doctoral et recherche (PEDR)!montants}
Trois niveaux de prime sont pr\'evus en fonction de la qualit\'e du dossier: 3500, 7000 et 10000 euros. En pratique, les primes sont souvent de 3500 euros, ce qui permet d'en distribuer plus. 
La campagne de candidature est ouverte \`a tou$\cdot$tes les chercheur$\cdot$ses, fonctionnaires (y compris les personnels
d\'etach\'es dans le corps des chercheur$\cdot$ses au CNRS) et fonctionnaires stagiaires en activit\'e et salari\'e$\cdot$es
 du CNRS au moment de la campagne. Chaque section du CN est libre
de l'organiser \`a sa convenance. Depuis 2017, la pr\'eselection des candidat$\cdot$es \`a la PEDR se fait par le CN \`a la session de printemps. Nous renvoyons au site\\
\lien{www.dgdr.cnrs.fr/drh/carriere/cherch/pedr.htm}\\
pour de plus amples informations concernant la PEDR.



%%%%%%%%%%%%%%%%%%%%%%%%%%%%%%%%%%%%
\subsection{Cumul d'activit\'es}
\label{sec. cumul CNRS}
\index{Cumul d'activit\'e ! au CNRS}

Les chercheur$\cdot$ses CNRS peuvent exercer une activit\'e accessoire (telle
que de l'enseignement, de l'expertise ou du conseil) \`a c\^ot\'e de
leur activit\'e principale de recherche, dans des conditions qui sont celles de tous les fonctionnaires (\`a savoir que cette activit\'e doit
\^etre compatible avec leurs fonctions, et ne doit pas nuire \`a
l'exercice de leurs missions, {\em cf.} par exemple~\ref{sec. cumul INRIA}).

Sur ce point, il peut \^etre int\'eressant de relever la coexistence de deux forces oppos\'ees :
\begin{itemize}
\item dans la fonction publique, la ``r\`egle" est l'interdiction du cumul mais des d\'erogations
sont possibles sur demande ;
\item tou$\cdot$tes les chercheur$\cdot$ses sont encourag\'e$\cdot$es (par le Comit\'e National, notamment) \`a exercer
 une activit\'e p\'edago\-gique, de quelque nature qu'elle soit... Encouragement maintenant
clairement \'enonc\'e comme une condition d'attribution de la PEDR.
\end{itemize}


\section{La mobilit\'e (sp\'ecifique aux chercheur$\cdot$ses au CNRS)}

\index{Mobilit\'e ! des chercheur$\cdot$ses CNRS}

La mobilit\'e des chercheur$\cdot$ses au CNRS est fortement encourag\'ee.
Cette mobilit\'e peut \^etre g\'eo\-gra\-phi\-que et/ou
th\'ematique. Nous tentons maintenant de recenser les
diff\'erents cadres dans lesquels cette mobilit\'e s'exerce.

Dans tous les cas, la chercheuse ou le chercheur doit adresser sa demande \`a son institut de rattachement au CNRS ; la d\'ecision est prise par le$\cdot$la directeur$\cdot$trice d'institut. En  cas de demande de mobilit\'e vers un laboratoire d\'ependant d'un autre institut que l'Insmi, la d\'ecision sera prise conjointement par les deux directeur$\cdot$trices d'institut. 

\subsection{Le changement d'affectation}

Le$\cdot$La chercheur$\cdot$se  souhaitant changer d'unit\'e d'affectation contacte la direction de l'Insmi qui le recevra. Lorsque la ou le chercheur$\cdot$se et la direction de l'Insmi sont d'accord sur le projet, celui-ci se valide administrativement par l'envoi \`a l'Insmi 
avec copie \`a la d\'el\'egation r\'egionale, d'un dossier (\'electronique)
comprenant une description (environ une page) du projet scientifique
justifiant la demande de changement, un avis du directeur ou de la directrice
de l'unit\'e d'origine, et un avis du ou de la directeur$\cdot$trice de la nouvelle
unit\'e envisag\'ee.


\subsection{Le stage}

Un$\cdot$e chercheur$\cdot$se au CNRS peut demander \`a passer une p\'eriode
temporaire dans une autre unit\'e que la sienne~: ceci s'appelle
un stage. La demande de stage se compose des m\^emes pi\`eces
que la demande de changement d'affectation et suit le m\^eme
parcours. Bien que le stage soit une bonne fa\c con de prendre
contact avec une unit\'e en vue d'une affectation, il peut pr\'esenter
un certain nombre d'inconv\'enients.

\begin{itemize}
\item Le ou la chercheur$\cdot$se en stage n'a pas droit au remboursement de
frais de transport (comme le Pass Navigo \`a Paris) ou aux
indemnit\'es d'habitation, m\^eme s'il ou elle effectue le stage dans
une ville y donnant droit. Inversement, un$\cdot$e chercheur$\cdot$se affect\'e$\cdot$e
dans une ville donnant droit \`a ces compensations les perd
automatiquement au moment du stage~: ainsi un$\cdot$e chercheur$\cdot$se de l'\^Ile
de France effectuant un stage de six mois en province perd son
indemnit\'e d'habitation (m\^eme s'il ou elle garde son logement).
\item Dans le cas o\`u un stage qui a dur\'e plus de neuf mois
aboutit \`a un changement d'affectation, le$\cdot$la chercheur$\cdot$se perd le
droit au remboursement des frais de d\'em\'enagement. En effet, le
CNRS consid\`erera que le$\cdot$la chercheur$\cdot$se a chang\'e de r\'esidence au
d\'ebut du stage. Au moment du changement d'affectation, le fait
d'avoir chang\'e de r\'esidence depuis plus de neuf mois fait
perdre le droit aux indemnit\'es de d\'em\'enagement.
\item Il vaut mieux \'egalement s'assurer aupr\`es des directeur$\cdot$trices d'unit\'e sur la mani\`ere dont on sera financ\'es lors de la p\'eriode de stage afin de ne pas se retrouver dans  
%\item Les unit\'es sont en grande partie financ\'ees en fonction
%du nombre de leurs membres actifs~: dans le cadre d'un stage, c'est
%l'unit\'e d'affectation qui per\c coit le financement relatif au
%chercheur et non l'unit\'e dans laquelle le chercheur effectue son
%stage. On peut devoir affronter 
la situation d\'esagr\'eable
o\`u aucune des deux unit\'es ne se sent concern\'ee
par les besoins de la chercheuse ou du chercheur, notamment en terme de financement de
missions, d'invitations ou d'achat de mat\'eriel.
\end{itemize}

\subsection{D\'etachement, disponibilit\'e et temps partiel}

Un$\cdot$e chercheur$\cdot$se au CNRS peut b\'en\'eficier d'un d\'etachement ou
d'une disponibilit\'e, la diff\'erence entre les deux situations
\'etant la m\^eme que pour tous les fonctionnaires ({\em cf.}  par exemple~\ref{sec. mobex INRIA}),
ou bien d'un temps partiel. Les deux sont accord\'es par d\'ecision du ou de la pr\'esident$\cdot$e du CNRS
 apr\`es avis de la direction de l'Institut auquel le chercheur ou la chercheuse est rattach\'e$\cdot$e.

\subsection{L'\'echange de postes} \label{echgecnrs}

%VRM 2010: mise a jour par P. Dehormoy, sous-direction de l'Insmi.
Un$\cdot$e chercheur$\cdot$se au CNRS qui souhaite avoir une activit\'{e}
p\'{e}dagogique, m\^{e}me \`{a} temps partiel, peut demander un
\'{e}change de postes. Il s'agit d'un accord entre le CNRS et une
universit\'{e} pour qu'un$\cdot$e chercheur$\cdot$se au CNRS et un
enseignant$\cdot$e-chercheur$\cdot$se \'{e}changent leurs fonctions. En substance,
l'enseignant$\cdot$e-chercheur$\cdot$se b\'{e}n\'{e}ficiera d'un semestre, ou plus,
de d\'{e}l\'{e}gation.  Ce type d'\'echange doit \^etre organis\'e par l'assembl\'ee
consultative de section ou l'instance qui en tient lieu
(ex-commission de sp\'ecialiste), et \^etre valid\'e par la direction de
l'Insmi et l'universit\'e concern\'ee. Formellement, une convention
d'\'echange est sign\'ee entre l'UFR et la d\'el\'egation r\'egionale du
CNRS concern\'ees.


\section{Le financement des projets de recherche et autres
opportunit\'es}

Les chercheur$\cdot$ses au CNRS peuvent participer aux projets de recherche
divers (ANR, PHC, {\em etc.}, {\em cf.} \ref{financement-projets})
au m\^eme titre que les enseignant$\cdot$es-chercheur$\cdot$ses.
Ils et elles ont de plus une opportunit\'e qui n'est probablement pas
assez exploit\'ee~: dans le cadre de la formation continue/formation permanente, ils$\cdot$elles
peuvent demander au CNRS de financer des projets non directement
li\'es \`a l'activit\'e scientifique {\em stricto sensu},
comme l'apprentissage d'une langue, une formation en informatique ou
 la participation \`a une \'ecole th\'ematique
{\em etc.}

Elles et ils peuvent aussi r\'epondre aux appels d'offre CNRS de type Peps, en particulier dans le cadre des appels d'offre de la Mission pour l'Interdisciplinarit\'e du CNRS ainsi qu'aux programmes Peps propos\'es par les instituts, que ce soit l'Insmi ou en coop\'eration avec d'autres instituts, ou encore dans le cadre des {\it Actions de site} que le CNRS met en place avec ses partenaires r\'egionaux (cf.~\ref{sec:insmi}). 


