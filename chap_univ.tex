%%%%%%%%%%%%%%%%%%%%%%%%%%%%%%%%%%%%%%%%%%%%
%%%%%%%%%%%%%%%%%%%%%%%%%%%%%%%%%%%%%%%%%%%%
\chapter{Les universit\'es}
\label{universite}
\index{Universit\'e}
Les universit\'es participent, en tant qu'\'etablissements d'enseignement sup\'erieur et de recherche,
au service public de l'enseignement sup\'erieur, dont les missions sont ainsi d\'efinies par la 
 \href{https://www.legifrance.gouv.fr/affichCodeArticle.do?cidTexte=LEGITEXT000006071191&idArticle=LEGIARTI000027747739&dateTexte=20170113}{loi n$ \textdegree$2013-660 du 22 juillet 2013 - art. 7 :}
\begin{enumerate}
 \item la formation initiale et continue,
 \item la recherche scientifique et technologique, la diffusion et la valorisation de ses r\'esultats,
 \item l'orientation, la promotion sociale et l'insertion professionnelle,
 \item la diffusion de la culture humaniste, scientifique, technique et industrielle,
 \item  la coop\'eration internationale.
\end{enumerate}

\quad

Les \'etablissements d'enseignement sup\'erieur et de recherche sont regroup\'es sous l'appellation Etablissements publics \`a
caract\`ere scientifique, culturel et professionnel (EPCSCP), et sont constitu\'es de~:
\begin{itemize}
\item 71 universit\'es,
\item 1 institut national polytechnique,
\item 19 instituts et \'ecoles ext\'erieurs aux universit\'es (INSA, \'Ecoles Centrales, Universit\'es Technologiques, {\em etc.}) ;
\item 20 grands \'etablissements de statuts divers~;
\item 4 \'ecoles normales sup\'erieures,
\item 5 \'ecoles fran\c caises \`a l'\'etranger~;
\item 21 communaut\'es d'universit\'es et \'etablissements  (COMUE ou ComUE)
\end{itemize}
\index{\'etablissements publics \`a caract\`ere scientifique, culturel et professionnel (EPCSCP)}
\index{communaut\'es d'universit\'es et \'etablissements (COMUE)}
Plus d'information sur le \href{http://www.enseignementsup-recherche.gouv.fr/cid20263/index.html}{site du minist\`ere}.\\

Nous utiliserons pour tous ces \'etablissement le mot \textit{universit\'es} pour plus de commodit\'e,
m\^eme s'il faut garder \`a l'esprit qu'il subsiste certaines diff\'erences au sein des EPSCP entre ceux
qui sont des universit\'es et les autres : un$\cdot$e pr\'esident$\cdot$e d'universit\'e est un$\cdot$e directeur$\cdot$trice pour d'autres
\'etablissements (ils ou elles ne sont pas nomm\'e$\cdot$es de la m\^eme mani\`ere, mais leurs pr\'erogatives sont
tr\`es proches), les noms et r\^oles des conseils peuvent diff\'erer, {\em etc.}\\

L'organisation et le fonctionnement des universit\'es sont r\'egis
par le \href{http://www.legifrance.gouv.fr/affichCode.do?cidTexte=LEGITEXT000006071191}{Code de l'\'education}. \\

Nous allons pr\'esenter maintenant bri\`evement les diff\'erentes instances de l'universit\'e.


%%%%%%%%%%%%%%%%%%%%%%%%%%%%%%%%%%%%%%%

\section{La Pr\'esidence}
Le ou la pr\'esident$\cdot$e de l'universit\'e est \'elu$\cdot$e \`a la majorit\'e absolue par les membres \'elu$\cdot$es du Conseil d'administration, 
pour un mandat de quatre ans (renouvelable une fois). 

Pour les d\'etails on peut se r\'ef\'erer au code de l'\'education :
\begin{enumerate}
 \item \href{https://www.legifrance.gouv.fr/affichCodeArticle.do;jsessionid=C8BFF801F2976E9272297AB33338C553.tpdila14v_3?idArticle=LEGIARTI000027747943&cidTexte=LEGITEXT000006071191&dateTexte=20170113}{Article L712-1, modifi\'e par Loi n$\textdegree$2013-660 du 22 juillet 2013 - art. 45}
 \item \href{https://www.legifrance.gouv.fr/affichCodeArticle.do;jsessionid=C8BFF801F2976E9272297AB33338C553.tpdila14v_3?idArticle=LEGIARTI000027747947&cidTexte=LEGITEXT000006071191&dateTexte=20170113}{Article L712-2, modifi\'e par Loi n$\textdegree$2013-660 du 22 juillet 2013 - art. 46}
\end{enumerate}

Au niveau national, les pr\'esident$\cdot$es d'universit\'e sont regroup\'e$\cdot$es en Conf\'erence des pr\'esidents d'universit\'e (CPU), cf. chapitre \ref{chapMinistere}.

\section{Conseil centraux}
Depuis la \href{https://www.legifrance.gouv.fr/affichTexte.do;jsessionid=C8BFF801F2976E9272297AB33338C553.tpdila14v_3?cidTexte=JORFTEXT000027735009&dateTexte=20170113}{loi du 22 juillet 2013} relative {\`a} l'enseignement sup{\'e}rieur et {\`a} la recherche, 
deux conseils contribuent \`a la gouvernance des universit\'es~: le Conseil d'administration et le Conseil acad{\'e}mique. Ces conseils sont consult\'es et votent sur 
l'orientation politique de l'universit{\'e}. Ils sont constitu\'es de repr{\'e}sentants des enseignant$\cdot$es, des chercheur$\cdot$ses, des personnels administratifs et techniques, 
des {\'e}tudiant$\cdot$es et de personnalit{\'e}s ext{\'e}rieures. 


\subsection{Conseil d'Administration (CA)}
Le r\^ole du Conseil d'administration est de d\'elib\'erer et de voter les d\'ecisions relevant de la politique de l'\'etablissement. Il doit ainsi se prononcer sur
le contrat d'\'etablissement, les accords et les conventions. Il lui revient de voter le budget et la r\'epartition des subventions et des emplois. 

Le nombre de ses membres est de l'ordre d'une trentaine. Les repr\'esentants des enseignant$\cdot$es-chercheur$\cdot$ses et enseignant$\cdot$es  sont pour moiti\'e des professeur$\cdot$es des universit\'es ou assimil\'es et pour moiti\'e des personnels d'autre statut (ma\^itres de conf\'erence, PRAG). Il comprend des personnalit\'es ext\'erieures, des repr\'esentant$\cdot$es des usagers (\'etudiant$\cdot$es et personnes b\'en\'eficiant de la formation continue inscrits dans l'\'etablissement),
ainsi que des repr\'esentant$\cdot$es des personnels IATOS (ing\'enieur$\cdot$es, administratifs, techniques et des biblioth\'eques), en exercice dans l'\'etablissement. 

\href{https://www.legifrance.gouv.fr/affichCodeArticle.do;jsessionid=C8BFF801F2976E9272297AB33338C553.tpdila14v_3?idArticle=LEGIARTI000027747951&cidTexte=LEGITEXT000006071191&dateTexte=20170113}{Article L712-3 modifi\'e par Loi n$\textdegree$2013-660 du 22 juillet 2013 - art. 47}

\subsection{Conseil acad{\'e}mique}
Lorsqu'elles se r\'eunissent ensemble, la Commission de la formation et de la vie universitaire 
(CFVU, ancien CEVU, Conseil des {\'e}tudes et de la vie universitaire) et la Commission de la recherche (ancien Conseil scientifique) constituent le Conseil acad\'emique.

Cet organisme est consult\'e par l'\'equipe de direction de l'universit\'e, pour se prononcer sur les orientations des politiques de formation, de recherche, ou sur tout autre sujet touchant la vie universitaire. Lorsqu'elles ont un impact budg\'etaire, les d{\'e}cisions du Conseil acad{\'e}mique doivent \^etre valid\'ees par le Conseil d'administration. 

\href{https://www.legifrance.gouv.fr/affichCodeArticle.do;jsessionid=C8BFF801F2976E9272297AB33338C553.tpdila14v_3?idArticle=LEGIARTI000027747976&cidTexte=LEGITEXT000006071191&dateTexte=20170113}{Article L712-4 modifi\'e par Loi n$\textdegree$2013-660 du 22 juillet 2013 - art. 49}


\subsubsection*{Commission de la formation et de la vie universitaire}
Tout ce qui touche aux formations d\'elivr\'ees par l'universit\'e est du ressort de la Commission de la Formation et de la Vie Universitaire (CFVU). La commission se prononce en particulier sur les programmes de formation des diff\'erentes composantes de l'universit\'e, mais aussi sur la r{\'e}partition des moyens attribu\'es \`a la formation au sein de l'enveloppe vot\'ee par le Conseil d'administration. Il lui revient aussi de soutenir et d\'evelopper les activit{\'e}s culturelles, sportives, sociales, associatives et veiller \`a la qualit\'e des conditions de vie et de travail des {\'e}tudiant$\cdot$es. Elle est garante des libert{\'e}s universitaires et des libert{\'e}s syndicales et politiques des {\'e}tudiant$\cdot$es. 

\href{https://www.legifrance.gouv.fr/affichCodeArticle.do;jsessionid=C8BFF801F2976E9272297AB33338C553.tpdila14v_3?idArticle=LEGIARTI000027747967&cidTexte=LEGITEXT000006071191&dateTexte=20170113}{Article L712-6 modifi{\'e} par Loi n$\textdegree$2013-660 du 22 juillet 2013 - art. 49}

\href{https://www.legifrance.gouv.fr/affichCodeArticle.do;jsessionid=C8BFF801F2976E9272297AB33338C553.tpdila14v_3?idArticle=LEGIARTI000027747991&cidTexte=LEGITEXT000006071191&dateTexte=20170113}{Article L712-6-1 modifi{\'e} par Loi n$\textdegree$2013-660 du 22 juillet 2013 - art. 50}

\subsubsection*{Commission de la recherche (CR)}
La r\'epartition des moyens destin{\'e}s \`a la recherche tels qu'allou{\'e}s par le Conseil d'administration est d\'ecid\'ee par la Commission de la Recherche (CR), qui d\'ecide \'egalement des r{\`e}gles de fonctionnement des laboratoires, et qui donne un avis consultatif sur les conventions avec les organismes de recherche. La CR a plus g\'en\'eralement la responsabilit\'e de favoriser le d\'eveloppement et la diffusion de la culture scientifique, technique et industrielle.

\href{https://www.legifrance.gouv.fr/affichCodeArticle.do;jsessionid=C8BFF801F2976E9272297AB33338C553.tpdila14v_3?idArticle=LEGIARTI000027747971&cidTexte=LEGITEXT000006071191&dateTexte=20170113}{Article L712-5   modifi{\'e} par Loi n$\textdegree$22013-660 du 22 juillet 2013 - art. 49}

\section{Agence comptable}
L'agent comptable, nomm\'e par deux minist\`eres \`a la fois (Education Nationale et Budget), a la responsabilit\'e de la comptabilit\'e de l'universit\'e dont il doit se porter garant. Il \'etablit le compte financier, et contr\^ole la gestion budg\'etaire (d{\'e}cret du 7 novembre 2012 relatif {\`a} la gestion budg{\'e}taire et comptable publique).


\section{Composantes}
Les composantes d'une universit\'e peuvent \^etre :
\begin{enumerate}
 \item des unit\'es de formation et de recherche (UFR), des d\'epartements, laboratoires et centres de recherche,
 \item des \'ecoles ou des instituts,
 \item des regroupements de composantes.
\end{enumerate}

Pour plus d'informations: 
\href{https://www.legifrance.gouv.fr/affichCode.do;jsessionid=6C65A5741617DEF72E5BAC5710E2EE64.tpdila20v_1?idSectionTA=LEGISCTA000006166682&cidTexte=LEGITEXT000006071191&dateTexte=20170114}{Article L713-1, modifi\'e par ORDONNANCE n$\textdegree$2014-807 du 17 juillet 2014 - art. 3} 

Ces composantes sont libres de fixer leur statut (qui doivent \^etre approuv\'es par le Conseil d'administration) et leur budget. 
Concernant les regroupements de composantes, les universit\'es fusionn\'ees sont organis\'ees en coll\`eges qui regroupent plusieurs UFR et instituts. 

A noter que les \'ecoles et les instituts, tels que les instituts universitaires de technologie (IUT), 
disposent de pr\'erogatives qui leur sont propres. Pour plus de d\'etails, un lecteur avis\'e pourra consulter :
\index{Institut universitaire de technologie (IUT)}
% Section 3 : Les instituts et les \'ecoles.
\href{https://www.legifrance.gouv.fr/affichCode.do;jsessionid=6C65A5741617DEF72E5BAC5710E2EE64.tpdila20v_1?idSectionTA=LEGISCTA000006182446&cidTexte=LEGITEXT000006071191&dateTexte=20170114}{Article L713-9 modifi\'e par Loi n$\textdegree$2005-380 du 23 avril 2005 - art. 44 JORF 24 avril 2005}

\subsection{Unit\'es de formation et de recherche (UFR)}
\index{Unit\'e de formation et de recherche (UFR)}
Les unit\'es de formation et de recherche (UFR) associent des d\'epartements de formation (par exemple, d\'epartement des trois ann\'ees de Licence) et des laboratoires ou centres de recherche. Elles correspondent \`a un projet \'educatif et \`a un programme de recherche mis en \oe uvre par des enseignant$\cdot$es-chercheur$\cdot$ses, des enseignant$\cdot$es et des chercheur$\cdot$ses relevant d'une ou de plusieurs disciplines fondamentales. Les UFR sont administr\'ees par un conseil \'elu et dirig\'ees par un$\cdot$e directeur$\cdot$trice \'elu$\cdot$e par ce conseil.

Pour de plus amples informations: 
\href{https://www.legifrance.gouv.fr/affichCode.do;jsessionid=6C65A5741617DEF72E5BAC5710E2EE64.tpdila20v_1?idSectionTA=LEGISCTA000006182444&cidTexte=LEGITEXT000006071191&dateTexte=20170114}{Article L713-3 modifi\'e par Loi n$\textdegree$2003-339 du 14 avril 2003 - art. 2 JORF 15 avril 2003}

\subsection{Laboratoire de recherche}
\index{Unit\'e de de recherche}
\index{Laboratoire de recherche}
\index{Unit\'e mixte de de recherche (UMR)}
Les structures permettant aux chercheur$\cdot$ses d'effectuer leur travail, en leur fournissant en particulier les moyens financiers, informatiques et administratifs) sont les unit\'es de recherche. Celles-ci peuvent \^etre des laboratoires relevant d'une ou de plusieurs universit\'es et d'organismes de recherche scientifique (CNRS, INRIA,\ldots), comportant des \'equipes ayant la responsabilit\'e de la vie scientifique (s\'eminaires, groupes de travail,\ldots). Ils comprennent donc des personnels de diff\'erents statuts et appartenances. Le statut de ces laboratoires d\'epend des organismes dont ils rel\`event (par exemple, unit\'e mixte de recherche ou UMR lorsqu'il y a un contrat d'association entre laboratoires, universit\'es et organismes de recherche). La direction du laboratoire est r\'egie par les statuts de ce laboratoire, avec la possibilit\'e d'un conseil de laboratoire qui est charg\'e de d\'efinir la strat\'egie de recherche.

La gestion des moyens financiers de la recherche, ne concerne g\'en\'eralement pas les salaires des personnels (qui sont \`a la charge des organismes employeurs). La question du financement de la recherche fait l'objet du chapitre \ref{financement-projets}. \\

\section{Regroupements}
\label{Regroupements}
\subsection{Communaut\'es, associations et fusions}
\index{communaut\'es d'universit\'es et \'etablissements (COMUE)}
Suite \`a la loi n$\circ$2013-660 du 22 juillet 2013 relative \`a  l'enseignement sup\'erieur et \`a la recherche,
tous les \'etablissements publics d'enseignement sup\'erieur sous tutelle du MESRI(Chapitre \ref{chapMinistere}) sont amen\'es \`a  se regrouper et \`a se coordonner \`a l'\'echelle de leur territoire, que ces regroupements aient statut de COMUE (pour communaut\'es d'universit\'es et \'etablissements), d'universit\'e fusionn\'ee ou d'association \`a  un EPSCP.
Ce dispositif succ\`ede aux P\^oles de recherche et d'enseignement sup\'erieur (PRES). 

Pour le lecteur ou la lectrice avide de plus amples informations, voici le lien minist\'eriel  \`a ce sujet, 
sous forme de FAQ et contenant \'egalement  une liste des \'etablissements membres des COMUE et des associations : \\
\lien{www.enseignementsup-recherche.gouv.fr/cid94756/index.html}

Et pour compl\'eter, des liens moins officiels : \\
\lien{www.campusfrance.org/fr/page/les-universites-et-les-comue} \\
\lien{www.sauvonsluniversite.com/spip.php?article6553}

Au moment de la mise \`a jour de ce livret, les projets de regroupements font partie de l'actualit\'e de la communaut\'e.

\subsection{Fondation de coop\'eration scientifique (FCS)}

Les Fondations de coop\'eration scientifique (FCS) succ\'edent aux R\'eseaux th\'ematiques de recherche avanc\'ee (RTRA), anciennes structures supprim\'ees par la loi relative \`a l'enseignement sup\'erieur et \`a la recherche de 2013.
\index{R\'eseau th\'ematique de recherche avanc\'ee (RTRA)}
\index{Fondation de coop\'eration scientifique (FCS)}

Leur noble objectif : rassembler, sur un th\`eme donn\'e, une masse critique de chercheur$\cdot$ses de tr\`es haut niveau, autour d'un noyau dur d'unit\'es de recherche g\'eographiquement proches, afin d'\^etre comp\'etitif avec les meilleurs centres de recherche au niveau mondial.

\href{https://www.legifrance.gouv.fr/affichCode.do?idSectionTA=LEGISCTA000027748290&cidTexte=LEGITEXT000006071190}{Article L344-11 modifi\'e par Loi n$\textdegree$2013-660 du 22 juillet 2013 - art. 66}

\subsubsection{La Fondation Sciences Math\'ematiques de Paris (FSMP)}
\lien{www.sciencesmaths-paris.fr/} \\

La Fondation Sciences Math\'ematiques de Paris (FSMP) a \'et\'e cr\'e\'ee en 2006, sur le feu statut de RTRA. 
Depuis 2011, la FSMP est \'egalement porteuse du LabEx Sciences Math\'ematiques de Paris. 

Les \'equipes participantes, rattach\'ees au CNRS, \`a INRIA, \`a l'ENS Paris,
au Coll\`ege de France et aux universit\'es Paris Descartes (Paris 5), Pierre et Marie Curie (Paris 6), Paris Panth\'eon-Sorbonne (P1), Paris Diderot (Paris 7), Paris-Dauphine (Paris 9) et Paris Nord (P13).

Les moyens mis en \oe uvre sont notamment des bourses, des chaires, des positions post-doctorales, des invitations de chercheur$\cdot$se$\cdot$s...

\subsubsection{La Fondation Math\'ematique Jacques Hadamard}
\lien{www.fondation-hadamard.fr} \\ \\
Cr\'e\'ee en 2011, la Fondation Math\'ematique Jacques Hadamard est h\'eberg\'ee par la FCS Campus Paris Saclay, charg\'ee de porter l'Op\'eration du m\^eme nom, cf. \ref{trucenex}.
Ses membres fondateurs sont l'Universit\'e Paris-Sud, l'Ecole Polytechnique, l'ENS Cachan, l'IHES et le CNRS. Elle porte le projet LabEx Math\'ematiques Hadamard. Ses objectifs
et moyens sont similaires \`a ceux de la Fondation Sciences Math\'ematiques de Paris.

\quad

\section*{Quelques liens :}
\begin{itemize}
\item \href{https://www.legifrance.gouv.fr/affichCode.do;jsessionid=C8BFF801F2976E9272297AB33338C553.tpdila14v_3?cidTexte=LEGITEXT000006071191&dateTexte=20170113}{Code de l'\'education} 
\item Le site de la Maison des universit\'es : \lien{www.amue.fr/}
\item Strat\'egie nationale dans l'enseignement sup\'erieur : \\
\lien{www.enseignementsup-recherche.gouv.fr/pid25092/strategie.html}
\end{itemize}
