%%%%%%%%%%%%%%%%%%%%%%%%%%%%%%%%%%%%%%%%%%%%%%
%%%%%%%%%%%%%%%%%%%%%%%%%%%%%%%%%%%%%%%%%%%%%%
\chapter{La communication}

%%%%%%%%%%%%%%%%%%%%%%%%%%%%%%%%%%%%%%%%%%%%%%
\section{Vulgarisation}

Vous vous demandez peut-\^etre pourquoi faire de la vulgarisation
\footnote{Nous utilisons ce terme de vulgarisation pour d\'ecrire l'activit\'e d'explication de travaux
 savants \`a des publics non sp\'ecialis\'es. On comprend bien que la notion de public non-sp\'ecialis\'e est relative :
 parler de ses travaux \`a une commission d'audition 25-26 n'est pas la m\^eme chose que dans un colloque sp\'ecialis\'e ;
 parler de son domaine de recherche \`a un public scientifiquement averti (ing\'enieur$\cdot$es, lecteur$\cdot$trice de la
{\it Recherche} ou de {\it Pour la science}...), \`a des enseignant$\cdot$es de math\'ematiques, \`a des
coll\'egien$\cdot$nes ou lyc\'een$\cdot$nes appellent des p\'edagogies diff\'erentes...  D'autres termes sont utilis\'es
au lieu de vulgarisation, par exemple : diss\'emination scientifique, communication scientifique.} scientifique,
alors que vos recherches, l'enseignement (et sans doute bient\^ot l'administration ?) prennent d\'ej\`a tout votre temps.
Les math\'ematicien$\cdot$nes ne sont pas habitu\'e$\cdot$es \`a expliquer leurs travaux au public.
Pourtant, que vous soyez enseignant$\cdot$e-chercheur$\cdot$se ou chercheur$\cdot$se, la diffusion de la culture et l'information
scientifique et technique fait partie de vos missions, et ce texte cherche \`a vous expliquer pourquoi c'est
 important\footnote{Il faut \^etre tout \`a fait clair : de telles
activit\'es ne sont pas correctement prises en compte dans les carri\`eres, ni au niveau national par le CNU
ou le comit\'e national du CNRS, ni au niveau local par les universit\'es.}.
De plus, vous vous apercevrez en tentant l'exp\'erience que communiquer son savoir et sa passion, et
par l\`a changer l'image que la soci\'et\'e a des math\'ematiques et des math\'ematicien$\cdot$nes, est aussi une activit\'e gratifiante.

Voici quelques fa\c cons de r\'eduire le manque de communication entre les math\'ematicien$\cdot$nes et le public.
Choisissez l'activit\'e qui vous convient selon vos pr\'ef\'erences, vos aptitudes et surtout votre disponibilit\'e.
Profitez des initiatives existantes !

\medskip\par
{\bf F\^ete de la Science} De plus en plus de laboratoires de math\'ematiques y participent en proposant
des conf\'erences, des ateliers, ou en animant un stand
\footnote{Voir \url{http://smf.emath.fr/content/fete-de-la-science-2016} pour quelques actions propos\'ees en 2016.}.
La plupart proposent des manipulations ludiques et des \'enigmes qui ne demandent aucune connaissance particuli\`ere.
Vous seriez surpris$\cdot$e de l'entrain suscit\'e par ce type d'activit\'es et du succ\`es qu'elles remportent aupr\`es du public.
Parlez-en \`a vos coll\`egues des autres universit\'es pour trouver des id\'ees de manipulations simples \`a mettre en place.

Tous les ans a lieu \`a Paris (en mai) le salon des jeux et de la culture math\'ematique, organis\'e par le CIJM\footnote{Comit\'e International des Jeux Math\'ematiques \url{http://www.cijm.org/}}, auquel vous pouvez aussi participer.

\medskip\par
{\bf Images des math\'ematiques\footnote{\url{http://images.math.cnrs.fr/}}}.
Ce site pr\'esente la recherche contemporaine et le m\'etier de math\'ematicien$\cdot$ne \`a l'ext\'erieur de la communaut\'e
scientifique, afin de rapprocher les chercheur$\cdot$ses en math\'ematiques et le public.
Tous les articles sont \'ecrits par des chercheur$\cdot$ses. Vos contributions seront donc les bienvenues.
Il est possible d'\'ecrire des articles \`a diff\'erents niveaux, mais l'id\'ee est toujours de parler de maths \`a des gens
qui n'en connaissent pas ou presque pas et d'essayer de montrer ce que fait un$\cdot$e math\'ematicien$\cdot$ne aujourd'hui.

\medskip\par
\textbf{Interstices\footnote{\url{http://interstices.info}}}
Ce site de culture scientifique a lui aussi \'et\'e cr\'e\'e par des chercheur$\cdot$ses, pour rendre accessibles \`a un large public
 les sciences et technologies de l'information et de la communication.

\medskip\par
\textbf{Culture math}\footnote{\url{http://www.math.ens.fr/culturemath/}}. Ce site s'adresse aux professeur$\cdot$es
de math\'ematiques du secondaire ; financ\'e par la direction g\'en\'erale de l'enseignement scolaire du minist\`ere de
 l'\'education nationale, il propose des documents permettant aux professeur$\cdot$es d'enrichir les contenus de leurs cours.
Culture Math accepte volontiers des textes de chercheur$\cdot$ses.

\medskip\par
\textbf{MADD Maths}\footnote{\url{http://maddmaths.smai.math.cnrs.fr/}}  MADD Maths est l'acronyme de Math\'ematiques Appliqu\'ees Divulgu\'ees et Didactiques, est une initiative de la SMAI en direction du grand public et notamment des lyc\'een$\cdot$nes.
L'objectif de MADD Maths est de montrer que les math\'ematiques constituent un domaine tr\`es dynamique o\`u il y a encore beaucoup de choses \`a d\'ecouvrir, qui est tr\`es utile, avec des applications parfois inattendues ou amusantes, et de donner envie de vouloir en savoir plus. Puisque les maths peuvent sembler quelquefois compliqu\'ees, le but du site web est de les rendre accessibles.
C'est aussi l'occasion de d\'ecouvrir de nouvelles facettes des maths que l'on n'a pas l'occasion d'apercevoir au lyc\'ee ou au coll\`ege.
Au menu, interviews de math\'ematicien$\cdot$nes passionnant$\cdot$es, math\'ematicien$\cdot$nes inattendu$\cdot$es, rubrique "culture maths", courts articles de vulgarisation de la recherche en maths.

\medskip
{\bf Presse}. Plusieurs magazines ou revues de vulgarisation scientifique publient des articles r\'edig\'es par des chercheur$\cdot$ses.
Il peut s'agir de publications grand public ou de revues destin\'ees \`a un lectorat plus restreint, mais non sp\'ecialis\'e.
Citons par exemple les magazines
\textit{La Recherche}\footnote{\url{http://www.larecherche.fr/}} et
\textit{Pour la Science}\footnote{\url{http://www.pourlascience.com/}}, ou encore
\textit{Science et vie}\footnote{\url{http://www.science-et-vie.com/}},
\textit{Science et vie junior}\footnote{\url{http://www.labosvj.fr/}},
\textit{Science et avenir}\footnote{\url{http://www.sciencesetavenir.fr/}},
\textit{Tangente}\footnote{\url{http://tangente.poleditions.com/}} (en kiosque)
et \textit{Quadrature}\footnote{\url{http://www.quadrature-journal.org/}}
(niveau TS ou Licence 1, sur abonnement seulement).

Tout comme pour les sites pr\'ec\'edents, il est conseill\'e de contacter les responsables du
 magazine pour proposer un sujet avant de se lancer dans l'\'ecriture d'un texte.

%\medskip
%{\bf Expos\'es}. Les Promenades math\'ematiques
%\footnote{\url{http://smf.emath.fr/MathGrandPublic/PromenadesMathematiques/}}
%sont une initiative destin\'ee \`a favoriser la diffusion de la culture math\'ematique aupr\`es
% de tous les publics en organisant des ateliers ou des conf\'erences de vulgarisation dans des cadres divers.
%Organis\'ees conjointement par la Soci\'et\'e Math\'emati\-que de France (cf \ref{smf}) et l'association Animath (cf \ref{animath}),
% elles b\'en\'eficient du soutien du CNRS et d'INRIA et s'appuient sur les laboratoires de
% math\'ematiques CNRS, INRIA et universitaires.

\medskip
{\bf Audimath}. Audimath \footnote{\url{http://audimath.math.cnrs.fr/}} est un r\'eseau cr\'e\'e par l’Institut National Sciences Math\'ematiques et de leurs Interactions (INSMI) du CNRS et destin\'e \`a apporter un soutien \`a tous les acteurs de la communaut\'e universitaire investis dans le d\'eveloppement des activit\'es de diffusion des math\'ematiques aupr\`es des publics extra-universitaires.



%%%%%%%%%%%%%%%%%%%%%%%%%%%%%%%%%%%%%%%%%%%%%
\section{Action vers les jeunes} \label{jeunes}

La communaut\'e de la recherche en math\'ematiques peut s'impliquer dans des actions en direction des jeunes et de
nos coll\`egues de l'enseignement secondaire, voire primaire.
On peut distinguer plusieurs types d'activit\'es dites ``p\'eriscolaires" qui permettent de toucher les jeunes :
\begin{itemize}
\item actions de culture math\'ematique : expositions, sites de culture math\'ema\-ti\-que, conf\'erences,
 rencontres avec des chercheur$\cdot$ses (voir ci-dessus) ;
\item comp\'etitions et concours en temps limit\'e (rallyes math\'ematiques, Championnat international
des jeux math\'ematiques et logiques, Kangourou...) ;
\item projets scientifiques permettant une initiation \`a la recherche, parfois sous forme de concours
ou comp\'etition (ateliers Maths en jeans\footnote{\url{http://www.mathenjeans.fr/}}, ateliers
 hippocampe maths \footnote{\lien{www.irem.univ-mrs.fr/-Hippocampe-.html}}, concours
Faites de la science\footnote{\url{http://www.faitesdelascience.fr/}}, concours C.G\'enial\footnote{\url{http://www.sciencesalecole.org/les-concours/concours-c-genial.html}}) ;
\item ateliers et clubs de math\'ematiques dans les coll\`eges et lyc\'ees ;
\item accompagnement de jeunes fortement motiv\'e$\cdot$es et au talent pr\'ecoce par un tutorat, des stages,
des clubs de math\'ematiques comme il en existe maintenant dans plusieurs universit\'es \footnote{Club de mathe\'ematiques discr\`etes (Lyon) \url{http://math.univ-lyon1.fr/~lass/club.html}, club math\'ematique de Nancy \url{http://depmath-nancy.univ-lorraine.fr/club/}, club Parismaths \url{http://www.parimaths.fr/}, cercle mathe\'ematique de Strasbourg \url{http://www-math.u-strasbg.fr/CercleMath/}, cercle Sofia Kovalevskaia de Toulouse \url{http://www2.animath.fr/spip.php?article2706}} ;
\item organisation de dispositifs sp\'ecifiques en direction de jeunes des zones d\'efavoris\'ees :
tutorat, stages pendant les vacances centr\'es sur les math\'ema\-ti\-ques\footnote{Par exemple, le Centre Galois (Orl\'eans Tours) \url{http://www.centre-galois.fr} ou Science ouverte (Paris 13) \url{http://scienceouverte.fr/-Stages-vacances-}, le stage Mat'les vacances \url{http://paestel.fr/} ; voir aussi les stages MathC2+ dans le paragraphe Animath} ;
\item organisation de tutorat, de mentorat, et de journ\'ees sp\'ecifiques... destin\'es aux filles  \footnote{\url{http://www.animath.fr/spip.php?rubrique160}} ;
\item Le Salon de la culture et des jeux math\'ematiques, qui a lieu tous les ans \`a la fin mai \`a Paris \url{https://www.cijm.org/salon} ;
\item Les diff\'erents \'ev\'enements qui sont organis\'es chaque dans le cadre de la Semaine des math\'ematiques, au mois de mars, ou de la F\^ete de la science au mois de septembre.
\end{itemize}

Voir aussi ci-dessus la pr\'esentation d'Animath (paragraphe \ref{animath}) et des autres acteurs. 

Le rapport Villani-Torossian \url{https://ww.education.gouv.fr/cid126423/21-mesures-pour-l-enseignement-des-mathematiques.html} pr\'econise, dans sa mesure 7 P\'eriscolaire et Clubs,	d'encourager les partenariats institutionnels avec le p\'eriscolaire et favoriser le d\'eveloppement de ce secteur, de recenser et de p\'erenniser les clubs en lien avec les math\'ematiques (de mod\'elisation, d’informatique, de jeux intelligents, etc.) ainsi que de r\'emun\'erer les intervenant$\cdot$es et adapter les emplois du temps des enseignant$\cdot$es. Le rappot apporte des pr\'ecisions dans les Recommandations 41 \`a 45.



%%%%%%%%%%%%%%%%%%%%%%%%%%%%%%%%%%%%%%%%
\section{Valorisation de la recherche}

La valorisation de la recherche comporte plusieurs aspects. Contrairement \`a la vulgarisation scientifique,
qui ressort d'une d\'emarche culturelle, la valorisation rel\`eve d'une d\'emarche plus utilitaire. Curieusement,
ce sont les m\^emes services des universit\'es et des organismes qui s'occupent de vulgarisation et de valorisation.

\subsection{Mise en valeur de travaux dans la communaut\'e math\'ematique}
Vous avez sans doute cr\'e\'e votre page web personnelle en vue de candidater.
Pensez maintenant \`a la mettre \`a jour r\'eguli\`erement : elle est la premi\`ere vitrine de vos recherches.
Pensez aussi \`a mettre vos articles sur \textbf{HAL}: \lien{hal.archives-ouvertes.fr/}.

\subsection{Mise en valeur de travaux en dehors de la communaut\'e math\'ematique}
Cette d\'emarche est compl\'ementaire de la vulgarisation de la recherche dont on a parl\'e plus haut.
Il est par exemple utile de promouvoir la recherche
en math\'ematiques pour que les tutelles prennent conscience de la valeur de leurs \'equipes de recherche
et puissent \`a leur tour utiliser cette information dans leur politique de communication ; il faut bien
comprendre que les r\'esultats de la recherche en math\'ematiques sont moins visibles et surtout moins
compr\'ehensibles que ceux de la quasi-totalit\'e des domaines de recherche ; en plus, et contrairement
 aux autres domaines, il est moins facile de justifier la recherche par des applications mirifiques \'eventuelles
comme gu\'erir le cancer, trouver des sources illimit\'ees d'\'energie etc.

La m\'ediatisation peut utiliser plusieurs angles : reconnaissance scientifique par la publication dans une
revue de premier plan, invitation dans un tr\`es grand congr\`es, r\'esultat facilement explicable ou donnant
 lieu \`a de belles images, preuve d'une conjecture un peu ancienne dont on peut mettre l'histoire en relief,
obtention d'une distinction particuli\`ere (IUF, prix...), collaboration internationale inhabituelle, contrats et brevets...

\medskip\par
Si vous pensez que vos \textbf{r\'esultats} peuvent \^etre \textbf{m\'ediatisables}, le ou la correspondant$\cdot$e
 communication de votre laboratoire\footnote{Coordonn\'ees des correspondant$\cdot$es communication des
 laboratoires de math\'ematiques \url{http://www.cnrs.fr/insmi/spip.php?article256}} vous aidera \`a prendre
 contact avec les services de communication\footnote{\`a l'INSMI \texttt{insmi-equipecom@cnrs-dir.fr}\\
Coordonn\'ees des communicants dans les d\'el\'egations r\'egionales du CNRS sur \url{http://www.cnrs.fr/fr/organisme/dircom/comdelegations.htm}} de vos tutelles qui chercheront \`a les valoriser
aupr\`es de la presse, des \'elu$\cdot$es, des jeunes, du grand public... (n'oubliez pas de pr\'evenir votre directeur$\cdot$trice d'unit\'e).

Pour cela, r\'edigez, si possible avant publication, un court texte en fran\c cais (environ une demi-page)
 repla\c cant le travail dans son contexte et explicitant votre r\'esultat.
Ce document, \'eventuellement accompagn\'e d'une illustration ou d'un sch\'ema, permettra de d\'eterminer
 l'audience susceptible d'\^etre int\'eress\'ee et de pouvoir b\'en\'eficier de diverses chambres de r\'esonances
 au niveau local, r\'egional ou national.
En effet, m\^eme si l'information ne fait pas l'objet d'un communiqu\'e de presse national, elle peut \^etre mise
en avant, par exemple dans :
\begin{itemize}
 \item des sites web (laboratoire ou institut, d\'el\'egation r\'egionale...) ;
 \item la lettre bi-mensuelle aux m\'edias "En direct des labos", diffusant les actualit\'es scientifiques
 des instituts du CNRS ;
 \item le journal du CNRS\footnote{\url{http://www2.cnrs.fr/presse/journal/}} (magazine mensuel
 tir\'e \`a 50 000 exemplaires, envoy\'e \`a tous les agents CNRS ainsi qu'\`a 2000 journalistes, \'elus, partenaires...) ;
 \item CNRS Hebdo (lettre \'electronique diffus\'ee par courriel chaque vendredi, regroupant des informations
nationales et les actualit\'es de la d\'el\'egation r\'egionale et de ses laboratoires) ;
 \item journal ou site web de l'universit\'e.
\end{itemize}

\smallskip\par
Il ne faut pas oublier que les sites web sont aujourd'hui la principale source d'information qu'utilisent
 les \'etudiants pour choisir une universit\'e et un laboratoire pour faire un master ou un doctorat. Avoir
 un site qui pr\'esente les activit\'es du laboratoire, avec une partie en anglais, est un atout important.
Cela peut aussi susciter des collaborations avec des scientifiques travaillant dans d'autres secteurs, des industriels.


\medskip
Si vous produisez des \textbf{images scientifiques}, elles peuvent \^etre d\'epos\'ees dans la banque d'images de CNRS Images\footnote{\url{http://phototheque.cnrs.fr/}}, en acc\`es libre sur Internet (exemples d'utilisations : exposition, presse, plaquette et marque-pages de l'INSMI, sites internet, etc.).

\medskip
Enfin, pour toute communication ou publication, n'oubliez pas de \textbf{mentionner
l'ensemble des tutelles} de votre laboratoire, et pas seulement votre organisme employeur.
