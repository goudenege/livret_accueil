%%%%%%%%%%%%%%%%%%%%%%%%%%%%%%%%%%%%%%%%%%%%%%%%%%%%%%
%%%%%%%%%%%%%%%%%%%%%%%%%%%%%%%%%%%%%%%%%%%%%%%%%%%%%%

\chapter{Le minist\`ere}
\label{chapMinistere}
\index{Minist{\`e}re de l'Enseignement Sup{\'e}rieur, de la Recherche et de l'Innovation (MESRI)}

Le Minist{\`e}re  de l'Enseignement Sup{\'e}rieur, de la Recherche et de l'Innovation (MESRI) a remplacé le Minist{\`e}re de l'\'Education Nationale,  l'Enseignement Sup{\'e}rieur et de la Recherche (MENESR) en mai 2017. Pour un organigramme complet, on peut se reporter \`a la page du minist\`ere.\url{https://www.enseignementsup-recherche.gouv.fr/fr/administration-centrale-49799}. Cet organigramme contient diff{\'e}rentes directions dont le r\^ole est de proposer et de mettre en \oe{}uvre, dans leur champ de comp\'etences, la politique du minist\`ere.

Le MESRI interagit avec de nombreux organismes, \'etablisse\-ments, agences et conseils tels que:
\begin{itemize}
 \item \textbf{Organismes sous tutelle :} \'etablissements d'enseignement sup{\'e}rieur, Grandes {\'e}coles, Universit{\'e}s,
 Centre National des Oeuvres Universitaires et Scolaires (CNOUS), Centres R{\'e}gionaux des Oeuvres Universitaires et Scolaires (CROUS).
 
 \item \textbf{Organismes de recherche :} \'Etablissements Publics {\`a} caract{\`e}re Scientifique et Technologique (EPST),
 \'Etablissements Publics {\`a} caract{\`e}re Industriel et Commercial (EPIC), \'Etablissements Publics {\`a} caract{\`e}re Administratif (EPCA),  Groupements d'Int{\'e}r{\^e}t Public (GIP), Fondations.
 
 \item \textbf{Haut conseil d'{\'e}valuation :} Haut Conseil de l'{\'e}valuation de la recherche et de l'enseignement sup{\'e}rieur (HCERES) ({\em cf.}, Chapitre \ref{HCERES}).
 
 \item \textbf{Agences de financement :} Bpifrance (\url{https://www.bpifrance.fr}) pour l'accompagnement des entreprises, Agence Nationale de la Recherche (ANR, \url{https://www.agence-nationale-recherche.fr}).

 \item \textbf{Structures de consultation :} Conf{\'e}rence des Pr{\'e}sident\mp e\mp s d'Universit{\'e} (CPU), 
 Conf{\'e}rence des Directeur\mp ice\mp s d'{\'E}coles Fran\c caises d'Ing{\'e}nieurs (CDEFI),
 Conseil National de l'Enseignement Sup{\'e}rieur et de la Recherche (CNESER), Haut conseil des biotechnologies, 
 Conseil Strat{\'e}gique de la Recherche (CSR) (a remplac{\'e} le Haut conseil de la science et de la technologie).
\end{itemize}
Pour plus d'exhaustivit\'e : \url{https://www.enseignementsup-recherche.gouv.fr/fr/organismes-independants-et-organismes-sous-tutelle-50921}.

On décrit ici certaines instances, internes ou externes au minist\`ere, 
intervenant directement sur les questions d'enseignement sup\'erieur et de recherche. Outre ces fonctions \og strat\'egiques\fg{}, le minist\`ere a \'egalement d'autres activit\'es qui concernent directement les jeunes math\'ematicien\mp ne\mp s comme l'expertise des dossiers de coop\'eration ({\em e.g.}, les PHC, {\em cf.}, Chapitre \ref{PHC}).

\section{La DGESIP} \label{DGESIP}
La Direction G\'en\'erale de l'Enseignement Sup\'erieur et de l'Insertion Professionnelle (DGESIP) a
pour principale mission l'\'elabo\-ration et la mise en \oe{}uvre de la politique relative \`a l'ensemble des formations
intiales post-baccalaur\'eat (Licence, Master, Doctorat) et continues relevant du ministre en charge de l'enseignement sup\'erieur.
\index{Direction G\'en\'erale de l'Enseignement Sup\'erieur et de l'Insertion Professionnelle\\(DGESIP)}
\index{Licence, Master, doctorat (LMD)}

\textbf{Pour plus de d\'etails\hspace{0.5em}} \url{https://www.enseignementsup-recherche.gouv.fr/fr/direction-generale-de-l-enseignement-superieur-et-de-l-insertion-professionnelle-dgesip-83714}

\section{La DGRI}
L'activit\'e de la Direction G\'en\'erale de la Recherche et de l'Innovation (DGRI) s'articule principalement autour de deux axes~: 
l'\'elaboration et la mise en \oe uvre de la politique de l'\'Etat en mati\`ere de recherche et d'emploi scientifique,
et le pilotage des programmes de la Mission Interminist\'erielle de Recherche et d'Enseignement Sup\'erieur (MIRES).
\index{Direction G\'en\'erale de la Recherche et de l'Innovation (DGRI)}
\index{Mission Interminist\'erielle de Recherche et d'Enseignement Sup\'erieur (MIRES)}

La DGRI veille d'abord \`a la coh\'erence et \`a la qualit\'e du
syst\`eme fran\c cais de recherche et d'innovation, en liaison avec
l'ensemble des minist\`eres concern\'es (finances, industrie,
affaires \'etrang\`eres, {\em etc.}). Elle d\'efinit les
orientations de la politique scientifique nationale ainsi que les
priorit\'es de recherche des \'etablissements d'enseignement
sup\'erieur. Elle assure leur mise en \oe uvre par la tutelle
strat\'egique des organismes relevant du minist\`ere en charge de la
recherche et contribue \`a la politique de l'innovation et de la
recherche industrielle.

Enfin, la DGRI assure le secr\'etariat permanent du Conseil Strat\'egique de la Recherche (CSR) dont elle pr\'epare les travaux.
Le CSR cr\'e\'e en 2013 est dépend du Premier ministre fran\c cais pour proposer les grandes orientations de la strat\'egie nationale de recherche scientifique, et participer \`a l'\'evaluation de leur mise en \oe uvre.
\index{Conseil Strat\'egique de la Recherche (CSR)}

\`A l'\'echelle europ\'eenne et internationale, la DGRI d\'efinit les mesures n\'ecessaires \`a la construction de l'espace europ\'een de l'enseignement sup\'erieur et de la recherche en liaison avec la DGESIP et la D\'el\'egation aux Relations Europ\'eennes, Internationales, et \`a la Coop\'eration (DREIC).
\index{DD\'el\'egation aux Relations Europ\'eennes, Internationales, et \`a la Coop\'eration (DREIC)}

Au titre de la politique territoriale de la recherche, la DGRI est charg\'ee de la politique d'organisation territoriale des activit\'es de recherche, en liaison avec la DGESIP. Elle assure le suivi des Contrats de Plan \'Etat-R\'egions pour ce qui concerne les \'etablissements de recherche dont elle a la tutelle.
\index{Contrats de Plan \'Etat-R\'egions (CPER)}
Elle coordonne aussi l'activit\'e des D\'el\'egu\'es R\'egionaux \`a la Recherche et \`a la Technologie charg\'es de l'action d\'econcentr\'ee de l'\'Etat pour la recherche et l'innovation.
\index{D\'el\'egation R\'egionale \`a la Recherche et \`a la Technologie (DRRT)}

La DGRI r\'epartit entre les organismes dont elle a la tutelle (la
plupart des EPST et EPIC) les moyens n\'ecessaires \`a
l'accomplissement de leurs missions, met en place et entretient en
concertation avec ces organismes les indicateurs de performance afin
de rendre compte de l'efficacit\'e des moyens engag\'es. Cela
concerne, entre autres, le BRGM, le CEA, le CNRS,
l'IFPEN, l'Ifremer, l'IFSTTAR, INRIA, INRAE, l'Inserm, l'IRD, l'Onera, {\em etc.}

\index{\'Etablissement Public \`a caract\`ere Scientifique \\et Technologique (EPST)}
\index{\'Etablissement Public \`a caract\`ere Industriel et Commercial (EPIC)}
\index{Bureau de Recherches G\'eologiques et Mini\`eres \\(BRGM)}
\index{Commissariat \`a l'\'Energie Atomique (CEA)}
\index{Conseil National de la Recherche Scientifique \\(CNRS)}
\index{IFP Energies Nouvelles (IFPEN)}
\index{Institut fran\c cais de recherche pour l'exploitation de la mer (Ifremer)}
\index{Institut Fran\c ais de Sciences et Technologies des Transports, de l'Am\'enagement et des R\'eseaux (IFSTTAR)}
\index{Institut National de Recherche pour l'Agriculture, l'alimentation et l'Environnement (INRAE)}
\index{Institut National de Recherche en Informatique et en Automatique (INRIA)}
\index{Institut National de la Sant\'e et de la Recherche M\'edicale (Inserm)}
\index{Institut de Recherche \\pour le D\'eveloppement (IRD)}
\index{Institut national de recherche en sciences et technologies pour l'environnement et l'agriculture (Irstea)}
\index{Office national d'\'etudes \\et de recherches a\'erospatiales (Onera)}

\textbf{Lien utile\hspace{0.5em}}\url{https://www.enseignementsup-recherche.gouv.fr/fr/direction-generale-de-la-recherche-et-de-l-innovation-dgri-83753}

\section{La DREIC}\label{DREIC}
La D\'el\'egation aux Relations Europ\'eennes, Internationales, et \`a la Coop\'eration (DREIC) d\'epend du Secr\'etariat g\'en\'eral du minist\`ere ({\em cf.}, Section \ref{sgm}). Elle coordonne le d\'eveloppement, les \'echanges et la coop\'eration avec les syst\`emes scolaires, universitaires et de recherche \'etrangers. \`A cette fin, elle contribue \`a la pr\'eparation des accords bilat\'eraux ({\em e.g.}, les partenariats Hubert-Curien (PHC), {\em cf.} Section \ref{PHC}), ainsi qu'\`a celle des projets conduits dans le cadre des organisations europ\'eennes ou internationales. Elle apporte son concours \`a la DGESIP et \`a la DGRI pour la d\'efinition des mesures n\'ecessaires \`a la construction de l'espace europ\'een de l'enseignement sup\'erieur et de la recherche. Elle pr\'epare les positions du minist\`ere et assure sa repr\'esentation dans les instances et rencontres internationales, notamment dans les conseils et comit\'es europ\'eens de l'\'education. La DREIC travaille en \'etroite collaboration avec le minist\`ere des affaires \'etrang\`eres.
\index{D\'el\'egation aux Relations Europ\'eennes, Internationales, et \`a la Coop\'eration (DREIC)}

\textbf{Lien utile\hspace{0.5em}}\url{https://www.education.gouv.fr/delegation-aux-relations-europeennes-et-internationales-et-la-cooperation-dreic-7418}

\section{La DEPP}
La Direction de l'\'Evaluation, de la Prospective et de la Performance (DEPP) est charg\'ee de la conception et de la gestion du syst\`eme d'information statistique en mati\`ere d'enseignement et de recherche. Elle con\c coit et met en \oe uvre, \`a la demande des autres directions, un programme d'\'evaluations, d'enqu\^etes et d'\'etudes sur tous les aspects du syst\`eme de recherche.
\index{Direction de l'\'Evaluation, de la Prospective et de la Performance (DEPP)}

\textbf{Lien utile\hspace{0.5em}}\url{https://www.education.gouv.fr/direction-de-l-evaluation-de-la-prospective-et-de-la-performance-depp-12389}

\section{Le Secr\'etariat G\'en\'eral} \label{sgm}

Le Secr\'etariat G\'en\'eral, plac\'e sous l'autorit\'e du minist\`ere, regroupe l'ensemble des directions et services venant en soutien des directions op\'erationnelles des minist\`eres (DGESIP, DGRI pour ce qui nous concerne). On y trouve aussi la DREIC, la DEPP, mais \'egalement la Direction G\'en\'erale des Ressources Humaines (DGRH), dont tous les enseignants-chercheurs d\'ependent, \textit{via} leur \'etablissement d'affectation. 
\index{Direction G\'en\'erale des Ressources Humaines\\(DGRH)}

\textbf{Lien utile\hspace{0.5em}}\url{https://www.education.gouv.fr/le-secretariat-general-sg-7601}

\section{Le CNESER}\label{CNESER}

\index{Conseil national de l'enseignement sup\'erieur et de la recherche (CNESER)}

Parmi les structures consultatives du minist\`ere cit\'ees dans le sch\'ema externe,
le Conseil National de l'Enseignement Sup\'erieur et de Recherche (CNESER) est l'instance de r\'ef\'erence pour le minist\`ere sur toutes les questions d'enseignement
sup\'erieur et de recherche \`a l'exception de celles touchant au statut des personnels. Y sont abord\'es, entre autres,
\begin{itemize}
\item la politique g\'en\'erale de l'enseignement sup\'erieur~;
\item les grands projets de r\'eforme (lors du passage au LMD par
exemple)~;
\index{Licence, Master, doctorat (LMD)}
\item les budgets des universit\'es, les programmes et
demandes de cr\'edits~;
\item les habilitations des divers
dipl\^omes (Licence, Master, {\em etc.})~;
\item les reconnaissances
des \'ecoles doctorales~;
\item l'ensemble des textes de loi et
d\'ecrets concernant l'enseignement sup\'erieur et la recherche.
\end{itemize}
\textbf{Lien utile\hspace{0.5em}}\url{https://www.enseignementsup-recherche.gouv.fr/fr/le-conseil-national-de-l-enseignement-superieur-et-de-la-recherche-cneser-87955}
% Outre le ministre, il comprend 68 membres, dont 45 repr\'esentants
% \'elus des universit\'es et \'etablissements assimil\'es, r\'epartis
% comme suit~:
% \begin{itemize}
% \item 5 repr\'esentants des chefs d'\'etablissements~;
% \item 22 enseignants-chercheurs, enseignants ou chercheurs (dont 11
% professeurs des universit\'es ou assimil\'es)~;
% \item 11 \'etudiants~;
% \item 7 repr\'esentants des personnels non-enseignants dont un
% conservateur des biblioth\`eques.
% \end{itemize}
% Il est pr\'esid\'e par
% le ministre charg\'e de l'enseignement sup\'erieur. Il regroupe en
% son sein une commission scientifique permanente charg\'ee de
% pr\'eparer les travaux du conseil en mati\`ere de recherche,
% d'enseignement et de dipl\^omes de 3\ieme{} cycle, et une
% section permanente qui assure l'ensemble des sessions du conseil
% national en dehors des sessions pl\'eni\`eres.

