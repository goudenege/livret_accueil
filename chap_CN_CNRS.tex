
%%%%%%%%%%%%%%%%%%%%%%%%%%%%%%%%%%%%%%%%%%%%%%%%
%%%%%%%%%%%%%%%%%%%%%%%%%%%%%%%%%%%%%%%%%%%%%%%%

\chapter[Le Comit\'e National]{Le Comit\'e national de la recherche scientifique}
\label{CN}

\index{Comit\'e national de la recherche scientifique \\ (CN)}

Le Comit\'e National de la recherche scientifique regroupe les instances de conseils et
d'\'evaluation de l'activit\'e des chercheur$\cdot$ses CNRS et des laboratoires dont le CNRS est
(co)tutelle.

%%%%%%%%%%%%%%%%%%%%%%%%%%%%%%%%%%%
\section{Sa composition}

Le Comit\'e National comprend
\begin{itemize}
\item 1 conseil scientifique (CS) ;
\item 10 conseils scientifiques d'instituts (CSI) ;
\item 41 sections recouvrant l'ensemble des disciplines scientifiques ;
\item et 5 commissions interdisciplinaires (CID) dont 4 commissions th\'ematiques et une commission consacr\'ee \`a la gestion
de la recherche.
\end{itemize}

Le conseil scientifique d'institut est constitu\'e de 24 membres, qui se r\'eunissent au moins deux fois par an, sur convocation du ou de la Directeur$\cdot$trice d'institut:
12 membres \'elu$\cdot$es  et  12 membres nomm\'e$\cdot$es par le ou la Pr\'esident$\cdot$e du CNRS, apr\`es avis du Conseil scientifique du CNRS, comprenant des personnalit\'es \'etrang\`eres.\\

Les math\'ematiques sont regroup\'ees dans la section 41, {\it
Math\'ematiques et interactions des math\'e\-ma\-ti\-ques}, qui
comporte 21 membres~: 14 \'elu$\cdot$es, soit 3 directeur$\cdot$trices de recherche
(DR), 3 charg\'e$\cdot$es de recherche (CR), 3 professeur$\cdot$es, 2 ma\^\i tres
de conf\'erences ou assimil\'es,  et 3 ing\'enieur$\cdot$es, technicien$\cdot$nes, administratifs
(ITA), ainsi que 7 autres membres nomm\'e$\cdot$es pour cinq ans par le ou la
ministre en charge de la recherche, apr\`es avis du ou de la pr\'esident$\cdot$e du CNRS. Il est \`a noter que les enseignant$\cdot$es-chercheur$\cdot$ses et les chercheur$\cdot$ses d'autres organismes
doivent s'inscrire pour \^etre \'electeur$\cdot$trices, de m\^eme que les
personnels IATOS qui sont alors \'electeur$\cdot$trices et \'eligibles dans le
coll\`ege ITA.\\

La Section 41 du Comit\'e National en place aujourd'hui a \'et\'e constitu\'e en
2016, pour un mandat de cinq ans. Il est actuellement pr\'esid\'e par
\verifier{Didier Bresch}, Directeur de recherche CNRS au Laboratoire de Math\'ematiques
de l'Universit\'e Savoie Mont Blanc.
On trouvera la composition de la section 41 sur la page suivante. \\
\lien{www.cnrs.fr/comitenational/contact/annuaire.php?inst=41}\\
et on pourra \'egalement consulter diverses informations sur la page \lien{cn.math.cnrs.fr}.
Il est aussi int\'eressant de constater que suite \`a la r\'eforme du CNRS, la section 41 est la seule
pr\'esente au sein de l'Insmi, contrairement  au MPPU  o\`u les math\'ematicien$\cdot$nes \'etaient regroup\'e$\cdot$es
avec les physicien(ne)s. \\

Outre cette section 41, les math\'ematiques peuvent \^etre pr\'esentes
dans les commissions interdisciplinaires (CID). C'est par exemple le cas de la CID 51, intitul\'ee {\it  Mod\'elisation, et analyse des donn\'ees et des syst\`emes biologiques~: approches informatiques, math\'ematiques et physiques}.


%%%%%%%%%%%%%%%%%%%%%%%%%%%%%%%%%%%
\section{Ses missions}


Tr\`es sch\'ematiquement,
les missions du Comit\'e National contiennent deux volets~: d'une part
une mission d'{\em \'evaluation}, d'autre part une mission de {\em conseil}.
La premi\`ere est confi\'ee aux diff\'erentes sections ainsi qu'aux commissions interdisciplinaires, tandis que les missions de conseil sont d\'evolues aux Conseils Scientifiques d'Institut, qui veillent
notamment \`a la coh\'erence de la politique scientifique de l'institut et donnent des avis sur les grandes orientations.


\subsection{Les concours de recrutement de chercheur$\cdot$ses}
\label{sec. conccnrs}

Alors que le changement de grade \`a l'int\'erieur d'un m\^eme corps (comme le fait de passer de CR2 \`a CR1,
ou DR2 \`a DR1) correspond \`a une promotion, l'acc\`es \`a un nouveau corps (i.e., le fait de devenir CR ou DR)
n\'ecessite de passer un {\em concours} dont le d\'eroulement est d\'ecrit dans la section \ref{sec. recrutCNRS}. \\

Il existe \'egalement certaines ann\'ees des concours externes de CR1 et DR1 ouverts aux membres ext\'erieurs au CNRS.
Le concours de DR2 est syst\'ematiquement ouvert \`a tou$\cdot$tes les chercheur$\cdot$ses ayant l'anciennet\'e requise.\\

Ce sont les sections du comit\'e national qui sont charg\'ees de la phase  d'admissibilit\'e de ces concours.
La phase d'admission est ensuite effectu\'ee par un jury constitu\'e pour parties de membres de sections et de personnalit\'es ext\'erieures propos\'ees par les instituts. 

\subsection{L'\'evaluation des chercheur$\cdot$ses au CNRS}
\label{sec. evalcnrs}
\index{Evaluation ! des chercheur$\cdot$ses CNRS}
\index{Compte rendu annuel d'activit\'e des chercheurs du CNRS (CRAC)}

Outre le CRAC (le ``compte-rendu annuel d'activit\'e des
chercheurs du CNRS", qui doit \^etre rempli tous les ans vers le mois de novembre
et sur lequel le$\cdot$la directeur$\cdot$trice de l'unit\'e \'emet un avis),
chaque chercheur$\cdot$se fournit tous les deux ans  et demi un rapport d'activit\'e personnel compl\'et\'e
par un programme de recherche pour les deux ann\'ees et demi \`a venir.
Sur la base de ce rapport et de l'avis du ou de la directeur$\cdot$trice d'unit\'e, la section \'evalue l'activit\'e scientifique du ou de la
chercheur$\cdot$se\footnote{\`A noter que les directeur$\cdot$tricess d'unit\'e ont acc\`es au
r\'esultat de l'\'evaluation des chercheur$\cdot$ses pr\'esent$\cdot$es dans leur
unit\'e.}.
L'avis de la section est transmis \`a la chercheuse ou au chercheur
{via} un portail informatique d\'enomm\'e \emph{Espace
Chercheur}~: c'est ce m\^eme portail qui, actuellement, permet la
compilation du CRAC, le d\'ep\^{o}t du rapport d'activit\'e ou
bien le d\'ep\^{o}t d'une demande particuli\`ere (mobilit\'e,
{\em etc.}). Il faut souligner que la section peut \'emettre un avis autre que favorable :
avis r\'eserv\'e ou avis d'alerte (qui d\'eclenchent des proc\'edures sp\'ecifiques impliquant la direction scientifique,
la DRH, la d\'el\'egation r\'egionale et le laboratoire, \`a des degr\'es divers). Enfin la section peut voter
une insuffisance professionnelle, ce qui d\'eclenche g\'en\'eralement une proc\'edure disciplinaire de passage en commission
paritaire\footnote{Pouvant conduire \'eventuellement \`a un licenciement, comme cela s'est d\'ej\`a produit en section 41.}.
\\

Les sections  font en outre un travail de classement des candidat$\cdot$es
demandant une promotion de grade (\`a l'int\'erieur d'un m\^eme corps).
Cette activit\'e fait explicitement partie de l'\'evaluation des chercheur$\cdot$ses assur\'ee par les sections ; les instituts  pr\'esentent ensuite un classement \`a la direction du CNRS qui
d\'ecide.\\

Enfin, les sections donnent leur avis sur un certain
nombre de d\'ecisions administratives concernant les chercheur$\cdot$ses
(leur affectation initiale, leur int\'egration apr\`es la p\'eriode de stage,
leur \'eventuelle reconstitution de carri\`ere apr\`es
int\'egration, {{\em etc.}}).

\subsection{L'\'evaluation des unit\'es de recherche associ\'ees au
CNRS}
\index{Evaluation ! des laboratoires}

Les sections du Comit\'e National donnent un avis scientifique sur les laboratoires propres ou associ\'es au CNRS et se
prononcent sur l'opportunit\'e de cr\'eer un laboratoire ou de l'associer
au CNRS ainsi que sur le renouvellement de l'association. Au moment du renouvellement des unit\'es, la section s'appuie sur les \'evaluations de l'HCERES (voir le chapitre \ref{HCERES}) pour conseiller
la direction sur la pertinence du soutien du CNRS \`a tel ou tel laboratoire. La section donne \'egalement
un avis sur les unit\'es mixtes de service (UMS) et sur les
groupements de recherche (GDR) ou de service (GDS).

\index{Unit\'e mixte de recherche (UMR)}
\index{Unit\'e mixte de services (UMS)}
\index{Groupement de recherche (GDR)}
\index{Groupement de services (GDS)}


\subsection{Les d\'el\'egations CNRS}
\index{D\'el\'egation au CNRS}

Les sections du Comit\'e National donnent \'egalement leur avis sur les demandes
de d\'el\'egation d\'epos\'ees par des enseignant$\cdot$es-chercheur$\cdot$ses et ayant re\c cu l'aval du conseil d'administration de leur tutelle. Ces dossiers sont 
 transmis par les d\'el\'egations r\'egionales  aux instituts du CNRS.
En concertation avec les directions d'instituts et la direction des diff\'erentes tutelles,
le ou la pr\'esident$\cdot$e du CNRS arr\^ete ensuite la liste des enseignant$\cdot$es-chercheur$\cdot$ses accueilli$\cdot$es au CNRS. Environ 500 ann\'ees de d\'el\'egation CNRS sont attribu\'ees chaque ann\'ee \`a des enseignant$\cdot$es-chercheur$\cdot$ses, leur permettant ainsi un accueil dans une unit\'e CNRS avec d\'echarge d'enseignement. Sur ces 500 d\'el\'egations, un peu plus d'une centaine d'ann\'ees vont aux math\'ematicien$\cdot$nes 


\subsection{Les postes de Chercheur$\cdot$se sur contrat longue dur\'ee}

La section donne son avis sur les demandes de postes de chercheur$\cdot$ses invit\'e$\cdot$es que font remonter la laboratoires lors de leur demande de moyens. Il s'agit d'invitation de trois mois dans les UMR de l'Insmi pour des chercheur$\cdot$ses en poste \`a l'\'etranger. 


\subsection{Les \'ecoles th\'ematiques et les Actions nationales de formation (ANF)}
\index{Subventions aux conf\'erences ! par le CNRS}
%
Le comit\'e national est consult\'e sur les projets d'\'ecoles th\'ematiques et d'ANF candidates \`a un financement du CNRS dans le cadre de ses activit\'es de formation interne.


