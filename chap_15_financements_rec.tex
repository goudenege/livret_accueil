%%%%%%%%%%%%%%%%%%%%%%%%%%%%%%%%%%%%%%%%%%%%%%%%%%%%
%%%%%%%%%%%%%%%%%%%%%%%%%%%%%%%%%%%%%%%%%%%%%%%%%%%%

%\chapter*{Les sources de financement}
%
%On peut ais\'ement distinguer deux types de financement pour un
%laboratoire~:
%
%\begin{itemize}
%\item[$\bullet$] les financements r\'ecurrents, qui peuvent \'emaner
%\begin{itemize}
%\item du(es) minist\`ere(s) de tutelle {\em via} le(s)
%\'etablissement(s) dont rel\`eve le laboratoire,
%\item des organismes de recherche (on \'evoquera ici seulement le CNRS
%et INRIA mais, dans d'autres disciplines, on peut trouver des
%financements r\'ecurrents provenant de l'Inserm, l'INRA, le CEA, {
%{\em etc.}})~;
%\end{itemize}
%\item[$\bullet$] les financements sur projet, venant (la liste n'est
%pas exhaustive!)
%\begin{itemize}
%\item de l'agence nationale de la recherche (ANR),
%\item du minist\`ere des affaires \'etrang\`eres,
%\item de la Communaut\'e europ\'eenne pour les diff\'erents programmes europ\'eens,
%\item de contrats avec des partenaires industriels.
%\end{itemize}
%\end{itemize}


\chapter[Les financements r\'ecurrents]{Les financements r\'ecurrents des \'etablissements d'enseignement sup\'erieur}

%\section{La loi LRU sur l'autonomie des universit\'es}
%\label{sec. quad}
%\index{loi LRU}

La  \href{www.legifrance.gouv.fr/affichTexte.do?cidTexte=JORFTEXT000000824315}{loi relative aux Libert\'es et Responsabilit\'es des Universit\'es}
(dite loi LRU ou loi P\'ecresse), r\'egit depuis le 10 ao\^ut 2007 les relations 
entre les \og grands \'etablissements\fg{} et l'\'Etat, donc notamment les relations financi\`eres. Il y est dit que les \'etablissements 
concluent avec l'\'etat un \og  contrat pluriannuel d'\'etablissement\fg{} (en pratique, pluriannuel signifie actuellement cinq ans.
C'\'etait quatre ans il y a quelques ann\'ees). Ces contrats pr\'ecisent des modalit\'es d'\'evaluation des personnels, et la mani\`ere dont 
l'\'etablissement contribue \`a un \og p\^ole de recherche et d'enseignement sup\'erieur\fg{}. Ils ne constituent en aucun cas un engagement financier 
pluriannuel de l'\'Etat, qui d\'etermine annuellement l'attribution des moyens par la loi de finances.

Il est dit dans la loi LRU que les \'etablissements rendent compte p\'eriodiquement de l'ex\'ecution de leurs engagements, 
qui est \'evalu\'ee par le Haut Conseil de l'\'Evaluation de la Recherche et de l'Enseignement Sup\'erieur ({\em cf.}, Chapitre \ref{HCERES}). 
Cette évaluation a des cons\'equences sur la vie des enseignant\mp e\mp s-chercheur\mp euse\mp s, d\'etaill\'ees ci-apr\`es. Il est alors dit dans la loi LRU que 
l'\'Etat tient compte des r\'esultats de cette \'evaluation pour d\'eterminer 
les engagements financiers qu'il prend envers les \'etablissements dans le cadre des contrats pluriannuels.
Ces engagements concernent en grande partie la masse salariale des personnels de l'universit\'e 
(qui peut \^etre de l'ordre de 70\% du budget total) et donc la possibilit\'e d'augmenter (ou l'obligation de diminuer) 
les effectifs des enseignant\mp e\mp s et des enseignant\mp e\mp s-chercheur\mp euse\mp s.

Pour la tr\`es grande majorit\'e des laboratoires de math\'ematiques,
le minist\`ere est le principal support
financier (\emph{via} les universit\'es).
Le financement r\'ecurrent doit permettre l'achat de mat\'eriel
(informatique et fournitures de bureau essentiellement), ainsi que
le paiement de frais de mission pour les membres permanents et non
permanents reconnus du laboratoire.

\section{Le BQR}
% \label{BQR}
\index{Bonus Qualit\'e Recherche (BQR)}

Historiquement, les financements r\'ecurrents dans les \'etablissements d'enseignement
sup\'erieur sont soumis au BQR (bonus qualit\'e recherche)~: ces
\'etablissements pr\'el\`event une quote-part repr\'esentant 15\,\% de
toutes les sommes vers\'ees par l'\'Etat et les organismes de
recherche, pour mener \`a bien leur politique scientifique.
Ainsi le BQR est pr\'elev\'e sur les subventions
minist\'erielles affect\'ees aux laboratoires, et il est redistribu\'e par
l'interm\'ediaire d'appels d'offres discut\'es et vot\'es au sein de l'universit\'e. 
Ces appels d'offres peuvent
proposer, par exemple, des soutiens \`a l'acquisition d'\'equipements de recherche,
\`a l'organisation de colloques, soutien aux jeunes arrivant\mp e\mp s (d\'echarge des jeunes EC).\\

Avec la loi LRU, il semble que le BQR ne soit plus systématique et ce au profit d'une organisation locale à l'établissement.
Ainsi il a parfois été remplacé par plusieurs dotations budgétaires, aux laboratoires, aux départements de formation, aux UFR, {\em etc}.
Il est semble alors difficile de donner une information générique sur ce sujet.

\section{Le financement par les organismes de recherche}

On \'evoque ici seulement le CNRS et INRIA mais, dans d'autres disciplines, on peut trouver des financements r\'ecurrents provenant de l'Inserm, INRAE, le CEA, {\em etc}.

\subsection{Le CNRS}
\index{CNRS}


En math\'ematiques, pr\`es des deux tiers des laboratoires (les UMR)
sont associ\'es au CNRS. Le CNRS est aussi signataire des contrats pluriannuels avec les
\'etablissements d'enseignement su\-p\'e\-rieur, lorsqu'il est
tutelle d'au moins un laboratoire de cet \'etablissement. Cela
signifie, entre autres, qu'il s'engage \`a fournir, pendant la
dur\'ee du contrat, une dotation dont le montant est revu
annuellement par la direction du CNRS. Le contrat entre l'\'etablissement et le CNRS peut \'eventuellement
\^etre renforc\'e, si le CNRS d\'ecide d'adjoindre aux moyens
financiers et aux agents admistratifs d'autres \'el\'ements, comme des
d\'el\'egations (voir \ref{delegation}).

\subsection{INRIA}
\index{INRIA}

INRIA peut financer des \'equipes de recherche de deux fa\c cons
diff\'erentes. Il peut s'associer \`a des \'etablissements d'enseignement
sup\'erieur et/ou d'autres organismes de recherche, auquel cas le
fonctionnement s'apparente au cas du CNRS.

Il peut financer des \'equipes propres, les \'equipes-projets, \`a dur\'ee
de vie plus limit\'ee (quatre ann\'ees, \'eventuel\-le\-ment
reconductibles), au sein de ses centres de recherche.