



\chapter{Le CNRS}
\label{CNRS}


\index{Centre national de recherche scientifique (CNRS)}

Avec environ 30 000 personnels statutaires (chercheur$\cdot$es, ing\'enieur$\cdot$es et technicien$\cdot$nes), le Centre National de la
Recherche Scientifique (CNRS) est le plus grand des
\'etablissements publics \`a caract\`ere scientifique et
technologique (EPST)~; on en trouvera une br\`eve pr\'esentation
(histoire, chiffres-clefs, budget, {\em etc.}) sur la page suivante.\\
\lien{www.cnrs.fr/fr/organisme/presentation.htm}\\
Depuis le 27 novembre 2008, le CNRS est organis\'e
en dix instituts dont trois nationaux (voir \lien{www.cnrs.fr/fr/recherche/instituts.htm})
\begin{itemize}
\item Institut des sciences biologiques (INSB) ;
\item Institut de chimie (INC) ;
\item Institut \'ecologie et environnement (INEE) ;
\item Institut des sciences humaines et sociales (INSHS) ;
\item Institut des sciences de l'information et de leurs interactions (INS2I) ;
\item Institut des sciences de l'ing\'enierie et des syst\`emes (INSIS) ;
\item Institut national des sciences math\'ematiques et de leurs interactions (Insmi) ;
\item Institut de physique (INP) ;
\item Institut national de physique nucl\'eaire et physique des particules (IN2P3) ;
\item Institut national des sciences de l'univers (Insu) ;
\end{itemize}
et en 19 d\'el\'egations r\'egionales. Les d\'el\'egations assurent
une gestion directe et locale des laboratoires et entretiennent les
liens avec les partenaires locaux et les collectivit\'es
territoriales. On peut consulter la carte des d\'el\'egations \`a
l'adresse suivante. 

\lien{www.dgdr.cnrs.fr/delegations/delegations.htm}

La gouvernance du CNRS est assur\'ee par \verifier{Antoine Petit}, pr\'esident du CNRS, assist\'e de deux
directeur$\cdot$trices g\'en\'eraux d\'el\'egu\'es, \verifier{Alain Schuhl} \`a la science et \verifier{Christophe Coudroy} aux
 ressources. Un organigramme est disponible \`a l'adresse suivante.
 
\lien{www.cnrs.fr/fr/organisme/organisation.htm}



%%%%%%%%%%%%%%%%%%%%%%%%%%%%%%%%%%%%%%%%%%%%
\section{Le CNRS et les math\'ematiques}\label{sec:insmi}

Les math\'ematiques
constituent l'un des dix instituts du CNRS, l'Insmi
(Institut National des Sciences Math\'ematiques et de
leurs Interactions). Comme l'Insu et l'IN2P3, l'Insmi est un institut national : l'Insmi a une mission de coordination des math\'ematiques \`a l'\'echelle nationale.\\
 \lien{www.cnrs.fr/Insmi/}\\
On trouvera un organigramme complet sur la page suivante.\\
\lien{www.cnrs.fr/Insmi/spip.php?article225}.
Par ailleurs des charg\'es de mission travaillent sur certains aspects particuliers : interdiciplinarit\'e, valorisation, formation, liens avec les Alliances, liens avec l'Europe. 
\\

L'Insmi conduit une politique nationale pour la recherche math\'ematique, notamment en structurant un r\'eseau de
\begin{itemize}
\item 42 Unit\'es mixtes de recherche (UMR) qui ont un ancrage g\'eographique et sont  en co-tutelle avec les Universit\'es ;
\item 13 F\'ed\'erations de recherche (FR) qui sont des associations r\'egionales de laboratoires (UMR ou Equipes d'accueil) ;
\item 25 Groupements de recherche (GDR) qui sont des structures nationales regroupant des math\'ematicien$\cdot$nes ou ing\'enieur$\cdot$es sur des th\`emes cibl\'es ;
\end{itemize}
sans oublier une Equipe de recherche labellis\'ee (ERL) et une Formation de recherche en \'evolution (FRE). Comme nous le verrons plus loin, l'Insmi assure \'egalement l'existence d'outils nationaux pour les math\'ematiques \`a travers 3 Groupements de service (GDS) et 6 Unit\'es mixtes de services (UMS). Par ailleurs, l'Insmi soutient les math\'ematiques fran\c{c}aises \`a l'international par un r\'eseau de 9 Unit\'es mixtes internationales (UMI) et d'un grand nombre de programmes internationaux dont 9 Laboratoires internationaux associ\'es (LIA) et 4 Groupes de recherche internationaux (GDRI).\\

L'Insmi dote ses unit\'es d'un 
budget r\'ecurrent et intervient aussi dans le cadre d'actions sp\'ecifiques ou d'appels d'offre, nationaux ou  internationaux. Il veille \`a r\'epartir entre les unit\'es les 
 moyens mat\'eriels et humains
(postes de chercheur$\cdot$ses, d'ing\'enieur$\cdot$es, de secr\'etaires, de biblioth\'ecaires, {\em
etc.}) qui sont allou\'es \`a l'Insmi par la direction du CNRS.\\

L'Insmi pilote aussi des appels d'offre permettant aux chercheur$\cdot$ses de disposer de financement pour une p\'eriode donn\'ee (un \`a deux ans) sur projet scientifique. Ces projets appel\'es {\it Projets exploratoires premiers soutiens} (Peps) sont l\'egers \`a monter. Ils peuvent \^etre totalement financ\'e par l'Insmi comme le projet {\it Peps-Jeunes chercheurs Jeunes chercheuses (JCJC)}, ou bien \^etre mont\'es avec d'autres instituts ainsi que la Mission interdisciplinarit\'e du CNRS (\lien{http://www.cnrs.fr/mi/}). Via la Direction de l'innovation et des relations avec les entreprises (\lien{http://www.cnrs.fr/dire/}), l'Insmi peut soutenir des projets innovants \`a fort potentiel.  Enfin, dans le cadre de la politique de site, le CNRS participe \`a des {\it Actions de site} avec ses partenaires r\'egionaux avec qui il lance des appels \`a projets que l'Insmi suit pour les math\'ematiques.\\

L'Insmi adresse une lettre d'information mensuelle aux membres de ses laboratoires. Celle-ci reprend les diff\'erentes nouvelles qui sont parues sur le site de l'Insmi durant le mois et permet de se tenir au courant des diff\'erents appels d'offre en cours ainsi que des nouvelles de la communaut\'e. L'Insmi est aussi pr\'esent sur Twitter et dispose d'une page web en anglais. 


%%%%%%%%%%%%%%%%%%%%%%%%%%%%%%%%%%%%%%%%%%%%%
\section{Les structures CNRS}



Outre les UMR qui constituent la brique de base de l'organisation de la recherche en math\'ematiques, un certain nombre de structures CNRS participent \`a la vie scientifique du ou de la math\'ematicien$\cdot$ne dans toutes ses dimensions. 




\subsection{Des r\'eseaux th\'ematiques : les GDR} Les groupements de recherche
(GDR), au nombre de 25 actuellement, sont des entit\'es du CNRS
regroupant des scientifiques de diverses universit\'es sous une
th\'ematique commune. Ces groupements sont constitu\'es et dot\'es
par le CNRS pendant quatre ans, durant lesquels se d\'eroulent des
manifestations \`a l'instigation du GDR. Les missions effectu\'ees par
les jeunes sont particuli\`erement encourag\'ees. Pour faire partie
d'un GDR, il faut en g\'en\'eral en contacter sa directrice ou son directeur. Si le
laboratoire dont on fait partie comprend des membres d'un GDR, cette
d\'emarche est facilit\'ee. Certains de ces groupements sont aussi des
Groupements de recherche internationaux (GDRI).

\subsection{Des instruments d'ouverture internationale}

Il faut mentionner ici les diff\'erentes op\'erations li\'ees \`a la
politique internationale du CNRS en ma\-th\'e\-ma\-tiques qui
s'appuie sur diff\'erents types de moyens~:
\begin{itemize}
\item 9 Unit\'es mixtes internationales (UMI) (Autriche, Br\'esil, Canada, Chili, Inde, Italie, Mexique, Pays-Bas) ;
\item 4 Groupements de recherche  Internationaux (GDRI) ;
\item 9 Laboratoires internationaux Associ\'es (LIA) ;
\item 16 Programmes internationaux de collaboration scientifique  (PICS).
\end{itemize}

On pourra consulter les relations internationales de l'Insmi sur la page suivante.\\
\lien{www.cnrs.fr/Insmi/spip.php?article219}

\index{Unit\'e mixte internationale (UMI)}


\subsection{Le soutien \`a la recherche math\'ematique}

Deux sortes de structures sont d\'edi\'ees \`a des activit\'es de soutien \`a la recherche. 
On distingue les Unit\'es mixtes de services (UMS)
constitu\'ees par le CNRS et un autre organisme, et les Groupements de services (GDS)
g\'er\'ees  uniquement par le CNRS.

\index{Unit\'e mixte de services (UMS)}
\index{Groupement de services (GDS)}
\index{Unit\'e propre de services (UPS)}

\subsubsection{Diffusion des connaissances}

\begin{itemize}
\item {\bf Le CIRM}.
Le centre international de rencontres math\'ematiques (CIRM) est une
Unit\'e mixte de service (UMS 822)  plac\'ee sous la cotutelle du CNRS, de la SMF et d'Aix Marseille Universit\'e. Il est aussi subventionn\'e par le minist\`ere de la recherche. Il est dirig\'e depuis septembre 2010
par Patrick Foulon (DR CNRS) et
accueille toute l'ann\'ee sur le campus de Luminy \`a Marseille des colloques, \'ecoles, petits groupes de travail,
recherche en bin\^ome. \\
\lien{www.cirm.univ-mrs.fr/}
\index{Centre international de rencontres \\ math\'ematiques (CIRM)}
\index{Soci\'et\'e math\'ematique de France (SMF)}
\item{\bf L'IHP}.
L'Institut Henri Poincar\'e (IHP) est la ``maison des
math\'ematicien$\cdot$nes et des physicien$\cdot$nes". C'est une Unit\'e mixte de service sous la cotutelle 
du CNRS et  de l'universit\'e Pierre et Marie Curie. Sa mission est d'offrir un lieu de rencontre pour les math\'ematicien$\cdot$nes et les physicien$\cdot$nes th\'eoricien$\cdot$nes. Le Centre Emile Borel qui y est h\'eberg\'e organise chaque semestre ou trimestre des enseignements, colloques et s\'eminaires centr\'es sur un th\`eme regroupant une centaine de participants de toutes nationalit\'es. L'IHP abrite \'egalement les soci\'et\'es savantes li\'ees aux math\'ematiques.
L'IHP, c'est aussi des bureaux d'accueil pour se rencontrer entre math\'ematicien$\cdot$nes, des publications (les Annales de l'IHP)
et une biblioth\`eque remarquable de math\'ematiques et de physique th\'eorique, d'histoire et de philosophie des sciences. Il est actuellement dirig\'e par Sylvie Benzoni (Professeure de math\'ematique à Lyon 1) second\'ee par R\'emi Monasson (DR CNRS de physique à l'\'Ecole Polytechnique).
\\
\lien{www.ihp.fr/}
\index{Institut Henri Poincar\'e (IHP)}
\item{\bf AuDiMath}. Le r\'eseau "Autour de la Diffusion des Math\'ematiques" (AuDiMath) est un Groupe de recherche (GDS 3745)  destin\'e \`a apporter un soutien \`a tous les acteurs de la communaut\'e universitaire investis dans le d\'eveloppement des activit\'es de diffusion des math\'ematiques aupr\`es des publics extra-universitaires ainsi que dans des activit\'es de communication.\\
\lien{http://audimath.math.cnrs.fr/}
\end{itemize}

\subsubsection{Relations avec le monde industriel et les entreprises}

\begin{itemize}
\item {\bf AMIES}.
Agence pour les Math\'ematiques en Interactions avec les Entreprises et la Soci\'et\'e (AMIES) est une Unit\'e mixte de services (UMS 3458) sous la cotutelle de l'Universit\'e Grenoble-Alpes et du CNRS. C'est aussi un Laboratoire d'Excellence (Labex) de l'Universit\'e Grenoble Alpes, du CNRS et d'Inria. Depuis 2011, la mission d'AMIES est de promouvoir les interactions entre les laboratoires de 	math\'ematiques, les \'etudiant$\cdot$es, et le monde de l'entreprise. Ses programmes, tant en formation et qu'en recherche, visent \`a donner aux entreprises, aux chercheur$\cdot$ses et aux \'etudiant$\cdot$es une meilleure 	visibilit\'e des opportunit\'es et de l'int\'er\^et de d\'evelopper des relations. Les principaux programmes d'AMIES sont :
	\begin{itemize}
	\item les Projets Exploratoires Premiers Soutiens (PEPS) qui co-financent des projets de recherche math\'ematiques - entreprises ;
	\item les Semaines d'Etude Math\'ematiques Entreprises (SEME) pendant lesquelles des doctorant$\cdot$es travaillent en groupe sur des sujets propos\'es par des entreprises ;
	\item le Forum Emploi Maths (FEM), c'est un forum national qui r\'eunit une fois par an, depuis 2013, \'etudiant$\cdot$es, entreprises, formations, laboratoires en math\'ematiques. 
	\end{itemize}
AMIES anime un r\'eseau national qui s'appuie sur des correspondants locaux et interagit \'egalement avec les initiatives \'equivalentes en Europe au travers du r\'eseau Eu-Maths-In et avec les structures	nationales (ANRT, p\^oles de comp\'etitivit\'e, SATT, ...). AMIES est actuellement dirig\'ee par V\'eronique Maume-Deschamps (Universit\'e Claude Bernard Lyon 1, Institut Camille Jordan).
\\Plus d'informations : \lien{http://www.agence-maths-entreprises.fr} 
\end{itemize}

\subsubsection{Ressources informatiques}
\begin{itemize}
\item{\bf Mathrice}.
Ce groupement de services (GDS 2754) 
regroupe la quasi-totalit\'e des administrateurs syst\`eme et r\'eseau des laboratoires de l'Insmi. Il a la double mission de r\'eseau m\'etier et de pilotage d'actions nationales structurantes. Mathrice est un lieu de communications et d'\'echange entre ses membres, ing\'enieur$\cdot$es, technicien$\cdot$nes ou math\'ematicien$\cdot$nes ; il propose de nombreux services \`a l'ensemble de la communaut\'e universitaire math\'ematique, via la plate-forme de services num\'eriques et documentaires {\it Portail Math} (\lien{https://portail.math.cnrs.fr/}). Ce portail a \'et\'e d\'evelopp\'e par Mathrice en collaboration avec Mathdoc et le RNBM. Parmi les services qu'il offre, citons l'annuaire, des jetons logiciels, un syst\`eme de bureau virtuel, une messagerie, l'h\'ebergement de fichiers et de sites, des moyens de calculs, des sessions interactives, un syst\`eme de visioconf\'erence,
l'acc\`es \`a certaines
revues \'electroniques\footnote{En principe, celles auxquelles votre unit\'e de rattachement est abonn\'ee.}
 ainsi qu'aux bases de donn\'ees Mathscinet et Zentralblatt. \\
\lien{www.mathrice.org/}\\

\index{Mathrice}
\index{PortailMath}
\end{itemize}




\subsubsection{Documentation}

\begin{itemize}
\item{\bf RNBM}. Le r\'eseau national des biblioth\`eques de
math\'ematiques est un groupement de services (GDS 2755). Il travaille au maintien de la qualit\'e, de
la sp\'ecificit\'e, et de la p\'erennit\'e de la documentation
math\'e\-ma\-ti\-que. Il participe aussi aux n\'egociations des accords
d'abonnement avec les \'editeurs scientifiques\footnote{Il est \`a noter que la part ``biblioth\`eque" d'un
budget de laboratoire est consid\'erable, voir par exemple la section \ref{sec. quad}.}.
\\
\lien{www.rnbm.org/}
\index{\bf R\'eseau national des biblioth\`eques de math\'e\-ma\-ti\-ques (RNBM)}
\item{\bf MathDoc}.
Cette unit\'e mixte de service (UMS 5638), en cotutelle entre le CNRS et l'Universit\'e Grenoble Alpes, est
la cellule de coordination documentaire nationale pour les
math\'ematiques. Elle s'occupe de num\'erisation (projet NUMDAM),
propose un service d'h\'ebergement pour les journaux acad\'emiques ainsi qu'un soutien \`a l'\'edition. Elle travaille  en \'etroite coordination avec Mathrice et le RNBM. \\
\lien{mathdoc.emath.fr/}
\index{Cellule de coordination documentaire nationale \\pour les math\'e\-ma\-ti\-ques (MathDoc)}
\index{NUMDAM}
\item{\bf Biblioth\`eque Jacques Hadamard}.
La biblioth\`eque Jacques Hadamard (UMS 1786) est une Unit\'e mixte de services sous la tutelle du CNRS et de l'Universit\'e Paris-Sud Orsay.
Cette biblioth\`eque de recherche destin\'ee aux math\'ematicien$\cdot$nes partage avec la biblioth\`eque de l'Universit\'e Paris-Sud Orsay (section Sciences) la fonction de Centre d'Acquisition et de Diffusion de l'Information Scientifique et Technique (CADIST) pour les sciences math\'ematiques. 
\lien{https://bibliotheque.math.u-psud.fr/}
\item{\bf Le CCSd}. Au niveau du CNRS et pour tous les instituts, 
le centre pour la communication scientifique directe (CCSd), qui est
une unit\'e propre de service (UPS) du CNRS, propose de nombreux
services en ligne~: service de pr\'epublications qui alimente
automatiquement ArXiv (HAL, hyper-articles en ligne), th\`eses en
ligne, cours en ligne, CIEL (Codes Informatiques en Ligne), {{\em etc.}}\\
\lien{ccsd.cnrs.fr/}
\index{Centre pour la communication scientifique\\ directe (CCSd)}
\index{ArXiv}
\end{itemize}




\subsubsection{Calcul}

\begin{itemize}
\item {\bf Le GDR Calcul}. Ce groupe de recherche (GDR 3275) rassemble les ing\'enieur$\cdot$es et les math\'ematicien$\cdot$nes travaillant en lien avec le calcul. Il joue le double r\^ole de r\'eseau m\'etier et de lieu de formation et d'\'echanges sur les aspects scientifiques du calcul (math\'ematiques et informatique). Il interagit avec tous les acteurs du calcul (m\'esocentres et le groupe GENCI par exemple).\\
\lien{calcul.math.cnrs.fr/}
\item{\bf Gricad}. Cette Unit\'e mixte de services (UMS  3758) d\'epend scientifiquement de la Mission Calcul et Donn\'ees (MiCaDo) et administrativement de l'Insmi. Gricad a une mission inter-institut sur le site de Grenoble, de suivi et de mutualisation des data center ainsi que d'animation de projets scientifiques autour de la mod\'elisation, du calcul ou du stockage des donn\'ees.\\
\lien{https://gricad.univ-grenoble-alpes.fr/}
\end{itemize}
Par ailleurs, il est bon de savoir qu'il existe quatre centres nationaux pour le calcul intensif.
\begin{itemize}
\item {\bf Idris} :
L'institut du d\'eveloppement et des ressources en informatique
scientifique, qui est une unit\'e propre de service (UPS) du
CNRS, est un centre majeur du CNRS pour le calcul num\'erique intensif
de tr\`es haute performance ; \\
\lien{www.idris.fr/}\\
\item {\bf CC-IN2P3} : Le Centre de calcul de l'IN2P3 qui est g\'er\'e par l'Institut national de physique nucl\'eaire et physique des particules du CNRS ; \\
\lien{http://cc.in2p3.fr/Le-CC-IN2P3}\\
\item {\bf CINES} : Le Centre informatique national de l'enseignement sup\'erieur,
\index{Centre informatique national de l'enseignement \\ sup\'erieur (CINES)}
qui d\'epend du minist\`ere et est rattach\'e aux universit\'es ;\\
\lien{https://www.cines.fr/}\\
\item {\bf TGCC} : Le Tr\`es  grand centre de calcul du CEA  qui abrite des supercalculateurs d\'echelle p\'etaflopique au CEA. \\
\lien{http://www-hpc.cea.fr/fr/complexe/tgcc.htm}
\end{itemize}
Ces centres sont accessibles via un portail unique \`a l'ensemble de la communaut\'e,
et sont regroup\'es au sein de la soci\'et\'e civile {\bf Genci}, Grand \'equipement national de calcul intensif.
\index{Grand \'equipement national de calcul intensif \\(Genci)}
\index{Institut du d\'eveloppement et des ressources en informatique
scientifique (IDRIS)}


