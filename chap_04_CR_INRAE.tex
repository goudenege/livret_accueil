%%%%%%%%%%%%%%%%%%%%%%%%%%%%%%%%%
\chapter{Le m\'etier de chercheur\mp euse \`a INRAE}
\emph{Attention, toutes les informations de ce chapitre ne sont peut-\^etre pas \`a jour.}

\section{L'Institut : statut, structures, personnels}

L'Institut National de Recherche pour l'Agriculture, l'alimentation et l'Environnement (INRAE, \url{https://www.inrae.fr}) est issu de la fusion au 1er janvier 2020 entre l'Institut national de Recherche en Sciences et Technologies pour l'Environnement et l'Agriculture (IRSTEA) et de l’Institut National de la Recherche Agronomique (INRA) qui avait \'et\'e fond\'e en 1946 ! 

INRAE est un Etablissement Public \`a Caract\`ere Scientifique et Technologique (EPST). Il est plac\'e sous la double tutelle du Minist\`ere charg\'e de l'Agriculture et du Minist\`ere charg\'e de la Recherche. Ses missions premi\`eres sont 
\begin{itemize}
\item produire et diffuser des connaissances scientifiques et des innovations, principalement dans les domaines de l'agriculture, de l'alimentation et de l'environnement ;
\item \oe uvrer au service de l'int\'er\^et public tout en maintenant l'\'equilibre entre les exigences de la recherche et les demandes de la soci\'et\'e ;
\item contribuer \`a l'expertise, \`a la formation, \`a la promotion de la culture scientifique et technique, au d\'ebat science/soci\'et\'e.
\end{itemize}
Il s'agit du 1er organisme de recherche sp\'ecialis\'e au monde en agriculture, alimentation et environnement.

INRAE est administr\'e par un Conseil d'Administration pr\'esid\'e par le\mp la Pr\'esident\mp e de l'Institut. Ce\mp tte dernier\mp e assure \'egalement la direction g\'en\'erale. Il\mp elle est assist\'e\mp e du Conseil Scientifique de l'Institut. Les recherches sont conduites au sein d'unit\'es de recherches. INRAE compte un peu plus de 200 unit\'es de recherche implant\'ees dans 18 centres sur toute la France. Ces unit\'es sont r\'eunies au sein de 14 d\'epartements scientifiques de recherche :
\og Action, transition et territoires \fg{}
\og  Agro\'ecosyst\`emes \fg{}
\og  Alimentation Humaine \fg{}
\og  Aliments, produis biosourc\'es et d\'echets \fg{}
\og  Biologie et am\'elioration des Plantes \fg{}
\og  \'Ecologie et biodiversit\'e \fg{}
\og  \'Economie et sciences sociales \fg{}
\og  \'Ecosyst\`emes aquatiques, ressources en eau et risques \fg{}
\og  G\'en\'etique Animale \fg{}
\og  Math\'ematiques et num\'erique \fg{}
\og  Microbiologie et cha\^ine alimentaire \fg{}
\og  Physiologie animale et syst\`emes d'\'elevage \fg{}
\og  Sant\'e animale \fg{}
\og  Sant\'e des plantes et environnement \fg{}
%ceux-ci sont eux-m\^emes coordonn\'es par 3 directeur\mp ices scientifiques \footnote{Alimentation et Bio\'economie, Agriculture, Environnement}. Dix-sept centres r\'egionaux INRAE et un centre si\`ege sont r\'epartis en 148 sites dans toute la France (m\'etropole et Antilles-Guyane). INRAE comprend \'egalement 49 unit\'es exp\'erimentales et 109 unit\'es d'appui et de service. Son budget \'etait de 882 millions d'euros en 2013.

Les personnels de INRAE sont des fonctionnaires de l'Etat r\'egis par le statut g\'en\'eral de la fonction publique. Ce statut est fix\'e par les lois \no83-634 du 13 juillet 1983 et \no84-16 du 11 janvier 1984 modifi\'ees relatives au statut g\'en\'eral des fonctionnaires, les d\'ecrets \no83-1260 du 30 d\'ecembre 1983 et \no84-1207 du 28 d\'ecembre 1984 modifi\'es relatifs respectivement aux fonctionnaires des EPST et \`a celles et ceux de INRAE. Ces textes r\'eglementent les diff\'erentes \'etapes de la carri\`ere des agents : recrutement, avancement, cong\'es, cessation de fonctions.

INRAE c'est, en 2021, un peu plus de 10 000 emplois, dont 2157 chercheur\mp euse\mp s et 3419 ing\'enieur\mp e\mp s. Chaque ann\'ee, INRAE accueille \'egalement plus de 2000 jeunes scientifiques parmi lesquel\mp le\mp s des doctorant\mp e\mp s et des jeunes docteur\mp e\mp s. Les jeunes scientifiques en contrat avec l'Institut sont des membres \`a part enti\`ere du personnel scientifique de l'unit\'e de recherche dans laquelle ils et elles \'evoluent. Elles et ils contribuent activement aux recherches conduites tout en se formant \`a la recherche. Les directeur\mp ices d'unit\'e veillent \`a ce que d\`es leur arriv\'ee, les doctorant\mp e\mp s et les jeunes chercheur\mp euse\mp s s'inscrivent dans un projet professionnel clair et construit.

\section{ Le recrutement}
Conform\'ement aux missions imparties aux personnels de la recherche, les chercheur\mp euse\mp s de l'organisme doivent contribuer non seulement \`a l'acquisition de connaissances nouvelles dans les domaines de leurs comp\'etences mais aussi au transfert des r\'esultats de leurs travaux dans la soci\'et\'e : valorisation \'economique et sociale, diffusion des informations scientifiques et techniques, formation \`a et par la recherche, d\'eveloppement des \'echanges scientifiques avec l'\'etranger. 
Quelle que soit leur discipline de formation, les chercheur\mp euse\mp s s'appuient sur des activit\'es de laboratoire ou de \og terrain\fg{}. Ils et elles sont fortement impliqu\'e\mp e\mp s dans des r\'eseaux scientifiques, r\'epondent \`a des questions environnementales, \'economiques, sociales. Recherche personnelle et projet collectif s'imbriquent \'etroitement pour faire progresser les connaissances et pour participer au d\'eveloppement de l'innovation.

INRAE emploie plus de 1900 chercheur\mp euse\mp s  r\'eparti\mp e\mp s selon deux cat\'egories : les directeur\mp ice\mp s de recherche (DR) et les charg\'e\mp e\mp s de recherche (CR). %Ils\mp elles \'evoluent dans des {\bf disciplines scientifiques} vari\'ees. 
En 2021, le d\'epartement MathNum comptait 190 agents titulaires dont 37 DR et 50 CR. La proportion de femmes parmi les CR et DR \'etait de 29,9\%. \`A l'inverse des sections 25 et 26 du CNU et du CNRS, on observe une disparit\'e moins importante en les CR (28\% de femmes) et les DR (32\% de femmes).

En sa qualit\'e d'\'etablissement public, INRAE recrute ses chercheur\mp euse\mp s par voie de concours. L'ouverture de chaque session de concours est fix\'ee par arr\^et\'e publi\'e au Journal officiel. Le nombre de postes propos\'es et la date limite de d\'ep\^ot des dossiers sont \'egalement fix\'es par arr\^et\'e.

\vspace{-.5\baselineskip}
\paragraph*{Lien utile} \url{https://jobs.inrae.fr/concours}

\subsection{ Les charg\'e\mp e\mp s de recherche}

Chaque ann\'ee, INRAE organise une campagne de concours pour le recrutement de  charg\'e\mp e\mp s de recherche de classe normale. Les CR sont recrut\'e\mp e\mp s par voie de concours organis\'es par discipline ou groupe de disciplines. 

Le recrutement s’effectue, en r\`egle g\'en\'erale, parmi les chercheur\mp euse\mp s et les chercheuses en d\'ebut de carri\`ere ayant soutenu une th\`ese (ou justifiant de titres et travaux scientifiques jug\'es \'equivalents). Les candidat\mp e\mp s sont recrut\'e\mp e\mp s pour leurs comp\'etences scientifiques qu'ils\mp elles mettront au service des grandes orientations d'INRAE en r\'epondant \`a une th\'ematique de recherche. Les candidats doivent avoir valoris\'e les r\'esultats de leur th\`ese par des publications. Les recrutements sont ouverts dans de nombreuses th\'ematiques scientifiques telles que la biologie cellulaire et mol\'eculaire, l'\'ecologie, l'\'economie, la g\'en\'etique, la g\'enomique et autres approches \og omiques\fg{}, l'informatique et l'intelligence artificielle, les math\'ematiques, la nutrition, la physiologie, la physico-chimie, les sciences m\'edicales et v\'et\'erinaires et la sociologie. 

Le calendrier de la campagne est en g\'en\'eral le suivant (sous r\'eserve de la publication de l'arr\^et\'e d'ouverture au Journal Officiel) : ouverture des inscriptions fin janvier, cl\^oture des inscriptions d\'ebut mars, admissibilit\'e (sur dossier) en avril-mai, admission (\'epreuve orale) entre fin mai et mi-juin. 

Les candidat\mp e\mp s admis\mp e\mp s prennent leur fonction en septembre et sont nomm\'e\mp e\mp s en qualit\'e de stagiaires pour une dur\'ee d'un an et sont titularis\'e\mp es, apr\`es avis de la Commission scientifique sp\'ecialis\'ee (CSS) comp\'etente. Toutefois, le stage peut \^etre prolong\'e de 18 mois au maximum ou il peut \^etre mis fin aux fonctions du chercheur ou de la chercheuse apr\`es avis de l'instance d'\'evaluation et de la Commission Administrative Paritaire (CAP) comp\'etente \`a l'\'egard du corps des Charg\'es de Recherche.


\subsection{ Les directeur\mp ice\mp s de recherche}

Le concours de directeur\mp ice\mp s de recherche de 2\`eme classe s'adresse \`a des chercheur\mp euse\mp s confirm\'e\mp e\mp s :
\begin{itemize}
\item candidat\mp e\mp s ayant au moins 3 ans d'anciennet\'e dans le grade des Charg\'e\mp e\mp s de Recherche ;
\item candidat\mp e\mp s titulaires d'un doctorat (ou \'equivalent) et justifiant de 8 ann\'ees d'exercice de la recherche ;
\item candidat\mp e\mp s justifiant de travaux scientifiques jug\'es \'equivalents.
\end{itemize}

Le calendrier de cette campagne est g\'en\'eralement le suivant : ouverture des inscriptions fin juin, cl\^oture des inscriptions fin ao\^ut, admissibilit\'e (sur dossier) octobre, admission (\'epreuve orale) en novembre-d\'ecembre. La prise de fonction se fait en janvier.

\section{L'\'evaluation}
Conform\'ement au d\'ecret qui r\'egit l'\'evaluation des chercheur\mp euse\mp s des EPST et \`a celui sp\'ecifiant les instances d'\'evaluation des chercheur\mp euse\mp s pour INRAE, des Commissions Scientifiques Sp\'ecialis\'ees (CSS) \'evaluent les chercheur\mp euse\mp s, \`a un rythme biennal, sur la base d'un dossier. Treize commissions \'evaluent les chercheur\mp euse\mp s de INRAE. Douze d'entre elles sont d\'efinies par les disciplines et les m\'ethodes de recherche et sont transversales aux d\'epartements. Une treizi\`eme commission \'evalue les chercheur\mp euse\mp s ayant des activit\'es de direction, d'animation ou de gestion de la recherche. Les p\'erim\`etres des CSS ont \'et\'e progressivement adapt\'es aux dynamiques scientifiques de l'Institut de fa\c{c}on \`a favoriser les interactions scientifiques jug\'ees strat\'egiques pour INRAE. Ces p\'erim\`etres sont valid\'es par le Conseil Scientifique de l'Institut. Chaque chercheur\mp euse choisit sa commission d'\'evaluation apr\`es une discussion avec son\mp sa directeur\mp ice d'unit\'e. Les chercheur\mp euse\mp s qui ont un profil pluridisciplinaire et dont les disciplines scientifiques ne sont pas suffisamment repr\'esent\'ees au sein d'une seule commission, peuvent soumettre leur dossier \`a deux commissions. Enfin, une commission peut demander \`a une chercheur\mp euse de soumettre son dossier \`a une autre commission, qu'elle jugera plus comp\'etente.

Ces commissions r\'ealisent une \'evaluation-conseil. Elles produisent pour la direction de l'Institut un avis sur chaque dossier \'evalu\'e. Ces avis sont utilis\'es par la direction pour diff\'erentes d\'ecisions concernant la gestion des personnels. Ils sont statutairement requis pour les demandes de titularisation des charg\'e\mp e\mp s de recherche, les candidatures de promotion en CR hors classe et en DR de classe exceptionnelle. Ils seront aussi disponibles pour la direction lors de son examen des candidatures \`a la prime d'excellence scientifique. Elles formulent des recommandations sur les aspects de l'activit\'e qui doivent \^etre am\'elior\'es. Elles r\'edigent un message personnel destin\'e \`a chaque chercheur\mp euse qui concr\'etise l'attention port\'ee \`a son profil d'activit\'e et \`a sa production et formulent d'\'eventuels conseils.

L'\'evaluation des chercheuses et chercheur\mp euse\mp s d'INRAE, r\'ealis\'ee par les CSS, porte sur l'ensemble des activit\'es des chercheur\mp euse\mp s et prend en compte leur environnement, les missions qui leur sont confi\'ees et les objectifs des collectifs auxquels elles et ils appartiennent. L'\'evaluation par les CSS est une \'evaluation ind\'ependante de la hi\'erarchie et de l'environnement proche des chercheur\mp euse\mp s. Enfin, cette \'evaluation est coll\'egiale : les avis et les messages sont le r\'esultat du travail de l'ensemble de la commission sous la responsabilit\'e de son\mp sa pr\'esident\mp e.

\section{ Les carri\`eres et les r\'emun\'erations}

Comme pour tout fonctionnaire, le traitement d'un\mp e chercheur\mp euse INRAE est constitu\'ee
d'une r\'emun\'eration principale \`a laquelle s'ajoutent la prime de recherche ou l'indemnit\'e de fonctions, sujetions et expertise

La r\'emun\'eration principale d'un\mp e d'un\mp e chercheur\mp euse INRAE augmente
p\'e\-rio\-di\-quement au fur et \`a mesure qu'il gravit les
\'echelons \`a l'int\'erieur de son grade. Cette progression se fait \`a l'anciennet\'e ({\em cf.}, Tables \ref{tab.avanc.CRINRAE} et \ref{tab.avanc.DRINRAE}).
A chaque \'echelon correspond en effet un indice qui d\'etermine le montant de la
r\'emun\'eration principale. L'indice de r\'emun\'eration auquel le\mp la candidat\mp e est recrut\'e\mp e est d\'etermin\'e en fonction de ses dipl\^omes et de ses activit\'es professionnelles ant\'erieures. Le calcul est simple~: un point
d'indice a une certaine valeur fiduciaire (au 1er janvier 2023, la valeur est fix\'ee \`a 4,85 \euro) et le traitement est
calcul\'e par simple multiplication du nombre de points d'indice par
cette valeur nominale. La valeur du point d'indice est
r\'e\'evalu\'ee de mani\`ere \'episodique.


\subsection{ Progression de carri\`ere pour les chercheur\mp euse\mp s}

\begin{center}
\begin{table}[b!]
\begin{center}
\subfloat[CR classe normale]{
\begin{tabular}{lcc}
\toprule
& Indice major\'e& Dur\'ee \\
\midrule
1\ier{} \'echelon &468&1 an \\
2\ieme{} \'echelon &504&2 ans\\
3\ieme{} \'echelon &554&2 ans et 3 mois\\
4\ieme{} \'echelon &594&2 ans et 3 mois\\
5\ieme{} \'echelon &637&2 ans et 6 mois\\
6\ieme{} \'echelon &687&2 ans et 6 mois\\
7\ieme{} \'echelon &733&3 ans\\
8\ieme{} \'echelon &763&3 ans\\
9\ieme{} \'echelon &797&2 ans et 9 mois\\
10\ieme{} \'echelon &830&\\
\bottomrule
\end{tabular}
}
\hfill
\subfloat[CR hors-classe]{
\begin{tabular}{lcc}
\toprule
& Indice major\'e& Dur\'ee\\
\midrule
1\ier{} \'echelon &637&1 an\\
2\ieme{} \'echelon &672&1 an\\
3\ieme{} \'echelon &710&1 an\\
4\ieme{} \'echelon &752&1 an\\
5\ieme{} \'echelon &797&2 ans\\
6\ieme{} \'echelon &830&5 ans\\
7\ieme{} \'echelon & & \\
~\hspace{1cm}1\ier{} chevron &890&1 an\\
~\hspace{1cm}2\ieme{} chevron &925&1 an\\
~\hspace{1cm}3\ieme{} chevron &972& avancement \\
\'Echelon exceptionnel & & \\
~\hspace{1cm}1\ier{} chevron &972&1 an\\
~\hspace{1cm}2\ieme{} chevron &1013&1 an\\
~\hspace{1cm}3\ieme{} chevron &1067&  \\
\bottomrule
\end{tabular}
}
\caption{Grilles d'avancement des CR INRAE}\label{tab.avanc.CRINRAE}
\end{center}
\end{table}
\end{center}

Le corps des CR comporte 2 classes
(\textit{grades}):
\begin{itemize}
\item une classe normale, qui comprend 10 \'echelons~;
\item une hors-classe, qui comprend 8 \'echelons.
\end{itemize}
Les Charg\'e\mp e\mp s de recherche de classe normale peuvent \^etre promu\mp e\mp s au choix \`a la hors classe, apr\`es avis de la Commission scientifique sp\'ecialis\'ee (CSS) comp\'etente, sous r\'eserve de justifier d'une d'anciennet\'e suffisante dans la classe normale (9\`eme \'echelon).

Une bonification d'anciennet\'e d'un an est accord\'ee aux Charg\'e\mp e\mp s de Recherche qui effectuent une mobilit\'e dont la dur\'ee est au moins \'egale \`a 2 ans :
\begin{itemize}
\item dans un autre organisme de recherche ou d'enseignement sup\'erieur \`a l'\'etranger,
\item aupr\`es d'une administration, d'une collectivit\'e locale ou d'une entreprise publique ou priv\'ee.
\end{itemize}

%\newpage
Le corps des DR comporte 3
classes (\textit{grades}):
\begin{itemize}
\item une seconde classe qui comprend 7 \'echelons~;
\item une premi\`ere classe qui comprend 3 \'echelons~;
\item une classe exceptionnelle qui comprend 2 \'echelons.
\end{itemize}
Les Directeur\mp trices de recherche de 2\`eme classe (DR2) peuvent acc\'eder \`a la 1\`ere classe (DR1) apr\`es examen de leur dossier de candidature par une commission d'avancement.

\begin{center}
\begin{table}[t!]
\begin{center}
\subfloat[DR seconde classe]{\small
\begin{tabular}{lcc}
\toprule
& Indice (INM)& Dur\'ee\\
\midrule
1\ier{} \'echelon &667&1 an et 3 mois\\
2\ieme{} \'echelon &705&1 an et 3 mois\\
3\ieme{} \'echelon &743&1 an et 3 mois\\
4\ieme{} \'echelon &785&1 an et 3 mois\\
5\ieme{} \'echelon &830&3 ans et 6 mois\\
6\ieme{} \'echelon & & \\
~\hspace{1cm}1\ier{} chevron &890&1 an\\
~\hspace{1cm}2\ieme{} chevron &925&1 an\\
~\hspace{1cm}3\ieme{} chevron &972& 1 an  et 6 mois\\
7\ieme{} \'echelon & &\\ 
~\hspace{1cm}1\ier{} chevron &972& 1 an\\
~\hspace{1cm}2\ieme{} chevron &1013& 1 an\\
~\hspace{1cm}3\ieme{} chevron &1067& \\
\bottomrule
\end{tabular}
}
\hfill
\subfloat[DR premi\`ere classe]{
\begin{tabular}{lcc}
\toprule
& Indice (INM)& dur\'ee\\
\midrule
1\ier{} \'echelon &830&3 ans\\
2\ieme{} \'echelon & &  \\
~\hspace{1cm}1\ier{} chevron &972&1 an\\
~\hspace{1cm}2\ieme{} chevron &1013&1 an\\
~\hspace{1cm}3\ieme{} chevron &1067&1 an \\
3\ieme{} \'echelon & &  \\
~\hspace{1cm}1\ier{} chevron &1124&1 an\\
~\hspace{1cm}2\ieme{} chevron &1148&1 an\\
~\hspace{1cm}3\ieme{} chevron &1173& \\
\bottomrule
\end{tabular}
}

\subfloat[DR classe exceptionnelle]{
\begin{tabular}{lcc}
\toprule
& Indice (INM)& dur\'ee\\
\midrule
1\ier{} \'echelon & &  \\
~\hspace{1cm}1\ier{} chevron &1173&1 an\\
~\hspace{1cm}2\ieme{} chevron &1226&1 an\\
~\hspace{1cm}3\ieme{} chevron &1279& avancement\\
2\ieme{} \'echelon & &  \\
~\hspace{1cm}1\ier{} chevron &1279&1 an\\
~\hspace{1cm}2\ieme{} chevron &1329&\\
\bottomrule
\end{tabular}
\normalsize}
\caption{Grilles d'avancement des DR INRAE}\label{tab.avanc.DRINRAE}
\end{center}
\end{table}
\end{center}

\subsection{Les primes et indemnit\'es}

Tout comme pour les EC, la politique indemnitaire des chercheur\mp euse\mp s est d\'efinie par la loi n° 2020-1674 du 24 d\'ecembre 2020 de programmation de la recherche pour les ann\'ees 2021 \`a 2030 (LPR). Les EC chercheur\mp euse\mp s \'eligibles \`a des indemnit\'es statutaires et \`a des primes fix\'ees par le d\'ecret n° 2021-1895 du 29 d\'ecembre 2021 ayant conduit \`a la cr\'eation du r\'egime indemnitaire des personnels enseignants et chercheurs (RIPEC). Ce dispositif comprends 3 composantes. Toutes les informations sont disponibles sur : \url{https://www.enseignementsup-recherche.gouv.fr/fr/bo/22/Hebdo10/ESRH2204566X.htm}.

En 2022, les chercheur\mp euse\mp s touchent une prime de recherche de 2800 \euro (correspondant \`a l'indemnit\'e statutaire C1 du RIPEC ({\em cf.}, Section \ref{ripec-C1}). 

Ils\mp elles peuvent \'egalement pr\'etendre \`a la composante C3 ({\em cf.}, Section \ref{ripec-C3}). Pour cette derni\`ere es dossiers sont \'evalu\'es par l'instance d'\'evaluation comp\'etente \`a l'\'egard du ou de la chercheur\mp euse concern\'e\mp e en application des r\`egles statutaires aff\'erentes \`a son corps. L'\'evaluation pr\'ecise au titre de quelle mission (au sens de l'article L. 411-1 du code de la recherche) la prime est propos\'ee. Il peut s'agir d'une de ces missions, de plusieurs ou de l'ensemble d'entre elles. Le b\'en\'efice de la prime peut \'egalement \^etre attribu\'e au titre de missions d'int\'er\^et g\'en\'eral. 

Le\mp la pr\'esident\mp e ou la\mp le directeur\mp ice de l'organisme arr\^ete les d\'ecisions individuelles d'attribution de la prime comprenant le montant individuel et la ou les missions au titre de laquelle ou desquelles la prime est attribu\'ee.

\section{La mobilit\'e}

La mobilit\'e, qu'elle soit th\'ematique, g\'eographique ou effectu\'ee vers d'autres \'etablissements ou entreprises du secteur public ou priv\'e, fait partie int\'egrante du parcours professionnel. Elle offre l'opportunit\'e de concilier les \'evolutions et les besoins de l'Institut avec les comp\'etences et les aspirations individuelles des agents. Diff\'erentes dispositions sont offertes au fonctionnaire pour effectuer une mobilit\'e. 

\subsection{La mise \`a disposition}

Pour ce dispositif le fonctionnaire demeure dans son corps d'origine. Elle\mp il est r\'eput\'e\mp e occuper son emploi et continue de percevoir sa r\'emun\'eration,
mais elle\mp il effectue tout ou partie de son service aupr\`es d'un ou de plusieurs organismes d'accueil. La mise \`a disposition peut \^etre prononc\'ee au profit d'administrations des trois fonctions publiques, ainsi que d'organismes contribuant \`a la mise en \oe uvre de la politique de l'Etat, des collectivit\'es territoriales ou de leurs \'etablissements publics administratifs, pour l'exercice des seules missions de service public confi\'ees \`a ces organismes. La mise \`a disposition intervient avec l'accord du fonctionnaire. Elle est prononc\'ee pour une dur\'ee maximale de trois ans et peut \^etre renouvel\'ee. Elle est g\'en\'eralement encadr\'ee par une convention ou un contrat  de partenariat sign\'e par les partenaires et les agents concern\'es. Ce document pr\'ecise notamment la nature des activit\'es qu'il va exercer et ses conditions d'emploi.

\subsection{Le d\'etachement}

Dans le cas du {\bf d\'etachement}, le\mp la fonctionnaire est plac\'e hors de son corps ou cadre d'emploi initial pour travailler au sein d'un autre organisme que son administration d'origine. Il\mp elle continue toutefois \`a jouir des droits \`a l'avancement et \`a la retraite attach\'es \`a son corps d'origine. D'un point de vue administratif, son \'evolution de carri\`ere se poursuit de mani\`ere parall\`ele dans les deux \'etablissements (\'etablissement d'origine et \'etablissement d'accueil). La dur\'ee du d\'etachement peut \^etre courte (6 mois port\'es \`a un an pour ceux qui exercent une mission \`a l'\'etranger) ou longue (jusqu'\`a 5 ans renouvelables).

\subsection{La mobilit\'e interne}

Pour les chercheur\mp euse\mp s, il n'existe pas de proc\'edure de mobilit\'e interne avec une campagne d'affichage des profils \`a pourvoir. Lorsqu'un chercheur\mp euse exprime un souhait de mobilit\'e (g\'eographique, th\'ematique), il\mp elle en informe son directeur ou sa directrice d'unit\'e et son ou sa chef de d\'epartement, puis cette mobilit\'e se construit sur la base d'un projet scientifique, en interaction avec l'unit\'e d'accueil. C'est \'egalement le cas lorsque les chercheur\mp euse\mp s r\'ealisent une mobilit\'e cons\'ecutive \`a la restructuration, la d\'elocalisation ou la fermeture de leur unit\'e. Si cette mobilit\'e se traduit par un changement de d\'epartement de recherche, elle n\'ecessitera aussi une n\'egociation entre les chefs des d\'epartements concern\'es, puis l'arbitrage final de la direction g\'en\'erale.
