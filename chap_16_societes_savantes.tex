%%%%%%%%%%%%%%%%%%%%%%%%%%%%%%%%%%%%%%%%%%%%%%%%%%
%%%%%%%%%%%%%%%%%%%%%%%%%%%%%%%%%%%%%%%%%%%%%%%%%%

\chapter{Les soci\'et\'es savantes}

Nous remercions les soci\'et\'es savantes qui ont aid\'e \`a la r\'edaction des textes ci-dessous et participe activement \`a l'organisation de cette journ\'ee.

%%%%%%%%%%%%%%%%%%%%%%%%%%%%%%%%%%%%%%
\section{La SMAI}

\emph{Pr\'esident\mp e actuel\mp le : \verifier{Olivier Goubet}} \hfill Site web : \url{http://smai.emath.fr}%, Facebook-Twitter : @smai-media
\smallskip

La SMAI (Soci\'et\'e de Math\'ematiques Appliqu\'ees et
Industrielles), une soci\'et\'e savante en math\'ematiques
appliqu\'ees... pour quoi faire ?

Quand on parle de \og soci\'et\'e savante\fg{}, en particulier chez
les jeunes scientifiques, il y a souvent deux types de
r\'eactions~:
\begin{itemize}%[leftmargin=2em,itemsep=0.4ex]
    \item \og Ouh l\`a, c'est du s\'erieux ...
pour en faire partie il doit falloir \^etre quelqu'un de tr\`es
tr\`es fort et \c{c}a doit parler de choses dr\^olement
compliqu\'ees ... \fg{},
\item \og C'est un peu vieillot et rempli de personnes qui
n'ont que \c ca \`a faire ...\fg{}
sans d'ailleurs savoir ce que \og \c ca\fg{}
repr\'esente.
\end{itemize}

Rien n'est plus faux. Les soci\'et\'es savantes en math\'ematiques
s'attachent \`a \^etre aux c\^ot\'es des math\'ematiciens dans
toutes leurs activit\'es professionnelles
(dans le monde acad\'emique ou industriel),
en particulier l\`a o\`u les structures font d\'efaut.

Malgr\'e toutes les actions que ces soci\'et\'es organisent
directement ou soutiennent, il y a encore beaucoup \`a
faire. Et c'est justement des jeunes que nos soci\'et\'es savantes ont
besoin... Parce qu'il s'agit bien de leur vie scientifique et
de leur avenir professionnel.

Le domaine de la recherche en math\'ematiques est fragile parce que difficile \`a
cerner et \`a expliquer. La SMAI est n\'ee du constat
que la sp\'ecificit\'e des math\'ematiques appliqu\'ees et industrielles devait \^etre
identifi\'ee et d\'efendue si la discipline voulait prosp\'erer. Les
applications des math\'ematiques s'entendent au sens le plus large,
en lien avec les autres sciences et avec les probl\'ematiques
soulev\'ees par des entreprises et qui n\'ecessitent des comp\'etences de math\'ematicien\mp ne\mp s.

L'un des premiers r\^oles de la SMAI est de les aider \`a nouer des contacts dans la communaut\'e
et en dehors, en France et \`a l'\'etranger \`a travers notamment
l'organisation
\begin{itemize}%[noitemsep,topsep=-\parskip,leftmargin=2em]
    \item des journ\'ees th\'ematiques Maths-Industrie ;
    \item des congr\`es (Congr\`es SMAI, journ\'ees des groupes th\'ematiques, CANUM, ...) ;
    \item du CEMRACS, une \'ecole d'\'et\'e sur 6 semaines au CIRM sur le calcul scientifique
	au sens large ;
    \item l'\'ecole franco-espagnole Jacques-Louis Lions sur la simulation num\'erique en
	physique et ing\'enierie ;
    \item des colloques co-organis\'es \`a l'\'etranger (Mexique, ...) ;
    \item ...
\end{itemize}
mais aussi \`a travers des actions communes avec des soci\'et\'es
s\oe urs fran\c caises (SFdS, SMF), \'etrang\`eres (en Italie, Espagne, ...)
des participations aux conseils d'instances et associations internationales (ICIAM,
EMS, ECCOMAS, ...), le parrainage de colloques, ...

La communication entre les membres de la SMAI se fait par le bulletin de
liaison \emph{Matapli} (3 num\'eros par an) et par la lettre \'electronique
\texttt{smai-info}, dont l'abonnement est ouvert \`a tous. De nombreuses
informations et actualit\'es se trouvent sur le site de la SMAI qui fait partie
du domaine \texttt{emath} regroupant divers sites de la communaut\'e
math\'ematique dont certains sont soutenus ou initi\'es par la SMAI~:
Agenda des Conf\'erences en Math\'ematiques, Op\'eration Postes, Carte des Masters, ...

Concernant l'emploi scientifique (postes acad\'emiques, dans les organismes de recherche ou dans les entreprises), la SMAI soutient des activit\'es
telles que
\begin{itemize}%[noitemsep,topsep=-\parskip,leftmargin=2em]
    \item le Forum Emploi Math\'ematiques, qu'elle co-organise ;
    \item l'Op\'eration Postes ;
\end{itemize}
et
\begin{itemize}%[noitemsep,topsep=-\parskip,leftmargin=2em,after={\\}]
    \item a publi\'e un \og livre blanc sur la valorisation dans l'industrie du dipl\^ome de docteur en math\'ematiques appliqu\'ees\fg{} ;
    \item joue le r\^ole d'expertise aupr\`es d'institutions
	nationales (minist\`ere, CNRS, ...) et europ\'eennes
	pour les questions li\'ees aux math\'ematiques appliqu\'e.
\end{itemize}

Pour assurer la diffusion des travaux des math\'ematiciens
appliqu\'ees, la SMAI \'edite les collections d'ouvrages
\emph{Math\'ematiques et applications} (Springer), \emph{Math\'ematiques appliqu\'ees
pour le Master/SMAI} (Dunod) ainsi que les revues de la collection
ESAIM (chez EDP Sciences)~: 
	\begin{itemize}
		\item M2AN (mod\'elisation math\'ematique et analyse num\'erique)
		\item COCV (contr\^ole optimal et calcul des variations)
		\item P\&S (probabilit\'es \& statistiques)
		\item ProcS (proceedings and surveys)
	\end{itemize}
ainsi que RAIRO-RO (avec la ROADEF) et Maths in Action. De plus, la SMAI a lanc\'e le \textbf{SMAI Journal of Computational Mathematics}
(SMAI-JCM), journal \'electronique gratuit.

De m\^eme, elle participe \`a des prix scientifiques,
dont plusieurs sont d\'ecern\'es par l'Acad\'emie des Sciences,
un autre au niveau international, ainsi que des prix
de th\`ese pour promouvoir les travaux de jeunes docteurs.

La SMAI ne se limite pas \`a cr\'eer
des liens entre les communaut\'es. Elle participe aussi largement aux questions
d'enseignement des math\'ematiques en lien avec les autres disciplines et les r\'eformes, du coll\`ege \`a l'universit\'e
et aux grandes \'ecoles,
apportant l\`a aussi son expertise en
participant \`a divers conseils et comit\'es, et en prenant
des positions publiques.

La SMAI est aussi fortement investie dans des actions
de promotion et d'explication au niveau du public,
souvent de fa\c con conjointe avec d'autres soci\'et\'es
(SMF, SFdS, ...) ou associations (Animath,~...).  Citons
par exemple, la publication de la brochure \og L'explosion
des math\'ematiques\fg{}, traduite en plusieurs langues
, la brochure \og Zoom sur les m\'etiers
des math\'ematiques\fg{} en partenariat avec l'Onisep (qui va \^etre r\'e-actualis\'ee),
des r\'eunions, d\'ebats et conf\'erences grand public, des actions
dans le cadre de l'ann\'ee internationale
\og Math\'ematiques pour la Plan\`ete Terre 2013\fg{},
l'organisation, conjointement avec INRIA, du
\og forum des laur\'eats des prix en informatique
et math\'ematiques appliqu\'ees\fg{}, ...

Parmi les actions grand public r\'ecentes de la SMAI, on peut citer:
\begin{itemize}
 \item l'organisation d'une journ\'ee Shannon (4 novembre 2016) \`a l'occasion du centi\`eme anniversaire de sa naissance.
 Cette rencontre grand public s'est d\'eroul\'ee au CIRM en collaboration avec le groupe SMAI-SIGMA.
\lien{www.fr-cirm-math.fr/hommage-claude-shannon.html}
Les conf\'erences on \'et\'e  film\'ees et sont disponibles sur la chaine YouTube du CIRM.
\item Semaine des math\'ematiques 2019. Le th\`eme choisi cette ann\'ee pour la semaine des math\'ematiques \'etait \textit{Jouons ensemble aux Math\'ematiques}.
\item Cycle SMAI/Mus\'ee des Arts et M\'etiers, dont les derniers expos\'es furent donn\'es par Ir\`ene Waldspurger et Patrick Joly. 
\item Salon de l'ONISEP o\`u la SMAI participe au stand tenu par les soci\'et\'es savantes.
\end{itemize}

A l'int\'erieur de la SMAI, les actions sont coordonn\'ees et d\'ecid\'ees par un
Conseil d'Administration de 24 membres, renouvel\'e par tiers tous
les ans, et par un bureau.

Certaines des activit\'es plus sp\'ecifiques
\`a des domaines de recherche sont organis\'ees
par les groupes th\'ematiques~:
\begin{itemize}%[noitemsep,topsep=-\parskip,leftmargin=2em,label={$\star$}]
    \item SMAI-GAMNI (Groupe th\'ematique pour l'Avancement des M\'ethodes Num\'eriques de l'Ing\'enieur) ;
    \item SMAI-MABIOME (Math\'ematiques Biologie M\'edecine) ;
    \item SMAI-MAIRCI (Math\'ematiques Appliqu\'ees, Informatique, R\'eseaux, Calcul, Industrie) ;
    \item SMAI-MAS (Mod\'elisation Al\'eatoire et Statistique) ;
    \item SMAI-MODE  (Math\'ematiques de l'Optimisation et de la D\'ecision) ;
    \item SMAI-SIGMA (Signal - Image - G\'eom\'etrie - Mod\'elisation - Approximation).
\end{itemize}
Il est possible d'appartenir \`a plusieurs de ces groupes ... ou \`a aucun.


La SMAI est une soci\'et\'e vivante, forte de 1\,200 adh\'erents
et repr\'esentative des math\'ematiques appliqu\'ees, mais
qui ne vit que par des membres actifs qui adh\`erent pour
\begin{itemize}%[noitemsep,topsep=-\parskip,leftmargin=2em,label={\ding{51}}]
    \item assurer sa repr\'esentativit\'e,
    \item \^etre force de proposition,
    \item am\'eliorer ses prises de d\'ecision,
    \item et soutenir ses actions.
\end{itemize}
Il existe de nombreuses fa\c cons de s'impliquer dans la SMAI
(participer au Conseil d'Administration, \^etre correspondant local,
proposer ou soutenir des actions, ...).

L'adh\'esion est \textbf{gratuite} pour les doctorants inscrits en th\`ese en France, 
ainsi que pour les docteurs ayant soutenu une th\`ese de math\'ematiques depuis moins de deux ans.

Elle est de \EUR{25} pour les moins de 35 ans.

%%%%%%%%%%%%%%%%%%%%%%%%%%%%%%%%%%%%%%%%%%%
\section{La SFdS}
\index{Soci\'et\'e fran\c caise de statistique (SFdS)}

\emph{Pr\'esident\mp e actuel\mp le~: \verifier{Anne Philippe}} \hfill Site web~: \url{https://www.sfds.asso.fr/}
\smallskip

La Soci\'et\'e Fran\c{c}aise de Statistique est une association d\'eclar\'ee au Journal Officiel du 23 ao\^ut 1997, qui a \'et\'e reconnue d'utilit\'e publique par d\'ecret du 3 d\'ecembre 1998. Elle compte \`a ce jour environ 1000 membres et une vingtaine d'adh\'erents personnes morales. Elle a vocation \`a rassembler tous les chercheur\mp ses, enseignant\mp es et utilisateur\mp trices de la statistique, quels que soient la nature de leurs fonctions et l'endroit o\`u ils les exercent : elle constitue ainsi un lieu privil\'egi\'e de rencontres, d'\'echanges et de r\'eflexions. La SFdS est aussi l'interlocuteur naturel des pouvoirs publics pour les diverses questions touchant \`a la science statistique et \`a la science des donn\'ees (enseignement, expertise, \'ethique, etc.). Elle vise \`a promouvoir l'utilisation de la statistique, \`a favoriser ses d\'eveloppements m\'ethodologiques et \`a d\'evelopper les \'echanges entre statisticiens travaillant dans les entreprises, les administrations ou des \'etablissements d'enseignement et de recherche.

Une grande partie des activit\'es de la SFdS est d\'evelopp\'ee au sein de groupes sp\'ecialis\'es dans un th\`eme ou un domaine d'application de la statistique (\verifier{au nombre de 15 en 2023}). Ses objectifs, son histoire et son mode de fonctionnement en groupes donnent ainsi \`a la SFdS, dans le paysage des soci\'et\'es savantes fran\c{c}aises une place particuli\`ere. En effet, elle r\'eunit en son sein aussi bien des universitaires que des praticiens issus de l'entreprise ou des administrations.

La SFdS et ses groupes organisent une cinquantaine de manifestations scientifiques annuelles mais aussi des manifestations grand public comme par exemple les Caf\'es de la Statistique, soir\'ees-d\'ebats favorisant la rencontre entre statisticiens et citoyens.

La soci\'et\'e publie trois revues scientifiques \'electroniques, le Journal de la SFdS, la revue Statistique et Enseignement et la revue Statistique 
et Soci\'et\'e. Par ailleurs, elle co-\'edite la revue Case Studies in Business and Industrial Statistics (CSBIGS).

La SFdS agit ainsi en fonction de deux axes compl\'ementaires. Tout d'abord, elle a vocation \`a promouvoir la recherche dans tous les domaines de la statistique (et de la science des donn\'ees). Outre les nombreuses journ\'ees th\'ematiques ou mini colloques organis\'es par les diff\'erents groupes de l'association sur des th\`emes pointus, elle r\'eunit, tous les ans, quatre \`a cinq cents personnes lors des Journ\'ees de Statistique, qui constituent un rendez-vous incontournable de la statistique francophone. Elles sont aussi un des premiers tremplins pour les doctorant\mp es et jeunes chercheur\mp ses dans la communaut\'e, avec les Rencontres des Jeunes Statisticien\mp nes qui ont lieu une ann\'ee sur deux les ann\'ees impaires.

L'autre axe de son action consiste \`a favoriser la divulgation des techniques r\'ecentes issues de la recherche : elle met ainsi en oeuvre des cours sp\'ecialis\'es sur des sujets \'emergents (Journ\'ees d'Etude en Statistique, Ateliers Statistiques) assurant l'interface entre les chercheur\mp ses et les utilisateurs ainsi que les rendez-vous M\'ethodes et Logiciels.

Au del\`a de toutes les actions d'animation de la communaut\'e statistique, actions qui font la richesse de notre association et qui sont men\'ees en 
grande partie par nos groupes, nous portons \`a tous les niveaux le message selon lequel c'est le statisticien qui est le plus \`a m\^eme de traiter correctement les millions de milliards d’octets de donn\'ees g\'en\'er\'ees chaque minute. Nous accompagnons les adaptations n\'ecessaires \`a ce nouveau paradigme. Nous nous effor\c{c}ons de d\'evelopper les activit\'es de la SFdS en dehors de territoire fran\c{c}ais et travaillons \`a l'\'emergence d'un journal de stature internationale.

Le statistique est au coeur des d\'efis de la soci\'et\'e de demain. D\'ecouvrez nos activit\'es en naviguant sur notre site internet et rejoignez-nous, une premi\`ere adh\'esion SFdS ne co\^ute que 10 euros.



%%%%%%%%%%%%%%%%%%%%%%%%%%%%%%%%%%%%%%%%%%%%
\section{La SMF}
\label{smf}
\index{Soci\'et\'e math\'ematique de France (SMF)}
\index{Centre international de rencontres \\ math\'ematiques (CIRM)}

\emph{Pr\'esident\mp e actuel\mp le : \verifier{Fabien Durand}} \hfill Site web :  \url{https://smf.emath.fr/}
\smallskip

La Soci\'et\'e\ Math\'ematique de France (SMF), cr\'e\'ee en 1872, est l'une des plus anciennes  soci\'et\'es savantes de math\'ematiques au monde. C'est une association loi 1901, reconnue d'utilit\'e publique, qui compte actuellement 1790 membres (essentiellement des membres individuels, mais aussi des membres institutionnels, c'est-\`a-dire des laboratoires de recherche, biblioth\`eques, institutions...). Elle est ouverte \`a tous les
math\'ematicien(ne)s, amateurs ou professionnels. Sa mission initiale, {\sl l'avancement et la propagation des \'etudes de Math\'ematiques pures et appliqu\'ees}, s'est \'elargie et adapt\'ee aux \'evolutions de notre \'epoque. La SMF s'int\'eresse aux math\'ematiques  dans leur diversit\'e et sous tous leurs aspects :  avanc\'ees de la recherche, interactions avec les sciences et techniques, \'edition de livres et revues,  structuration de la vie scientifique, enseignement \`a tous niveaux.

Elle est en relation avec les soci\'et\'es et institutions qui poursuivent les m\^emes buts et interagit avec beaucoup d'entre elles. En particulier beaucoup de ses actions sont communes avec la SFdS et la SMAI, \'{e}galement avec Animath, Femmes et Math\'{e}matiques, et bient\^ot la Fondation Blaise Pascal. Elle  travaille \'egalement avec de nombreuses soci\'et\'es math\'ematiques \'{e}trang\`{e}res et organise des congr\`{e}s bilat\'{e}raux.



Le r\^{o}le de la SMF aujourd'hui est multiple, et parfois pas assez visible. {\bf La SMF est sollicit\'ee sur un grand nombre de sujets, et est \`a l'initiative de beaucoup d'autres} (que je vais d\'evelopper ci-dessous). Pour pouvoir r\'epondre aux nombreuses sollicitations, et d\'evelopper de belles actions, le r\^ole et la participation de la communaut\'e et des adh\'erents est fondamental, et c'est pourquoi {\bf votre soutien nous est important.}
\\Pour y adh\'erer, vous pouvez notamment visiter notre site web. Les adh\'esions sont gratuites pendant 3 ans pour toutes les personnes en th\`ese, et seulement \`a 25 euros par an pour les moins de 35 ans. Cette adh\'esion donne droit aux 4 exemplaires de la Gazette, l’acc\`es \'electronique gratuit au expos\'es Bourbaki, 30 pour cent de r\'eduction sur nos livres. Surtout, {\bf votre adh\'esion est un soutien moral fondamental et une source de motivation pour tous les b\'en\'evoles qui s’impliquent dans la SMF - elle nous donne \'egalement une grande cr\'edibilit\'e lorsque nous d\'efendons les math\'ematiques aupr\`es des institutions, administrations, m\'edias.}


\subsection*{Une source d'informations}

 Un des r\^oles de la SMF est de faire diffuser les informations li\'ees aux math\'ematiques au sein de la communaut\'e fran\c caise. Pour ceci, nous utilisons:
 \begin{itemize}
 \item  \emph{la Gazette des math\'{e}\-maticiens}, que re\c{c}oivent tous les adh\'{e}rent(e)s tous les trois mois. La {\em Gazette} a \'et\'e enti\`erement repens\'ee il y a deux ans pour \^etre plus conviviale.
 \item la lettre \'{e}lectronique mensuelle, 
 \item notre compte Twitter @SocMathFr (tr\`es actif!), avec plus de 3300 abonn\'es\
 \item notre site web. Comme notre syst\`eme informatique dans son ensemble a \'et\'e remis \`a jour pour permettre de mieux r\'epondre aux attentes des adh\'erent\mp es et \^etre conformes avec toutes les nouvelles r\'eglementations. Surtout, il nous fournit une assise technique solide pour les ann\'ees qui viennent. N’h\'esitez pas \`a le parcourir et \`a le consulter, de nombreux articles originaux et nos manifestes y paraissent r\'eguli\`erement.
\end{itemize}

Nous parrainons beaucoup de rencontres, dont les annonces sont diffus\'ees sur nos canaux.

\subsection*{Un lieu de r\'{e}flexion} 
\`{A} titre  d'exemple, la SMF s'engage dans les d\'ebats sur la formation des enseignant\mp es ou l'\'{e}volution des licences et masters aussi bien que de l'enseignement secondaire.

Elle prend position sur les structures de l'enseignement sup\'{e}rieur et la recherche.

Elle r\'{e}fl\'{e}chit aux moyens d'attirer les jeunes vers les sciences et vers les m\'{e}tiers scientifiques.  


\subsection*{Un porte-parole de la communaut\'{e} math\'{e}matique}
Sur toutes les questions mentionn\'ees plus haut, la SMF n'est bien s\^ur pas  seule \`a mener des r\'eflexions, et l'un de nos plus importants  r\^oles est celui d'\^etre organisateur  et facilitateur de d\'ebats, comme lors du dernier congr\`es SMF, ou \`a l'initiative de tribunes dont la vocation est de porter la voix des math\'ematiciennes et math\'ematiciens.

 La SMF a vocation, avec ses partenaires, \`{a} \^{e}tre porte-parole de la communaut\'{e} math\'{e}matique fran\c{c}aise aupr\`{e}s des pouvoirs publics, au niveau fran\c cais et international, mais \'egalement aupr\`es de tous les acteurs de la vie publique: m\'edias, entreprises, recruteurs,...

Ainsi, la SMF participe \`a beaucoup de groupes de r\'eflexions, de comit\'es, de jury, ... qu'il serait trop long de mentionner (mais qui n'est pas secr\`ete !!).

\subsection*{Un organisateur de   conf\'erences pour les math\'ematicien(ne)s} 

\begin{itemize}
 \item
Des sessions {\bf Etats de la recherche}   permettent chaque ann\'ee  de s'initier  et de faire un \og \'etat de l'art\fg{} d'un domaine des math\'{e}matiques en pleine expansion. 
\item
Depuis 2017, la SMF a d\'ecid\'e de financer deux fois par an {\bf une semaine de conf\'erence au CIRM} dont les sp\'ecificit\'es sont d'\^etre orient\'ees vers les jeunes (vous!): organisation de mini-cours, journ\'ee r\'eserv\'ee pour des expos\'es aux th\'esard(e)s, post-docs, et jeunes recrut\'e(e)s, avec un financement sp\'ecifique.
\item
En 2016, le {\bf Premier congr\`es de la SMF, SMF2016,} regroupant l'ensemble de la communaut\'e, a \'et\'e organis\'e \`a Tours. Cela a \'et\'e un moment scientifique remarquable  et l'occasion d'intenses \'echanges, toutes les math\'ematiques \'etant repr\'esent\'ees. Devant ce succ\`es, il a \'et\'e  d\'ecid\'e de l'organiser tous les deux ans, le prochain aura lieu \`a Lille en 2018. Venez nombreux vous tenir au courant des derni\`eres avanc\'ees pr\'esent\'ees par celles et ceux qui les ont produites!

\item
La SMF organise \'egalement des {\bf congr\`es joints} avec des soci\'et\'es \'etrang\`eres. Par exemple, un congr\`es avec l'AMS  est en cours de pr\'eparation pour les ann\'ees qui viennent. 

\item Nous participons \`a beaucoup d'autres \'ev\'enements: cette journ\'ee d'accueil des nouveaux ma\^itres de
conf\'erences et charg\'es de recherche bien s\^ur, la remise de prix de l'Acad\'emie des sciences, et bien d'autres...

\end{itemize}


\subsection*{L'organisation et le soutien d'\'{e}v\`{e}nements destin\'es \`a un plus large public} 

\begin{itemize}
 \item
 La SMF coorganise les deux cycles de conf\'erences
{\bf Un texte, un math\'ematicien}   \`a la Biblioth\`{e}que nationale de France  et {\bf Une question, un chercheur} que vous connaissez s\^urement, et qui sont destin\'es aux lyc\'eens et aux classes pr\'eparatoires.
\item
La SMF lance un nouveau cycle {\bf Math\'ematiques \'etonnantes} durant lequel des conf\'erenciers feront d\'ecouvrir, seuls ou en duo avec leur complice, une interaction inattendue entre diff\'erents domaines math\'ematiques ou entre math\'ematiques et applications. Ces conf\'erences de math\'ematiques seront donn\'ees plusieurs fois par an au niveau licence. Le public vis\'e est celui des \'etudiant\mp es d'universit\'e et \'el\`eves de grandes \'ecoles, des professeur\mp es du second degr\'e et des chercheur\mp ses et ing\'enieur\mp es de tout domaine.

\item
Depuis 2017, la SMF organise un {\bf Concours SMF Junior} destin\'e aux \'etudiants encore \`a l'universit\'e. Le but est de promouvoir aupr\`es de ces jeunes la recherche en math\'ematiques sous une forme originale, celle d'un concours par \'equipe sur 10 jours. Je vous invite fortement \`a y participer en proposant des sujets et en motivant vos \'etudiant(e)s !!

\item
La SMF d\'ecerne tous les 2 ans le {\bf Prix d'Alembert}, qui r\'ecompense des initiatives visant \`a la diffusion des math\'ematiques. En 2016, le prix a \'et\'e remis \`a l'association {\em Pi Day} qui organise un spectacle original et scientifique \`a Marseille et en 2018 le laur\'eat f\^ut Mickael Launay.

\item 
La SMF d\'ecerne \'egalement le nouveau {\bf Prix Jacqueline Ferrand} qui r\'ecompense des initiatives p\'edagogiques remarquables autour des math\'ematiques. Le	premier prix Jacqueline Ferrand a \'et\'e attribu\'e en 2018 \`a l’association Maths en vie qui met l’accent mis sur les premiers cycles de l’apprentissage en math\'ematiques, dont l’importance est fondamentale, et l’utilisation parcimonieuse mais extr\^emement pertinente des outils num\'eriques. 

\item
Comme dit plus haut, nous parrainons un tr\`es grand nombre de rencontres et d'\'ev\'enements (Prix Decerf pour la p\'edagogie, Prix Mandelbrot en association avec l'ambassade de Pologne,...), qui sont trop longs \`a lister, mais que vous pouvez consulter \`a tout moment sur notre site!

\end{itemize}


\subsection*{Une maison d'\'{e}dition}
La SMF est une maison d'\'edition ind\'ependante qui publie de nombreux revues et collections. Parlons chiffre:\begin{itemize}
\item
4 journaux: {\bf Bulletins de la SMF, M\'emoires de la SMF, Revue d'Histoire des Math\'ematiques}, et {\bf Ast\'erisque},
\item
Nous sommes les diffuseurs des {\bf Annales de l'ENS}
\item
4 collections d'ouvrages: {\bf Cours sp\'ecialis\'es,
Documents math\'ematiques, Panoramas et Synth\`eses, S\'eminaires et Congr\`es},

\item
d'autres collections: SMF/AMS Texts and Monographs, S\'erie T, S\'erie Chaire Morlet, Fascicules des journ\'ees annuelles.
\item
 8.000 pages uniques par an

\item
 environ 4.000.000 pages imprim\'ees, 
 \item
 environ 50.000 volumes, envoy\'es partout dans le monde,

\item
 plusieurs centaines d'abonn\'es dans le monde
\end{itemize}

Nous g\'erons l'organisation des comit\'es de r\'edaction (qui sont bien s\^ur de tr\`es grande qualit\'e et {\bf ind\'ependants}), la composition des ficheirs (i.e. la remise en format Latex smf et la relecture), l'impression et l'envoi de l'{\bf ensemble de nos revues et collections}. Les salari\'es de la SMF de Paris et Marseille, ainsi que les nombreux b\'en\'tevoles de la SMF (dont vous pouvez faire partie si vous le souhaitez) effectuent cet \'enorme travail depuis de nombreuses ann\'ees, et leur travail garantit la qualit\'e de nos publications.



{\bf La SMF s'engage depuis plusieurs ann\'ees vers le num\'erique et la diffusion} vers le plus grand nombre de son patrimoine scientifique. Nous sommes en train de num\'eriser nos 380 volumes d'{\bf Ast\'erisque} pour les rendre accessibles en ligne,  d'apposer des DOI \`a tous nos documents, de r\'efl\'echir \`a des revues plus \og ouvertes\fg{}, ... Cela demande beaucoup de travail, mais {\bf ces sujets fondamentaux et passionnants ne pourront \'evoluer qu'avec votre participation!}. Rejoignez-nous ! 

\subsection*{Une tutelle du CIRM}  La SMF a \'{e}t\'{e} \`{a} l'origine de la cr\'{e}ation du CIRM. Elle en est aujourd'hui tutelle, conjointement avec le CNRS et Aix-Marseille-Universit\'e. Le CIRM est aujourd'hui le centre international de rencontres math\'{e}matiques qui accueille le plus de visiteurs par an au monde (plus de 3500). Les math\'{e}maticiennes et math\'ematiciens fran\c{c}ais ont eu ou auront tous l'occasion d'y s\'{e}journer lors d'un colloque, d'un groupe de travail, ou d'une {\sl recherche en bin\^{o}me}: consultez   son site \lien{www.cirm.univ-mrs.fr/} et tous les documents notamment vid\'eos disponibles.

La SMF joue un grand r\^ole dans la gestion du CIRM, en termes financier et humain (plusieurs salari\'es de la SMF travaillent au CIRM et/ou pour le CIRM), et elle participe activement aux projets ambitieux de d\'eveloppement du CIRM qui sont en cours.
 
 
 

\subsection*{Notre organisation}

Vous pourrez d\'ecouvrir sur notre site web toute notre organisation. Mais il est important de comprendre que pour faire fonctionner la SMF, beaucoup de personnes sont impliqu\'ees, des salari\'es \'evidemment mais beaucoup de b\'en\'evoles dont j'esp\`ere vous ferez bient\^ot partie. 

L\`a encore, focalisons-nous sur les traits saillants de la SMF:
\begin{itemize}
\item
1850 adh\'erents, qui {\bf votent chaque ann\'ee}
 
 
\item un CA de 24 membres (contactez-nous si vous voulez y participer!)
\item un Bureau de 8 membres issus du CA
\item un Conseil Scientifique (12 membres)
 
\item une Commission d'Enseignement (12 membres)


\end{itemize}
Les personnes (enseignant\mp e\mp s-chercheur\mp se\mp s, chercheur\mp e\mp s, professeurs, industriels, ...) ci-dessus sont des b\'en\'evoles.

Mais la SMF, c'est aussi (et surtout!):

\begin{itemize} 
 \item des bureaux au 4\`eme \'etage de l'IHP, avec 3 salari\'ees \`a temps plein et une salari\'ee en alternance
\item la \og Maison de la SMF\fg{} \`a c\^ot\'e du CIRM, l\`a o\`u nous stockons tous nos volumes (des centaines de milliers) et d'o\`u la diffusion est organis\'ee
\item 2 salari\'es \`a temps plein dans cette Maison de la SMF
\item 3 salari\'es \`a temps plein au CIRM
\end{itemize}

Bref, la SMF rassemble beaucoup de monde, et est le centre de beaucoup d'activit\'es et de r\'eflexion autour des math\'ematiques. Il ne tient qu'\`a vous d'en faire partie et d'y participer, car la SMF est avant tout votre Soci\'et\'e.

\subsection*{Une conclusion: votre implication}





La SMF ne peut vivre que gr\^ace \`a ses adh\'{e}rents ! Elle a en particulier besoin du soutien actif des jeunes math\'ematiciennes et math\'ematiciens. 

Le soutien, cela peut \'evidemment passer par {\bf une adh\'esion}: pour nous rejoindre, c'est facile via notre site, ou m\^eme en allant directement au 4\`e \'etage de l'IHP (pas d'excuses aujourd'hui!).

Mais le soutien se pr\'esente aussi sous la forme de participation \`a nos activit\'es: soumettre vos articles aux revues de la SMF, participer aux conf\'erences et \'ev\'enements de la SMF, r\'epondre positivement \`a nos sollicitations (notamment pour le {\bf concours SMF junior}, aller au CIRM - et r\'ealiser que la SMF est tr\`es active l\`a-bas - et \`a la maison de la SMF):  {\bf cela est fondamental pour nous, et nous conforte dans  notre engagement et notre b\'en\'evolat}.

Pour conclure, soutenez la SMF, et/ou d'autres associations et soci\'et\'es savantes. Tous ensemble (avec la SMAI et la SFdS par exemple), nous sommes des acteurs importants de la vie de la communaut\'e, et beaucoup de nos coll\`egues passent un temps pr\'ecieux \`a des actions fondamentales mais parfois pas assez visibles ou reconnues.

Et quand vous h\'esiterez \`a renouveler votre adh\'esion (si vous h\'esitez!), reprenez ce fascicule, parcourez sur nos sites web, et vous serez convaincu(e), j'esp\`ere, de cliquer sur le bon bouton.




%%%%%%%%%%%%%%%%%%%%%%%%%%%%%%%%%%%%
\section{Soci\'et\'e Math\'ematique Europ\'eenne}
\index{Soci\'et\'e Math\'ematique Europ\'eenne (SME)}

Mentionnons \'egalement la Soci\'et\'e Math\'ematique Europ\'eenne
(SME). Pour plus d'information, consultez son site web \lien{www.emis.de/}.


