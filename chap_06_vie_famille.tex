%%%%%%%%%%%%%%%%%%%%%%%%%%%%%%%%%%%

\chapter{Concilier travail et vie de famille}

Vous pouvez trouver la plupart des informations r\'esum\'ees ici sur
le portail de l'administration fran\c{c}aise:\\
\lien{www.service-public.fr/}

%%%%%%%%%%%%%%%%%%%%%%%%%%%%%%%%%%%%%%
\section{Le cong\'e de maternit\'e}

Toutes les salari\'ees, du priv\'e comme du public, ont droit au
cong\'e de maternit\'e.
Il est \`a noter que vous pouvez d\'ecaler ce cong\'e. Ceci veut dire
que, par exemple pour un premier enfant, vous n'\^etes pas oblig\'ee de
respecter six semaines d'arr\^et pr\'enatal et dix semaines d'arr\^et
postnatal~: vous pouvez reporter une partie du cong\'e pr\'enatal en
cong\'e postnatal apr\`es accord de votre m\'edecin et
\`a condition de conserver un minimum de
2 semaines d'arr\^et pr\'enatal.\\

Si vous \^etes enseignante-chercheuse, vous vous inqui\'eterez ensuite
de savoir quel volume horaire vous aurez \`a enseigner l'ann\'ee de
votre cong\'e. Quelle que soit la date d'accouchement, la d\'echarge de service est de 96h \'eq TD. Une fiche r\'ecapitulative est ici\\
{\footnotesize\lien{www.snesup.fr/conge-de-maternite}}\\
et elle fait r\'ef\'erence \`a la circulaire 2012-0009 du 30-4-2012 valable pour les cong\`es de maternit\'e, de paternit\'e et les cong\`es de maladie

\url{http://www.enseignementsup-recherche.gouv.fr/pid20536/bulletin-officiel.html?cid_bo=60265&cbo=1},


%Auparavant, chaque \'etablissement faisait sa propre cuisine interne. Le seul et unique texte qui pouvait faire r\'ef\'erence \'etait la circulaire DPE A2/FD 892 du 7 novembre 2001, que vous pouvez t\'el\'echarger \`a l'adresse \\ \lien{www.snesup.fr/webuploads/download/492\_0.}\\

%Ce texte vous garantit que, si votre cong\'e tombe int\'egralement pendant l'ann\'ee scolaire, on ne peut vous demander plus de la moiti\'e de votre charge si c'est votre premier ou votre deuxi\`eme enfant,  plus du cinqui\`eme de votre charge si c'est au moins votre troisi\`eme enfant, et aucun service s'il s'agit de naissances multiples. Les probl\`emes commencent bien \'evidemment quand votre cong\'e tombe \`a cheval sur les vacances scolaires...{}\\

% Voici une petite synth\`ese de ce que nous avons pu constater.
%\begin{itemize}
%\item Le d\'egr\`evement horaire simple~: on vous calcule une moyenne
%$N$ d'heures par semaine ``ouverte" (hors vacances
%scolaires). Votre cong\'e de maternit\'e peut alors contenir un minimum
%de six semaines ouvertes et vous obtiendrez donc une d\'echarge de
%$6N$ heures. Appliqu\'e tel quel, ce calcul respecte rarement la
%circulaire.

%\item Le d\'egr\`evement horaire forfaitaire~: quel que soit le moment
%auquel tombe votre cong\'e, on vous retire un nombre forfaitaire
%d'heures, typiquement un tiers de service (cas fr\'equemment
%rencontr\'e). Ce calcul peut \^etre avantageux, sauf si votre
%cong\'e tombe int\'egralement pendant l'ann\'ee scolaire~; il est
%alors en opposition avec la circulaire (sauf dans les rares cas o\`u
%le d\'egr\`evement est d'un demi-service).

%\item Le d\'egr\`evement horaire avec coefficient multiplicateur~: c'est le
%seul qui semble respecter la circulaire, mais c'est de loin le moins
%appliqu\'e. Le calcul est le m\^eme que pour le d\'egr\`evement
%simple, mais le volume horaire restant est multipli\'e par un
%coefficient (0,75 par exemple) pour \^etre en conformit\'e avec la
%circulaire. Ce coefficient est parfois justifi\'e par un
%am\'enagement du temps de travail de la femme enceinte.
%\end{itemize}
%Quel que soit le mode de calcul retenu par votre \'etablissement,
%n'h\'esitez pas \`a contacter le service du personnel pour faire
%respecter vos droits et l'application de la circulaire.

Enfin, sachez que vous pouvez pr\'etendre \`a un CRCT de 6 mois \`a la suite d'un cong\'e maternit\'e, voir la section  \ref{CRCT}. Dans le d\'ecret 84-431 du 6 juin 1984 - Article 19, il est mentionn\'e que ``Un cong\'e pour recherches ou conversions th\'ematiques, d'une dur\'ee de six mois, peut \^etre accord\'e apr\`es un cong\'e maternit\'e ou un cong\'e parental, \`a la demande de l'enseignant$\cdot$e-chercheur$\cdot$se."\\

Nous terminons cette section par quelques liens int\'eressants :\\
%\lien{postes.smai.emath.fr/parite/journee/ConfLBroze.pdf}\\
\lien{postes.smai.emath.fr/apres/parite/}\\
\lien{listes.mathrice.fr/math.cnrs.fr/info/forum-parite}\\
%\lien{postes.smai.emath.fr/parite/ipa.php}\\
%\lien{postes.smai.emath.fr/parite/journee/journee\_parite.php}\\

%%%%%%%%%%%%%%%%%%%%%%%%%%%%%%%%%
\section{Cong\'e parental et temps partiel}

Tout salari\'e a droit de demander un cong\'e parental (dans les trois
premi\`eres ann\'ees suivant une naissance ou une adoption) ou \`a travailler \`a temps
partiel. Ceci est bien s\^ur valable pour les chercheur$\cdot$ses et les
enseignant$\cdot$es-chercheur$\cdot$ses. En cas de cong\'e parental, vous n'\^etes plus
r\'emun\'er\'e$\cdot$e mais vos ann\'ees de cong\'e compteront pour la
retraite. En cas de temps partiel, vous \^etes alors pay\'e$\cdot$e au {\em
pro rata} de votre temps de travail, \`a une exception pr\`es~: si
vous souhaitez vous mettre \`a 80\,\%. Dans ce cas, vous toucherez
85,7\,\% de votre salaire. Il est \`a noter que les primes (par exemple
d'enseignement sup\'erieur et de recherche) ou le suppl\'ement
familial de traitement seront aussi calcul\'es au {\em pro rata}.

Dans le cas o\`u vous avez des enfants en bas \^age, votre Caisse
d'allocations familiales (CAF) pourra vous verser un compl\'ement de r\'emun\'eration. En d\'ebut
de carri\`ere, il est parfois plus avantageux financi\`erement de travailler \`a
80\,\% tant que le compl\'ement CAF peut vous \^etre vers\'e. Une bonne fa\c con
de reprendre l'enseignement en douceur apr\`es un cong\'e maternit\'e, un cong\'e
parental ou simplement l'arriv\'ee d'un enfant puisque le compl\'ement CAF peut \^etre
vers\'e aux jeunes mamans comme aux jeunes papas!

De m\^eme qu'\`a la suite d'un cong\'e maternit\'e, vous pouvez pr\'etendre,
apr\`es un cong\'e parental, \`a un CRCT de 6 mois, voir la section \ref{CRCT}.

%%%%%%%%%%%%%%%%%%%%%%%%%%%%%%%%
\section{Arr\^et maladie ou cong\'e de paternit\'e}

Toujours dans la circulaire 2012-0009 du 30-4-2012 cit\'ee plus haut, il est pr\'ecis\'e qu'on ne peut demander \`a un$\cdot$e
enseignant$\cdot$e-chercheur$\cdot$se de rattraper les heures qu'il n'aurait pu
effectuer suite \`a un arr\^et maladie. Typiquement, si vous \^etes
malade un jour o\`u vous deviez effectuer 10 heures d'enseignement,
ces heures sont consid\'er\'ees comme ayant \'et\'e effectu\'ees et
doivent vous \^etre comptabilis\'ees, tout comme \`a la personne qui
vous a remplac\'e$\cdot$e le cas \'ech\'eant. Et toute heure effectu\'ee en
plus de votre service doit vous \^etre pay\'ee en heure
suppl\'ementaire. Nous ne pouvons donc que vous conseiller de
d\'eposer vos arr\^ets maladie, m\^eme de courte dur\'ee.

La dur\'ee du cong\'e paternit\'e est fix\'ee à 25 jours calendaires. Sur ces 25 jours calendaires, 4 doivent obligatoirement \^etre pris cons\'ecutivement et imm\'ediatement apr\`es le cong\'e de naissance de 3 jours. 
Vous pouvez choisir de prendre la p\'eriode restante de 21 jours calendaires de mani\`ere continue ou fractionn\'ee en 2 p\'eriodes maximum d'au moins 5 jours chacune.
Ces 21 jours doivent \^etre pris dans les 6 mois suivant la naissance.
Pendant votre cong\'e de paternit\'e, vous continuez de toucher en totalit\'e votre traitement indiciaire.

