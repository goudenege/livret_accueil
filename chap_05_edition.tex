\chapter{L'{\'e}dition scientifique\protect\footnote{Chapitre r\'edig\'e par Fr\'ed\'eric H\'elein, Directeur scientifique du \href{http://www.rnbm.org/}{R\'eseau National des Biblioth\`eques de Math\'ematiques}.}}
 
\section{L'{\'e}dition scientifique en pleine mutation}

L'{\'e}dition scientifique et, notamment, le syst{\`e}me des revues publiant des articles de recherche {\'e}voluent constamment
depuis 30 ans, et cette {\'e}volution est loin d'{\^e}tre termin{\'e}e.

L'{\'e}v{\'e}nement majeur des ann{\'e}es 1980 fut l'introduction des logiciels de traitement de texte TeX et LaTeX. On passa ainsi de la frappe
{\`a} la machine {\`a} {\'e}crire des pr{\'e}publications {\`a} la saisie de fichiers {\'e}lectroniques que l'on peut transmettre directement aux {\'e}diteurs des
revues. Puis, au milieu des ann{\'e}es 90, le d{\'e}veloppement d'Internet permit la naissance des premi{\`e}res revues {\'e}lectroniques publiant les
articles en ligne. Ce mode de diffusion s'est g{\'e}n{\'e}ralis{\'e} et est devenu la norme durant la d{\'e}cennie 2000 suivant plusieurs {\'e}tapes~: dans un
premier temps les {\'e}diteurs ont continu{\'e} {\`a} vendre aux biblioth{\`e}ques des abonnements sur papier, en proposant, contre un suppl{\'e}ment financier,
un acc{\`e}s en ligne {\`a} une version {\'e}lectronique des articles. Dans un deuxi{\`e}me temps, les usages ont {\'e}t{\'e} invers{\'e}s et les {\'e}diteurs ont vendu des
abonnements {\'e}lectroniques avec la possibilit{\'e} de recevoir des fascicules imprim{\'e}s sur papier, {\`a} nouveau contre paiement d'un suppl{\'e}ment.
Enfin, dans une troisi{\`e}me phase, qui est arriv{\'e}e {\`a} terme dans de nombreuses disciplines mais pas encore en math{\'e}matiques, on a assist{\'e} {\`a}
une d{\'e}saffection des fascicules papier, les biblioth{\`e}ques concentrant l'essentiel de leurs moyens au r{\`e}glement des acc{\`e}s {\'e}lectroniques,
sauf pour certaines biblioth{\`e}ques soucieuses de constituer un patrimoine d'archives sur papier. Ainsi en pratique lorsqu'un chercheur
ou un {\'e}tudiant veut lire ou t{\'e}l{\'e}charger un article {\'e}lectronique, il faut qu'il soit reconnu comme {\'e}tant un <<~ayant-droit~>> par le portail
{\'e}lectronique de l'{\'e}diteur, ce qui signifie que l'institution ou la biblioth{\`e}que via laquelle il se connecte a pay{\'e} un abonnement. Sinon
l'acc{\`e}s lui sera refus{\'e}, {\`a} moins qu'il accepte de payer en ligne.

Cependant le fonctionnement {\'e}ditorial de la revue n'a pas chang{\'e}, du moins en ce qui concerne les <<~vraies~>> revues scientifiques (ce qui
exclut les revues <<~pirates~>> dont nous parlerons plus loin)~: chaque revue est supervis{\'e}e scientifiquement par un comit{\'e} {\'e}ditorial, constitu{\'e}
de chercheurs, dont la responsabilit{\'e} est de solliciter des rapporteurs anonymes, en leur demandant d'expertiser les articles soumis. Le but
est bien s{\^u}r de s{\'e}lectionner les articles qui seront publi{\'e}s, en veillant {\`a} ce qu'ils soient corrects, originaux et dignes d'int{\'e}r{\^e}t,
t{\^a}che difficile et parfois tributaire de crit{\`e}res subjectifs ou sociologiques. L'organisation de ce travail n{\'e}cessite une part de secr{\'e}tariat
assez importante qui, pour les revues les mieux dot{\'e}es, est assur{\'e} par un ou une secr{\'e}taire. Ce ou cette secr{\'e}taire est le plus souvent r{\'e}mun{\'e}r{\'e}(e)
par des institutions acad{\'e}miques, avec, parfois, une aide financi{\`e}re de l'{\'e}diteur. Mais dans de nombreux cas ce travail est assur{\'e} par des chercheurs
membres du comit{\'e} {\'e}ditorial, parfois indemnis{\'e}s par une petite r{\'e}tribution financi{\`e}re de la part de l'{\'e}diteur\footnote{Cet aspect
 des relations entre comit{\'e}s {\'e}ditoriaux et maisons d'{\'e}dition est toutefois assez opaque en g{\'e}n{\'e}ral.}

Ces {\'e}volutions ont permis de r{\'e}duire consid{\'e}rablement les co{\^u}ts de fonctionnement pour les {\'e}diteurs, ceux-ci n'ayant plus {\`a} composer les fascicules
en imprimerie et n'ayant pratiquement plus {\`a} les imprimer, ni {\`a} les stocker et les exp{\'e}dier. N{\'e}anmoins ces co{\^u}ts restent non nuls car les {\'e}diteurs
continuent {\`a} contribuer partiellement au fonctionnement du comit{\'e} {\'e}ditorial de certaines revues, {\`a} mettre aux normes les fichiers {\'e}lectroniques
avant leur publication. Ils doivent aussi concevoir des plate-formes pour la mise en ligne (y compris les syst{\`e}mes de p{\'e}age ou d'identification
des <<~ayant-droit~>> et excluant les <<~pas-ayant-droit~>> !)
et, pour les revues soucieuses d'un bon niveau de qualit{\'e}, assurer une relecture de la langue des articles. De plus comme
le chercheur du 21{\`e}me si{\`e}cle veut pouvoir acc{\'e}der {\`a} un article en quelques clics {\`a} partir de donn{\'e}es partielles (l{\`a} o{\`u} son anc{\^e}tre du 20{\`e}me si{\`e}cle
devait faire des d{\'e}marches dans sa biblioth{\`e}que et parfois attendre que celle-ci lui procure ce dont il avait besoin), les {\'e}diteurs doivent aussi
produire une sorte de carte d'identit{\'e} {\'e}lectronique de l'article contenant des informations sur l'article et ses auteurs, que l'on appelle <<~m{\'e}tadonn{\'e}es~>>.
Ces co{\^u}ts peuvent repr{\'e}senter une charge non n{\'e}gligeable pour les petits {\'e}diteurs (lesquels, dans les cas o{\`u}, par exemple, ils ne peuvent pas assurer
le travail de mise en ligne de leur revues, sont oblig{\'e}s de payer des prestataires de service comme JSTOR). En revanche les gros {\'e}diteurs, publiant des
centaines de revues, r{\'e}alisent de tr{\`e}s grandes {\'e}conomies d'{\'e}chelle sur ces co{\^u}ts de mise en ligne. 

Cette baisse des co{\^u}ts de production et la d{\'e}mat{\'e}rialisation des publications sous une forme {\'e}lectronique ont eu plusieurs cons{\'e}quences. 

\subsection{Les avantages des bouquets}

\subsubsection{Pour les chercheurs et leurs institutions} 
La premi{\`e}re cons{\'e}quence fut un gain pour les institutions de recherche (mais nous verrons le revers de la m{\'e}daille un peu plus loin... )~:
au lieu de s'abonner {\`a} une liste restreinte de revues, {\`a} la mesure de leur budget et choisies suivant les priorit{\'e}s scientifiques,
les biblioth{\`e}ques, en se regroupant en consortia, ont pu s'abonner {\`a} des bouquets de revues (dont le principe est analogue {\`a}
celui des bouquets de cha{\^\i}nes de t{\'e}l{\'e}vision) pour un co{\^u}t qui sembla raisonnable au d{\'e}but. Cela fut particuli{\`e}rement b{\'e}n{\'e}fique aux
petites universit{\'e}s ou aux petites institutions qui eurent ainsi acc{\`e}s {\`a} des dizaines, voire des centaines de revues {\'e}lectroniques,
alors qu'auparavant, elle ne pouvaient s'offrir que quelques revues au mieux. Cela repose sur des accords qui sont n{\'e}goci{\'e}s en amont
par le consortium et dans lesquels chaque biblioth{\`e}que adh{\'e}rente s'engage {\`a} payer une partie de la facture globale. En France les
premiers accords de ce type furent conclus pour les math{\'e}matiques par le \href{http://www.rnbm.org/}{RNBM}
(R{\'e}seau National des Biblioth{\`e}ques de Math{\'e}matiques)
avec l'{\'e}diteur Springer. Aujourd'hui les n{\'e}gociations men{\'e}es par le RNBM sont chapeaut{\'e}es par le Consortium
\href{http://www.couperin.org/}{\emph{Couperin}}, qui regroupe
toutes les Universit{\'e}s et la plupart des Institutions de recherche en France. En effet le RNBM est maintenant un Groupement de Service de
l'INSMI, qui est lui-m{\^e}me un Institut au sein du CNRS, lequel CNRS est membre du consortium \emph{Couperin}...


\subsubsection{Pour les {\'e}diteurs}
La deuxi{\`e}me cons{\'e}quence fut un gain pour les gros {\'e}diteurs: gr{\^a}ce aux importantes {\'e}conomies d'{\'e}chelle qu'elles ont pu faire,
les tr{\`e}s grosses
compagnies comme \emph{Reed Elsevier} (maintenant \emph{RELX Group}), \emph{Springer Nature}, \emph{Wiley}, \emph{Wolters Kluwer},
\emph{Informa} (\emph{Taylor \& Francis})
r{\'e}alisent toutes aujourd'hui des marges op{\'e}rationnelles sup{\'e}rieures {\`a} 24\% (l'industrie pharmaceutique est d{\'e}pass{\'e}e!)
et d{\'e}passant m{\^e}me 37\% pour
\emph{Reed Elsevier} et \emph{Springer Nature}, ce qui constitue un record toutes cat{\'e}gories (on pourra consulter  {\`a} ce sujet la
\href{http://www.eprist.fr/wp-content/uploads/2016/03/I-IST_16_R{\'e}sultatsFinanciers2015EditionScientifique.pdf}{note de l'EPRIST du 30 mars 2016}).

Mais un des probl{\`e}mes, c'est que ces b{\'e}n{\'e}fices spectaculaires ne s'expliquent pas uniquement par la baisse des co{\^u}ts de production, car ils sont
r{\'e}alis{\'e}s sur le dos des institutions publiques (universit{\'e}s, organismes de recherche) ou de certaines industries de pointe. En effet ils doivent
beaucoup aux augmentations fortes et incessantes des prix des abonnements, lesquelles semblent difficiles {\`a} justifier. Comment se fait-il
que les biblioth{\`e}ques acceptent de payer chaque ann{\'e}e des sommes toujours plus grandes ? La r{\'e}ponse est que, bien que le march{\'e} soit
partag{\'e} entre une multitude d'entreprises, des plus petites aux plus grosses comme \emph{Reed Elsevier} ou \emph{Springer Nature},
ce march{\'e} est sans concurrence,
car chaque revue est unique.

\subsection{Les effets pervers des bouquets}

De plus, les grosses entreprises tirent profit du syst{\`e}me de bouquets de revues. La strat{\'e}gie consiste {\`a} proposer aux biblioth{\`e}ques le choix entre
des abonnements {\`a} la carte aux revues qui les int{\'e}ressent, mais {\`a} des tarifs prohibitifs, et un abonnement {\`a} un bouquet (<<~big deal~>>). Pour plusieurs
raisons, les n{\'e}gociateurs choisissent le plus souvent la seconde option. La premi{\`e}re de ces raisons est que la diff{\'e}rence de prix est {\'e}norme~: pour une
somme comparable, les biblioth{\`e}ques ont ainsi acc{\`e}s {\`a} des centaines de revues au lieu de quelques unes ou de quelques dizaines (suivant la taille de 
l'universit{\'e}). Enfin, c'est beaucoup plus simple pour le n{\'e}gociateur (et il faut reconna{\^\i}tre que les contrats propos{\'e}s par les {\'e}diteurs sont au moins
aussi opaques que les forfaits des op{\'e}rateurs t{\'e}l{\'e}phoniques). Mais une fois que les n{\'e}gociateurs sont ainsi <<~guid{\'e}s~>> (pour ne pas dire <<~forc{\'e}s~>>) vers
le choix d'un <<~big deal~>>, il devient alors tr{\`e}s difficile de n{\'e}gocier son montant global, puisque la discussion porte sur <<~tout ou rien~>>~: le n{\'e}gociateur
ne peut pas prendre la responsabilit{\'e} de revenir vers les universit{\'e}s en disant qu'il n'y aura pas d'acc{\`e}s aux revues Elsevier ou Springer l'ann{\'e}e
prochaine et l'{\'e}diteur\footnote{N{\'e}anmoins les Pays-Bas ont r{\'e}cemment boycott{\'e} Elsevier et l'Allemagne fait de m{\^e}me depuis d{\'e}but 2017,
mais il s'agit plus de moyens de pression pour une n{\'e}gociation que de v{\'e}ritables boycotts.}
le sait tr{\`e}s bien... Voir par exemple  \href{http://alambic.hypotheses.org/6245}{L'Alambic num{\'e}rique du 6 d{\'e}cembre 2016}

Un autre effet ind{\'e}sirable du syst{\`e}me de bouquets est que, comme les contrats en question avec les gros {\'e}diteurs sont en g{\'e}n{\'e}ral sign{\'e}s pour plusieurs
ann{\'e}es (parce qu'ainsi l'{\'e}diteur propose un prix plus bas sur une plus longue p{\'e}riode) et comme les budgets des biblioth{\`e}ques sont {\`a} la baisse d'une
ann{\'e}e sur l'autre, les biblioth{\`e}ques sont oblig{\'e}es de se d{\'e}sabonner aux revues des petits {\'e}diteurs pour {\'e}quilibrer leur budget. Du coup, ce sont pr{\'e}cis{\'e}ment
ces petits {\'e}diteurs, lesquels n'ont pas acc{\`e}s aux {\'e}conomies d'{\'e}chelle, qui font les frais de ce syst{\`e}me in fine. Cela concourt {\`a} faire dispara{\^\i}tre les
petites maisons d'{\'e}dition, ainsi rachet{\'e}es par les gros {\'e}diteurs, qui s'en trouvent ainsi renforc{\'e}s, alimentant de la sorte un cercle vicieux.

Les bouquets n'ont pas seulement des effets pervers sur les d{\'e}penses des institutions qui financent la recherche et sur les petites revues, mais aussi
sur le plan scientifique. Auparavant les biblioth{\`e}ques se devaient de s{\'e}lectionner rigoureusement les revues auxquelles elles s'abonnaient, ce qui
obligeait les revues {\`a} maintenir un niveau et une qualit{\'e} scientifique suffisantes pour survivre. Aujourd'hui ce m{\'e}canisme de s{\'e}lection naturelle des
revues n'existe plus et on assiste {\`a} une prolif{\'e}ration des revues, dont certaines n'auraient pas fait long feu dans l'ancien {\'e}cosyst{\`e}me, ce que l'on
peut regretter.

Beaucoup de chercheurs et de biblioth{\'e}caires, ainsi que les consortia et les {\'e}tablissements, sont aujourd'hui pleinement
conscients de ces probl{\`e}mes et {\`a} la recherche de solutions.

\subsection{Un autre effet ind{\'e}sirable de l'{\'e}lectronique~: la r{\'e}tention de l'information}

En th{\'e}orie le passage {\`a} l'{\'e}lectronique permet de d{\'e}cupler les possibilit{\'e}s de diffusion des informations scientifiques et d'offrir l'acc{\`e}s aux r{\'e}sultats
de la recherche {\`a} plus de chercheurs et de citoyens. C'est en grande partie vrai, mais, paradoxalement, {\c c}a n'est pas forc{\'e}ment le cas pour beaucoup
d'articles publi{\'e}s dans des revues scientifiques. En effet une biblioth{\`e}que ne paye plus pour acqu{\'e}rir et conserver ind{\'e}finiment un document sur papier,
mais pour acc{\'e}der {\`a} une information d{\'e}mat{\'e}rialis{\'e}e, dont certains {\'e}diteurs gardent abusivement le contr{\^o}le. Ceux-ci peuvent alors demander un droit de p{\'e}age
pour chaque usage~: lire les publications de l'ann{\'e}e en cours, consulter des archives des ann{\'e}es pr{\'e}c{\'e}dentes ou effectuer de la fouille de donn{\'e}es.

\subsection{La r{\'e}action du monde de la recherche}

Face {\`a} ces abus des chercheurs ont cherch{\'e} {\`a} r{\'e}agir. En 2012 le math{\'e}maticien Tim Gowers a lanc{\'e} une p{\'e}tition et un appel au boycott de l'{\'e}diteur
Elsevier (\href{http://www.thecostofknowledge.com/}{\emph{The Cost of Knowledge}}~: l'engagement {\`a}
ne plus publier, ni accepter d'{\^e}tre rapporteur ou {\'e}diteur pour une revue Elsevier), qui a
recueilli une assez forte adh{\'e}sion, mais dont les effets restent limit{\'e}s. En effet la situation ne pourra pas {\'e}voluer tant que des mod{\`e}les
{\'e}conomiques stables permettant de s'affranchir du joug des {\'e}diteurs commerciaux ne seront pas en place. Diff{\'e}rents projets ont vu le jour en ce sens.

\section{L'\emph{Open Access} ou l'acc{\`e}s libre}

\subsection{L'\emph{Open Access} r{\^e}v{\'e} par les chercheurs}

La r{\'e}ponse id{\'e}ale consisterait {\`a} profiter des possibilit{\'e}s d'internet pour rendre accessible {\`a} tout le monde tous les  contenus des revues scientifiques.
Etant donn{\'e} que les co{\^u}ts de mise en ligne sont beaucoup plus bas que ceux de l'{\'e}dition traditionnelle, ceux-ci pourraient {\^e}tre support{\'e}s par les
institutions publiques et celle-ci pourraient ainsi r{\'e}aliser des {\'e}conomies. Ce r{\^e}ve, qui avait pour nom \emph{Open Access}, avait {\'e}t{\'e} publiquement formul{\'e}
en 2001 dans la d{\'e}claration de Budapest (\href{http://www.budapestopenaccessinitiative.org/}{\emph{Budapest Open Access Initiative}}),
laquelle d{\'e}claration avait
suscit{\'e} une belle frayeur chez les {\'e}diteurs.

\subsection{L'\emph{Open Access} revisit{\'e} par les {\'e}diteurs}

Des ann{\'e}es plus tard, les {\'e}diteurs se sont appropri{\'e}s le concept d'\emph{Open Access} et l'ont retourn{\'e} {\`a} leur avantage. Leur id{\'e}e est de rendre les articles
accessibles gratuitement en ligne, certes, mais en faisant payer l'auteur, ou son institution des frais de publication, appel{\'e}s souvent Publications
fees ou APC, pour Article Processing Charges. Pr{\'e}cisons que les frais de publication sont le plus souvent autour de 2000 {\`a} 3000 \euro (mais peuvent
atteindre 6 000 ou 7 000 \euro pour certaines revues). Mentionnons {\'e}galement le premier probl{\`e}me, {\'e}vident, que pose le paiement des APC par l'auteur~:
il cr{\'e}e une in{\'e}galit{\'e} entre les chercheurs pour faire reconna{\^\i}tre leur travaux, in{\'e}galit{\'e} reposant sur les finances de leur laboratoire, de leur universit{\'e},
de leur pays ou des plans de financement nationaux ou europ{\'e}ens dont ils peuvent b{\'e}n{\'e}ficier. Ce point crucial est malheureusement souvent omis dans beaucoup
d'analyses que l'on peut lire.

Nous sommes donc en face de plusieurs projets se r{\'e}clamant tous de l'\emph{Open Access}~: d'une part, ceux propos{\'e}s par les {\'e}diteurs, d'autre part, ceux que les
chercheurs ou leurs institutions tentent de mettre en place, afin d'{\'e}viter un mod{\`e}le dans lequel les chercheurs sont oblig{\'e}s de payer ou, au moins,
d'en limiter les d{\'e}g{\^a}ts. De plus nous sommes en Europe plus ou moins oblig{\'e}s de trouver une ou plusieurs solutions, car, parmi les objectifs fix{\'e}s dans
\href{https://ec.europa.eu/research/science-society/document_library/pdf_06/recommendation-access-and-preservation-scientific-information_fr.pdf}{Horizon 2020}
par la communaut{\'e} europ{\'e}enne, figurent celui de diffuser tous les r{\'e}sultats de la recherche en \emph{Open Access}!


\subsection{O{\`u} en est-on?}

La situation est en fait plus complexe encore que ce que l'on pourrait croire d'apr{\`e}s ce qui pr{\'e}c{\`e}de, tant pour les chercheurs et leurs institutions
que pour les {\'e}diteurs. A nouveau dans ce qui suit, il convient de distinguer parmi les {\'e}diteurs les groupes importants, disposant de moyens financiers
gigantesques pour investir et s'adapter {\`a} toutes ces {\'e}volutions, des petits {\'e}diteurs, qui ont le plus grand mal {\`a} suivre.

Le premier constat, paradoxal, est que, malgr{\'e} le d{\'e}veloppement rapide de l'\emph{Open Access} depuis une d{\'e}cennie, la principale d{\'e}pense des institutions et
principale source de revenues pour les {\'e}diteurs reste, de loin, les abonnements aux revues. Cela est vrai m{\^e}me au Royaume-Uni, pourtant engag{\'e} depuis
2013 dans une politique de publication syst{\'e}matique des articles de ses chercheurs en \emph{Open Access}. Le mod{\`e}le {\'e}conomique traditionnel, fond{\'e} sur les abonnements,
est donc remarquablement stable. On peut d{\'e}celer plusieurs raisons pour expliquer la lenteur d'une transition de l'ancien mod{\`e}le vers un ou plusieurs
mod{\`e}les \emph{Open Access}. Du c{\^o}t{\'e} des {\'e}diteurs, cette lenteur n'est pas un probl{\`e}me mais une b{\'e}n{\'e}diction, car la transition a un co{\^u}t mais, comme nous le
verrons plus loin, ce co{\^u}t suppl{\'e}mentaire est support{\'e} par les institutions publiques, ce qui se traduit par une majoration des b{\'e}n{\'e}fices qu'ils r{\'e}alisent.

Pour les institutions, cette lenteur est souvent due au fait que les d{\'e}cideurs tardent {\`a} prendre position, comme c'est le cas notamment en France.
Il faut {\`a} ce sujet reconna{\^\i}tre la difficult{\'e} de mettre en place une politique~: d'abord parce que nous sommes dans une situation transitoire et qu'il
est difficile de prendre du recul et encore plus d'anticiper sur les {\'e}volutions futures. Ensuite parce que les diff{\'e}rents acteurs ne sont pas toujours
d'accord~: le probl{\`e}me ne se pose de la m{\^e}me fa{\c c}on pour les universit{\'e}s, pour les organismes comme le CNRS et, \emph{a fortiori}, pour les industries de pointe
(quant {\`a} l'Acad{\'e}mie des Sciences, laquelle n'est pas concern{\'e}e par le r{\`e}glement des factures des biblioth{\`e}ques, mais au contraire per{\c c}oit de l'argent
d'un {\'e}diteur comme Elsevier, ses avis trahissent une vision un peu trop simplifi{\'e}e du probl{\`e}me). De plus, {\`a} l'int{\'e}rieur de chacune de ces instances,
plusieurs sensibilit{\'e}s diff{\'e}rentes peuvent s'opposer. Enfin les points de vue, les besoins et les habitudes peuvent varier radicalement selon des
disciplines comme la biologie et la m{\'e}decine, d'un c{\^o}t{\'e}, et les math{\'e}matiques et les sciences humaines et sociales, dont les budgets sont beaucoup
plus chiches, de l'autre.

En tr{\`e}s gros, pour les institutions, le d{\'e}bat porte sur le choix entre confier aux {\'e}diteurs priv{\'e}s la gestion de la transition vers l'\emph{Open Access},
en n{\'e}gociant avec eux pour tenter d'obtenir les meilleurs conditions financi{\`e}res possibles, ou bien construire des mod{\`e}les d'{\'e}dition g{\'e}r{\'e}s par les
institutions publiques. Pour cette deuxi{\`e}me voie, les bonnes volont{\'e}s et les id{\'e}es ne manquent pas parmi les chercheurs et les professionnels de
l'information scientifique (biblioth{\'e}caires, documentalistes, informaticiens), mais, en g{\'e}n{\'e}ral, des moyens financiers importants et une r{\'e}elle
volont{\'e} politique font d{\'e}faut.

Du point de vue des {\'e}diteurs, {\'e}tant donn{\'e} qu'il est en g{\'e}n{\'e}ral difficile de basculer instantan{\'e}ment une revue financ{\'e}e par des abonnements en une
revue \emph{Open Access} et, ce, d'autant plus que cette revue est prestigieuse, deux types de strat{\'e}gie sont mises en place. La premi{\`e}re consiste {\`a} cr{\'e}er
\emph{ex nihilo} de nouvelles revues \emph{Open Access}, financ{\'e}es par les paiements des APC. Mais alors la difficult{\'e} est que ces revues ne b{\'e}n{\'e}ficient pas
\emph{a priori} de la renomm{\'e}e et du prestige des revues d{\'e}j{\`a} bien {\'e}tablies, et donc les {\'e}diteurs ne peuvent pr{\'e}tendre que les auteurs accepteront de payer
des APC {\'e}lev{\'e}s pour y publier (mais cette transition a toutefois {\'e}t{\'e} <<~r{\'e}ussie~>> en m{\'e}decine). De ce fait, le montant des APC pour ces revues se situe
en moyenne entre 300 et 600 \euro. La deuxi{\`e}me strat{\'e}gie est une pratique beaucoup plus contestable, qui consiste {\`a} transformer des revues anciennement
financ{\'e}es par les abonnements en revues hybrides. Expliquons cela.

\subsection{Les revues hybrides ou comment les institutions payent deux fois des articles offerts gratuitement par ses chercheurs}

Une revue hybride est une revue {\`a} laquelle il est n{\'e}cessaire de payer un abonnement, si l'on veut acc{\'e}der {\`a} la totalit{\'e} de son contenu, mais qui contient
des articles en OpenAccess, {\`a} condition que les auteurs de ces articles aient vers{\'e} des APC pour cela. En somme, non seulement l'auteur c{\`e}de gratuitement
les droits patrimoniaux de son article {\`a} l'{\'e}diteur, et non seulement sa biblioth{\`e}que doit payer pour qu'il puisse acc{\'e}der {\`a} tous les articles de la revue
en question, mais encore l'auteur paye pour {\^e}tre lu gratuitement par d'autres. Les {\'e}diteurs ont coutume de justifier cette pratique en expliquant que les
versements des APC contribuent {\`a} baisser les co{\^u}ts des abonnements, mais en r{\'e}alit{\'e}, on n'observe pas de baisses de ces co{\^u}ts. C'est pourquoi,
comme l'a
\href{http://www.cnrs.fr/comitenational/doc/recommandations/2016/Recommandation-csi-INSMI-au-sujet-des-frais-de-publication-(APC).pdf}{recommand{\'e} le Conseil Scientifique de l'INSMI}
en 2016, \textbf{cette option doit {\^e}tre {\'e}vit{\'e}e},
d'autant plus que, comme nous le verrons plus loin, il existe
un moyen pour rendre gratuitement accessible les contenus de ses publications~: la voie verte.

Ajoutons que le syst{\`e}me des revues hybrides pr{\'e}sente pour les gros {\'e}diteurs d'autres avantages que ceux imm{\'e}diats d'une source compl{\'e}mentaire de revenus,
dans la mesure o{\`u} il permet d'{\'e}viter une transition brutale du mod{\`e}le avec abonnements vers un mod{\`e}le exclusivement \emph{Open Access} avec paiement d'APC.
En effet, dans une telle transition, si des entreprises comme Elsevier souhaitaient garder leur chiffre d'affaire, elles seraient oblig{\'e}es de r{\'e}clamer des
APC sup{\'e}rieurs {\`a} 7~000 \euro par article, ce qui mettrait au grand jour la r{\'e}alit{\'e}
des prix auxquels on est arriv{\'e} aujourd'hui et causerait une certaine {\'e}motion. Cela comporterait ainsi
le risque de remettre en question ces tarifs, que les {\'e}diteurs pr{\'e}f{\`e}rent certainement {\'e}viter de courir. 

\textbf{Attention~!} Le choix de publier un article sous une forme hybride, contre paiement d'un APC, n'est pas toujours clairement expliqu{\'e} lorsque l'{\'e}diteur
vous demande de remplir en ligne un contrat d'{\'e}dition. Cette option est appel{\'e}e <<~Open choice~>> chez certains {\'e}diteurs. 
De plus \textbf{il est parfois difficile de revenir sur ce choix, et certains {\'e}diteurs interdisent
de le faire}, ce qui semble contredire l'article
\href{https://www.legifrance.gouv.fr/affichCodeArticle.do?cidTexte=LEGITEXT000006069565&idArticle=LEGIARTI000006292075&dateTexte=&categorieLien=cid}{L121-20-12}
du Code de la consommation. 
Il convient donc d'{\^e}tre vigilant au moment o{\`u} l'on renseigne un contrat de
publication en ligne, surtout d{\`e}s qu'on voit le mot <<~\emph{Open Access}~>>.
Si le mal est fait, il est pr{\'e}f{\'e}rable de contacter l'{\'e}diteur commercial avant 14 jours pour lui demander d'annuler la commande
(cf. les articles
\href{https://www.legifrance.gouv.fr/affichCodeArticle.do?cidTexte=LEGITEXT000006069565&idArticle=LEGIARTI000006292075&dateTexte=&categorieLien=cid}{L121-20-12}
et \href{https://www.legifrance.gouv.fr/affichCodeArticle.do?cidTexte=LEGITEXT000006069565&idArticle=LEGIARTI000024039758}{L122-3}
du Code de la Consommation).

\subsection{Les revues pirates}

Un autre effet ind{\'e}sirable de l'\emph{Open Access} est le d{\'e}veloppement de <<~revues pirates~>>~: dans les pires des cas, il s'agit de revues tout {\`a} fait
factices, sans comit{\'e} de r{\'e}daction, ni processus de review et qui donc publient n'importe quoi, dans un simple but commercial. Il est clair que
l'existence m{\^e}me de ces revues n'est possible que parce que des chercheurs (ou des personnes souhaitant se faire passer pour des chercheurs) payent
pour publier. Nous devons mettre en garde contre ces revues, qui choisissent des noms ronflants {\'e}voquant ceux de revues prestigieuses et qui
affichent des adresses qui inspirent confiance (comme par exemple, une ville universitaire du
monde anglo-saxon), lesquelles peuvent n'{\^e}tre que de simple bo{\^\i}tes {\`a} lettres.
Enfin, entre ces revues pirates et des revues tout {\`a} fait recommandables, s'{\'e}tale une zone grise de revues, qui ne m{\'e}ritent pas le qualificatif
de <<~pirate~>> et dont le fonctionnement satisfait plus ou moins les r{\`e}gles habituelles, mais dont le niveau scientifique et l'exigence sont largement
discutables. On peut retrouver des revues de ce type dans les bouquets auxquelles les biblioth{\`e}ques sont abonn{\'e}es, mais aussi bien s{\^u}r parmi les revues
\emph{Open Access}. La prolif{\'e}ration de ces revues pose probl{\`e}me, dans la mesure o{\`u}, le plus souvent, elles contribuent {\`a} diluer la connaissance
dans un corpus o{\`u} l'on ne se retrouve plus (et que personne ne lit), augmentant le risque de publier des r{\'e}p{\'e}titions, des plagiats, quand il ne
s'agit pas d'articles faux.

\section{Que font les institutions en France et dans le Monde?}

\subsection{D{\'e}velopper des revues Open Access sans frais de publications}

\subsubsection{En France}

En math{\'e}matiques, l'INSMI, via la cellule Mathdoc, d{\'e}veloppe la plate-forme  \href{http://www.cedram.org}{\emph{Cedram}}
proposant des revues de math{\'e}matiques
en \emph{Open Access}~: ainsi les \emph{Annales Blaise Pascal}, de l'Universit{\'e} de Clermont-Ferrand,
les \emph{Annales de l'Institut Fourier} sont devenues d'acc{\`e}s libres,
sans paiement d'APC. Le \emph{Journal de l'Ecole Polytechnique} est publi{\'e} {\`a} nouveau, suivant le m{\^e}me mod{\`e}le et d'autres revues acad{\'e}miques de math{\'e}matiques
devraient suivre le mouvement. Cependant le nombre de revues concern{\'e}es est pour l'instant limit{\'e}, peut-{\^e}tre parce que les moyens mis en place sont pour
l'instant plus modestes qu'en Sciences Humaines (voir plus loin). 

Le projet \href{https://www.episciences.org/}{\emph{Episciences}} a pour ambition de proposer des revues dans toutes les disciplines. La particularit{\'e} de ce
projet est d'utiliser les d{\'e}p{\^o}ts de pr{\'e}publications comme HAL et arXiv comme support de publication~: un {\'e}pijournal est une structure dot{\'e}e d'un comit{\'e}
{\'e}ditorial fonctionnant suivant les m{\^e}mes r{\`e}gles qu'un journal traditionnel, mais dont les articles sont simplement mis en ligne sur une banque de
pr{\'e}publications. Cela permet de r{\'e}duire les co{\^u}ts de publication au minimum (m{\^e}me s'ils restent non nuls). Ce projet est mis en {\oe}uvre par le CCSD,
une Unit{\'e} Mixte de Service du CNRS (dont par ailleurs l'activit{\'e} principale est la gestion du portail HAL), avec un soutien d'Inria. Cependant,
malgr{\'e} une certaine publicit{\'e}, ce projet d{\'e}marre lentement, faute, pour l'instant, de moyens suffisants et donc d'une r{\'e}elle volont{\'e} politique de le soutenir.

Mais le plus grand succ{\`e}s en France est celui des sciences humaines, qui ont r{\'e}ussi {\`a} se doter de moyens importants et {\`a} d{\'e}velopper un mod{\`e}le
d'{\'e}dition en \emph{Open Access} qui f{\'e}d{\`e}re la moiti{\'e} de revues francophones (regroup{\'e}es dans le portail Revues.org, lui-m{\^e}me int{\'e}gr{\'e} au plus vaste
projet \href{https://www.openedition.org/}{\emph{OpenEdition}}.

\subsubsection{\`A l'{\'e}tranger}

Ce sont {\`a} nouveau les Sciences Humaines qui sont {\`a} la pointe, avec, par exemple, le projet \href{http://www.knowledgeunlatched.org/}{\emph{Knowledge Unlatched}}, dont le but est la publication de
livres {\'e}lectroniques (e-books) d'acc{\`e}s gratuits, financ{\'e}s par des souscriptions aupr{\`e}s de biblioth{\`e}ques ou encore \href{https://www.openlibhums.org/}{\emph{Open Library of Humanities}},
un projet similaire pour les revues. En math{\'e}matiques, la fondation \href{https://compositio.nl/#}{\emph{Compositio Mathematica}}, n{\'e}e aux Pays-Bas, poursuit le m{\^e}me objectif. Enfin
le projet le plus important est  \href{http://www.scielo.org}{\emph{SciELO}}, initiative br{\'e}silienne {\`a} laquelle ont adh{\'e}r{\'e} la plupart des pays d'Am{\'e}rique
latine, qui h{\'e}berge plus d'un millier de revues dont beaucoup sont \emph{Open Access} sans frais de publication. Ce projet souffre n{\'e}anmoins du fait
que beaucoup de revues y sont nouvelles, b{\'e}n{\'e}ficient donc d'un prestige limit{\'e} et peinent {\`a} attirer les bons articles, notamment en math{\'e}matiques.

Afin de se rep{\'e}rer dans ce paysage complexe, en pleine {\'e}volution, on pourra consulter des sites proposant des listes de revues \emph{OpenAccess}
avec des informations sur leurs pratiques comme \href{http://www.sherpa.ac.uk/romeo/index.php}{\emph{SHERPA/RoMEO}}.

\subsection{La voie verte et la loi <<~pour une R{\'e}publique Num{\'e}rique~>>}

Il s'agit d'une solution pour rendre accessibles les r{\'e}sultats de la recherche et r{\'e}pondre ainsi aux objectifs de l'Horizon 2020, sans avoir {\`a}
payer des APC pour cela, mais en perp{\'e}tuant le financement des revues par des abonnements. L'id{\'e}e est de poster sur des portails comme arXiv ou
HAL les contenus des articles publi{\'e}s par ailleurs dans des revues. Pour permettre aux revues de continuer {\`a} vendre des abonnements et, ainsi,
de vivre, ces contenus sont mis en lignes une fois qu'une p{\'e}riode minimale, appel{\'e}e p{\'e}riode d'embargo, s'est {\'e}coul{\'e}e depuis la publication de l'article.
C'est ce qu'on appelle la voie verte ou celle du Green \emph{Open Access}.

Suivant d'autres pays, la France s'est dot{\'e}e en 2016 d'un texte l{\'e}gislatif pr{\'e}cisant le mode d'application de ce principe pour les chercheurs financ{\'e}s
au moins pour moiti{\'e} par les deniers publics fran{\c c}ais. Il s'agit de l'article 30 de la loi <<~pour une R{\'e}publique Num{\'e}rique~>> promulgu{\'e}e le 7 octobre 2016
devenu \href{https://www.legifrance.gouv.fr/affichCodeArticle.do;jsessionid=B2842ED6626DE6921F9FD32EA69A3C93.tpdila15v_2?idArticle=LEGIARTI000033205794&cidTexte=LEGITEXT000006071190&dateTexte=20170124}{Article L533-4}
du \emph{Code de la recherche}.
Ainsi, voici ce que vous avez le droit de faire~:
\begin{itemize}
\item lorsque vous avez {\'e}crit une pr{\'e}publication, vous pouvez la poster sur HAL, arXiv ou toute autre banque d'articles accessible {\`a} tous imm{\'e}diatement
et vous pouvez y laisser ce texte ind{\'e}finiment. Vous pouvez remplacer cet article par une version plus r{\'e}cente, tant que celle-ci est ant{\'e}rieure {\`a} la date
{\`a} laquelle vous signez de contrat de cession de droits {\`a} un {\'e}diteur pour le publier. Cette disposition {\'e}tait d{\'e}j{\`a} valable avant la loi du 7 octobre 2016,
car elle d{\'e}coule du code de la propri{\'e}t{\'e} intellectuelle.
\item lorsque, une fois votre article r{\'e}vis{\'e} par le comit{\'e} de r{\'e}daction de la revue et une fois celui-ci accept{\'e}, vous signez un contrat pour sa publication,
alors, le plus souvent (mais cela d{\'e}pend de la politique de l'{\'e}diteur), vous n'avez pas le droit de mettre tout de suite en ligne le contenu de votre article
mot pour mot (et formule pour formule). Mais (et c'est l{\`a} une disposition de la nouvelle loi), pass{\'e} un d{\'e}lai maximum de 6 mois, vous avez le droit de le faire
(ce d{\'e}lai maximal est de 12 mois pour les Sciences Humaines). La loi ne s'applique pas  au fichier produit par l'{\'e}diteur (il est possible que vous n'ayez
jamais le droit de mettre en ligne).
\end{itemize}
\medskip
\noindent
Cette disposition est-elle r{\'e}tro-active~? 
En g{\'e}n{\'e}ral ce n'est pas le cas. Cependant il existe des exceptions\footnote{notamment si
\emph{la loi est d'ordre public et r{\'e}pond {\`a} des motifs imp{\'e}rieux d'int{\'e}r{\^e}t g{\'e}n{\'e}ral}}
et une {\'e}tude juridique montre que l'article
\href{https://www.legifrance.gouv.fr/affichCodeArticle.do;jsessionid=B2842ED6626DE6921F9FD32EA69A3C93.tpdila15v_2?idArticle=LEGIARTI000033205794&cidTexte=LEGITEXT000006071190&dateTexte=20170124}{L533-4}
peut {\^e}tre consid{\'e}r{\'e} comme en faisant partie. Ainsi
le Conseil Scientifique du CNRS a vot{\'e} le \href{http://www.cnrs.fr/comitenational/doc/recommandations/2017/Reco_Interpetation_de_la_loi_numerique.pdf}{24 janvier 2017}, entre autres
recommandations, celle d'appliquer cet article 
de fa{\c c}on r{\'e}tro-active~!


La m{\^e}me loi contient un article (num{\'e}ro 38, int{\'e}gr{\'e} au \emph{Code de la propri{\'e}t{\'e} intellectuelle} dans l'article
\href{https://www.legifrance.gouv.fr/affichCodeArticle.do;jsessionid=6B4C28C0CE24D0F735EB9C7F3B081B16.tpdila15v_3?cidTexte=LEGITEXT000006069414&idArticle=LEGIARTI000033219347&dateTexte=20170124&categorieLien=cid#LEGIARTI000033219347}{L342-3})
dont le but est d'autoriser un chercheur {\`a} pratiquer la fouille de textes et de donn{\'e}es (\emph{Text \& Data Mining})
dans les contenus auxquels son institution est abonn{\'e}e~:  d{\`e}s qu'un abonnement sera conclu, les {\'e}diteurs devront mettre {\`a} la disposition d'organismes publics
d{\'e}sign{\'e}s par d{\'e}cret les donn{\'e}es concern{\'e}es par cet abonnement, afin de permettre cette fouille de textes et de donn{\'e}es. Toutefois les modalit{\'e}s d'application
de cet article ne sont pas encore connues, tant que le d{\'e}cret d'application ne sera pas publi{\'e} (cela est pr{\'e}vu courant janvier 2017).

\subsection{Accord globaux pour financer l'\emph{Open Access}}

Plut{\^o}t {\`a} l'oppos{\'e} des projets pr{\'e}c{\'e}dents, une tendance se dessine dans les pays du nord de l'Europe (le Royaume-Uni depuis
2013, suivi, {\`a} partir de 2016, par les Pays-Bas, l'Allemagne\footnote{Ces m{\^e}mes pays sont aussi ceux o{\`u} les g{\'e}ants
de l'{\'e}dition sont implant{\'e}s.}, l'Autriche...)~:
n{\'e}gocier au niveau national ou d'une institution des accords commerciaux avec les {\'e}diteurs pr{\'e}voyant de payer une somme forfaitaire une fois pour
toute {\`a} un {\'e}diteur, pour que les chercheurs de l'institution concern{\'e}e puisse publier chez cet {\'e}diteur en \emph{Open Access}.
Ces accords peuvent {\'e}galement
inclure des abonnements aux bouquets. Ce type d'accord pr{\'e}sente quelques avantages~: {\'e}liminer les possibles in{\'e}galit{\'e}s entre chercheurs au sein d'une m{\^e}me
institution, contr{\^o}ler le co{\^u}t de l'\emph{Open Access} (m{\^e}me si, en g{\'e}n{\'e}ral, il n'emp{\^e}che pas une augmentation globale des co{\^u}ts
comme cela est \href{http://www.eprist.fr/wp-content/uploads/2016/11/I-IST_24EtudeJISC.pdf}{observ{\'e}} au Royaume-Uni). Cependant ils comportent
bien des risques~: les plans de <<~basculement~>> propos{\'e}s reposent sur des analyses macro-{\'e}conomiques grossi{\`e}res, sans disposer de donn{\'e}es fines et pr{\'e}cises
(les montants des abonnements pay{\'e}s par les institutions sont en g{\'e}n{\'e}ral tenus secrets, quant aux prix que co{\^u}tent les APC pour les articles en
\emph{Open Access}, aucune institution n'est capable d'en donner une estimation! sauf peut-{\^e}tre au Royaume-Uni --- seuls les {\'e}diteurs connaissent
les chiffres). De plus, du fait que ces contrats sont pr{\'e}vus prioritairement avec certains {\'e}diteurs (en l'occurrence les plus gros) et pas les autres,
cela risque de fausser le march{\'e} de l'\emph{Open Access} en d{\'e}faveur des petits {\'e}diteurs (une fois de plus), puisque les APC pour ceux-ci devraient
{\^e}tre pay{\'e}s s{\'e}par{\'e}ment. Tout cela ne ferait qu'accro{\^\i}tre davantage la concentration de l'industrie de l'{\'e}dition
contre laquelle nous mettent en garde la COAR et
l'UNESCO dans leur 
\href{http://www.unesco.org/new/fileadmin/MULTIMEDIA/HQ/CI/CI/pdf/news/coar_unesco_oa_statement.pdf}{appel}.

Ainsi ces accords engageraient de fa{\c c}on irr{\'e}versible et massive les budgets des biblioth{\`e}ques, d{\'e}tournant ainsi ces moyens
de politiques de d{\'e}veloppement des mod{\`e}les d'{\'e}dition plus vertueux que nous avons mentionn{\'e}s plus haut et aboutissant {\`a}
une situation dans laquelle on n'aura pas rem{\'e}di{\'e} aux effets ind{\'e}sirables observ{\'e}s actuellement, notamment sur le plan scientifique,
et on aura confi{\'e} la gestion de ces probl{\`e}mes {\`a} de grandes entreprises commerciales.

\section{Au del{\`a} des publications}


\subsection{L'{\'e}valuation}

Apr{\`e}s cet {\'e}tat des lieux, il est bon de se demander pourquoi nous publions dans des revues dont le fonctionnement est si on{\'e}reux. Il appara{\^\i}t
clairement aujourd'hui que la raison premi{\`e}re n'est plus la diffusion des connaissances et des r{\'e}sultats de la recherche comme on pouvait le
clamer nagu{\`e}re, puisque, pour cela, il suffit de d{\'e}poser nos articles sur une banque de pr{\'e}publications comme \href{https://hal.archives-ouvertes.fr/}{HAL}
ou \href{https://arxiv.org/}{arXiv}. La raison est donc
la n{\'e}cessit{\'e} d'{\^e}tre {\'e}valu{\'e} par un comit{\'e} de r{\'e}daction et d'{\^e}tre ainsi reconnu. Une autre raison essentielle et r{\'e}elle pour publier dans des revues
est la constitution d'un corpus de connaissances stable et auquel les g{\'e}n{\'e}rations futures pourront se r{\'e}f{\'e}rer sans ambigu{\"\i}t{\'e}, mais il faut reconna{\^\i}tre
que cette seconde raison, beaucoup plus noble, n'est certainement pas la motivation premi{\`e}re du chercheur qui soumet un article {\`a} une revue.
Repenser le processus de l'{\'e}valuation, en s'affranchissant du joug des {\'e}diteurs priv{\'e}s, vendant cher journaux et outils d'{\'e}valuation <<~clefs en main~>>
mais mal ficel{\'e}s (par la bibliom{\'e}trie), est le d{\'e}fi des scientifiques de demain!

\subsection{Les r{\'e}seaux sociaux}

Les r{\'e}seaux sociaux scientifiques comme \emph{ResearchGate} (ou \emph{Academia}) offrent des possibilit{\'e}s tr{\`e}s int{\'e}ressantes pour acc{\'e}der
{\`a} des articles, des pr{\'e}publications~: l'inscription {\`a} ces r{\'e}seaux donne acc{\`e}s {\`a} un nombre croissant de tels documents,
ainsi qu'{\`a} des projets et des {\'e}changes scientifiques et permet d'y participer. Ces r{\'e}seaux sont tr{\`e}s efficaces, ainsi ils
op{\`e}rent automatiquement une fouille des publications se rapportant {\`a} un auteur sur la toile, l'aidant ainsi {\`a} constituer une banque
de textes dont il est l'auteur. On peut donc les utiliser avec profit.

Mais il faut cependant rester prudent et s'interroger sur certains points. Par exemple~: si un auteur d{\'e}pose une pr{\'e}publication sur un
tel r{\'e}seau, en conserve-t-il la propri{\'e}t{\'e}? S'il s'agit de la propri{\'e}t{\'e} intellectuelle et si le droit fran{\c c}ais s'applique,
la r{\'e}ponse est claire, car, gr{\^a}ce au code de la propri{\'e}t{\'e} intellectuelle, l'auteur est prot{\'e}g{\'e} et garde ind{\'e}finiment
la propri{\'e}t{\'e} d'un texte ou d'une {\oe}uvre.
En revanche la situation est plus floue en ce qui concerne la propri{\'e}t{\'e} patrimoniale~: l'auteur a-t-il le droit de signer un contrat de
publication avec un {\'e}diteur pour publier un texte r{\'e}dig{\'e} sur un tel r{\'e}seau social? Et inversement, le r{\'e}seau social peut-il
pr{\'e}tendre avoir des droits de publication sur ce texte? Il n'y a pas de r{\'e}ponse claire {\`a} ces questions {\`a} cause du vide juridique
sur le statut patrimonial de ces documents. Un risque est que ces r{\'e}seaux sociaux, dont l'usage est gratuit pour l'instant soient un jour
vendus {\`a} un gros {\'e}diteur, qui r{\'e}cup{\'e}rera ainsi des donn{\'e}es pr{\'e}cieuses comme les contenus scientifiques et aussi des
indices d'{\'e}valuation des chercheurs (tels que ceux produits automatiquement \emph{ResearchGate}). Le cas s'est d{\'e}j{\`a} produit avec notamment
le r{\'e}seau social \emph{Mendeley}, rachet{\'e} par Elsevier. 

D'autres moyens <<~libres~>>, mis au point par des {\'e}quipes qui n'ont pas de finalit{\'e} commerciale, sont offerts au chercheur. Ceux-ci sont
encore {\`a} l'{\'e}tat de projets et on peut esp{\'e}rer qu'ils se d{\'e}velopperont, si les institutions publiques donnent un petit coup de pouce.
L'un d'eux est le projet  \href{http://dissem.in/}{\emph{dissemin}}, il permet {\`a} un chercheur de se constituer tr{\`e}s rapidement une banque
d'articles dont il est l'auteur. Un autre projet int{\'e}ressant est le  \href{http://sjscience.org/}{\emph{Self Journal of Science}}, qui repose
sur un concept original et qui pourrait {\^e}tre une alternative int{\'e}ressante au processus d'{\'e}valuation traditionnel.

\subsection{Les portails}
Outre les syst{\`e}mes d'acc{\`e}s {\'e}lectronique {\`a} la documentation mis {\`a} la disposition des chercheurs
par leurs biblioth{\`e}ques ou leurs Services Communs de la Documentation, des portails nationaux ou europ{\'e}ens leur sont {\'e}galement propos{\'e}s.
Le \href{https://portail.math.cnrs.fr/}{\emph{Portail Math}} est d{\'e}velopp{\'e} par l'INSMI via la cellule \href{http://www.mathdoc.fr/}{\emph{Mathdoc}},
avec le soutien du r{\'e}seau \href{https://www.mathrice.fr/}{\emph{Mathrice}} et du \href{http://www.rnbm.org/}{\emph{RNBM}}.
Un de ses objectifs est d'offrir un acc{\`e}s personnalis{\'e} et simple {\`a} la documentation math{\'e}matique.
Il offre {\'e}galement des services num{\'e}riques.

A un niveau interdisciplinaire, le portail \href{https://bib.cnrs.fr/}{\emph{BibCnrs}},
refond{\'e} r{\'e}cemmment, donne acc{\`e}s aux ressources documentaires acquises par
l'\href{http://www.inist.fr/}{\emph{Inist}} pour le compte du CNRS.

Enfin le portail \href{https://eudml.org/}{\emph{EuDML}} offre une collection d'articles
en acc{\`e}s libre mise {\`a} disposition par un r{\'e}seau europ{\'e}en d'institutions.
